Quantum many-body physics is a very challenging research field that investigates systems populated by multiple particles interacting between each other. Among its numerous application fields we can find quantum chemistry, condensed matter physics and materials science, which brought to an increasing interest towards this subject in the scientific community. A further push in this direction was provided by the improvement in the computational performance of modern machines achieved in the last decades: this contribution revealed to be fundamental, since the complexity of the treated systems requires computational approaches. 

A quantum many-body problem can be faced with different methods, the choice of which depends on the features of the system in exam. One of the most famous approaches is constituted by the Hartree-Fock method, which was introduced for the first time by Hartree in 1928 \cite{hartree_1928}. Two years later the procedure was generalized independently by Slater \cite{slater} and Fock \cite{fock}, who included in the mathematical treatment the anti-symmetry of the wavefunction required by the Pauli exclusion principle for a system of fermions. The approach essentially allows to find an approximated solution to the Schr\"odinger equation, providing thus with an estimate for both the wavefunction and consequently the energy of the considered system. This is achieved by mean of the minimization of the energy functional with respect to the single-particle wavefunctions: this produces a system of equations in which each particle is under the action of the mean-field potential generated by the others. A self-consistency requirement is then imposed, leading thus to an iterative procedure that returns the single-particle wavefunctions entering in the construction of the Slater determinant for the system.

However, the simple Hartree-Fock method fails when it aims to describe more complex systems, involving for example the dynamics of a group of electrons illuminated by an intense laser source. An attempt to face this lack of accuracy was faced by Zanghellini et al. in \cite{Zanghellini_2003} with the introduction of the so-called multi-configuration time-dependent Hartree-Fock (MCTDHF) method. As suggested by the name, this new approach is based on the adoption of multiple Slater determinants (or configurations) for the description of the system. The accuracy achievable with the introduction of this hypothesis was tested again by Zanghellini et al. in \cite{Zanghellini_2004}: here the MCTDHF was adopted to access the properties of two systems, respectively constituted by a couple of electrons in a harmonic oscillator potential and a Helium atom. In both the cases a time-dependent potential representing an external laser source was also added. \\

In the context of this project, we adopted the standard time-dependent Hartree-Fock method to reproduce the results obtained by Zanghellini et al. for $\eta=1$, namely when just one Slater determinant is employed for the description of the system. Limiting our analysis to the 2-electron case, we first accessed the main features of the apparatus in its ground-state at $t=0$. The so-obtained results served as seed at a later stage, when the time-dependence represented by the laser field in the Hamiltonian was included. Further analysis not directly related to the content of the paper were also conducted: a more detailed description will be provided in the next section. The results were achieved on our side through the implementation of a Python code relying on the modules provided in \cite{gitOyvind}: this library allows us to face the problem by expanding the unknown single-particle wavefunctions in series of eigenstates of the harmonic oscillator (HO) hamiltonian. The task was then reduced to the determination and manipulation of the coefficients entering in this expansion.

\subsection{OVERVIEW OF THE WORKFLOW}
After this brief general introduction, we will proceed in the next section with the description of the main features of our system, together with the theory elements exploited in the context of this project. In particular, we will focus on the derivation of the time-independent and time-dependent Hartree-Fock equations, together with the main aspects behind the Fourier analysis of the time-dependent dipole moment and overlap. Later, Section \ref{sec:methods} introduces the expansion of the single particle states on a properly chosen basis, arriving then to the Ruthaan-Hall equations and their time-dependent generalization. The structure of the code and the main features of the \texttt{quantum-system} library appear in Section \ref{sec:code}, followed then by the description of the results and related discussions respectively in Section \ref{sec:results} and Section \ref{sec:discussion}. Finally, some possible further developments of this project are described in Section \ref{sec:improvements}, concluding then with the main learning points in Section \ref{sec:conclusions}.
