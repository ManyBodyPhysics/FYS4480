\begin{abstract}
    The main aim of this project was to reproduce the results obtained by Zanghellini et al. in \cite{Zanghellini_2004}: here they exploited the Hartree-Fock equations for the analysis of a system of two interacting electrons trapped in a harmonic oscillator potential and illuminated by a laser source. We accessed the ground state properties by mean of a time-independent solver, implemented using both the generalized and restricted spin representation for the single-particle wavefunctions. Their radial part was expanded using the eigenstates of an harmonic hamiltonian. The convergence features associated to the two representations were also analyzed, highlighting the possibility for the general system to explore lower energy regions. A time-dependent solver was also implemented, but limited to the general representation. All the results in terms of ground state energy, one-body density and overlap integral $\bracket{\Psi(t)}{\Psi(0)}$ (being $\Psi$ the Slater determinant for the system) were consistent with the corresponding estimates contained in the mentioned article. The time dependent dipole operator $\bracketOP{\Psi(t)}{\Psi(t)}{\hat{x}}$ was also analyzed for multiple values of the laser frequency $\omega$.
    
    We proceeded further by evolving the system illuminated by a laser for a limited time $T$, then switching the source off. A Fourier analysis was performed on both the signals $\bracketOP{\Psi(t)}{\Psi(t)}{\hat{x}}$ and $\vert \bracket{\Psi(t)}{\Psi(T)} \vert^2$ considered for $t>T$, revealing the relations between the main peaks in the so-obtained spectra and both the frequency of the harmonic trap $\Omega$ and of the laser $\omega$.
\end{abstract}