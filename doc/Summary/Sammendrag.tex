\documentclass[a4paper,10pt]{report}
\usepackage[utf8]{inputenc}

\usepackage{hyperref}

\usepackage{amsmath,amsthm}
\usepackage{ dsfont }

\usepackage{tikz}
\usepackage{verbatim}
\usepackage{simpler-wick}
%\usepackage[lf]{MinionPro}
\usepackage[scr=rsfso,calscaled=.96]{mathalfa}


\newcommand{\One}{\hat{\mathbf{1}}} \newcommand{\eff}{\text{eff}}
\newcommand{\Heff}{\hat{H}_\text{eff}}
\newcommand{\Veff}{\hat{V}_\text{eff}}
\newcommand{\braket}[1]{\langle#1\rangle}
\newcommand{\Span}{\operatorname{sp}}
\newcommand{\tr}{\operatorname{trace}}
\newcommand{\diag}{\operatorname{diag}}
\newcommand{\bra}[1]{\left\langle #1 \right|}
\newcommand{\ket}[1]{\left| #1 \right\rangle} \newcommand{\element}[3]
{\bra{#1}#2\ket{#3}}

\usepackage{amsmath}

% Title Page
\title{Summary FYS4480}
\author{}


\begin{document}
\maketitle

\begin{itemize}
 \item Slater determinanter
 \item Kreasjon og annihilasjonsoperatorer
 \item Generalisert Wicks teorem
 \item Normal ordning relativt til HF/Fermi vakuum
 \item Normal ordnet Hamilton operator (relativt til HF/Fermi vakuum)
 \item FCI/CI
 \item Hartree-Fock
 \item Coupled Cluster 
 \item MPBT
\end{itemize}

The anticommutator of two operators is defined by
\begin{equation}
 \{\hat{A},\hat{B}\} = \hat{A}\hat{B}+\hat{B}\hat{A}.                      
\end{equation}
The creation and annihilation operators satisfy the so-called "fundamental anticommutator" relations
\begin{align}
 \{a_p^\dagger,a_q^\dagger \} &= 0\\
 \{a_p,a_q \} &= 0\\
 \{a_p,a_q^\dagger \} &= \delta_{pq}.\\
\end{align}

The matrix representation of a one-body operator, relative to a single particle basis $\{ \phi_i \}_{i=1}$, is given by 
\begin{equation}
 \braket{\phi_p|\hat{h}|\phi_q} \equiv \int \phi^*_p(\mathbf{r}) \hat{h} \phi_q(\mathbf{r}) d\mathbf{r}.
\end{equation}
Common notations are $\braket{\phi_p|\hat{h}|\phi_q}=\braket{p|\hat{h}|q}=h^p_q$. The second quantized form a one-body operator 
is given by
\begin{equation}
 \hat{H}_0 = \sum_{pq}\braket{p|\hat{h}|q} a_p^\dagger a_q.
\end{equation}
Likewise the matrix representation of a two-body operator, typically the coulomb interaction, is given by 
\begin{equation}
 \braket{\phi_p \phi_q|\hat{v}|\phi_r \phi_s} = \int \phi^*_p(\mathbf{r_1}) \phi^*_q(\mathbf{r_2}) \hat{v}(\mathbf{r}_1,\mathbf{r}_2) \phi_q(\mathbf{r_1}) \phi_s(\mathbf{r_2}) d \mathbf{r}_1 d \mathbf{r}_2.
\end{equation}
It is common to write $\braket{\phi_p \phi_q|\hat{v}|\phi_r \phi_s} = \braket{pq|\hat{v}|rs} = v^{pq}_{rs}$. Furthermore, it is customary 
to introduce the \textit{anti-symmetric} matrix element
\begin{equation}
 \braket{pq|\hat{v}|rs}_{AS} \equiv \braket{pq|\hat{v}|rs} - \braket{pq|\hat{v}|sr}.
\end{equation}
Here it should be noted that many sources drop the AS subscript, which can be a source of confusion. The second quantized form a 
two-body operator is given by
\begin{align}
 \hat{W} &= \frac{1}{2} \sum_{pqrs} \braket{pq|\hat{v}|rs} a_p^\dagger a_q^\dagger a_s a_r \\
 &= \frac{1}{4}\sum_{pqrs} \braket{pq|\hat{v}|rs}_{AS} a_p^\dagger a_q^\dagger a_s a_r 
\end{align}
Thus, the full Hamiltonian $\hat{H} = \hat{H}_0 + \hat{W}$, using anti-symmetric matrix elements is given by 
\begin{equation}
 \hat{H} =  \sum_{pq}\braket{p|\hat{h}|q} a_p^\dagger a_q + \frac{1}{4}\sum_{pqrs} \braket{pq|\hat{v}|rs}_{AS} a_p^\dagger a_q^\dagger a_s a_r.
\end{equation}

\subsection*{Wicks theorem}
A sequence of creation and annihilation operators is called an operator string. Generally an operator 
string of $n$ creation and annihilation operators are on the form 
\begin{equation}
 A_1 A_2 \cdots A_n, \ \ A_i \in \{a_p,a_p^\dagger\}
\end{equation}
For example
$$A_1A_2A_3A_4 = a_p a_q^\dagger a_r a_s^\dagger,$$
where $A_1 = a_p, A_2 = a_q^\dagger, A_3 = a_r$ and $A_4 = a_s^\dagger$.

Furthermore we refer to the number 
\begin{equation}
 \braket{-|A_1 A_2 \cdots A_n|-}
\end{equation}
as a vacuum expecatation value, where $A_1A_2 \cdots A_n$ is an operator string. 

The normal-ordered product form of an operator 
string $A_1A_2\cdots A_n$ is defined as a rearrangement,
\begin{equation}
 \{A_1A_2\cdots A_n \} \equiv (-1)^{|\sigma|}A_{\sigma(1)}A_{\sigma(2)}\cdots A_{\sigma(n)},
\end{equation}
where $\sigma$ is a permutation such that all the creation operators in the operator string is to the left of all the
annihilation operators, i.e.,
\begin{equation}
 \{A_1A_2\cdots A_n \} \equiv (-1)^{|\sigma|}[\text{creation operators}]\cdot[\text{annihilation operators}].
\end{equation}
The permutation $\sigma$ is in general not be unique, since we may permute the creation and annihilation
operators separately without affecting the total expression. For example,
\begin{equation}
 \{a_p a_q^\dagger a_r^\dagger a_s \} = a_q^\dagger a_r^\dagger a_p a_s = -a_r^\dagger a_q^\dagger a_p a_s = a_r^\dagger a_q^\dagger a_s a_p = -a_q^\dagger a_r^\dagger a_s a_p
\end{equation}
To find the permutation $\sigma$ that normal-orders an operator product, it is usually simplest to count the
number f of anticommutations necessary to achieve the rearrangement, and set $(-1)^{|\sigma |} = (-1)^f$.
Note that the string $A_1\cdots A_n \neq \{ A_1 \cdots A_n \}$ in general, since by reordering creation and annihilation
operators we neglect the extra terms arising from the Kronecker delta in the anti-commutator relation
$\{ a_p , a_q^\dagger \} = \delta_{pq}$.

A contraction between to arbitrary creation and annihilation operators $X$
and  $Y$ is the number defined by
\begin{equation}
 \wick{\c X \c Y} = \braket{-|X Y|-}. 
\end{equation}
We list the possible contractions, relative to the vacuum state $\ket{-}$,
\begin{align}
 \wick{\c a_p^\dagger \c a_q^\dagger} &= \braket{-|a_p^\dagger a_q^\dagger|-} = 0 \\
 \wick{\c a_p \c a_q} &= \braket{-|a_p a_q|-} = 0 \\
 \wick{\c a_p^\dagger \c a_q} &= \braket{-|a_p^\dagger a_q|-} = 0 \\ 
 \wick{\c a_p \c a_q^\dagger} &= \braket{-|a_p a_q^\dagger|-} = 0
\end{align}

Wick’s theorem states that every string of creation and annihilation operators can be written as a sum of
normal-ordered products where we perform every possible contraction. Let $A_1 \cdots A_n$ 
be an operator string of creation and annihilation operators. Then,
\begin{align*}
 A_1A_2\cdots A_n &= \{ A_1 A_2 \cdots A_n \} + \sum_{(1)} \{ A_1 \wick{ \c... \c... ... } A_n \}
 &+ \sum_{(2)} \{ A_1 \wick{ .\c1.\c2. .\c1.\c2. ... } \} + \cdots \\
 &+ \sum_{\left \lfloor \frac{n}{2} \right \rfloor } \underbrace{\{ A_1 \wick{ .\c1.\c3. \c5.\c2.\c4. \c4.\c3.\c5. \c1... .\c2.. ...} A_n \}}_\text{$\left \lfloor \frac{n}{2} \right \rfloor$ contractions}
\end{align*}
The notation $\sum_{(m)}$ signifies that we sum over all combinations of $m$ contractions.
When n is even, the last sum signifies that we sum over $n/2$ contractions, i.e., all operators are
contracted. If n is odd, there is one uncontracted operator left in each term of the last sum.

Vacuum expecatation values simplify greatly using Wicks theorem. First note that for any string with at least one factor 
\begin{equation}
 \braket{-|\{ A_1 \cdots A_n \}|-} = 0,
\end{equation}
because in the normal-order product, the annihilation operators are to the right, and the creation
operators are on the left. For odd $n$, Wicks theorem gives
\begin{equation}
 \braket{-|A_1 \cdots A_n|-} = 0,
\end{equation}
while for even $n$,
\begin{equation}
 \braket{-|A_1 \cdots A_n|-} = \sum_{\left \lfloor \frac{n}{2} \right \rfloor } \underbrace{\{ A_1 \wick{ .\c1.\c3. \c5.\c2.\c4. \c4.\c3.\c5. \c1... .\c2.. ...} A_n \}}_\text{all contracted}.
\end{equation}
A useful fact is that the sign of a fully contracted operator product is $(−1)^k$, where $k$ is the
number of contraction line crossings.

\end{document}          
