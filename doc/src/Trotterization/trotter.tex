\documentclass{beamer}
\usepackage[utf8]{inputenc}
\usepackage{amsmath,amssymb}
\usepackage{listings}
\usepackage{xcolor}

\title{Trotterization and Suzuki Decomposition}
\subtitle{Quantum Time Evolution Simulation}
\author{}
\date{}

\begin{document}

\begin{frame}
\titlepage
\end{frame}

\begin{frame}{Outline}
\tableofcontents
\end{frame}

\section{Introduction}

\begin{frame}{Quantum Hamiltonian Evolution}

The time evolution operator for a quantum system is $U(t)=e^{-iHt}$, solving the Schrödinger equation $i,\frac{d}{dt}|\psi(t)\rangle = H|\psi(t)\rangle$ . Simulating $U(t)$ is essential in physics and chemistry .
Many Hamiltonians are a sum of terms, $H=\sum_j H_j$ . If all terms commute, time evolution factorizes exactly: e.g. for $H=H_1+H_2$ with $[H_1,H_2]=0$, we have $e^{-i(H_1+H_2)t}=e^{-iH_1 t}e^{-iH_2 t}$.
In general $H_j$ \textit{do not} commute, so $e^{-i(H_1+H_2)t}\neq e^{-iH_1t}e^{-iH_2t}$. We need to approximate the evolution by alternating the non-commuting pieces in small time slices.
\end{frame}


\begin{frame}{Trotter Product Formula}
e^{-i(H_1+H_2)t} = \lim_{N\to\infty}\Big(e^{-iH_1 \frac{t}{N}}\;e^{-iH_2 \frac{t}{N}}\Big)^N\,.

This is the basic Trotter-Suzuki decomposition (first-order splitting) . In the infinite step limit, it becomes exact (also known as the Lie product formula or Trotter formula ).
For finite $N$, $(e^{-iH_1 t/N}e^{-iH_2 t/N})^N$ approximates $e^{-i(H_1+H_2)t}$ with some error. Using a finite $N$ steps is called Trotterization, and the approximation error can be bounded by a desired $\epsilon$ .
\end{frame}


\begin{frame}{Higher-Order Suzuki Decompositions}

By symmetrizing the sequence, we can cancel lower-order errors. For example, a second-order formula uses half-step kicks of $H_1$:
S_{2}(\Delta) \;=\; e^{-iH_1 \Delta/2}\;e^{-iH_2 \Delta}\;e^{-iH_1 \Delta/2}\,,
which yields $e^{-i(H_1+H_2)\Delta}$ up to $O(\Delta^3)$ error . This symmetric Trotter-Suzuki formula eliminates the $O(\Delta^2)$ term.
In general, there are higher even-order formulas ($4$th, $6$th, …) that achieve errors $O(\Delta^{p+1})$ for any desired order $p$ . These higher-order decompositions (derived recursively by Suzuki) require more instances of the exponential operators (and sometimes negative-time coefficients) to cancel lower-order commutator errors.
\end{frame}


\section{Detailed Examples and Derivations}

\begin{frame}{First-Order Trotter Expansion (Derivation)}
Using the Baker–Campbell–Hausdorff (BCH) formula, one finds:
e^{A}e^{B} = \exp\!\Big(A + B + \tfrac{1}{2}[A,B] + \tfrac{1}{12}[A,[A,B]] - \tfrac{1}{12}[B,[A,B]] + \cdots\Big) .
For $A=-iH_1 \Delta,;B=-iH_2 \Delta$:
e^{-iH_1 \Delta}e^{-iH_2 \Delta} = \exp\!\Big(-i(H_1+H_2)\Delta \;-\; \tfrac{1}{2}[H_1,H_2]\,\Delta^2 + O(\Delta^3)\Big)\,.
Thus, a single Trotter step incurs a local error term $-\frac{1}{2}[H_1,H_2]\Delta^2$. The leading error scales as $O(\Delta^2)$, so after $N=t/\Delta$ steps the total error is $O(t,\Delta)$ (first order in $\Delta$).
\end{frame}

\begin{frame}{Second-Order Trotter Expansion (Insight)}

In the symmetric product $S_2(\Delta)=e^{-iH_1\Delta/2}e^{-iH_2\Delta}e^{-iH_1\Delta/2}$, the first-order commutator terms cancel out. Intuitively, the $[H_1,H_2]$ error from the first half-step is negated by the second half-step.
The leading error in $S_2$ involves double commutators like $[H_1,[H_1,H_2]]$ (and $[H_2,[H_1,H_2]]$) , which enter at order $O(\Delta^3)$. Thus the second-order scheme has local error $O(\Delta^3)$ (global error $O(\Delta^2)$), a significant improvement over first order.
\end{frame}


\begin{frame}{Example: Single-Qubit $H = X + Z$}

Consider a single qubit with Hamiltonian $H = \sigma_X + \sigma_Z$ (Pauli $X$ and $Z$). Here $[X,Z] = 2iY \neq 0$, so $X$ and $Z$ do not commute . We cannot implement $e^{-i(X+Z)t}$ as one gate, but must Trotterize.
Trotter strategy: alternate short rotations about the $X$-axis and $Z$-axis. For small $\Delta t$, $e^{-iX \Delta t}$ and $e^{-iZ \Delta t}$ are simpler rotations. Repeating them approximates the full evolution $e^{-i(X+Z)t}$ .
In this case, $e^{-iX \theta} = R_x(2\theta)$ and $e^{-iZ \theta} = R_z(2\theta)$, standard single-qubit rotations . Thus each Trotter step can be directly realized as two orthogonal axis rotations on the qubit.
\end{frame}


\begin{frame}[fragile]{Trotterization in Python (First-Order)}
\lstset{basicstyle=\ttfamily\footnotesize, language=Python}
\begin{lstlisting}
import numpy as np
from numpy.linalg import norm
from scipy.linalg import expm


Define Pauli matrices


X = np.array([[0, 1],
[1, 0]])
Z = np.array([[1, 0],
[0,-1]])
H = X + Z

t = 1.0
N = 4
dt = t/N


First-order Trotter approximation


U_trot = np.eye(2)
for k in range(N):
U_trot = expm(-1j * X * dt) @ expm(-1j * Z * dt) @ U_trot


Exact evolution


U_exact = expm(-1j * H * t)
error = norm(U_trot - U_exact)
print(error)
\end{lstlisting}
\end{frame}

\begin{frame}{Results: Trotter Approximation Error}

With $N=4$ time steps, the first-order Trotter approximation gives $|U_{\text{trot}} - U_{\text{exact}}| \approx 2.5\times10^{-1}$. Increasing to $N=16$ steps reduces the error to $\sim6\times10^{-2}$. Doubling $N$ roughly halves the error, consistent with $O(1/N)$ convergence (global error $\sim O(t/N)$ for first order).
A second-order Trotter scheme yields far smaller error for the same $N$. For example, at $N=4$ steps, the symmetric formula gives error $\sim2.4\times10^{-2}$ (about 10× smaller than first order). This faster convergence (error $\sim O(1/N^2)$) is evident in practice.
In general, each $e^{-iH_j \Delta t}$ corresponds to a quantum gate implementing that term. In this 1-qubit example, $e^{-iX\Delta t}$ and $e^{-iZ\Delta t}$ are rotations about $X$ and $Z$ axes. Thus the Trotterized $e^{-i(X+Z)t}$ can be realized as a sequence of short rotations, which becomes exact in the limit of fine steps .
\end{frame}


\begin{frame}{Error Scaling Comparison}
Error norm versus number of Trotter steps $N$ for first-order (Lie–Trotter) and second-order (symmetric) decomposition of $H=X+Z$. On a log–log plot, the first-order errors (yellow, circles) decrease linearly (slope $-1$), while second-order errors (red, squares) decrease with slope $-2$, confirming the $1/N$ vs $1/N^2$ scaling.
\end{frame}

\begin{frame}{Scaling of Trotter Steps with Accuracy}

The number of Trotter steps required grows as a function of the simulation time $t$ and desired accuracy $\epsilon$:

First order: global error $\sim O(t^2/N)$, so to achieve error $\epsilon$ one needs $N = O(t^2/\epsilon)$ steps (gate operations) .
Second order: global error $\sim O(t^3/N^2)$, so one needs $N = O!\big((t^3/\epsilon)^{1/2}\big) = O(t^{3/2}/\sqrt{\epsilon})$ steps for error $\epsilon$.

Higher-order Suzuki formulas further reduce the scaling. In practice, there is a trade-off: higher order means more gates per step. One chooses an order that minimizes total error (Trotter error + hardware errors) for a given quantum hardware .
\end{frame}


\section{Exercises}

\begin{frame}{Exercises}
\begin{enumerate}
\item Use the BCH expansion to show the leading correction term for $U_{\text{trot}}(\Delta) = e^{-iH_1\Delta}e^{-iH_2\Delta}$ is $-\tfrac{i}{2}[H_1,H_2]\Delta^2$.  (Hint: Expand $e^{-iH_1\Delta}e^{-iH_2\Delta}$ to second order in $\Delta$.)
\item Verify that in the second-order formula $S_2(\Delta)=e^{-iH_1\Delta/2}e^{-iH_2\Delta}e^{-iH_1\Delta/2}$, the $[H_1,H_2]$ term cancels out. What commutator(s) govern the leading error term of $S_2$?
\item Write a Python script (using NumPy) to simulate $U(t)=e^{-iHt}$ for a simple $2\times2$ Hamiltonian $H=H_1+H_2$ with and without Trotterization. Compare the norm error $|U_{\text{trot}} - U|$ for different $N$ and for first vs second-order Trotterization.
\end{enumerate}
\end{frame}

\end{document}
