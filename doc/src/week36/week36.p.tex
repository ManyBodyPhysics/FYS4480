%%
%% Automatically generated file from DocOnce source
%% (https://github.com/doconce/doconce/)
%% doconce format latex week36.do.txt --minted_latex_style=trac --latex_admon=paragraph --no_mako
%%
% #ifdef PTEX2TEX_EXPLANATION
%%
%% The file follows the ptex2tex extended LaTeX format, see
%% ptex2tex: https://code.google.com/p/ptex2tex/
%%
%% Run
%%      ptex2tex myfile
%% or
%%      doconce ptex2tex myfile
%%
%% to turn myfile.p.tex into an ordinary LaTeX file myfile.tex.
%% (The ptex2tex program: https://code.google.com/p/ptex2tex)
%% Many preprocess options can be added to ptex2tex or doconce ptex2tex
%%
%%      ptex2tex -DMINTED myfile
%%      doconce ptex2tex myfile envir=minted
%%
%% ptex2tex will typeset code environments according to a global or local
%% .ptex2tex.cfg configure file. doconce ptex2tex will typeset code
%% according to options on the command line (just type doconce ptex2tex to
%% see examples). If doconce ptex2tex has envir=minted, it enables the
%% minted style without needing -DMINTED.
% #endif

% #define PREAMBLE

% #ifdef PREAMBLE
%-------------------- begin preamble ----------------------

\documentclass[%
oneside,                 % oneside: electronic viewing, twoside: printing
final,                   % draft: marks overfull hboxes, figures with paths
10pt]{article}

\listfiles               %  print all files needed to compile this document

\usepackage{relsize,makeidx,color,setspace,amsmath,amsfonts,amssymb}
\usepackage[table]{xcolor}
\usepackage{bm,ltablex,microtype}

\usepackage[pdftex]{graphicx}

\usepackage[T1]{fontenc}
%\usepackage[latin1]{inputenc}
\usepackage{ucs}
\usepackage[utf8x]{inputenc}

\usepackage{lmodern}         % Latin Modern fonts derived from Computer Modern

% Hyperlinks in PDF:
\definecolor{linkcolor}{rgb}{0,0,0.4}
\usepackage{hyperref}
\hypersetup{
    breaklinks=true,
    colorlinks=true,
    linkcolor=linkcolor,
    urlcolor=linkcolor,
    citecolor=black,
    filecolor=black,
    %filecolor=blue,
    pdfmenubar=true,
    pdftoolbar=true,
    bookmarksdepth=3   % Uncomment (and tweak) for PDF bookmarks with more levels than the TOC
    }
%\hyperbaseurl{}   % hyperlinks are relative to this root

\setcounter{tocdepth}{2}  % levels in table of contents

% --- fancyhdr package for fancy headers ---
\usepackage{fancyhdr}
\fancyhf{} % sets both header and footer to nothing
\renewcommand{\headrulewidth}{0pt}
\fancyfoot[LE,RO]{\thepage}
% Ensure copyright on titlepage (article style) and chapter pages (book style)
\fancypagestyle{plain}{
  \fancyhf{}
  \fancyfoot[C]{{\footnotesize \copyright\ 1999-2023, Morten Hjorth-Jensen. Released under CC Attribution-NonCommercial 4.0 license}}
%  \renewcommand{\footrulewidth}{0mm}
  \renewcommand{\headrulewidth}{0mm}
}
% Ensure copyright on titlepages with \thispagestyle{empty}
\fancypagestyle{empty}{
  \fancyhf{}
  \fancyfoot[C]{{\footnotesize \copyright\ 1999-2023, Morten Hjorth-Jensen. Released under CC Attribution-NonCommercial 4.0 license}}
  \renewcommand{\footrulewidth}{0mm}
  \renewcommand{\headrulewidth}{0mm}
}

\pagestyle{fancy}


% prevent orhpans and widows
\clubpenalty = 10000
\widowpenalty = 10000

% --- end of standard preamble for documents ---


% insert custom LaTeX commands...

\raggedbottom
\makeindex
\usepackage[totoc]{idxlayout}   % for index in the toc
\usepackage[nottoc]{tocbibind}  % for references/bibliography in the toc

%-------------------- end preamble ----------------------

\begin{document}

% matching end for #ifdef PREAMBLE
% #endif

\newcommand{\exercisesection}[1]{\subsection*{#1}}


% ------------------- main content ----------------------



% ----------------- title -------------------------

\thispagestyle{empty}

\begin{center}
{\LARGE\bf
\begin{spacing}{1.25}
Week 36: Operators in second quantization, computation of expectation values and Wick's theorem
\end{spacing}
}
\end{center}

% ----------------- author(s) -------------------------

\begin{center}
{\bf Morten Hjorth-Jensen${}^{1, 2}$} \\ [0mm]
\end{center}

\begin{center}
% List of all institutions:
\centerline{{\small ${}^1$Department of Physics and Center for Computing in Science Education, University of Oslo, Norway}}
\centerline{{\small ${}^2$Department of Physics and Astronomy and Facility for Rare Isotope Beams, Michigan State University, USA}}
\end{center}
    
% ----------------- end author(s) -------------------------

% --- begin date ---
\begin{center}
Week 36, September 4-8, 2023
\end{center}
% --- end date ---

\vspace{1cm}


\subsection{Week 36}

\begin{itemize}
\item Topics to be covered
\begin{enumerate}

 \item Thursday: Second quantization, operators in second quantization and diagrammatic representation

 \item \href{{https://youtu.be/CQQ6DIuw0R8}}{Video of lecture}

 \item Friday: Second quantization and Wick's theorem

 \item \href{{https://youtu.be/}}{Video of lecture}

\end{enumerate}

\noindent
\item Lecture Material: These slides, handwritten notes and	Szabo and Ostlund sections 2.3 and 2.4.

\item \href{{https://github.com/ManyBodyPhysics/FYS4480/blob/master/doc/Exercises/2023/ExercisesWeek36.pdf}}{Second exercise set}
\end{itemize}

\noindent
\subsection{Second quantization, brief summary from week 35}

We can summarize  our findings from last week as 
\begin{equation}
	\{a_\alpha^{\dagger},a_\beta \} = \delta_{\alpha\beta} \label{eq:2-20}
\end{equation}
with  $\delta_{\alpha\beta}$ is the Kroenecker $\delta$-symbol.

The properties of the creation and annihilation operators can be summarized as (for fermions)
\[
	a_\alpha^{\dagger}|0\rangle \equiv  |\alpha\rangle,
\]
and
\[
	a_\alpha^{\dagger}|\alpha_1\dots \alpha_n\rangle_{\mathrm{AS}} \equiv  |\alpha\alpha_1\dots \alpha_n\rangle_{\mathrm{AS}}. 
\]
from which follows
\[
        |\alpha_1\dots \alpha_n\rangle_{\mathrm{AS}} = a_{\alpha_1}^{\dagger} a_{\alpha_2}^{\dagger} \dots a_{\alpha_n}^{\dagger} |0\rangle.
\]

The hermitian conjugate has the folowing properties
\[
        a_{\alpha} = ( a_{\alpha}^{\dagger} )^{\dagger}.
\]
Finally we found 
\[
	a_\alpha\underbrace{|\alpha_1'\alpha_2' \dots \alpha_{n+1}'}\rangle_{\neq \alpha} = 0, \quad
		\textrm{in particular } a_\alpha |0\rangle = 0,
\]
and
\[
 a_\alpha |\alpha\alpha_1\alpha_2 \dots \alpha_{n}\rangle = |\alpha_1\alpha_2 \dots \alpha_{n}\rangle,
\]
and the corresponding commutator algebra
\[
	\{a_{\alpha}^{\dagger},a_{\beta}^{\dagger}\} = \{a_{\alpha},a_{\beta}\} = 0 \hspace{0.5cm} 
\{a_\alpha^{\dagger},a_\beta \} = \delta_{\alpha\beta}.
\]

\subsection{One-body operators in second quantization}

A very useful operator is the so-called number-operator.  Most physics
cases we will study in this text conserve the total number of
particles.  The number operator is therefore a useful quantity which
allows us to test that our many-body formalism conserves the number of
particles.  In for example $(d,p)$ or $(p,d)$ reactions it is
important to be able to describe quantum mechanical states where
particles get added or removed.  A creation operator
$a_\alpha^{\dagger}$ adds one particle to the single-particle state
$\alpha$ of a give many-body state vector, while an annihilation
operator $a_\alpha$ removes a particle from a single-particle state
$\alpha$.

Let us consider an operator proportional with $a_\alpha^{\dagger} a_\beta$ and 
$\alpha=\beta$. It acts on an $n$-particle state 
resulting in
\begin{equation}
	a_\alpha^{\dagger} a_\alpha |\alpha_1\alpha_2 \dots \alpha_{n}\rangle = 
	\begin{cases}
		0  &\alpha \notin \{\alpha_i\} \\
		\\
		|\alpha_1\alpha_2 \dots \alpha_{n}\rangle & \alpha \in \{\alpha_i\}
	\end{cases}
\end{equation}
Summing over all possible one-particle states we arrive at
\begin{equation}
	\left( \sum_\alpha a_\alpha^{\dagger} a_\alpha \right) |\alpha_1\alpha_2 \dots \alpha_{n}\rangle = 
	n |\alpha_1\alpha_2 \dots \alpha_{n}\rangle \label{eq:2-21}
\end{equation}

The operator 
\begin{equation}
	\hat{N} = \sum_\alpha a_\alpha^{\dagger} a_\alpha \label{eq:2-22}
\end{equation}
is called the number operator since it counts the number of particles in a give state vector when it acts 
on the different single-particle states.  It acts on one single-particle state at the time and falls 
therefore under category one-body operators.
Next we look at another important one-body operator, namely $\hat{H}_0$ and study its operator form in the 
occupation number representation.

We want to obtain an expression for a one-body operator which conserves the number of particles.
Here we study the one-body operator for the kinetic energy plus an eventual external one-body potential.
The action of this operator on a particular $n$-body state with its pertinent expectation value has already
been studied in coordinate  space.
In coordinate space the operator reads
\begin{equation}
	\hat{H}_0 = \sum_i \hat{h}_0(x_i) \label{eq:2-23}
\end{equation}
and the anti-symmetric $n$-particle Slater determinant is defined as 
\[
\Phi(x_1, x_2,\dots ,x_n,\alpha_1,\alpha_2,\dots, \alpha_n)= \frac{1}{\sqrt{n!}} \sum_p (-1)^p\hat{P}\psi_{\alpha_1}(x_1)\psi_{\alpha_2}(x_2) \dots \psi_{\alpha_n}(x_n).
\]

Defining
\begin{equation}
	\hat{h}_0(x_i) \psi_{\alpha_i}(x_i) = \sum_{\alpha_k'} \psi_{\alpha_k'}(x_i) \langle\alpha_k'|\hat{h}_0|\alpha_k\rangle \label{eq:2-25}
\end{equation}
we can easily  evaluate the action of $\hat{H}_0$ on each product of one-particle functions in Slater determinant.
From Eq.~(\ref{eq:2-25})  we obtain the following result without  permuting any particle pair 
\begin{align}
	&& \left( \sum_i \hat{h}_0(x_i) \right) \psi_{\alpha_1}(x_1)\psi_{\alpha_2}(x_2) \dots \psi_{\alpha_n}(x_n) \nonumber \\
	& =&\sum_{\alpha_1'} \langle \alpha_1'|\hat{h}_0|\alpha_1\rangle 
		\psi_{\alpha_1'}(x_1)\psi_{\alpha_2}(x_2) \dots \psi_{\alpha_n}(x_n) \nonumber \\
	&+&\sum_{\alpha_2'} \langle \alpha_2'|\hat{h}_0|\alpha_2\rangle
		\psi_{\alpha_1}(x_1)\psi_{\alpha_2'}(x_2) \dots \psi_{\alpha_n}(x_n) \nonumber \\
	&+& \dots \nonumber \\
	&+&\sum_{\alpha_n'} \langle \alpha_n'|\hat{h}_0|\alpha_n\rangle
		\psi_{\alpha_1}(x_1)\psi_{\alpha_2}(x_2) \dots \psi_{\alpha_n'}(x_n) \label{eq:2-26}
\end{align}

If we interchange particles $1$ and $2$  we obtain
\begin{align}
	&& \left( \sum_i \hat{h}_0(x_i) \right) \psi_{\alpha_1}(x_2)\psi_{\alpha_1}(x_2) \dots \psi_{\alpha_n}(x_n) \nonumber \\
	& =&\sum_{\alpha_2'} \langle \alpha_2'|\hat{h}_0|\alpha_2\rangle 
		\psi_{\alpha_1}(x_2)\psi_{\alpha_2'}(x_1) \dots \psi_{\alpha_n}(x_n) \nonumber \\
	&+&\sum_{\alpha_1'} \langle \alpha_1'|\hat{h}_0|\alpha_1\rangle
		\psi_{\alpha_1'}(x_2)\psi_{\alpha_2}(x_1) \dots \psi_{\alpha_n}(x_n) \nonumber \\
	&+& \dots \nonumber \\
	&+&\sum_{\alpha_n'} \langle \alpha_n'|\hat{h}_0|\alpha_n\rangle
		\psi_{\alpha_1}(x_2)\psi_{\alpha_1}(x_2) \dots \psi_{\alpha_n'}(x_n) \label{eq:2-27}
\end{align}

We can continue by computing all possible permutations. 
We rewrite also our Slater determinant in its second quantized form and skip the dependence on the quantum numbers $x_i.$
Summing up all contributions and taking care of all phases
$(-1)^p$ we arrive at 
\begin{align}
	\hat{H}_0|\alpha_1,\alpha_2,\dots, \alpha_n\rangle &=& \sum_{\alpha_1'}\langle \alpha_1'|\hat{h}_0|\alpha_1\rangle
		|\alpha_1'\alpha_2 \dots \alpha_{n}\rangle \nonumber \\
	&+& \sum_{\alpha_2'} \langle \alpha_2'|\hat{h}_0|\alpha_2\rangle
		|\alpha_1\alpha_2' \dots \alpha_{n}\rangle \nonumber \\
	&+& \dots \nonumber \\
	&+& \sum_{\alpha_n'} \langle \alpha_n'|\hat{h}_0|\alpha_n\rangle
		|\alpha_1\alpha_2 \dots \alpha_{n}'\rangle \label{eq:2-28}
\end{align}

In Eq.~(\ref{eq:2-28}) 
we have expressed the action of the one-body operator
of Eq.~(\ref{eq:2-23}) on the  $n$-body state in its second quantized form.
This equation can be further manipulated if we use the properties of the creation and annihilation operator
on each primed quantum number, that is
\begin{equation}
	|\alpha_1\alpha_2 \dots \alpha_k' \dots \alpha_{n}\rangle = 
		a_{\alpha_k'}^{\dagger}  a_{\alpha_k} |\alpha_1\alpha_2 \dots \alpha_k \dots \alpha_{n}\rangle \label{eq:2-29}
\end{equation}
Inserting this in the right-hand side of Eq.~(\ref{eq:2-28}) results in
\begin{align}
	\hat{H}_0|\alpha_1\alpha_2 \dots \alpha_{n}\rangle &=& \sum_{\alpha_1'}\langle \alpha_1'|\hat{h}_0|\alpha_1\rangle
		a_{\alpha_1'}^{\dagger}  a_{\alpha_1} |\alpha_1\alpha_2 \dots \alpha_{n}\rangle \nonumber \\
	&+& \sum_{\alpha_2'} \langle \alpha_2'|\hat{h}_0|\alpha_2\rangle
		a_{\alpha_2'}^{\dagger}  a_{\alpha_2} |\alpha_1\alpha_2 \dots \alpha_{n}\rangle \nonumber \\
	&+& \dots \nonumber \\
	&+& \sum_{\alpha_n'} \langle \alpha_n'|\hat{h}_0|\alpha_n\rangle
		a_{\alpha_n'}^{\dagger}  a_{\alpha_n} |\alpha_1\alpha_2 \dots \alpha_{n}\rangle \nonumber \\
	&=& \sum_{\alpha, \beta} \langle \alpha|\hat{h}_0|\beta\rangle a_\alpha^{\dagger} a_\beta 
		|\alpha_1\alpha_2 \dots \alpha_{n}\rangle \label{eq:2-30a}
\end{align}

In the number occupation representation or second quantization we get the following expression for a one-body 
operator which conserves the number of particles
\begin{equation}
	\hat{H}_0 = \sum_{\alpha\beta} \langle \alpha|\hat{h}_0|\beta\rangle a_\alpha^{\dagger} a_\beta \label{eq:2-30b}
\end{equation}
Obviously, $\hat{H}_0$ can be replaced by any other one-body  operator which preserved the number
of particles. The stucture of the operator is therefore not limited to say the kinetic or single-particle energy only.

The opearator $\hat{H}_0$ takes a particle from the single-particle state $\beta$  to the single-particle state $\alpha$ 
with a probability for the transition given by the expectation value $\langle \alpha|\hat{h}_0|\beta\rangle$.

It is instructive to verify Eq.~(\ref{eq:2-30b}) by computing the expectation value of $\hat{H}_0$ 
between two single-particle states
\begin{equation}
	\langle \alpha_1|\hat{h}_0|\alpha_2\rangle = \sum_{\alpha\beta} \langle \alpha|\hat{h}_0|\beta\rangle
		\langle 0|a_{\alpha_1}a_\alpha^{\dagger} a_\beta a_{\alpha_2}^{\dagger}|0\rangle \label{eq:2-30c}
\end{equation}

Using the commutation relations for the creation and annihilation operators we have 
\begin{equation}
a_{\alpha_1}a_\alpha^{\dagger} a_\beta a_{\alpha_2}^{\dagger} = (\delta_{\alpha \alpha_1} - a_\alpha^{\dagger} a_{\alpha_1} )(\delta_{\beta \alpha_2} - a_{\alpha_2}^{\dagger} a_{\beta} ), \label{eq:2-30d}
\end{equation}
which results in
\begin{equation}
\langle 0|a_{\alpha_1}a_\alpha^{\dagger} a_\beta a_{\alpha_2}^{\dagger}|0\rangle = \delta_{\alpha \alpha_1} \delta_{\beta \alpha_2} \label{eq:2-30e}
\end{equation}
and
\begin{equation}
\langle \alpha_1|\hat{h}_0|\alpha_2\rangle = \sum_{\alpha\beta} \langle \alpha|\hat{h}_0|\beta\rangle\delta_{\alpha \alpha_1} \delta_{\beta \alpha_2} = \langle \alpha_1|\hat{h}_0|\alpha_2\rangle \label{eq:2-30f}
\end{equation}

\subsection{Two-body operators in second quantization}

Let us now derive the expression for our two-body interaction part, which also conserves the number of particles.
We can proceed in exactly the same way as for the one-body operator. In the coordinate representation our
two-body interaction part takes the following expression
\begin{equation}
	\hat{H}_I = \sum_{i < j} V(x_i,x_j) \label{eq:2-31}
\end{equation}
where the summation runs over distinct pairs. The term $V$ can be an interaction model for the nucleon-nucleon interaction
or the interaction between two electrons. It can also include additional two-body interaction terms. 

The action of this operator on a product of 
two single-particle functions is defined as 
\begin{equation}
	V(x_i,x_j) \psi_{\alpha_k}(x_i) \psi_{\alpha_l}(x_j) = \sum_{\alpha_k'\alpha_l'} 
		\psi_{\alpha_k}'(x_i)\psi_{\alpha_l}'(x_j) 
		\langle \alpha_k'\alpha_l'|\hat{v}|\alpha_k\alpha_l\rangle \label{eq:2-32}
\end{equation}

We can now let $\hat{H}_I$ act on all terms in the linear combination for $|\alpha_1\alpha_2\dots\alpha_n\rangle$. Without any permutations we have
\begin{align}
	&& \left( \sum_{i < j} V(x_i,x_j) \right) \psi_{\alpha_1}(x_1)\psi_{\alpha_2}(x_2)\dots \psi_{\alpha_n}(x_n) \nonumber \\
	&=& \sum_{\alpha_1'\alpha_2'} \langle \alpha_1'\alpha_2'|\hat{v}|\alpha_1\alpha_2\rangle
		\psi_{\alpha_1}'(x_1)\psi_{\alpha_2}'(x_2)\dots \psi_{\alpha_n}(x_n) \nonumber \\
	& +& \dots \nonumber \\
	&+& \sum_{\alpha_1'\alpha_n'} \langle \alpha_1'\alpha_n'|\hat{v}|\alpha_1\alpha_n\rangle
		\psi_{\alpha_1}'(x_1)\psi_{\alpha_2}(x_2)\dots \psi_{\alpha_n}'(x_n) \nonumber \\
	& +& \dots \nonumber \\
	&+& \sum_{\alpha_2'\alpha_n'} \langle \alpha_2'\alpha_n'|\hat{v}|\alpha_2\alpha_n\rangle
		\psi_{\alpha_1}(x_1)\psi_{\alpha_2}'(x_2)\dots \psi_{\alpha_n}'(x_n) \nonumber \\
	 & +& \dots \label{eq:2-33}
\end{align}
where on the rhs we have a term for each distinct pairs. 

For the other terms on the rhs we obtain similar expressions  and summing over all terms we obtain
\begin{align}
	H_I |\alpha_1\alpha_2\dots\alpha_n\rangle &=& \sum_{\alpha_1', \alpha_2'} \langle \alpha_1'\alpha_2'|\hat{v}|\alpha_1\alpha_2\rangle
		|\alpha_1'\alpha_2'\dots\alpha_n\rangle \nonumber \\
	&+& \dots \nonumber \\
	&+& \sum_{\alpha_1', \alpha_n'} \langle \alpha_1'\alpha_n'|\hat{v}|\alpha_1\alpha_n\rangle
		|\alpha_1'\alpha_2\dots\alpha_n'\rangle \nonumber \\
	&+& \dots \nonumber \\
	&+& \sum_{\alpha_2', \alpha_n'} \langle \alpha_2'\alpha_n'|\hat{v}|\alpha_2\alpha_n\rangle
		|\alpha_1\alpha_2'\dots\alpha_n'\rangle \nonumber \\
	 &+& \dots \label{eq:2-34}
\end{align}

We introduce second quantization via the relation
\begin{align}
	&& a_{\alpha_k'}^{\dagger} a_{\alpha_l'}^{\dagger} a_{\alpha_l} a_{\alpha_k} 
		|\alpha_1\alpha_2\dots\alpha_k\dots\alpha_l\dots\alpha_n\rangle \nonumber \\
	&=& (-1)^{k-1} (-1)^{l-2} a_{\alpha_k'}^{\dagger} a_{\alpha_l'}^{\dagger} a_{\alpha_l} a_{\alpha_k}
		|\alpha_k\alpha_l \underbrace{\alpha_1\alpha_2\dots\alpha_n}_{\neq \alpha_k,\alpha_l}\rangle \nonumber \\
	&=& (-1)^{k-1} (-1)^{l-2} 
	|\alpha_k'\alpha_l' \underbrace{\alpha_1\alpha_2\dots\alpha_n}_{\neq \alpha_k',\alpha_l'}\rangle \nonumber \\
	&=& |\alpha_1\alpha_2\dots\alpha_k'\dots\alpha_l'\dots\alpha_n\rangle \label{eq:2-35}
\end{align}

Inserting this in (\ref{eq:2-34}) gives
\begin{align}
	H_I |\alpha_1\alpha_2\dots\alpha_n\rangle
	&=& \sum_{\alpha_1', \alpha_2'} \langle \alpha_1'\alpha_2'|\hat{v}|\alpha_1\alpha_2\rangle
		a_{\alpha_1'}^{\dagger} a_{\alpha_2'}^{\dagger} a_{\alpha_2} a_{\alpha_1}
		|\alpha_1\alpha_2\dots\alpha_n\rangle \nonumber \\
	&+& \dots \nonumber \\
	&=& \sum_{\alpha_1', \alpha_n'} \langle \alpha_1'\alpha_n'|\hat{v}|\alpha_1\alpha_n\rangle
		a_{\alpha_1'}^{\dagger} a_{\alpha_n'}^{\dagger} a_{\alpha_n} a_{\alpha_1}
		|\alpha_1\alpha_2\dots\alpha_n\rangle \nonumber \\
	&+& \dots \nonumber \\
	&=& \sum_{\alpha_2', \alpha_n'} \langle \alpha_2'\alpha_n'|\hat{v}|\alpha_2\alpha_n\rangle
		a_{\alpha_2'}^{\dagger} a_{\alpha_n'}^{\dagger} a_{\alpha_n} a_{\alpha_2}
		|\alpha_1\alpha_2\dots\alpha_n\rangle \nonumber \\
	&+& \dots \nonumber \\
	&=& \sum_{\alpha, \beta, \gamma, \delta} ' \langle \alpha\beta|\hat{v}|\gamma\delta\rangle
		a^{\dagger}_\alpha a^{\dagger}_\beta a_\delta a_\gamma
		|\alpha_1\alpha_2\dots\alpha_n\rangle \label{eq:2-36}
\end{align}

Here we let $\sum'$ indicate that the sums running over $\alpha$ and $\beta$ run over all
single-particle states, while the summations  $\gamma$ and $\delta$ 
run over all pairs of single-particle states. We wish to remove this restriction and since
\begin{equation}
	\langle \alpha\beta|\hat{v}|\gamma\delta\rangle = \langle \beta\alpha|\hat{v}|\delta\gamma\rangle \label{eq:2-37}
\end{equation}
we get
\begin{align}
	\sum_{\alpha\beta} \langle \alpha\beta|\hat{v}|\gamma\delta\rangle a^{\dagger}_\alpha a^{\dagger}_\beta a_\delta a_\gamma &=& 
		\sum_{\alpha\beta} \langle \beta\alpha|\hat{v}|\delta\gamma\rangle 
		a^{\dagger}_\alpha a^{\dagger}_\beta a_\delta a_\gamma \label{eq:2-38a} \\
	&=& \sum_{\alpha\beta}\langle \beta\alpha|\hat{v}|\delta\gamma\rangle
		a^{\dagger}_\beta a^{\dagger}_\alpha a_\gamma a_\delta \label{eq:2-38b}
\end{align}
where we  have used the anti-commutation rules.

Changing the summation indices 
$\alpha$ and $\beta$ in (\ref{eq:2-38b}) we obtain
\begin{equation}
	\sum_{\alpha\beta} \langle \alpha\beta|\hat{v}|\gamma\delta\rangle a^{\dagger}_\alpha a^{\dagger}_\beta a_\delta a_\gamma =
		 \sum_{\alpha\beta} \langle \alpha\beta|\hat{v}|\delta\gamma\rangle 
		  a^{\dagger}_\alpha a^{\dagger}_\beta  a_\gamma a_\delta \label{eq:2-38c}
\end{equation}
From this it follows that the restriction on the summation over $\gamma$ and $\delta$ can be removed if we multiply with a factor $\frac{1}{2}$, resulting in 
\begin{equation}
	\hat{H}_I = \frac{1}{2} \sum_{\alpha\beta\gamma\delta} \langle \alpha\beta|\hat{v}|\gamma\delta\rangle
		a^{\dagger}_\alpha a^{\dagger}_\beta a_\delta a_\gamma \label{eq:2-39}
\end{equation}
where we sum freely over all single-particle states $\alpha$, 
$\beta$, $\gamma$ og $\delta$.

With this expression we can now verify that the second quantization form of $\hat{H}_I$ in Eq.~(\ref{eq:2-39}) 
results in the same matrix between two anti-symmetrized two-particle states as its corresponding coordinate
space representation. We have  
\begin{equation}
	\langle \alpha_1 \alpha_2|\hat{H}_I|\beta_1 \beta_2\rangle =
		\frac{1}{2} \sum_{\alpha\beta\gamma\delta}
			\langle \alpha\beta|\hat{v}|\gamma\delta\rangle \langle 0|a_{\alpha_2} a_{\alpha_1} 
			 a^{\dagger}_\alpha a^{\dagger}_\beta a_\delta a_\gamma 
			 a_{\beta_1}^{\dagger} a_{\beta_2}^{\dagger}|0\rangle. \label{eq:2-40}
\end{equation}

Using the commutation relations we get 
\begin{align}
	&& a_{\alpha_2} a_{\alpha_1}a^{\dagger}_\alpha a^{\dagger}_\beta 
		a_\delta a_\gamma a_{\beta_1}^{\dagger} a_{\beta_2}^{\dagger} \nonumber \\
	&=& a_{\alpha_2} a_{\alpha_1}a^{\dagger}_\alpha a^{\dagger}_\beta 
		( a_\delta \delta_{\gamma \beta_1} a_{\beta_2}^{\dagger} - 
		a_\delta  a_{\beta_1}^{\dagger} a_\gamma a_{\beta_2}^{\dagger} ) \nonumber \\
	&=& a_{\alpha_2} a_{\alpha_1}a^{\dagger}_\alpha a^{\dagger}_\beta 
		(\delta_{\gamma \beta_1} \delta_{\delta \beta_2} - \delta_{\gamma \beta_1} a_{\beta_2}^{\dagger} a_\delta -
		a_\delta a_{\beta_1}^{\dagger}\delta_{\gamma \beta_2} +
		a_\delta a_{\beta_1}^{\dagger} a_{\beta_2}^{\dagger} a_\gamma ) \nonumber \\
	&=& a_{\alpha_2} a_{\alpha_1}a^{\dagger}_\alpha a^{\dagger}_\beta 
		(\delta_{\gamma \beta_1} \delta_{\delta \beta_2} - \delta_{\gamma \beta_1} a_{\beta_2}^{\dagger} a_\delta \nonumber \\
		&& \qquad - \delta_{\delta \beta_1} \delta_{\gamma \beta_2} + \delta_{\gamma \beta_2} a_{\beta_1}^{\dagger} a_\delta
		+ a_\delta a_{\beta_1}^{\dagger} a_{\beta_2}^{\dagger} a_\gamma ) \label{eq:2-41}
\end{align}

The vacuum expectation value of this product of operators becomes
\begin{align}
	&& \langle 0|a_{\alpha_2} a_{\alpha_1} a^{\dagger}_\alpha a^{\dagger}_\beta a_\delta a_\gamma 
		a_{\beta_1}^{\dagger} a_{\beta_2}^{\dagger}|0\rangle \nonumber \\
	&=& (\delta_{\gamma \beta_1} \delta_{\delta \beta_2} -
		\delta_{\delta \beta_1} \delta_{\gamma \beta_2} ) 
		\langle 0|a_{\alpha_2} a_{\alpha_1}a^{\dagger}_\alpha a^{\dagger}_\beta|0\rangle \nonumber \\
	&=& (\delta_{\gamma \beta_1} \delta_{\delta \beta_2} -\delta_{\delta \beta_1} \delta_{\gamma \beta_2} )
	(\delta_{\alpha \alpha_1} \delta_{\beta \alpha_2} -\delta_{\beta \alpha_1} \delta_{\alpha \alpha_2} ) \label{eq:2-42b}
\end{align}

Insertion of 
Eq.~(\ref{eq:2-42b}) in Eq.~(\ref{eq:2-40}) results in
\begin{align}
	\langle \alpha_1\alpha_2|\hat{H}_I|\beta_1\beta_2\rangle &=& \frac{1}{2} \big[ 
		\langle \alpha_1\alpha_2|\hat{v}|\beta_1\beta_2\rangle- \langle \alpha_1\alpha_2|\hat{v}|\beta_2\beta_1\rangle \nonumber \\
		&& - \langle \alpha_2\alpha_1|\hat{v}|\beta_1\beta_2\rangle + \langle \alpha_2\alpha_1|\hat{v}|\beta_2\beta_1\rangle \big] \nonumber \\
	&=& \langle \alpha_1\alpha_2|\hat{v}|\beta_1\beta_2\rangle - \langle \alpha_1\alpha_2|\hat{v}|\beta_2\beta_1\rangle \nonumber \\
	&=& \langle \alpha_1\alpha_2|\hat{v}|\beta_1\beta_2\rangle_{\mathrm{AS}}. \label{eq:2-43b}
\end{align}

The two-body operator can also be expressed in terms of the anti-symmetrized matrix elements we discussed previously as
\begin{align}
	\hat{H}_I &=& \frac{1}{2} \sum_{\alpha\beta\gamma\delta}  \langle \alpha \beta|\hat{v}|\gamma \delta\rangle
		a_\alpha^{\dagger} a_\beta^{\dagger} a_\delta a_\gamma \nonumber \\
	&=& \frac{1}{4} \sum_{\alpha\beta\gamma\delta} \left[ \langle \alpha \beta|\hat{v}|\gamma \delta\rangle -
		\langle \alpha \beta|\hat{v}|\delta\gamma \rangle \right] 
		a_\alpha^{\dagger} a_\beta^{\dagger} a_\delta a_\gamma \nonumber \\
	&=& \frac{1}{4} \sum_{\alpha\beta\gamma\delta} \langle \alpha \beta|\hat{v}|\gamma \delta\rangle_{\mathrm{AS}}
		a_\alpha^{\dagger} a_\beta^{\dagger} a_\delta a_\gamma \label{eq:2-45}
\end{align}

The factors in front of the operator, either  $\frac{1}{4}$ or 
$\frac{1}{2}$ tells whether we use antisymmetrized matrix elements or not. 

We can now express the Hamiltonian operator for a many-fermion system  in the occupation basis representation
as  
\begin{equation}
	H = \sum_{\alpha, \beta} \langle \alpha|\hat{t}+\hat{u}_{\mathrm{ext}}|\beta\rangle a_\alpha^{\dagger} a_\beta + \frac{1}{4} \sum_{\alpha\beta\gamma\delta}
		\langle \alpha \beta|\hat{v}|\gamma \delta\rangle a_\alpha^{\dagger} a_\beta^{\dagger} a_\delta a_\gamma. \label{eq:2-46b}
\end{equation}
This is the form we will use in the rest of these lectures, assuming that we work with anti-symmetrized two-body matrix elements.

\subsection{Wick's theorem}

\href{{https://github.com/ManyBodyPhysics/FYS4480/tree/master/doc/HandwrittenNotes/2022}}{See handwritten notes for week 36}


% ------------------- end of main content ---------------

% #ifdef PREAMBLE
\end{document}
% #endif

