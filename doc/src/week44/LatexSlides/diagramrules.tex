\documentclass[aspectratio=169]{beamer}

\usetheme{Madrid}
\usecolortheme{dolphin}
\setbeamertemplate{navigation symbols}{}

\usepackage{amsmath,amssymb,amsfonts,physics,bm,braket,mathtools}
\usepackage{tikz}
\usetikzlibrary{arrows.meta,decorations.pathmorphing,calc}

\title[Diagram Rules for Fermions]{Diagram Rules for Fermions in Many-Body Theory:\\
Time-Independent Perturbation Theory in Second Quantization}
\author{FYS4480/9480}
\institute{Addition to lecture notes week 44}
\date{\today}

%%%%%%%%%%%%%%%%%%%%%%%%%%%%%%%%%%%%%%%%
\begin{document}
%%%%%%%%%%%%%%%%%%%%%%%%%%%%%%%%%%%%%%%%

\begin{frame}
    \titlepage
\end{frame}


%--------------------------------------------------
\begin{frame}{Reminder on fermionic Creation and Annihilation Operators}
We work in second quantization with a chosen single-particle basis
$\{ \phi_p(\mathbf{x}) \}$.
\begin{align}
    \hat{a}^\dagger_p &: \text{creates a fermion in state } \phi_p, \\
    \hat{a}_p &: \text{annihilates a fermion in state } \phi_p.
\end{align}
They satisfy canonical anticommutation relations
\begin{equation}
    \{ \hat{a}_p, \hat{a}_q^\dagger \} = \delta_{pq}, \qquad
    \{ \hat{a}_p, \hat{a}_q \} = 0, \qquad
    \{ \hat{a}^\dagger_p, \hat{a}^\dagger_q \} = 0.
\end{equation}
A generic $A$-particle Slater determinant (Fermi sea, reference, etc.) is
\begin{equation}
    \ket{\Phi_0} = \prod_{i \in \text{occ}} \hat{a}^\dagger_i \ket{0},
\end{equation}
where $\text{occ}$ labels occupied single-particle states in the reference.
\end{frame}

%--------------------------------------------------
\begin{frame}{Hamiltonian in Second Quantization}
We split the Hamiltonian into an unperturbed part $\hat{H}_0$ and an interaction $\hat{V}$:
\begin{equation}
    \hat{H} = \hat{H}_0 + \hat{V}.
\end{equation}
A common choice:
\begin{align}
    \hat{H}_0 &= \sum_{pq} t_{pq} \, \hat{a}^\dagger_p \hat{a}_q, \\
    \hat{V} &= \frac{1}{4} \sum_{pqrs} \bar{v}_{pqrs} \,
        \hat{a}^\dagger_p \hat{a}^\dagger_q \hat{a}_s \hat{a}_r,
\end{align}
where $\bar{v}_{pqrs}$ are antisymmetrized two-body matrix elements
\[
\bar{v}_{pqrs} = \matrixel{pq}{\hat{v}}{rs}
- \matrixel{pq}{\hat{v}}{sr}.
\]
We treat $\hat{H}_0$ as solvable and $\hat{V}$ as the perturbation.
The reference state $\ket{\Phi_0}$ is usually an eigenstate of $\hat{H}_0$.
\end{frame}

%--------------------------------------------------
\begin{frame}{Normal Ordering (Wick Setup)}
\textbf{Normal ordering} $:\,\cdots\,:$ means: move all creation operators
to the left of all annihilation operators \emph{with a sign} for each fermionic swap.

For example,
\begin{equation}
    : \hat{a}^\dagger_p \hat{a}_q : \;=\; \hat{a}^\dagger_p \hat{a}_q,
\qquad
    : \hat{a}_q \hat{a}^\dagger_p :
    \;=\; - \hat{a}^\dagger_p \hat{a}_q.
\end{equation}

Important subtlety: normal ordering must be defined with respect to a
reference state. In many-body theory we \emph{do not} always normal order
relative to the true vacuum $\ket{0}$, but often relative to a filled Fermi
sea $\ket{\Phi_0}$.

We denote normal ordering with respect to $\ket{\Phi_0}$ by
$N[ \cdots ]$ when we need to stress it.
\end{frame}

%--------------------------------------------------
\begin{frame}{Contractions (Fermions)}
Given two fermionic operators $\hat{A}$ and $\hat{B}$, their
\textbf{contraction} with respect to $\ket{\Phi_0}$ is defined as
\begin{equation}
    \contraction{}{\hat{A}}{}{\hat{B}}
    \hat{A}\hat{B}
    \;\equiv\;
    \langle \Phi_0 | \, T(\hat{A}\hat{B}) \, | \Phi_0 \rangle
    - \langle \Phi_0 | \, N[\hat{A}\hat{B}] \, | \Phi_0 \rangle,
\end{equation}
where $T$ is an ordering operation appropriate to the context.

For \textbf{time-independent} perturbation theory,
we typically consider static (equal-time) operators in products like
$\hat{V}\hat{V}\cdots\hat{V}$.
Then the contraction between $\hat{a}_p$ and $\hat{a}^\dagger_q$ is
\begin{equation}
    \contraction{}{\hat{a}_p}{}{\hat{a}^\dagger_q}
    \hat{a}_p \hat{a}^\dagger_q
    \;=\;
    \langle \Phi_0 | \hat{a}_p \hat{a}^\dagger_q | \Phi_0 \rangle
    =
    \delta_{pq} \times
    \begin{cases}
        1, & p \text{ unoccupied in } \ket{\Phi_0}, \\
        0, & p \text{ occupied in } \ket{\Phi_0}.
    \end{cases}
\end{equation}
Similarly,
\begin{equation}
    \contraction{}{\hat{a}^\dagger_p}{}{\hat{a}_q}
    \hat{a}^\dagger_p \hat{a}_q
    \;=\;
    \langle \Phi_0 | \hat{a}^\dagger_p \hat{a}_q | \Phi_0 \rangle
    =
    \delta_{pq} \times
    \begin{cases}
        1, & p \text{ occupied in } \ket{\Phi_0}, \\
        0, & p \text{ unoccupied in } \ket{\Phi_0}.
    \end{cases}
\end{equation}

In words: a contraction ``projects'' whether an index is occupied
(hole line) or unoccupied (particle line) in $\ket{\Phi_0}$.
\end{frame}

%--------------------------------------------------
\begin{frame}{Wick's Theorem (Static Form)}
\textbf{Wick's theorem} for fermions states:

Any product of fermionic creation and annihilation operators can be written
as a sum of
\begin{itemize}
    \item a normal-ordered product $N[\cdots]$ (with respect to $\ket{\Phi_0}$),
    \item plus all possible normal-ordered products with one contraction,
    \item plus all with two contractions,
    \item etc.,
    \item up to the fully contracted term(s).
\end{itemize}

Symbolically,
\begin{equation}
    \hat{A}_1 \hat{A}_2 \cdots \hat{A}_n
    =
    N[\hat{A}_1 \hat{A}_2 \cdots \hat{A}_n]
    + \sum_{\text{1 contraction}} N[\cdots]
    + \sum_{\text{2 contractions}} N[\cdots]
    + \cdots
\end{equation}
with appropriate fermionic signs for each reordering needed to realize
the contractions.
\end{frame}

%--------------------------------------------------
\begin{frame}{Expectation Values and Full Contractions}
Take the expectation value in $\ket{\Phi_0}$.
Because all normal-ordered terms annihilate $\ket{\Phi_0}$ on the right
(or $\bra{\Phi_0}$ on the left), only \emph{fully contracted} terms survive:
\begin{equation}
    \mel{\Phi_0}{\hat{A}_1 \hat{A}_2 \cdots \hat{A}_n}{\Phi_0}
    =
    \sum_{\text{all complete pairings}}
    (\pm)\,
    \prod_{\text{pairs }(i,j)} 
        \contraction{}{\hat{A}_i}{}{\hat{A}_j}
        \hat{A}_i \hat{A}_j .
\end{equation}
Key point:
\begin{itemize}
    \item The sum runs over all distinct ways of pairing operators
    into contractions.
    \item Each pairing comes with a fermionic sign determined by how many
    swaps are needed to bring operators next to each other.
\end{itemize}

This is the algebraic origin of Feynman- or Goldstone-like diagrams:
each full contraction pattern $\leftrightarrow$ one diagram.
\end{frame}

%--------------------------------------------------
\begin{frame}{Many-Body Perturbation Theory (Energy)}
For a non-degenerate reference state $\ket{\Phi_0}$ with unperturbed
energy $E_0^{(0)}$,
the Rayleigh--Schr\"odinger perturbation series for the ground-state
energy is
\begin{equation}
    E_0 = E_0^{(0)} + E_0^{(1)} + E_0^{(2)} + \cdots 
\end{equation}
with
\begin{align}
    E_0^{(1)} &= \mel{\Phi_0}{\hat{V}}{\Phi_0}, \\
    E_0^{(2)} &= 
        \sum_{n\neq 0}
        \frac{
        \abs{\mel{\Phi_n}{\hat{V}}{\Phi_0}}^2
        }{
        E_0^{(0)} - E_n^{(0)}
        }, \quad \text{etc.}
\end{align}
In second quantization, these matrix elements are operator strings of
$\hat{a}^\dagger \hat{a}^\dagger \hat{a}\hat{a}$ etc.

\textbf{Wick’s theorem} reduces them to sums of complete contractions,
with denominators from intermediate states. Each such contraction
pattern corresponds to one diagram in the standard diagrammatic
expansion of the ground-state energy.
\end{frame}

%--------------------------------------------------
\begin{frame}{Short summary}
\begin{itemize}
    \item Fermionic operators anticommute, and states are built as Slater determinants.
    \item The Hamiltonian is expressed in normal-ordered two-body form.
    \item Contractions encode occupied/unoccupied structure of the reference state.
    \item Wick's theorem rewrites any operator product as a sum over
    normal-ordered pieces plus contractions.
    \item Only fully contracted pieces survive in expectation values,
    and each full contraction $\leftrightarrow$ a diagram.
\end{itemize}

\bigskip
Next  we will prove the \textbf{diagram rules}: how to translate any
fully contracted term into a sign, symmetry factor, energy denominator,
and algebraic expression.
\end{itemize}
\end{frame}


%--------------------------------------------------
\begin{frame}{From Contractions to Lines}
Consider a two-body interaction vertex from $\hat{V}$:
\begin{equation}
    \hat{V} = \frac{1}{4} \sum_{pqrs} \bar{v}_{pqrs} \,
    \hat{a}^\dagger_p \hat{a}^\dagger_q \hat{a}_s \hat{a}_r.
\end{equation}
A single insertion of $\hat{V}$ acting on $\ket{\Phi_0}$ can
\emph{excite} two particles from occupied (hole) states $i,j$ to
unoccupied (particle) states $a,b$.
Schematically:
\[
    \hat{a}^\dagger_a \hat{a}^\dagger_b \hat{a}_j \hat{a}_i \ket{\Phi_0}
    \quad \Rightarrow \quad
    \ket{\Phi_{ij}^{ab}}.
\]
In diagrams:
\begin{itemize}
    \item Each $\hat{a}^\dagger$ corresponds to an outgoing particle line.
    \item Each $\hat{a}$ corresponds to an incoming hole line.
\end{itemize}
A contraction between $\hat{a}_i$ in one vertex and $\hat{a}^\dagger_a$
in another vertex becomes a \textbf{line} connecting two vertices.
\end{frame}

%--------------------------------------------------
\begin{frame}{Sign Structure: Fermionic Minus Signs}
When we evaluate a string like
\[
\hat{a}^\dagger_{p_1} \hat{a}^\dagger_{p_2} \hat{a}_{q_2} \hat{a}_{q_1}
\hat{a}^\dagger_{r_1} \hat{a}^\dagger_{r_2} \hat{a}_{s_2} \hat{a}_{s_1}
\cdots
\]
we must bring operators next to each other to form contractions.
Every swap of two fermionic operators contributes a factor $(-1)$.

\textbf{Result:} Each complete contraction pattern produces
\begin{equation}
(\text{sign}) = (-1)^{N_{\text{perm}}},
\end{equation}
where $N_{\text{perm}}$ is the number of permutations needed to realize
that pairing.

\textbf{Diagrammatically}: following standard fermionic diagram
conventions, internal lines that ``cross'' encode these permutations.
Thus, the sign of a diagram is the sign of the underlying antisymmetry
of the many-body wave function.
\end{frame}

%--------------------------------------------------
\begin{frame}{Energy Denominators (Time-Independent PT)}
In ordinary Rayleigh--Schr\"odinger perturbation theory, an $n$th-order
energy correction involves $n$ insertions of $\hat{V}$ and $(n-1)$
sums over intermediate states:
\begin{equation}
E_0^{(n)} =
\sum_{\text{int. states}}
\frac{
\mel{\Phi_0}{\hat{V}}{\Phi_{1}}
\mel{\Phi_{1}}{\hat{V}}{\Phi_{2}}
\cdots
\mel{\Phi_{n-1}}{\hat{V}}{\Phi_0}
}{
(E_0^{(0)}-E_1^{(0)})
(E_0^{(0)}-E_2^{(0)})
\cdots
(E_0^{(0)}-E_{n-1}^{(0)})
}.
\label{eq:EnPert}
\end{equation}

Each intermediate state $\ket{\Phi_k}$ is a Slater determinant with
some set of particle-hole excitations. Its unperturbed energy
$E_k^{(0)}$ is just the sum of single-particle energies of occupied
orbitals in that determinant.

\textbf{Diagram rule:}
Every diagram at order $n$ carries a product of denominators,
one for each ``time slice'' (or intermediate configuration) in which a
set of particle-hole excitations propagates.
\end{frame}

%--------------------------------------------------
\begin{frame}{Putting It Together: Generic Fermion Diagram Rule}
For ground-state energy corrections in time-independent MBPT:

\textbf{Rule 1: Vertices.}
Each interaction vertex contributes a factor
$\bar{v}_{pqrs}$ with two incoming (hole) lines and two outgoing
(particle) lines, summed over all internal indices.

\textbf{Rule 2: Lines.}
Each internal line corresponds to a contraction and implies
\begin{itemize}
    \item a Kronecker delta enforcing index matching,
    \item whether that index is a particle (unoccupied in $\ket{\Phi_0}$)
    or a hole (occupied in $\ket{\Phi_0}$),
    \item an energy associated with that orbital.
\end{itemize}

\textbf{Rule 3: Signs.}
Include a global factor $(-1)^{N_{\text{perm}}}$ coming from the
reordering of fermion operators needed to realize the contraction pattern.
Equivalently: track the number of fermionic line exchanges.

\textbf{Rule 4: Denominators.}
For an $n$th-order diagram, write down the sequence of intermediate
Slater determinants generated as you ``move through'' the vertices.
For each nontrivial intermediate determinant, include a denominator
$[E_0^{(0)}-E_{\text{int}}^{(0)}]^{-1}$.
\end{frame}

%--------------------------------------------------
\begin{frame}{Example: Second-Order Correlation Energy}
Consider the standard second-order (MP2-like) correlation energy.
Occupied indices: $i,j$.
Unoccupied (virtual) indices: $a,b$.

The algebra from Eq.~\eqref{eq:EnPert} gives
\begin{equation}
E_0^{(2)} =
\frac{1}{4}
\sum_{ijab}
\frac{
\abs{ \bar{v}_{ijab} }^2
}{
\epsilon_i + \epsilon_j - \epsilon_a - \epsilon_b
},
\end{equation}
where $\epsilon_p$ are single-particle energies from $\hat{H}_0$.

\bigskip
Diagrammatically:
\begin{itemize}
    \item One vertex excites $ij \to ab$,
    \item The other vertex de-excites $ab \to ij$,
    \item Internal lines connect $i\leftrightarrow i$, $j\leftrightarrow j$,
    $a\leftrightarrow a$, $b\leftrightarrow b$,
    \item The denominator is the energy difference of the intermediate
    $2p$--$2h$ excitation,
    \item The sign is $+$ for this canonical ordering.
\end{itemize}
This diagram encodes the full sum over $(ijab)$.
\end{frame}

%--------------------------------------------------
\begin{frame}{Symmetry / Combinatorial Factors}
In higher orders, multiple distinct contraction patterns may generate
the \emph{same} topological diagram.

\textbf{Rule 5 (Symmetry factor).}
If $m$ different full contraction patterns reduce to the same
topological diagram, that diagram receives a prefactor $m$ (possibly
with signs already accounted for).

Example:
\begin{itemize}
    \item Two vertices with identical structure can sometimes be interchanged
    without changing the topology.
    \item Then both orderings appear separately in Wick expansions,
    so the diagram gains an extra factor $2$.
\end{itemize}

This is how purely algebraic counting in Wick's theorem becomes
combinatorics of diagrams.
\end{frame}

%--------------------------------------------------
\begin{frame}{Final Recipe (Ground-State Energy Diagrams)}
To evaluate an $n$th-order fermionic diagram contributing to the
ground-state energy:
\begin{enumerate}
    \item Assign indices ($i,j,\dots$ for occupied/hole,
          $a,b,\dots$ for unoccupied/particle) to each line.
    \item For each vertex, write a matrix element $\bar{v}_{pqrs}$
          with appropriate indices from attached lines.
    \item Multiply by the sign $(-1)^{N_{\text{perm}}}$ determined by
          fermion line permutations.
    \item Sum over all internal indices (Einstein-like summation).
    \item Include one energy denominator for every intermediate
          particle-hole configuration.
    \item Multiply by the symmetry/combinatorial factor for that
          topological diagram.
\end{enumerate}

\bigskip
\textbf{Claim:} These rules are \emph{exactly} what you get by
applying Wick’s theorem to $\hat{V}^n$ between Slater determinants,
collecting only fully contracted terms, and organizing them by
topology.
\end{frame}

%--------------------------------------------------
\begin{frame}{Conceptual Proof Structure (Why This Works)}
\begin{itemize}
    \item Wick's theorem ensures: only full contractions survive in
    $\mel{\Phi_0}{\hat{V}^n}{\Phi_0}$.
    \item Each full contraction uniquely pairs annihilators with
    creators across the $n$ interaction insertions.
    \item Each such pairing defines:
        \begin{itemize}
            \item which particle-hole excitations appear,
            \item in which order they propagate,
            \item and how they recombine to $\ket{\Phi_0}$.
        \end{itemize}
    \item The fermionic anticommutation algebra enforces the correct sign.
    \item The intermediate-state sums naturally generate the
    denominators in Rayleigh--Schr\"odinger perturbation theory.
    \item Grouping algebraically equivalent contraction patterns
    gives topological diagrams plus symmetry factors.
\end{itemize}

Therefore, the diagram rules are not assumptions --- they are a compact
repackaging of Wick's theorem and standard many-body perturbation theory.
\end{frame}


\end{document}

