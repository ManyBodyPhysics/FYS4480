% Slides for FYS4480, Green's function theories


\documentclass[compress]{beamer}


% Try the class options [notes], [notes=only], [trans], [handout],
% [red], [compress], [draft], [class=article] and see what happens!

% For a green structure color use:
%\colorlet{structure}{green!50!black}

\mode<article> % only for the article version
{
  \usepackage{beamerbasearticle}
  \usepackage{fullpage}
  \usepackage{hyperref}
}

\beamertemplateshadingbackground{red!10}{blue!10}

%\beamertemplateshadingbackground{red!10}{blue!10}
\beamertemplatetransparentcovereddynamic
%\usetheme{Hannover}

\setbeamertemplate{footline}[page number]


%\usepackage{beamerthemeshadow}

%\usepackage{beamerthemeshadow}
\usepackage{ucs}


\usepackage{pgf,pgfarrows,pgfnodes,pgfautomata,pgfheaps,pgfshade}
\usepackage{graphicx}
\usepackage{simplewick}
\usepackage{amsmath,amssymb}
\usepackage[latin1]{inputenc}
\usepackage{colortbl}
\usepackage[english]{babel}
\usepackage{listings}
\usepackage{shadow}
\lstset{language=c++}
\lstset{alsolanguage=[90]Fortran}
\lstset{basicstyle=\small}
%\lstset{backgroundcolor=\color{white}}
%\lstset{frame=single}
\lstset{stringstyle=\ttfamily}
%\lstset{keywordstyle=\color{red}\bfseries}
%\lstset{commentstyle=\itshape\color{blue}}
\lstset{showspaces=false}
\lstset{showstringspaces=false}
\lstset{showtabs=false}
\lstset{breaklines}
\usepackage{times}

% Use some nice templates
\beamertemplatetransparentcovereddynamic

% own commands
\newcommand*{\cre}[1]{a^{\dagger}_{#1}}
\newcommand*{\an}[1]{a_{#1}}
\newcommand*{\crequasi}[1]{b^{\dagger}_{#1}}
\newcommand*{\anquasi}[1]{b_{#1}}
\newcommand*{\for}[3]{\langle#1|#2|#3\rangle}
\newcommand{\be}{\begin{equation}}
\newcommand{\ee}{\end{equation}}
\newcommand*{\kpr}[1]{\left\{#1\right\}}
\newcommand*{\ket}[1]{|#1\rangle}
\newcommand*{\bra}[1]{\langle#1|}
%\newcommand{\bra}[1]{\left\langle #1 \right|}
%\newcommand{\ket}[1]{\left| # \right\rangle}
\newcommand{\braket}[2]{\left\langle #1 \right| #2 \right\rangle}
\newcommand{\OP}[1]{{\bf\widehat{#1}}}
\newcommand{\matr}[1]{{\bf \cal{#1}}}
\newcommand{\beN}{\begin{equation*}}
\newcommand{\bea}{\begin{eqnarray}}
\newcommand{\beaN}{\begin{eqnarray*}}
\newcommand{\eeN}{\end{equation*}}
\newcommand{\eea}{\end{eqnarray}}
\newcommand{\eeaN}{\end{eqnarray*}}
\newcommand{\bdm}{\begin{displaymath}}
\newcommand{\edm}{\end{displaymath}}
\newcommand{\bsubeqs}{\begin{subequations}}
\newcommand*{\fpr}[1]{\left[#1\right]}
\newcommand{\esubeqs}{\end{subequations}}
\newcommand*{\pr}[1]{\left(#1\right)}
\newcommand{\element}[3]
        {\bra{#1}#2\ket{#3}}
\newcommand{\md}{\mathrm{d}}
\newcommand{\e}[1]{\times 10^{#1}}
\renewcommand{\vec}[1]{\mathbf{#1}}
\newcommand{\gvec}[1]{\boldsymbol{#1}}
\newcommand{\dr}{\, \mathrm d^3 \vec r}
\newcommand{\dk}{\, \mathrm d^3 \vec k}
\def\psii{\psi_{i}}
\def\psij{\psi_{j}}
\def\psiij{\psi_{ij}}
\def\psisq{\psi^2}
\def\psisqex{\langle \psi^2 \rangle}
\def\psiR{\psi({\bf R})}
\def\psiRk{\psi({\bf R}_k)}
\def\psiiRk{\psi_{i}(\Rveck)}
\def\psijRk{\psi_{j}(\Rveck)}
\def\psiijRk{\psi_{ij}(\Rveck)}
\def\ranglep{\rangle_{\psisq}}
\def\Hpsibypsi{{H \psi \over \psi}}
\def\Hpsiibypsi{{H \psii \over \psi}}
\def\HmEpsibypsi{{(H-E) \psi \over \psi}}
\def\HmEpsiibypsi{{(H-E) \psii \over \psi}}
\def\HmEpsijbypsi{{(H-E) \psij \over \psi}}
\def\psiibypsi{{\psii \over \psi}}
\def\psijbypsi{{\psij \over \psi}}
\def\psiijbypsi{{\psiij \over \psi}}
\def\psiibypsiRk{{\psii(\Rveck) \over \psi(\Rveck)}}
\def\psijbypsiRk{{\psij(\Rveck) \over \psi(\Rveck)}}
\def\psiijbypsiRk{{\psiij(\Rveck) \over \psi(\Rveck)}}
\def\EL{E_{\rm L}}
\def\ELi{E_{{\rm L},i}}
\def\ELj{E_{{\rm L},j}}
\def\ELRk{E_{\rm L}(\Rveck)}
\def\ELiRk{E_{{\rm L},i}(\Rveck)}
\def\ELjRk{E_{{\rm L},j}(\Rveck)}
\def\Ebar{\bar{E}}
\def\Ei{\Ebar_{i}}
\def\Ej{\Ebar_{j}}
\def\Ebar{\bar{E}}
\def\Rvec{{\bf R}}
\def\Rveck{{\bf R}_k}
\def\Rvecl{{\bf R}_l}
\def\NMC{N_{\rm MC}}
\def\sumMC{\sum_{k=1}^{\NMC}}
\def\MC{Monte Carlo}
\def\adiag{a_{\rm diag}}
\def\tcorr{T_{\rm corr}}
\def\intR{{\int {\rm d}^{3N}\!\!R\;}}
\def\ul{\underline}
\def\beq{\begin{eqnarray}}
\def\eeq{\end{eqnarray}}
\newcommand{\eqbrace}[4]{\left\{
\begin{array}{ll}
#1 & #2 \\[0.5cm]
#3 & #4
\end{array}\right.}
\newcommand{\eqbraced}[4]{\left\{
\begin{array}{ll}
#1 & #2 \\[0.5cm]
#3 & #4
\end{array}\right\}}
\newcommand{\eqbracetriple}[6]{\left\{
\begin{array}{ll}
#1 & #2 \\
#3 & #4 \\
#5 & #6
\end{array}\right.}
\newcommand{\eqbracedtriple}[6]{\left\{
\begin{array}{ll}
#1 & #2 \\
#3 & #4 \\
#5 & #6
\end{array}\right\}}
\newcommand{\mybox}[3]{\mbox{\makebox[#1][#2]{$#3$}}}
\newcommand{\myframedbox}[3]{\mbox{\framebox[#1][#2]{$#3$}}}
%% Infinitesimal (and double infinitesimal), useful at end of integrals
%\newcommand{\ud}[1]{\mathrm d#1}
\newcommand{\ud}[1]{d#1}
\newcommand{\udd}[1]{d^2\!#1}
%% Operators, algebraic matrices, algebraic vectors
%% Operator (hat, bold or bold symbol, whichever you like best):
\newcommand{\op}[1]{\widehat{#1}}
%\newcommand{\op}[1]{\mathbf{#1}}
%\newcommand{\op}[1]{\boldsymbol{#1}}
%% Vector:
\renewcommand{\vec}[1]{\boldsymbol{#1}}
%% Matrix symbol:
%\newcommand{\matr}[1]{\boldsymbol{#1}}
%\newcommand{\bb}[1]{\mathbb{#1}}
%% Determinant symbol:
\renewcommand{\det}[1]{|#1|}
%% Means (expectation values) of varius sizes
\newcommand{\mean}[1]{\langle #1 \rangle}
\newcommand{\meanb}[1]{\big\langle #1 \big\rangle}
\newcommand{\meanbb}[1]{\Big\langle #1 \Big\rangle}
\newcommand{\meanbbb}[1]{\bigg\langle #1 \bigg\rangle}
\newcommand{\meanbbbb}[1]{\Bigg\langle #1 \Bigg\rangle}
%% Shorthands for text set in roman font
\newcommand{\prob}[0]{\mathrm{Prob}} %probability
\newcommand{\cov}[0]{\mathrm{Cov}}   %covariance
\newcommand{\var}[0]{\mathrm{Var}}   %variancd
%% Big-O (typically for specifying the speed scaling of an algorithm)
\newcommand{\bigO}{\mathcal{O}}
%% Real value of a complex number
\newcommand{\real}[1]{\mathrm{Re}\!\left\{#1\right\}}
%% Quantum mechanical state vectors and matrix elements (of different sizes)
%\newcommand{\bra}[1]{\langle #1 |}
\newcommand{\brab}[1]{\big\langle #1 \big|}
\newcommand{\brabb}[1]{\Big\langle #1 \Big|}
\newcommand{\brabbb}[1]{\bigg\langle #1 \bigg|}
\newcommand{\brabbbb}[1]{\Bigg\langle #1 \Bigg|}
%\newcommand{\ket}[1]{| #1 \rangle}
\newcommand{\ketb}[1]{\big| #1 \big\rangle}
\newcommand{\ketbb}[1]{\Big| #1 \Big\rangle}
\newcommand{\ketbbb}[1]{\bigg| #1 \bigg\rangle}
\newcommand{\ketbbbb}[1]{\Bigg| #1 \Bigg\rangle}
%\newcommand{\overlap}[2]{\langle #1 | #2 \rangle}
\newcommand{\overlapb}[2]{\big\langle #1 \big| #2 \big\rangle}
\newcommand{\overlapbb}[2]{\Big\langle #1 \Big| #2 \Big\rangle}
\newcommand{\overlapbbb}[2]{\bigg\langle #1 \bigg| #2 \bigg\rangle}
\newcommand{\overlapbbbb}[2]{\Bigg\langle #1 \Bigg| #2 \Bigg\rangle}
\newcommand{\bracket}[3]{\langle #1 | #2 | #3 \rangle}
\newcommand{\bracketb}[3]{\big\langle #1 \big| #2 \big| #3 \big\rangle}
\newcommand{\bracketbb}[3]{\Big\langle #1 \Big| #2 \Big| #3 \Big\rangle}
\newcommand{\bracketbbb}[3]{\bigg\langle #1 \bigg| #2 \bigg| #3 \bigg\rangle}
\newcommand{\bracketbbbb}[3]{\Bigg\langle #1 \Bigg| #2 \Bigg| #3 \Bigg\rangle}
\newcommand{\projection}[2]
{| #1 \rangle \langle  #2 |}
\newcommand{\projectionb}[2]
{\big| #1 \big\rangle \big\langle #2 \big|}
\newcommand{\projectionbb}[2]
{ \Big| #1 \Big\rangle \Big\langle #2 \Big|}
\newcommand{\projectionbbb}[2]
{ \bigg| #1 \bigg\rangle \bigg\langle #2 \bigg|}
\newcommand{\projectionbbbb}[2]
{ \Bigg| #1 \Bigg\rangle \Bigg\langle #2 \Bigg|}
%% If you run out of greek symbols, here's another one you haven't
%% thought of:
\newcommand{\Feta}{\hspace{0.6ex}\begin{turn}{180}
        {\raisebox{-\height}{\parbox[c]{1mm}{F}}}\end{turn}}
\newcommand{\feta}{\hspace{-1.6ex}\begin{turn}{180}
        {\raisebox{-\height}{\parbox[b]{4mm}{f}}}\end{turn}}
\title[FYS4480]{Slides from FYS4480/9480 Lectures}
\author[Quantum mechanics for many-particle systems]{%
  Morten Hjorth-Jensen}
\institute[ORNL, University of Oslo and MSU]{
  Department of Physics and Center for Computing in Science Education\\
  University of Oslo, N-0316 Oslo, Norway and\\
  Department of Physics and Astronomy and Facility for Rare Isotope Beams, Michigan State University, East Lansing, MI 48824, USA }
  
\date[UiO]{Fall  2023}
\subject{FYS4480/9480 Quantum mechanics for  many-particle systems}
\begin{document}
\newcommand{\braket}[2]
    {\langle #1 | #2 \rangle}
\newcommand{\ket}[1]
    {|#1\rangle}
\newcommand{\bra}[1]
    {\langle #1 |}
\newcommand{\element}[3]
    {\bra{#1}#2\ket{#3}}
\newcommand{\vbraket}[2]
    {\langle \mathbf{#1} | \mathbf{#2} \rangle}
\newcommand{\vket}[1]
    {|\mathbf{#1}\rangle}
\newcommand{\vbra}[1]
    {\langle \mathbf{#1} |}
\newcommand{\velement}[3]
    {\vbra{#1}\mathbf{#2}\vket{#3}}
\newcommand{\ud}{\mathrm{d}}
\newcommand{\nn}{\notag \\}

\newcommand{\vect}[1]
    {\mathbf{#1}}
\newcommand{\op}[1]
    {\hat{\mathrm{#1}}}

\newcommand{\SE}{Schr\"{o}dinger equation }
\newcommand{\SED}{Schr\"{o}dinger equation. }
\newcommand{\barh}{\bar{\mathrm{H}}}

\newcommand{\normord}[1]{
    \left\{#1\right\}
}
\newcommand{\BCH}{Baker-Campbell-Hausdorff formula}

% Box for sketching algorithms
\newsavebox{\fmbox}
\newenvironment{fmpage}[1]
    {\begin{lrbox}{\fmbox}\begin{minipage}{#1}}
    {\end{minipage}\end{lrbox}\fbox{\usebox{\fmbox}}}

\numberwithin{equation}{subsection}
\numberwithin{figure}{section}
\renewcommand{\theequation}{\arabic{section}.\arabic{subsection}.\arabic{equation}}
\renewcommand{\thefigure}{\arabic{section}.\arabic{figure}}
\renewcommand{\thempfootnote}{\arabic{mpfootnote}}
%\renewcommand{\thesubfigure}{\alph{subfigure}}
\newcommand{\sphelaplop}{
    \frac{1}{r^2} \frac{\partial}{\partial r}
    \left(r^2 \frac{\partial}{\partial r} \right) + \frac{1}{r^2 \sin{\theta}}
    \frac{\partial}{\partial \theta} \left( \sin{\theta} \frac{\partial}
    {\partial \theta} \right) + \frac{1}{r^2 \sin^2\theta} \left(
    \frac{\partial^2}{\partial \phi^2}\right)}

\newcommand{\sphelapl}[1]{
    \frac{1}{r^2} \frac{\partial}{\partial r}
    \left(r^2 \frac{\partial #1}{\partial r} \right) + \frac{1}{r^2 \sin{\theta}}
    \frac{\partial}{\partial \theta} \left( \sin{\theta} \frac{\partial #1}
    {\partial \theta} \right) + \frac{1}{r^2 \sin^2\theta} \left(
    \frac{\partial^2 #1}{\partial \phi^2}\right)}


%\pagenumbering{plain}
\frame{\titlepage}
\frame
{
\frametitle{Schr\"odinger picture}
\begin{small}
{\scriptsize
The time-dependent Schr\"odinger equation (or equation of motion) reads
\[
\imath \hbar\frac{\partial }{\partial t}|\Psi_S(t)\rangle = \hat{H}\Psi_S(t)\rangle,
\]
where the subscript $S$ stands for Schr\"odinger here.
A formal solution is given by 
\[
|\Psi_S(t)\rangle = \exp{(-\imath\hat{H}(t-t_0)/\hbar)}|\Psi_S(t_0)\rangle.
\]
The Hamiltonian $\hat{H}$ is hermitian and the exponent represents a unitary 
operator with an operation carried ut on the wave function at a time $t_0$.
}
\end{small}
}
\frame
{
\frametitle{Interaction picture}
\begin{small}
{\scriptsize
Our Hamiltonian is normally written out as the sum of an unperturbed part $\hat{H}_0$ and an interaction part $\hat{H}_I$, that is
\[
\hat{H}=\hat{H}_0+\hat{H}_I.
\]
In general we have $[\hat{H}_0,\hat{H}_I]\ne 0$ since $[\hat{T},\hat{V}]\ne 0$.
We wish now to define a unitary transformation in terms of $\hat{H}_0$ by defining
\[
|\Psi_I(t)\rangle = \exp{(\imath\hat{H}_0t/\hbar)}|\Psi_S(t)\rangle,
\]
which is again a unitary transformation carried out now at the time $t$ on the 
wave function in the Schr\"odinger picture. 
}
\end{small}
}
\frame
{
\frametitle{Interaction picture}
\begin{small}
{\scriptsize
We can easily find the equation of motion by taking the time derivative
\[
\imath \hbar\frac{\partial }{\partial t}|\Psi_I(t)\rangle = -\hat{H}_0\exp{(\imath\hat{H}_0t/\hbar)}\Psi_S(t)\rangle+\exp{(\imath\hat{H}_0t/\hbar)}
\imath \hbar\frac{\partial }{\partial t}\Psi_S(t)\rangle.
\]
}
\end{small}
}
\frame
{
\frametitle{Interaction picture}
\begin{small}
{\scriptsize
Using the definition of the Schr\"odinger equation, we can rewrite the last equation as 
\[
\imath \hbar\frac{\partial }{\partial t}|\Psi_I(t)\rangle = \exp{(\imath\hat{H}_0t/\hbar)}\left[-\hat{H}_0+\hat{H}_0+\hat{H}_I\right]\exp{(-\imath\hat{H}_0t/\hbar)}\Psi_I(t)\rangle,
\]
which gives us
\[
\imath \hbar\frac{\partial }{\partial t}|\Psi_I(t)\rangle = \hat{H}_I(t)\Psi_I(t)\rangle,
\]
 with 
\[
\hat{H}_I(t)=
\exp{(\imath\hat{H}_0t/\hbar)}\hat{H}_I\exp{(-\imath\hat{H}_0t/\hbar)}.
\]
}
\end{small}
}
\frame
{
\frametitle{Interaction picture}
\begin{small}
{\scriptsize
The order of the operators is important since $\hat{H}_0$ and $\hat{H}_I$ do generally not commute.
The expectation value of
an arbitrary operator in the interaction picture can now be written as
\[
\langle \Psi'_S(t)|\hat{O}_S|\Psi_S(t)\rangle = 
\langle \Psi'_I(t) |\exp{(\imath\hat{H}_0t/\hbar)}\hat{O}_I
\exp{(-\imath\hat{H}_0t/\hbar)}|\Psi_I(t)\rangle,
\]
and using the definition
\[
\hat{O}_I(t)=
\exp{(\imath\hat{H}_0t/\hbar)}\hat{O}_I\exp{(-\imath\hat{H}_0t/\hbar)},
\]
we obtain
\[
\langle \Psi'_S(t)|\hat{O}_S|\Psi_S(t)\rangle = 
\langle \Psi'_I(t) |\hat{O}_I(t)|\Psi_I(t)\rangle,
\]
stating that a unitary transformation does not change expectation values!
}
\end{small}
}
\frame
{
\frametitle{Interaction picture}
\begin{small}
{\scriptsize
If the take the time derivative of the operator in the interaction picture we arrive at the following equation of motion
\[
\imath \hbar\frac{\partial }{\partial t}\hat{O}_I(t) = \exp{(\imath\hat{H}_0t/\hbar)}\left[\hat{O}_S\hat{H}_0-\hat{H}_0\hat{O}_S\right]\exp{(-\imath\hat{H}_0t/\hbar)}=\left[\hat{O}_I(t),\hat{H}_0\right].
\]
Here we have used the time-independence of the Schr\"odinger equation
together with the observation that any function of an operator commutes with the operator itself. 
}
\end{small}
}
\frame
{
\frametitle{Interaction picture}
\begin{small}
{\scriptsize
In order to solve the equation of motion equation in the interaction picture, we define a unitary operator
time-development operator $\hat{U}(t,t')$. Later we will derive its
connection with the linked-diagram theorem, which yields a
linked expression for the actual operator. 
The action of the operator on the wave function is
\[
|\Psi_I(t) \rangle = \hat{U}(t,t_0)|\Psi_I(t_0)\rangle,
\]
with the obvious value $\hat{U}(t_0,t_0)=1$.
}
\end{small}
}
\frame
{
\frametitle{Interaction picture}
\begin{small}
{\scriptsize
The time-development operator $U$ has the
properties that
\[
     \hat{U}^{\dagger}(t,t')\hat{U}(t,t')=\hat{U}(t,t')\hat{U}^{\dagger}(t,t')=1,
\]
which implies that $U$ is unitary
\[
     \hat{U}^{\dagger}(t,t')=\hat{U}^{-1}(t,t').
\]
Further,
\[
    \hat{U}(t,t')\hat{U}(t't'')=\hat{U}(t,t'')
\]
and
\[
    \hat{U}(t,t')\hat{U}(t',t)=1,
\]
which leads to
\[
    \hat{U}(t,t')=\hat{U}^{\dagger}(t',t).
\]
}
\end{small}
}
\frame
{
\frametitle{Interaction picture}
\begin{small}
{\scriptsize
Using our definition of Schr\"odinger's equation in the interaction picture, we can then construct the operator $\hat{U}$. We have defined
\[
|\Psi_I(t)\rangle = \exp{(\imath\hat{H}_0t/\hbar)}|\Psi_S(t)\rangle,
\]
which can be rewritten as 
\[
|\Psi_I(t)\rangle = \exp{(\imath\hat{H}_0t/\hbar)}\exp{(-\imath\hat{H}(t-t_0)/\hbar)}|\Psi_S(t_0)\rangle,
\]
or
\[
|\Psi_I(t)\rangle = \exp{(\imath\hat{H}_0t/\hbar)}\exp{(-\imath\hat{H}(t-t_0)/\hbar)}\exp{(-\imath\hat{H}_0t_0/\hbar)}|\Psi_I(t_0)\rangle.
\]
}
\end{small}
}
\frame
{
\frametitle{Interaction picture}
\begin{small}
{\scriptsize
From the last expression we can define
\[
\hat{U}(t,t_0)=\exp{(\imath\hat{H}_0t/\hbar)}\exp{(-\imath\hat{H}(t-t_0)/\hbar)}\exp{(-\imath\hat{H}_0t_0/\hbar)}.
\]
It is then easy to convince oneself that the properties defined above are satisfied by the definition of $\hat{U}$. 
}
\end{small}
}
\frame
{
\frametitle{Interaction picture}
\begin{small}
{\scriptsize
We derive the equation of motion for $\hat{U}$ using the above definition.
This results in
\[
\imath \hbar\frac{\partial }{\partial t}\hat{U}(t,t_0) = \hat{H}_I(t)\hat{U}(t,t_0),
\]
which we integrate from $t_0$ to a time $t$ resulting in
\[
\hat{U}(t,t_0)-\hat{U}(t_0,t_0)=\hat{U}(t,t_0)-1=-\frac{\imath}{\hbar}\int_{t_0}^t dt' \hat{H}_I(t')\hat{U}(t',t_0),
\]
which can be rewritten as
\[
\hat{U}(t,t_0)=1-\frac{\imath}{\hbar}\int_{t_0}^t dt' \hat{H}_I(t')\hat{U}(t',t_0).
\]
}
\end{small}
}
\frame
{
\frametitle{Interaction picture}
\begin{small}
{\scriptsize
We can solve this equation iteratively keeping in mind the time-ordering of the of the operators
\[
\hat{U}(t,t_0)=1-\frac{\imath}{\hbar}\int_{t_0}^t dt' \hat{H}_I(t')+\left(\frac{-\imath}{\hbar}\right)^2\int_{t_0}^t dt'\int_{t_0}^{t'} dt'' \hat{H}_I(t')\hat{H}_I(t'')+\dots
\]
The third term can be written as 
\[
\int_{t_0}^t dt'\int_{t_0}^{t'} dt'' \hat{H}_I(t')\hat{H}_I(t'')=
\frac{1}{2}\int_{t_0}^t dt'\int_{t_0}^{t'} dt'' \hat{H}_I(t')\hat{H}_I(t'')
+\frac{1}{2}\int_{t_0}^t dt''\int_{t''}^{t} dt' \hat{H}_I(t')\hat{H}_I(t'').
\]
}
\end{small}
}
\frame
{
\frametitle{Interaction picture}
\begin{small}
{\scriptsize
We obtain this expression by changing the integration order in the second term
via a change of the integration variables $t'$ and $t''$  in 
\[
\frac{1}{2}\int_{t_0}^t dt'\int_{t_0}^{t'} dt'' \hat{H}_I(t')\hat{H}_I(t'').
\]
We can rewrite the terms which contain the double integral as
\[
\int_{t_0}^t dt'\int_{t_0}^{t'} dt'' \hat{H}_I(t')\hat{H}_I(t'')=\]
\[
\frac{1}{2}\int_{t_0}^t dt'\int_{t_0}^{t'} dt''\left[\hat{H}_I(t')\hat{H}_I(t'')\Theta(t'-t'')
+\hat{H}_I(t')\hat{H}_I(t'')\Theta(t''-t')\right],
\]
with $\Theta(t''-t')$ being the standard Heavyside or step function. The step function allows us to give a specific time-ordering to the above expression.
}
\end{small}
}
\frame
{
\frametitle{Interaction picture}
\begin{small}
{\scriptsize
With the $\Theta$-function we can rewrite the last expression as 
\[
\int_{t_0}^t dt'\int_{t_0}^{t'} dt'' \hat{H}_I(t')\hat{H}_I(t'')=
\frac{1}{2}\int_{t_0}^t dt'\int_{t_0}^{t'} dt''\hat{T}\left[\hat{H}_I(t')\hat{H}_I(t'')\right],
\]
where $\Hat{T}$ is the so-called time-ordering operator. 
}
\end{small}
}
\frame
{
\frametitle{Interaction picture}
\begin{small}
{\scriptsize
With this definition, we can rewrite the expression for $\hat{U}$ as 
\[
\hat{U}(t,t_0)=\sum_{n=0}^{\infty}\left(\frac{-\imath}{\hbar}\right)^n\frac{1}{n!}
\int_{t_0}^t dt_1\dots \int_{t_0}^t dt_N \hat{T}\left[\hat{H}_I(t_1)\dots\hat{H}_I(t_n)\right]=\hat{T}\exp{\left[\frac{-\imath}{\hbar}
\int_{t_0}^t dt' \hat{H}_I(t')\right]}.
\]
The above time-evolution operator in the interaction picture will be used
to derive various contributions to many-body perturbation theory. See also exercise 26 for a discussion of the various time orderings.
}
\end{small}
}
\frame
{
\frametitle{Heisenberg picture}
\begin{small}
{\scriptsize
We wish now to define a unitary transformation in terms of $\hat{H}$ by defining
\[
|\Psi_H(t)\rangle = \exp{(\imath\hat{H}t/\hbar)}|\Psi_S(t)\rangle,
\]
which is again a unitary transformation carried out now at the time $t$ on the 
wave function in the Schr\"odinger picture. If we combine this equation with 
Schr\"odinger's equation we obtain the following equation of motion
\[
\imath \hbar\frac{\partial }{\partial t}|\Psi_H(t)\rangle = 0,
\]
meaning that $|\Psi_H(t)\rangle$ is time independent. An operator in this picture is defined as
\[
\hat{O}_H(t)=
\exp{(\imath\hat{H}t/\hbar)}\hat{O}_S\exp{(-\imath\hat{H}t/\hbar)}.
\]
}
\end{small}
}
\frame
{
\frametitle{Heisenberg picture}
\begin{small}
{\scriptsize
The time dependence is then in the operator itself, and this yields in turn the
following equation of motion
\[
\imath \hbar\frac{\partial }{\partial t}\hat{O}_H(t) = \exp{(\imath\hat{H}t/\hbar)}\left[\hat{O}_H\hat{H}-\hat{H}\hat{O}_H\right]\exp{(-\imath\hat{H}t/\hbar)}=\left[\hat{O}_H(t),\hat{H}\right].
\]
We note that an operator in the Heisenberg picture can be related to the corresponding
operator in the interaction picture as 
\[
\hat{O}_H(t)=
\exp{(\imath\hat{H}t/\hbar)}\hat{O}_S\exp{(-\imath\hat{H}t/\hbar)}=\]
\[
\exp{(\imath\hat{H}_It/\hbar)}\exp{(-\imath\hat{H}_0t/\hbar)}\hat{O}_I\exp{(\imath\hat{H}_0t/\hbar)}\exp{(-\imath\hat{H}_It/\hbar)}.
\]
}
\end{small}
}
\frame
{
\frametitle{Heisenberg picture}
\begin{small}
{\scriptsize
With our definition of the time evolution operator we see that
\[
\hat{O}_H(t)=\hat{U}(0,t)\hat{O}_I\hat{U}(t,0),
\]
which in turn implies that $\hat{O}_S=\hat{O}_I(0)=\hat{O}_H(0)$, all operators are equal at $t=0$. The wave function in the Heisenberg formalism is 
related to the other pictures as 
\[
|\Psi_H\rangle=|\Psi_S(0)\rangle=|\Psi_I(0)\rangle,
\]
since the wave function in the Heisenberg picture is time independent. 
We can relate this wave function to that a given time $t$ via the time evolution operator as
\[
|\Psi_H\rangle=\hat{U}(0,t)|\Psi_I(t)\rangle.
\]
}
\end{small}
}
\frame
{
\frametitle{Adiabatic hypothesis}
\begin{small}
{\scriptsize
We assume that the interaction term is switched on gradually. Our wave function at time $t=-\infty$ and $t=\infty$ is supposed to represent a non-interacting system
given by the solution to the unperturbed part of our Hamiltonian $\hat{H}_0$.
We assume the ground state is given by $|\Phi_0\rangle$, which could be a Slater determinant.
We define our Hamiltonian as
\[
\hat{H}=\hat{H}_0+\exp{(-\varepsilon t/\hbar)}\hat{H}_I,
\]
where $\varepsilon$ is a small number. The way we write the Hamiltonian 
and its interaction term is meant to simulate the switching of the interaction.
}
\end{small}
}
\frame
{
\frametitle{Adiabatic hypothesis}
\begin{small}
{\scriptsize
The time evolution of the wave function in the interaction picture is then
\[
|\Psi_I(t) \rangle = \hat{U}_{\varepsilon}(t,t_0)|\Psi_I(t_0)\rangle,
\]
with 
\[
\hat{U}_{\varepsilon}(t,t_0)=\sum_{n=0}^{\infty}\left(\frac{-\imath}{\hbar}\right)^n\frac{1}{n!}
\int_{t_0}^t dt_1\dots \int_{t_0}^t dt_N \exp{(-\varepsilon(t_1+\dots+t_n)/\hbar)}\hat{T}\left[\hat{H}_I(t_1)\dots\hat{H}_I(t_n)\right]
\]
}
\end{small}
}
\frame
{
\frametitle{Adiabatic hypothesis}
\begin{small}
{\scriptsize
In the limit $t_0\rightarrow -\infty$, the solution ot Schr\"odinger's equation is
$|\Phi_0\rangle$, and the eigenenergies are given by 
\[
\hat{H}_0|\Phi_0\rangle=W_0|\Phi_0\rangle,
\]
meaning that 
\[
|\Psi_S(t_0)\rangle = \exp{(-\imath W_0t_0/\hbar)}|\Phi_0\rangle,
\]
with the corresponding interaction picture wave function given by
\[
|\Psi_I(t_0)\rangle = \exp{(\imath \hat{H}_0t_0/\hbar)}|\Psi_S(t_0)\rangle=|\Phi_0\rangle.
\]
}
\end{small}
}
\frame
{
\frametitle{Adiabatic hypothesis}
\begin{small}
{\scriptsize
The solution becomes time independent in the limit $t_0\rightarrow -\infty$.
The same conclusion can be reached by looking at 
\[
\imath \hbar\frac{\partial }{\partial t}|\Psi_I(t)\rangle =
\exp{(\varepsilon |t|/\hbar)}\hat{H}_I|\Psi_I(t)\rangle 
\]
and taking the limit $t\rightarrow -\infty$.
We can rewrite the equation for the wave function at a time $t=0$ as
\[
|\Psi_I(0) \rangle = \hat{U}_{\varepsilon}(0,-\infty)|\Phi_0\rangle.
\]
}
\end{small}
}
\frame
{
\frametitle{Goldstone's theorem and Gell-Mann and Low theorem on the ground state}
\begin{small}
{\scriptsize
Our wave function for ground state (after Gell-Mann and Low, see Phys.~Rev.~{\bf 84}, 350 (1951)) is then
\[
        \frac{\ket{\Psi_0}}{\left\langle\Phi_0 | \Psi_0 \right\rangle}=
    \lim_{\epsilon \rightarrow 0}
   \lim_{t'\rightarrow -\infty}
   \frac{U(0,-\infty )\ket{\Phi_0} }
   { \bra{\Phi_0} U(0,-\infty )\ket{\Phi_0} },
\]
and we ask whether this quantity exists to all orders in perturbation theory.
Goldstone's theorem states that only linked diagrams enter the expression for the final binding energy. It means that energy difference reads now
\[
\Delta E=\sum_{i=0}^{\infty}\langle \Phi_0|\hat{H}_I\left\{\frac{\hat{Q}}{W_0-\hat{H}_0}\hat{H}_I\right\}^i|\Phi_0\rangle_L,
\]
where the subscript $L$ indicates that only linked diagrams are included. In our Rayleigh-Schr\"odinger expansion, the energy difference included also unlinked diagrams. 
}
\end{small}
}
\frame
{
\frametitle{Goldstone's theorem and Gell-Mann and Low theorem on the ground state}
\begin{small}
{\scriptsize
If it does, Gell-Mann and Low showed that it is an eigenstate of $\hat{H}$ with eigenvalue
\[
 \hat{H}\frac{\ket{\Psi_0}}{\left\langle\Phi_0 | \Psi_0 \right\rangle}= E\frac{\ket{\Psi_0}}{\left\langle\Phi_0 | \Psi_0 \right\rangle}
\]
and multiplying from the left with $\langle \Phi_0|$ we can rewrite the last equation
as
\[
E-W_0=\frac{\langle \Phi_0|\hat{H}_I\ket{\Psi_0}}{\left\langle\Phi_0 | \Psi_0 \right\rangle},
\]
since $\hat{H}_0|\Phi_0\rangle = W_0|\Phi_0\rangle$. The numerator and the denominators of the last equation do not exist separately. The theorem of Gell-Mann and Low asserts that this ratio exists. 
}
\end{small}
}
\frame
{
\frametitle{Goldstone's theorem and Gell-Mann and Low theorem on the ground state}
\begin{small}
{\scriptsize
We wish to link the above expression with the corresponding expression from time-dependent perturbation theory. We write our expression as
\[
E-W_0=\Delta E= \lim_{\epsilon \rightarrow 0^+}
   \frac{\bra{\Phi_0(0)}\hat{H}_IU_{\epsilon }(0,-\infty )\ket{\Phi_0(-\infty)} }
   { \bra{\Phi_0(0)} U_{\epsilon}(0,-\infty )\ket{\Phi_0(-\infty)} },
\]
with a numerator 
\[
N=\lim_{\epsilon \rightarrow 0^+}\bra{\Phi_0(0)}\hat{H}_IU_{\epsilon}(0,-\infty )\ket{\Phi_0(-\infty)}, 
\]
which we rewrite as
\[
 N=\lim_{\epsilon \rightarrow 0^+}\bra{\Phi_0(0)}\hat{H}_I(t=0)\displaystyle\sum_{n=0}^{\infty}\frac{(-i)^n}{n!}
   \int_{-\infty}^{0}dt_1  \int_{-\infty}^{0}dt_2\dots  \int_{-\infty}^{0}dt_n}\exp{(\epsilon/\hbar(t_1+\dots+t_n))} \hat{T}\left[H_1(t_1)H_1(t_2)\dots H_1(t_n)\right] \ket{\Phi_0(-\infty)}. 
\]
}
\end{small}
}
\frame
{
\frametitle{Goldstone's theorem and Gell-Mann and Low theorem on the ground state}
\begin{small}
{\scriptsize
A linked diagram (or connected diagram) is a diagram which is linked to the last interaction vertex at $t=0$.
We divide the diagrams into linked and unlinked. 
In general, the way we can distribute $\mu$ unlinked diagrams among the total of $n$ diagrams is given by the combinatorial factor
\[
\left(\begin{array}{c} n \\ \mu \end{array}\right) = \frac{n!}{\mu!\nu!},
\]
and using the following relation
\[
\sum_{n=0}^{\infty}\frac{1}{n!}\sum_{\mu+\nu=n}^{\infty}\frac{n!}{\mu!\nu!}=\sum_{\mu=0}^{\infty}\frac{1}{\mu!}\sum_{\nu}^{\infty}\frac{1}{\nu!},
\]
we can rewrite the numerator $N$ as 
\[
N=\bra{\Phi_0(0)}\hat{H}_IU_{\epsilon}(0,-\infty )\ket{\Phi_0(-\infty)}_L\bra{\Phi_0(0)}U_{\epsilon}(0,-\infty )\ket{\Phi_0(-\infty)},  
=N_LD,
\]
with $N_L$ now containing only linked terms
\[
 N_L=\lim_{\epsilon \rightarrow 0^+}\bra{\Phi_0(0)}\hat{H}_I(t=0)\displaystyle\sum_{\nu=0}^{\infty}\frac{(-i)^{\nu}}{\nu!}
   \int_{-\infty}^{0}dt_1  \int_{-\infty}^{0}dt_2\dots  \int_{-\infty}^{0}dt_n}\exp{(\epsilon/\hbar(t_1+\dots+t_n))} \hat{T}\left[H_1(t_1)H_1(t_2)\dots H_1(t_n)\right] \ket{\Phi_0(-\infty)}_L, 
\]
with the subscript $L$ indicating that only linked diagrams appear, that is those diagrams which are linked to the last interaction vertex.
}
\end{small}
}
\frame
{
\frametitle{Goldstone's theorem and Gell-Mann and Low theorem on the ground state}
\begin{small}
{\scriptsize
A linked diagram (or connected diagram) is a diagram which is linked to the last interaction vertex at $t=0$.
We divide the diagrams into linked and unlinked. 
In general, the way we can distribute $\mu$ unlinked diagrams among the total of $n$ diagrams is given by the combinatorial factor
\[
\left(\begin{array}{c} n \\ \mu \end{array}\right) = \frac{n!}{\mu!\nu!},
\]
and using the following relation
\[
\sum_{n=0}^{\infty}\frac{1}{n!}\sum_{\mu+\nu=n}^{\infty}\frac{n!}{\mu!\nu!}=\sum_{\mu=0}^{\infty}\frac{1}{\mu!}\sum_{\nu}^{\infty}\frac{1}{\nu!},
\]
we can rewrite the numerator $N$ as 
\[
N=\bra{\Phi_0(0)}\hat{H}_IU_{\epsilon}(0,-\infty )\ket{\Phi_0(-\infty)}_L\bra{\Phi_0(0)}U_{\epsilon}(0,-\infty )\ket{\Phi_0(-\infty)},  
=N_LD,
\]
with $N_L$ now containing only linked terms
\[
 N_L=\lim_{\epsilon \rightarrow 0^+}\bra{\Phi_0(0)}\hat{H}_I(t=0)\displaystyle\sum_{\nu=0}^{\infty}\frac{(-i)^{\nu}}{\nu!}
   \int_{-\infty}^{0}dt_1  \int_{-\infty}^{0}dt_2\dots  \int_{-\infty}^{0}dt_n}\exp{(\epsilon/\hbar(t_1+\dots+t_n))} \hat{T}\left[H_1(t_1)H_1(t_2)\dots H_1(t_n)\right] \ket{\Phi_0(-\infty)}_L, 
\]
with the subscript $L$ indicating that only linked diagrams appear, that is those diagrams which are linked to the last interaction vertex.
}
\end{small}
}
\frame
{
\frametitle{Goldstone's theorem and Gell-Mann and Low theorem on the ground state}
\begin{small}
{\scriptsize
We note that also that the term $D$ is nothing but the denominator of the equation for the energy. We obtain then the following expression for the energy
\[
E-W_0=\Delta E=N_L= \bra{\Phi_0(0)}\hat{H}_IU_{\epsilon}(0,-\infty )\ket{\Phi_0(-\infty)}_L,
\]
and Goldstone's theorem is then proved. 
The corresponding expression from Rayleigh-Schr\"odinger perturbation theory is given by
\[
\Delta E=\langle \Phi_0|\left(\hat{H}_I+\hat{H}_I\frac{\hat{Q}}{W_0-\hat{H}_0}\hat{H}_I+
\hat{H}_I\frac{\hat{Q}}{W_0-\hat{H}_0}\hat{H}_I\frac{\hat{Q}}{W_0-\hat{H}_0}\hat{H}_I+\dots\right)|\Phi_0\rangle_C.
\]
}
\end{small}
}
\frame
{
\frametitle{Goldstone's theorem and Gell-Mann and Low theorem on the ground state}
\begin{small}
{\scriptsize
An important point in the derivation of the Gell-Mann and Low theorem
\[
E-W_0=\frac{\langle \Phi_0|\hat{H}_I\ket{\Psi_0}}{\left\langle\Phi_0 | \Psi_0 \right\rangle},
\]
is that the numerator and the denominators of the last equation do not exist separately. The theorem of Gell-Mann and Low asserts that this ratio exists. To prove it we proceed as follows. Consider the expression
\[
(\hat{H}_0-E)U_{\epsilon }(0,-\infty )\ket{\Phi_0}=\left[\hat{H}_0,U_{\epsilon }(0,-\infty )\right]\ket{\Phi_0}.
\]
}
\end{small}
}
\frame
{
\frametitle{Goldstone's theorem and Gell-Mann and Low theorem on the ground state}
\begin{small}
{\scriptsize
To evaluate the commutator 
\[
(\hat{H}_0-E)U_{\epsilon }(0,-\infty )\ket{\Phi_0}=\left[\hat{H}_0,U_{\epsilon }(0,-\infty )\right]\ket{\Phi_0}.
\]
we write the associate commutator as
\[
\left[\hat{H}_0,\hat{H}_I(t_1)\hat{H}_I(t_2)\dots \hat{H}_I(t_n)\right]=
\left[\hat{H}_0,\hat{H}_I(t_1)\right]\hat{H}_I(t_2)\dots \hat{H}_I(t_n)+
\]
\[
\dots+\hat{H}_I(t_1)\left[\hat{H}_0,\hat{H}_I(t_2)\right]\hat{H}_I(t_3)\dots \hat{H}_I(t_n)+\dots
\]
Using the equation of motion for an operator in the interaction picture we have
\[
\imath \hbar\frac{\partial }{\partial t}\hat{H}_I(t) = \left[\hat{H}_I(t),\hat{H}_0\right].
\]
Each of the above commutators yield then a time derivative!
}
\end{small}
}
\frame
{
\frametitle{Goldstone's theorem and Gell-Mann and Low theorem on the ground state}
\begin{small}
{\scriptsize
We have then
\[
\left[\hat{H}_0,\hat{H}_I(t_1)\hat{H}_I(t_2)\dots \hat{H}_I(t_n)\right]=\imath \hbar\left(\frac{\partial }{\partial t_n}+\frac{\partial }{\partial t_1}+\dots+\frac{\partial }{\partial t_n}\right) \hat{H}_I(t_1)\hat{H}_I(t_2)\dots\hat{H}_I(t_n),
\]
meaning that we can rewrite
\[
(\hat{H}_0-E)U_{\epsilon }(0,-\infty )\ket{\Phi_0}=\left[\hat{H}_0,U_{\epsilon }(0,-\infty )\right]\ket{\Phi_0},
\]
as
\[
(\hat{H}_0-E)U_{\epsilon }(0,-\infty )\ket{\Phi_0}=-\sum_{n=1}^{\infty}\left(\frac{-\imath}{\hbar}\right)^{n-1}\frac{1}{n!}
\int_{t_0}^t dt_1\dots \int_{t_0}^t dt_N \exp{(-\varepsilon(t_1+\dots+t_n)/\hbar)}
\]
\[
\times\sum_{i=1}^n(\frac{\partial }{\partial t_i} )\hat{T}\left[\hat{H}_I(t_1)\dots\hat{H}_I(t_n)\right].
\]
}
\end{small}
}
\frame
{
\frametitle{Goldstone's theorem and Gell-Mann and Low theorem on the ground state}
\begin{small}
{\scriptsize
All the time derivatives in this equation 
\[
(\hat{H}_0-E)U_{\epsilon }(0,-\infty )\ket{\Phi_0}=-\sum_{n=1}^{\infty}\left(\frac{-\imath}{\hbar}\right)^{n-1}\frac{1}{n!}
\int_{t_0}^t dt_1\dots \int_{t_0}^t dt_N \exp{(-\varepsilon(t_1+\dots+t_n)/\hbar)}
\]
\[
\times\sum_{i=1}^n(\frac{\partial }{\partial t_i} )\hat{T}\left[\hat{H}_I(t_1)\dots\hat{H}_I(t_n)\right],
\]
make the same contribution, as can be seen by changing dummy variables. We can therefore retain just one time derivative $\partial/\partial t$ and multiply with $n$. Integrating by parts wrt $t_1$  we obtain two terms. 
}
\end{small}
}
\frame
{
\frametitle{Goldstone's theorem and Gell-Mann and Low theorem on the ground state}
\begin{small}
{\scriptsize
Integrating by parts wrt $t_1$  one can finally show that
\[
        \frac{\ket{\Psi_0}}{\left\langle\Phi_0 | \Psi_0 \right\rangle}=
    \lim_{\epsilon \rightarrow 0}
   \lim_{t'\rightarrow -\infty}
   \frac{U(0,-\infty )\ket{\Phi_0} }
   { \bra{\Phi_0} U(0,-\infty )\ket{\Phi_0} },
\]
For more details about the derivation, see Gell-Mann and Low, Phys.~Rev.~{\bf 84}, 350  (1951). See also chapter 6.2 of Raimes or Fetter and Walecka, chapter 3.6.
}
\end{small}
}
\frame
{
\frametitle{Goldstone's theorem and Gell-Mann and Low theorem on the ground state}
\begin{small}
{\scriptsize
In the present discussion of the time-dependent theory we will make
use of the so-called complex-time approach to describe the time
evolution operator $U$.
This means that we
allow the time $t$ to be rotated by a small angle $\epsilon$
relative to the real time axis. The complex time $t$ is then
related to the real time $\tilde{t}$ by
\[
t=\tilde{t}(1-i\epsilon ).
\]
Let us first study the true eigenvector $\Psi_{\alpha}$ which evolves
from the unperturbed eigenvectors $\Phi_{\alpha}$ through the action of the
time development operator
\[
   U_{\varepsilon}(t,t')=\lim_{\epsilon \rightarrow 0}
   \lim_{t'\rightarrow -\infty}
   {\displaystyle\sum_{n=0}^{\infty}\frac{(-i)^n}{n!}
   \int_{t'}^{t}dt_1  \int_{t'}^{t}dt_2\dots  \int_{t'}^{t}dt_n}
\]
\[
	      \times T\left[H_1(t_1)H_1(t_2)\dots H_1(t_n)\right],
\]
where $T$ stands for the correct time-ordering.
}
\end{small}
}
\frame
{
\frametitle{Goldstone's theorem and Gell-Mann and Low theorem on the ground state}
\begin{small}
{\scriptsize
In time-dependent
perturbation theory we let $\Psi_{\alpha}$ develop from $\Phi_{\alpha}$ in the
remote past to a given time $t$
\[
    \frac{\ket{\Psi_{\alpha}}}
    {\left\langle\psi_{\alpha} | \Psi_{\alpha} \right\rangle}=
    \lim_{\epsilon \rightarrow 0}
   \lim_{t'\rightarrow -\infty}
   \frac{U_{\varepsilon}(t,t' )\ket{\psi_{\alpha}} }
   { \bra{\psi_{\alpha}} U(t,t' )\ket{\Phi_{\alpha}} },
\]
and similarly, we let
$\Psi_{\beta}$ develop from $\Phi_{\beta}$ in the remote future
\[
    \frac{\bra{\Psi_{\beta}}}{\left\langle
    \psi_{\beta} | \Psi_{\beta} \right\rangle}=
    \lim_{\epsilon \rightarrow 0}
    \lim_{t'\rightarrow \infty}
    \frac{\bra{\psi_{\beta}}U_{\varepsilon}(t' ,t) }
    { \bra{\psi_{\beta}} U_{\varepsilon}(t' ,t)\ket{\Phi_{\beta}} }.
\]
}
\end{small}
}
\frame
{
\frametitle{Goldstone's theorem and Gell-Mann and Low theorem on the ground state}
\begin{small}
{\scriptsize
Here we are interested in the expectation value of a given
operator ${\cal O}$ acting at a time $t=0$. This can be achieved
from the two previous equations defining
\[
     \ket{\Psi_{\alpha ,\beta}'}=
     \frac{\ket{\Psi_{\alpha ,\beta}}}
     {\left\langle\Phi_{\alpha ,\beta} | \Psi_{\alpha ,\beta} \right\rangle}
\]
we have
\[
   {\cal O}_{\alpha\beta}
  =\frac{N_{\beta\alpha}}{D_{\beta}D_{\alpha}},
\]
where we have introduced
\[
   N_{\beta\alpha}=
   \bra{\Phi_{\beta}}U_{\varepsilon}(\infty ,0){\cal O}U_{\varepsilon}(0,-\infty )\ket{\Phi_{\alpha}} ,
\]
and 
\[
   D_{\alpha ,\beta}=
   \sqrt{\bra{\psi_{\alpha ,\beta}}
   U_{\varepsilon}(\infty ,0)U_{\varepsilon}(0,-\infty )\ket{\Phi_{\alpha ,\beta}}}. 
\]
}
\end{small}
}
\frame
{
\frametitle{Goldstone's theorem and Gell-Mann and Low theorem on the ground state}
\begin{small}
{\scriptsize
If the operator ${\cal O}$ stands for the hamiltonian $H$ we obtain
\[
    {\displaystyle  \frac{\bra{\Psi_{\lambda}'}H\ket{\Psi_{\lambda}'} }
   { \left\langle\Psi_{\lambda}' | \Psi_{\lambda}' \right\rangle} }
\]
At this stage, {\em it is important to observe} that our 
expression for the expectation value of a given operator ${\cal O}$
{\em is hermitian} insofar ${\cal O}^{\dagger}={\cal O}$. This is readily 
demonstrated. The above equation is of the general form
\[
U(t,t_0){\cal O}U(t_0,-t),
\]
and noting that 
\[
   U^{\dagger}(t,t_0)=
   \left({\displaystyle e^{iH_0t}e^{-iH(t-t_0)}e^{-iH_0t}}\right)^{\dagger}
   =U(t_0,-t),
\]
since $H^{\dagger}=H$ and $H_0^{\dagger}=H_0$, we have that
\[
    \left(U(t,t_0){\cal O}U(t_0,-t)\right)^{\dagger}
    =U(t,t_0){\cal O}U(t_0,-t).
\]
The question we pose now is what happens in the limit $\varepsilon\rightarrow 0$?
Do we get results which are meaningful?
}
\end{small}
}
\frame
{
\frametitle{Goldstone's theorem and Gell-Mann and Low theorem on the ground state}
\begin{small}
{\scriptsize
Our wave function for ground state is then
\[
        \frac{\ket{\Psi_0}}{\left\langle\Phi_0 | \Psi_0 \right\rangle}=
    \lim_{\epsilon \rightarrow 0}
   \lim_{t'\rightarrow -\infty}
   \frac{U(0,-\infty )\ket{\Phi_0} }
   { \bra{\Phi_0} U(0,-\infty )\ket{\Phi_0} },
\]
meaning that the energy difference is given by
\[
E_0-W_0=\Delta E_0= \lim_{\epsilon \rightarrow 0}
   \lim_{t'\rightarrow -\infty}
   \frac{\bra{\Phi_0}\hat{H}_IU_{\varepsilon}(0,-\infty )\ket{\Phi_0} }
   { \bra{\Phi_0} U_{\varepsilon}(0,-\infty )\ket{\Phi_0} },
\]
and we ask whether this quantity exists to all orders in perturbation theory.
}
\end{small}
}
\frame
{
\frametitle{Goldstone's theorem and Gell-Mann and Low theorem on the ground state}
\begin{small}
{\scriptsize
If it does, Gell-Mann and Low showed that it is an eigenstate of $\hat{H}$ with eigenvalue
\[
 \hat{H}\frac{\ket{\Psi_0}}{\left\langle\Phi_0 | \Psi_0 \right\rangle}= E_0\frac{\ket{\Psi_0}}{\left\langle\Phi_0 | \Psi_0 \right\rangle}
\]
and multiplying from the left with $\langle \Phi_0|$ we can rewrite the last equation
as
\[
E_0-W_0=\frac{\langle \Phi_0|\hat{H}_I\ket{\Psi_0}}{\left\langle\Phi_0 | \Psi_0 \right\rangle},
\]
since $\hat{H}_0|\Phi_0\rangle = W_0|\Phi_0\rangle$. The numerator and the denominators of the last equation do not exist separately. The theorem of Gell-Mann and Low asserts that this ratio exists. 
}
\end{small}
}
\frame
{
\frametitle{Goldstone's theorem and Gell-Mann and Low theorem on the ground state}
\begin{small}
{\scriptsize
Goldstone's theorem states that only linked diagrams enter the expression for the final binding energy. It means that energy difference reads now
\[
\Delta E_0=\sum_{i=0}^{\infty}\langle \Phi_0|\hat{H}_I\left\{\frac{\hat{Q}}{W_0-\hat{H}_0}\hat{H}_I\right\}^i|\Phi_0\rangle_L,
\]
where the subscript $L$ indicates that only linked diagrams are included. In our Rayleigh-Schr\"odinger expansion, the energy difference included also unlinked diagrams. 
}
\end{small}
}
\frame
{
\frametitle{Goldstone's theorem and Gell-Mann and Low theorem on the ground state}
\begin{small}
{\scriptsize
We wish to link the above expression with the corresponding expression from time-dependent perturbation theory. We write our expression as
\[
E_0-W_0=\Delta E_0= \lim_{\epsilon \rightarrow 0^+}
   \frac{\bra{\Phi_0(0)}\hat{H}_IU_{\epsilon }(0,-\infty )\ket{\Phi_0(-\infty)} }
   { \bra{\Phi_0(0)} U_{\epsilon}(0,-\infty )\ket{\Phi_0(-\infty)} },
\]
with a numerator 
\[
N=\lim_{\epsilon \rightarrow 0^+}\bra{\Phi_0(0)}\hat{H}_IU_{\epsilon}(0,-\infty )\ket{\Phi_0(-\infty)}, 
\]
which we rewrite as
\[
 N=\lim_{\epsilon \rightarrow 0^+}\bra{\Phi_0(0)}\hat{H}_I(t=0)\displaystyle\sum_{n=0}^{\infty}\frac{(-i)^n}{n!}
   \int_{-\infty}^{0}dt_1  \int_{-\infty}^{0}dt_2\dots  \int_{-\infty}^{0}dt_n}\exp{(\epsilon/\hbar(t_1+\dots+t_n))} \hat{T}\left[H_1(t_1)H_1(t_2)\dots H_1(t_n)\right] \ket{\Phi_0(-\infty)}. 
\]
}
\end{small}
}
\frame
{
\frametitle{Goldstone's theorem and Gell-Mann and Low theorem on the ground state}
\begin{small}
{\scriptsize
From this term we can obtain both linked and unlinked contributions. Goldstone's theorem states that only linked diagrams enter the expression for the final binding energy. 
A linked diagram (or connected diagram) is a diagram which is linked to the last interaction vertex at $t=0$.
We label the number of linked diagrams with the variable $\nu$ and the number of unlinked with $\mu$  with $n=\nu+\mu$.  The number of unlinked diagrams is then $\mu=n-\nu$. 
}
\end{small}
}
\frame
{
\frametitle{Goldstone's theorem and Gell-Mann and Low theorem on the ground state}
\begin{small}
{\scriptsize
In general, the way we can distribute $\mu$ unlinked diagrams among the total of $n$ diagrams is given by the combinatorial factor
\[
\left(\begin{array}{c} n \\ \mu \end{array}\right) = \frac{n!}{\mu!\nu!},
\]
and using the following relation
\[
\sum_{n=0}^{\infty}\frac{1}{n!}\sum_{\mu+\nu=n}^{\infty}\frac{n!}{\mu!\nu!}=\sum_{\mu=0}^{\infty}\frac{1}{\mu!}\sum_{\nu}^{\infty}\frac{1}{\nu!},
\]
we can rewrite the numerator $N$ as 
\[
N=\bra{\Phi_0(0)}\hat{H}_IU_{\epsilon}(0,-\infty )\ket{\Phi_0(-\infty)}_L\bra{\Phi_0(0)}U_{\epsilon}(0,-\infty )\ket{\Phi_0(-\infty)}=N_LD.
\]
}
\end{small}
}
\frame
{
\frametitle{Goldstone's theorem and Gell-Mann and Low theorem on the ground state}
\begin{small}
{\scriptsize
We define  $N_L$ to contain only linked terms
%\[
% N_L=\lim_{\epsilon \rightarrow 0^+}\bra{\Phi_0(0)}\hat{H}_I(t=0)\sum_{\nu=0}^{\infty}\frac{(-i)^{\nu}}{\nu!}\int_{-\infty}^{0}dt_1  \int_{-\infty}^{0}dt_2\dots  \int_{-\infty}^{0}dt_n}\exp{(\epsilon/\hbar(t_1+\dots+t_n))} \hat{T}\left[H_1(t_1)H_1(t_2)\dots H_1(t_n)\right] \ket{\Phi_0(-\infty)}_L, 
%\]
with the subscript $L$ indicating that only linked diagrams appear, that is those diagrams which are linked to the last interaction vertex.
}
\end{small}
}
\frame
{
\frametitle{Goldstone's theorem and Gell-Mann and Low theorem on the ground state}
\begin{small}
{\scriptsize
We note that also that the term $D$ is nothing but the denominator of the equation for the energy. We obtain then the following expression for the energy
\[
E_0-W_0=\Delta E_0=N_L= \bra{\Phi_0(0)}\hat{H}_IU_{\epsilon}(0,-\infty )\ket{\Phi_0(-\infty)}_L,
\]
and Goldstone's theorem is then proved. 
The corresponding expression from Rayleigh-Schr\"odinger perturbation theory is given by
\[
\Delta E_0=\langle \Phi_0|\left(\hat{H}_I+\hat{H}_I\frac{\hat{Q}}{W_0-\hat{H}_0}\hat{H}_I+
\hat{H}_I\frac{\hat{Q}}{W_0-\hat{H}_0}\hat{H}_I\frac{\hat{Q}}{W_0-\hat{H}_0}\hat{H}_I+\dots\right)|\Phi_0\rangle_C.
\]
}
\end{small}
}


% Green's function theory

\frame
{
\frametitle{Single-particle Green's functions}
\begin{small}
{\scriptsize
Let us consider a particle in free space described by a single particle Hamiltonian $h_{1}$. Its eigenstates and eigenenergies are

$$
h_{1}\left|\phi_{n}\right\rangle=\varepsilon_{n}\left|\phi_{n}\right\rangle
$$

In general, if we put the particle in one of its $\left|\phi_{n}\right\rangle$ orbits, it will remain in the same state forever. Instead, we immagine to prepare the system in a generic state $\left|\psi_{t r}\right\rangle$ (tr stands for 'trial') and then follow its time evolution. If the trial state is created at time $t=0$, the wavefunction at a later time $t$ is given by [see Eq. (1.47)]

$$
\begin{aligned}
|\psi(t)\rangle & =e^{-i h_{1} t / \hbar}\left|\psi_{t r}\right\rangle \\
& =\sum_{n}\left|\phi_{n}\right\rangle e^{-i \varepsilon_{n} t / \hbar}\left\langle\phi_{n} \mid \psi_{t r}\right\rangle
\end{aligned}
$$

}
\end{small}
}


\frame
{
\frametitle{Single-particle Green's functions}
\begin{small}
{\scriptsize
The second line shows that if one knows the eigentstates $\left|\phi_{n}\right\rangle$, it is relatively simple to compute the time evution: one expands $\left|\psi_{t r}\right\rangle$ in this basis and let every component propagate independently. Eventually, at time $t$, we want to know the probability amplitude that a measurement would find the particle at position $\mathbf{r}$,

$$
\begin{aligned}
\langle\mathbf{r} \mid \psi(t)\rangle & =\left\langle\mathbf{r}\left|e^{-i h_{1} t / \hbar}\right| \psi_{t r}\right\rangle \\
& =\int d \mathbf{r}^{\prime}\left\langle\mathbf{r}\left|e^{-i h_{1} t / \hbar}\right| \mathbf{r}^{\prime}\right\rangle\left\langle\mathbf{r}^{\prime} \mid \psi_{t r}\right\rangle
\end{aligned}
$$

$$
\begin{aligned}
& =\int d \mathbf{r}^{\prime} \sum_{n}\left\langle\mathbf{r} \mid \phi_{n}\right\rangle e^{-i \varepsilon_{n} t / \hbar}\left\langle\phi_{n} \mid \mathbf{r}^{\prime}\right\rangle\left\langle\mathbf{r}^{\prime} \mid \psi_{t r}\right\rangle \\
& \equiv \int d \mathbf{r}^{\prime} G\left(\mathbf{r}, \mathbf{r}^{\prime} ; t\right) \psi_{t r}\left(\mathbf{r}^{\prime}\right),
\end{aligned}
$$

which defines the propagator $G$.
}
\end{small}
}


\frame
{
\frametitle{Single-particle Green's functions}
\begin{small}
{\scriptsize
 Obviously, once $G\left(\mathbf{r}, \mathbf{r}^{\prime} ; t\right)$
 is known it can be used to calculate the evolution of any initial
 state. However,there is more information included in the
 propagator. This is apparent from the expansion in the third line of
 Eq (2.3): first, the braket $\left\langle\phi_{n} \mid
 \mathbf{r}\right\rangle=\left\langle\phi_{n}\left|\psi^{\dagger}(\mathbf{r})\right|
 0\right\rangle$ gives us the probability that putting a particle at
 position $\mathbf{r}$ and mesuring its energy rgiht away, would make
 the system to collapse into the eigenstate
 $\left|\phi_{n}\right\rangle$. Second, the time evolution is a
 superposition of waves propagating with different energies and could
 be inverted to find the eigenspectrum. Immagine an experiment in
 which the particle is put at position $\mathbf{r}$ and picked up at
 $\mathbf{r}^{\prime}$ after some time t. If one can do this for
 different positions and elapsed timesand with good resolution-then a
 Fourier transform would simply give back the full eigenvalue
 spectrum. Such an experiment is a lot of work to carry out! But would
 give us complete information on our particle.

}
\end{small}
}


\frame
{
\frametitle{Single-particle Green's functions}
\begin{small}
{\scriptsize
We now want to apply the above ideas to see what we can learn by
adding and removing a particle in an environment when many others are
present. This can cause the particle to behave in an unxepected way,
induce collective excitations of the full systems, and so
on. Moreover, the role played by the physical vacuum in the above
example, is now taken by an many-body state (usually its ground
state). Thus, it is also possible to probe the system by removing
particles.

}
\end{small}
}


\frame
{
\frametitle{Single-particle Green's functions}
\begin{small}
{\scriptsize
In the following we consider the Heisenberg description of the field operators,

$$
\psi_{s}^{\dagger}(\mathbf{r}, t)=e^{i H t / \hbar} \psi_{s}^{\dagger}(\mathbf{r}) e^{-i H t / \hbar}
$$

where the subscript $s$ serves to indicate possible internal degrees of freedom (spin, isospin, etc...). We omit the superscrips $\mathrm{H}$ (Hiesenberg) and S (Scrödinger) from the operators since the two pictures can be distinguished from the presence of the time variable, which appears only in the first case. Similarly,

$$
\psi_{s}(\mathbf{r}, t)=e^{i H t / \hbar} \psi_{s}(\mathbf{r}) e^{-i H t / \hbar},
$$

}
\end{small}
}


\frame
{
\frametitle{Single-particle Green's functions}
\begin{small}
{\scriptsize
For the case of a general single-particle basis $\left\{u_{\alpha}(\mathbf{r})\right\}$ one uses the following creation and annihilation operators

$$
\begin{aligned}
& c_{\alpha}^{\dagger}(t)=e^{i H t / \hbar} c_{\alpha}^{\dagger} e^{-i H t / \hbar}, \\
& c_{\alpha}(t)=e^{i H t / \hbar} c_{\alpha} e^{-i H t / \hbar}
\end{aligned}
$$

which are related to $\psi_{s}^{\dagger}(\mathbf{r}, t)$ and $\psi_{s}(\mathbf{r}, t)$ through Eqs. (1.14) and (1.15).

}
\end{small}
}


\frame
{
\frametitle{Single-particle Green's functions}
\begin{small}
{\scriptsize
In most applications the Hamiltonian is split in a unperturbed part $H_{0}$ and a residual interaction

$$
H=H_{0}+V .
$$

The N-body eigenstates of the full Hamiltonian are indicated with $\left|\Psi_{n}^{N}\right\rangle$, while $\left|\Phi_{n}^{N}\right\rangle$ are the unperturbed ones

$$
\begin{aligned}
H\left|\Psi_{n}^{N}\right\rangle & =E_{n}^{N}\left|\Psi_{n}^{N}\right\rangle \\
H_{0}\left|\Phi_{n}^{N}\right\rangle & =E_{n}^{(0), N}\left|\Phi_{n}^{N}\right\rangle
\end{aligned}
$$

The definitions given in the following are general and do not depend on the type of interaction being used. Thus, most properties of Green's functions result from genaral principles of quantum mechanics and are valid for any system.
}
\end{small}
}


\frame
{
\frametitle{Single-particle Green's functions}
\begin{small}
{\scriptsize
The two-points Green's function describies the propagation of one particle or one hole on top of the ground state $\left|\Psi_{0}^{N}\right\rangle$. This is defined by

$$
g_{s s^{\prime}}\left(\mathbf{r}, t ; \mathbf{r}^{\prime}, t^{\prime}\right)=-\frac{i}{\hbar}\left\langle\Psi_{0}^{N}\left|T\left[\psi_{s}(\mathbf{r}, t) \psi_{s^{\prime}}^{\dagger}\left(\mathbf{r}^{\prime}, t^{\prime}\right)\right]\right| \Psi_{0}^{N}\right\rangle
$$

where $T[\cdots]$ is the time ordering operator that imposes a change of sing for each exchange of two fermion operators

$$
T\left[\psi_{s}(\mathbf{r}, t) \psi_{s^{\prime}}^{\dagger}\left(\mathbf{r}^{\prime}, t^{\prime}\right)\right]= \begin{cases}\psi_{s}(\mathbf{r}, t) \psi_{s^{\prime}}^{\dagger}\left(\mathbf{r}^{\prime}, t^{\prime}\right), & t>t^{\prime} \\ \pm \psi_{s^{\prime}}^{\dagger}\left(\mathbf{r}^{\prime}, t^{\prime}\right) \psi_{s}(\mathbf{r}, t), & t^{\prime}>t\end{cases}
$$

where the upper (lower) sign is for bosons (fermions). A similar definition can be given for the non interacting state $\left|\Phi_{0}^{N}\right\rangle$, in this case the Heisenberg operators (2.4) to (2.7) must evolve only according to $H_{0}$ and the notation $g^{(0)}$ is used.
}
\end{small}
}


\frame
{
\frametitle{Single-particle Green's functions}
\begin{small}
{\scriptsize
If the Hamiltonian does not depend on time, the propagator (2.11) depends only on the difference $t-t^{\prime}$

$$
\begin{aligned}
g_{s s^{\prime}}\left(\mathbf{r}, \mathbf{r}^{\prime} ; t-t^{\prime}\right)= & -\frac{i}{\hbar} \theta\left(t-t^{\prime}\right)\left\langle\Psi_{0}^{N}\left|\psi_{s}(\mathbf{r}) e^{-i\left(H-E_{0}^{N}\right)\left(t-t^{\prime}\right) / \hbar} \psi_{s^{\prime}}^{\dagger}\left(\mathbf{r}^{\prime}\right)\right| \Psi_{0}^{N}\right\rangle \\
& \mp \frac{i}{\hbar} \theta\left(t^{\prime}-t\right)\left\langle\Psi_{0}^{N}\left|\psi_{s^{\prime}}^{\dagger}\left(\mathbf{r}^{\prime}\right) e^{i\left(H-E_{0}^{N}\right)\left(t-t^{\prime}\right) / \hbar} \psi_{s}(\mathbf{r})\right| \Psi_{0}^{N}\right\rangle .
\end{aligned}
$$

In this case it is useful to Fourier transform with respect to time and define

$$
g_{s s^{\prime}}\left(\mathbf{r}, \mathbf{r}^{\prime} ; \omega\right)=\int d \tau e^{i \omega \tau} g_{s s^{\prime}}\left(\mathbf{r}, \mathbf{r}^{\prime} ; \tau\right)
$$

}
\end{small}
}


\frame
{
\frametitle{Single-particle Green's functions}
\begin{small}
{\scriptsize
By using the relation

$$
\theta( \pm \tau)=\mp \lim _{\eta \rightarrow 0^{+}} \frac{1}{2 \pi i} \int_{-\infty}^{+\infty} d \omega \frac{e^{-i \omega \tau}}{\omega \pm i \eta}
$$

one obtains

$$
\begin{aligned}
g_{s s^{\prime}}\left(\mathbf{r}, \mathbf{r}^{\prime} ; \omega\right)= & g_{s s^{\prime}}^{p}\left(\mathbf{r}, \mathbf{r}^{\prime} ; \omega\right)+g_{s s^{\prime}}^{h}\left(\mathbf{r}, \mathbf{r}^{\prime} ; \omega\right) \\
= & \left\langle\Psi_{0}^{N}\left|\psi_{s}(\mathbf{r}) \frac{1}{\hbar \omega-\left(H-E_{0}^{N}\right)+i \eta} \psi_{s^{\prime}}^{\dagger}\left(\mathbf{r}^{\prime}\right)\right| \Psi_{0}^{N}\right\rangle \\
& \mp\left\langle\Psi_{0}^{N}\left|\psi_{s^{\prime}}^{\dagger}\left(\mathbf{r}^{\prime}\right) \frac{1}{\hbar \omega+\left(H-E_{0}^{N}\right)-i \eta} \psi_{s}(\mathbf{r})\right| \Psi_{0}^{N}\right\rangle,
\end{aligned}
$$
}
\end{small}
}


\frame
{
\frametitle{Single-particle Green's functions}
\begin{small}
{\scriptsize
In Eq. (2.16), $g^{p}$ propagates a particle from $\mathbf{r}^{\prime}$ to $\mathbf{r}$, while $g^{h}$ propagates a hole from $\mathbf{r}$ to $\mathbf{r}^{\prime}$. Note that the interpretation is that a particle is added at $\mathbf{r}^{\prime}$, and later on some (indistiguishable) particle is removed from $\mathbf{r}^{\prime}$ (and similarly for holes). In the mean time, it is the fully correlated $(N \pm 1)$ body system that propagates. We will discuss in the next chapter that in many cases-and especially in the vicinity of the Fermi surface-this motion mantains many characteristics that are typical of a particle moving in free space, even if the motion itself could actually be a collective excitation of many constituents. But since it looks like a single particle state we may still refer to it as quasiparticle.

}
\end{small}
}
\frame
{
\frametitle{Single-particle Green's functions}
\begin{small}
{\scriptsize
The same definitions can be made for any orthonormal basis $\{\alpha\}$, leading to the realtions

$$
g_{\alpha \beta}\left(t, t^{\prime}\right)=-\frac{i}{\hbar}\left\langle\Psi_{0}^{N}\left|T\left[c_{\alpha}(t) c_{\beta}^{\dagger}\left(t^{\prime}\right)\right]\right| \Psi_{0}^{N}\right\rangle
$$

where

$$
g_{s s^{\prime}}\left(\mathbf{r}, t ; \mathbf{r}^{\prime}, t^{\prime}\right)=\sum_{\alpha \beta} u_{\alpha}(\mathbf{r}, s) g_{\alpha \beta}\left(t, t^{\prime}\right) u_{\beta}^{*}\left(\mathbf{r}^{\prime}, s^{\prime}\right)
$$

and

$$
\begin{aligned}
g_{\alpha \beta}(\omega)= & \left\langle\Psi_{0}^{N}\left|c_{\alpha} \frac{1}{\hbar \omega-\left(H-E_{0}^{N}\right)+i \eta} c_{\beta}^{\dagger}\right| \Psi_{0}^{N}\right\rangle \\
& \mp\left\langle\Psi_{0}^{N}\left|c_{\beta}^{\dagger} \frac{1}{\hbar \omega+\left(H-E_{0}^{N}\right)-i \eta} c_{\alpha}\right| \Psi_{0}^{N}\right\rangle .
\end{aligned}
$$

}
\end{small}
}


\frame
{
\frametitle{Single-particle Green's functions}
\begin{small}
{\scriptsize
Equations (2.17) and (2.19) are completely equivalent to the previous
ones. These may look a bit more abstract than the corresponding
Eqs. (2.11) and (2.16) but are more general since they show that the
formalism can be developed and applied in any orthonormal basis,
without restricting oneself to coordindate space.

}
\end{small}
}


\frame
{
\frametitle{Single-particle Green's functions}
\begin{small}
{\scriptsize
As discussed in Sec. 2.1 for the one particle case, the information
contained in the propagators becomes more clear if one Fourier
transforms the time variable and inserts a completness for the
intermediate states. This is so because it makes the spectrum and the
transition amplitudes to apper explicitely. Using the completeness
relations for the $(N \pm 1)$-body systems in Eq. (2.19), one has

$$
\begin{aligned}
g_{\alpha \beta}(\omega)= & \sum_{n} \frac{\left\langle\Psi_{0}^{N}\left|c_{\alpha}\right| \Psi_{n}^{N+1}\right\rangle\left\langle\Psi_{n}^{N+1}\left|c_{\beta}^{\dagger}\right| \Psi_{0}^{N}\right\rangle}{\hbar \omega-\left(E_{n}^{N+1}-E_{0}^{N}\right)+i \eta} \\
& \mp \sum_{k} \frac{\left\langle\Psi_{0}^{N}\left|c_{\beta}^{\dagger}\right| \Psi_{k}^{N-1}\right\rangle\left\langle\Psi_{k}^{N-1}\left|c_{\alpha}\right| \Psi_{0}^{N}\right\rangle}{\hbar \omega-\left(E_{0}^{N}-E_{k}^{N-1}\right)-i \eta} .
\end{aligned}
$$

which is known as the Lehmann repressentation of a many-body Green's
function ${ }^{1}$. Here, the first and second terms on the left hand
side describe the propagation of a (quasi)particle and a (quasi)hole
excitations.

}
\end{small}
}
\frame
{
\frametitle{Single-particle Green's functions}
\begin{small}
{\scriptsize
The poles in Eq. (2.20) are the energies relatives to the
$\left|\Psi_{0}^{N}\right\rangle$ ground state. Hence they give the
energies actually relased in a capture reaction experiment to a bound
state of $\left|\Psi_{n}^{N+1}\right\rangle$. The residues are
transition amplitudes for the addition of a particle and take the name
of spectroscopic amplitudes. They play the same role of the
$\left\langle\phi_{n} \mid \mathbf{r}\right\rangle$ wave function in
Eq. (2.3). In fact these energies and amplitudes are solutions of a
Schrödinger-like equation: the Dyson equation. The hole part of the
propagator gives instead information on the process of particle
emission, the poles being the exact energy absorbed in the
process. For example, in the single particle Green's function of a
molecule, the quasiparticle and quasihole poles are respectively the
electron affinities an ionization energies.

We will look at the physical significance of spectroscopic amplitudes
below and derive the Dyson equation (which is the
fundamental equation in many-body Green's function theory) only later
on.

}
\end{small}
}


\frame
{
\frametitle{Single-particle Green's functions}
\begin{small}
{\scriptsize
As a last definition, we rewrite the contents of Eq. (2.20) in a form
that can compared more easily to experiments. By using the relation

$$
\frac{1}{x \pm i \eta}=\mathcal{P} \frac{1}{x} \mp i \pi \delta(x)
$$
\footnotetext{${ }^{1}$ H. Lehmann, Nuovo Cimento 11, 324 (1954).
}
it is immediate to extract the one-body spectral function

$$
S_{\alpha \beta}(\omega)=S_{\alpha \beta}^{p}(\omega)+S_{\alpha \beta}^{h}(\omega),
$$

where the partcle and hole components are

$$
\begin{aligned}
S_{\alpha \beta}^{p}(\omega) & =-\frac{1}{\pi} \operatorname{Im} g_{\alpha \beta}^{p}(\omega) \\
& =\sum_{n}\left\langle\Psi_{0}^{N}\left|c_{\alpha}\right| \Psi_{n}^{N+1}\right\rangle\left\langle\Psi_{n}^{N+1}\left|c_{\beta}^{\dagger}\right| \Psi_{0}^{N}\right\rangle \delta\left(\hbar \omega-\left(E_{n}^{N+1}-E_{0}^{N}\right)\right) \\
S_{\alpha \beta}^{h}(\omega) & =\frac{1}{\pi} \operatorname{Im} g_{\alpha \beta}^{h}(\omega) \\
& =\mp \sum_{k}\left\langle\Psi_{0}^{N}\left|c_{\beta}^{\dagger}\right| \Psi_{k}^{N-1}\right\rangle\left\langle\Psi_{k}^{N-1}\left|c_{\alpha}\right| \Psi_{0}^{N}\right\rangle \delta\left(\hbar \omega-\left(E_{0}^{N}-E_{k}^{N-1}\right)\right) .
\end{aligned}
$$

}
\end{small}
}


\frame
{
\frametitle{Single-particle Green's functions}
\begin{small}
{\scriptsize
The diagonal part of the spectral function is interpreted as the
probability of adding $\left[S_{\alpha \alpha}^{p}(\omega)\right]$ or
removing $\left[S_{\alpha \alpha}^{h}(\omega)\right]$ one particle in
the state $\alpha$ leaving the residual system in a state of energy
$\omega$.

By comparing Eqs. (2.23) and (2.24) to the Lehmann representation
(2.20), it is seen that the propagator is completely constrained by
its imaginary part. Indeed,

$$
g_{\alpha \beta}(\omega)=\int d \omega^{\prime} \frac{S_{\alpha \beta}^{p}\left(\omega^{\prime}\right)}{\omega-\omega^{\prime}+i \eta}+\int d \omega^{\prime} \frac{S_{\alpha \beta}^{h}\left(\omega^{\prime}\right)}{\omega-\omega^{\prime}-i \eta} .
$$

In general the single particle propagator of a finite system has isolated poles in correspondence to the bound eigenstates of the $(N+1)$-body system. For larger enegies, where $\left|\Psi_{n}^{N+1}\right\rangle$ are states in the continuum, it develops a branch cut. The particle propagator $g^{p}(\omega)$ is analytic in the upper half of the complex plane, and so is the full propagator (2.16) for $\omega \geq E_{0}^{N+1}-E_{0}^{N}$. Analogously, the hole propagator has poles for $\omega \leq E_{0}^{N}-E_{0}^{N-1}$ and is analytic in the lower complex plane. Note that high excitation energies in the (N-1)body system correspond to negative values of the poles $E_{0}^{N}-E_{k}^{N-1}$, so $g^{h}(\omega)$ develops a branch cut for large negative energies.

}
\end{small}
}
\frame
{
\frametitle{Single-particle Green's functions}
\begin{small}
{\scriptsize
The one-body density matrix (1.40) can be obtained from the one-body
propagator. One simply chooses the appropriate time ordering in
Eq. (2.17)

$$
\rho_{\alpha \beta}=\left\langle\Psi_{0}^{N}\left|c_{\beta}^{\dagger} c_{\alpha}\right| \Psi_{0}^{N}\right\rangle= \pm i \hbar \lim _{t^{\prime} \rightarrow t^{+}} g_{\alpha \beta}\left(t, t^{\prime}\right)
$$

(where the upper sign is for bosons and the lower one is for fermions). Alternatively, the hole spectral function can be used

$$
\rho_{\alpha \beta}=\mp \int d \omega S_{\alpha \beta}^{h}(\omega) \text {. }
$$

}
\end{small}
}


\frame
{
\frametitle{Single-particle Green's functions}
\begin{small}
{\scriptsize
Thus, the expectation value of a one-body operator, Eq. (1.41), on the
ground states $\left|\Psi_{0}^{N}\right\rangle$ is usually written in
one the following ways

$$
\begin{aligned}
\left\langle\Psi_{0}^{N}|O| \Psi_{0}^{N}\right\rangle & =\mp \sum_{\alpha \beta} \int d \omega o_{\alpha \beta} S_{\beta \alpha}^{h}(\omega) \\
& = \pm i \hbar \lim _{t^{\prime} \rightarrow t^{+}} \sum_{\alpha \beta} o_{\alpha \beta} g_{\beta \alpha}\left(t, t^{\prime}\right)
\end{aligned}
$$

which are equivalent.

}
\end{small}
}


\frame
{
\frametitle{Single-particle Green's functions}
\begin{small}
{\scriptsize
From the particle spectral function, one can extract the quantity

$$
d_{\alpha \beta}=\left\langle\Psi_{0}^{N}\left|c_{\alpha} c_{\beta}^{\dagger}\right| \Psi_{0}^{N}\right\rangle=\int d \omega S_{\beta \alpha}^{p}(\omega)
$$

which leads to the following sum rule

$$
\int d \omega S_{\alpha \beta}(\omega)=d_{\alpha \beta} \mp \rho_{\alpha \beta}=\left\langle\Psi_{0}^{N}\left|\left[c_{\alpha}, c_{\beta}^{\dagger}\right]_{\mp}\right| \Psi_{0}^{N}\right\rangle=\delta_{\alpha \beta} .
$$

}
\end{small}
}
\frame
{
\frametitle{Single-particle Green's functions}
\begin{small}
{\scriptsize
For the case of an Hamiltonian containing only two-body interactions,

$$
\begin{aligned}
H & =U+V \\
& =\sum_{\alpha \beta} t_{\alpha \beta} c_{\alpha}^{\dagger} c_{\beta}+\frac{1}{4} \sum_{\alpha \beta \gamma \delta} v_{\alpha \beta, \gamma \delta} c_{\alpha}^{\dagger} c_{\beta}^{\dagger} c_{\delta} c_{\gamma},
\end{aligned}
$$

there exist an important sum rule that relates the total energy of the state $\left|\Psi_{0}^{N}\right\rangle$ to its one-body Green's function. To derive this, one makes use of the equation of motion for Heisenberg operators (1.51), which gives

$$
i \hbar \frac{d}{d t} c_{\alpha}(t)=e^{i H t / \hbar}\left[c_{\alpha}, H\right] e^{-i H t / \hbar},
$$

with $^{2}$

$$
\left[c_{\alpha}, H\right]=\sum_{\beta} t_{\alpha \beta} c_{\beta}+\frac{1}{2} \sum_{\beta \gamma \delta} v_{\alpha \beta \gamma \delta} c_{\beta}^{\dagger} c_{\delta} c_{\gamma}
$$

which is valid for both fermions and bosons.

}
\end{small}
}


\frame
{
\frametitle{Single-particle Green's functions}
\begin{small}
{\scriptsize
If one uses Eq. (2.33) and derives the propagator (2.17) with respect to time,

$$
\begin{aligned}
i \hbar \frac{\partial}{\partial t} g_{\alpha \beta}\left(t-t^{\prime}\right)= & \delta\left(t-t^{\prime}\right) \delta_{\alpha \beta}+\sum_{\gamma} t_{\alpha \gamma} g_{\gamma \beta}\left(t-t^{\prime}\right) \\
& -\frac{i}{\hbar} \sum_{\eta \gamma \zeta} \frac{1}{2} v_{\alpha \eta, \gamma \zeta}\left\langle\Psi_{0}^{N}\left|T\left[c_{\eta}^{\dagger}(t) c_{\zeta}(t) c_{\gamma}(t) c_{\beta}^{\dagger}\left(t^{\prime}\right)\right]\right| \Psi_{0}^{N}\right\rangle
\end{aligned}
$$

}
\end{small}
}


\frame
{
\frametitle{Single-particle Green's functions}
\begin{small}
{\scriptsize
The braket in the last line contains the four points Green's function [see Eq. (2.38)], which can describe the simultaneous propagation of two particles. Thus, one sees that applying the equation of motion to a propagator leads to relations which contain Green's functions of higher order. This result is particularly important because it shows there exist a hierarchy between propagators, so that the exact equations that determine the one-body function will depend on the two-body one, the two-body function will contain contributions from three-body propagators, and so on.

For the moment we just want to select a particular order of the operators in Eq. (2.34) in order to extract the one- and two-body density matrices. To do this, we chose $t^{\prime}$ to be a later time than $t$ and take its limit to the latter from above. This yields

$$
\pm i \hbar \lim _{t^{\prime} \rightarrow t^{+}} \sum_{\alpha} \frac{\partial}{\partial t} g_{\alpha \alpha}\left(t-t^{\prime}\right)=\langle T\rangle+2\langle V\rangle
$$

(note that for $t \neq t^{\prime}$, the term $\delta\left(t-t^{\prime}\right)=0$ and it does not contribute to the limit). This result can also be expressed in energy representation by inverting the Fourier transformation (2.14), which gives

$$
\lim _{\tau \rightarrow 0^{-}} \frac{\partial}{\partial \tau} g_{\alpha \beta}(\tau)=-\int d \omega \omega S_{\alpha \beta}^{h}(\omega)
$$

}
\end{small}
}
\frame
{
\frametitle{Single-particle Green's functions}
\begin{small}
{\scriptsize
By combining (2.35) with Eq. (2.28) one finally obtains

$$
\begin{aligned}
\langle H\rangle=\langle U\rangle+\langle V\rangle & = \pm i \hbar \frac{1}{2} \lim _{t^{\prime} \rightarrow t^{+}} \sum_{\alpha \beta}\left\{\delta_{\alpha \beta} \frac{\partial}{\partial t}+t_{\alpha \beta}\right\} g_{\beta \alpha}\left(t-t^{\prime}\right) \\
& =\mp \frac{1}{2} \sum_{\alpha \beta} \int d \omega\left\{\delta_{\alpha \beta} \omega+t_{\alpha \beta}\right\} S_{\beta \alpha}^{h}(\omega)
\end{aligned}
$$
We use the relation $[A, B C]_{-}=[A, B] C-B[C, A]=\{A, B\} C-B\{C, A\}$ which is valid for both commutators and anticommutators


Surprisingly, for an Hamiltonian containing only two-body forces it is possible to extract the ground state energy by knowing only the one-body propagator. This result was derived independently by Galitski and Migdal ${ }^{3}$ and by Kolutn ${ }^{4}$. When interactions among three or more particles are present, this relation has to be augmented to include additional terms. In these cases higher order Green's functions will appear explicitly.

}
\end{small}
}
\frame
{
\frametitle{Single-particle Green's functions}
\begin{small}
{\scriptsize
The definition (2.17) can be extended to Green's functions for the
propagation of more than one particle. In general, for each additional
particle it will be necessary to introduce one additional creation and
one annihilation operator. Thus a $2 n$-points Green's function will
propagate a maximum of $n$ quasiparticles. The explicit definition of
the 4-points propagator is

$$
g_{\alpha \beta, \gamma \delta}^{4-p t}\left(t_{1}, t_{2} ; t_{1}^{\prime}, t_{2}^{\prime}\right)=-\frac{i}{\hbar}\left\langle\Psi_{0}^{N}\left|T\left[c_{\beta}\left(t_{2}\right) c_{\alpha}\left(t_{1}\right) c_{\gamma}^{\dagger}\left(t_{1}^{\prime}\right) c_{\delta}^{\dagger}\left(t_{2}^{\prime}\right)\right]\right| \Psi_{0}^{N}\right\rangle
$$

while the 6 -point case is

$$
\begin{aligned}
& g_{\alpha \beta \gamma, \mu \nu \lambda}^{6-p t}\left(t_{1}, t_{2}, t_{3} ; t_{1}^{\prime}, t_{2}^{\prime}, t_{3}^{\prime}\right)= \\
& \quad-\frac{i}{\hbar}\left\langle\Psi_{0}^{N}\left|T\left[c_{\gamma}\left(t_{3}\right) c_{\beta}\left(t_{2}\right) c_{\alpha}\left(t_{1}\right) c_{\mu}^{\dagger}\left(t_{1}^{\prime}\right) c_{\nu}^{\dagger}\left(t_{2}^{\prime}\right) c_{\lambda}^{\dagger}\left(t_{3}^{\prime}\right)\right]\right| \Psi_{0}^{N}\right\rangle,
\end{aligned}
$$

}
\end{small}
}
\frame
{
\frametitle{Single-particle Green's functions}
\begin{small}
{\scriptsize
It should be noted that the actual number of particles that are
propagated by these objects depends on the ordering of the time
variables. Therefore the information on transitions between
eigenstates of the systems with $N, N \pm 1$ and $N \pm 2$ bodies are
all encoded in Eq. (2.38), while additional states of $N \pm 3$-body
states are included in Eq. (2.38). Obviously, the presence of so many
time variables makes the use of these functions extremely difficult
(and even impossible, in many cases). However, it is still useful to
consider only certain time orderings which allow to extract the
information not included in the 2-point propagator.

}
\end{small}
}
\frame
{
\frametitle{Single-particle Green's functions}
\begin{small}
{\scriptsize
The two-particle-two-hole propagator is a two-times Green's function
defined as

$$
g_{\alpha \beta, \gamma \delta}^{I I}\left(t, t^{\prime}\right)=-\frac{i}{\hbar}\left\langle\Psi_{0}^{N}\left|T\left[c_{\beta}(t) c_{\alpha}(t) c_{\gamma}^{\dagger}\left(t^{\prime}\right) c_{\delta}^{\dagger}\left(t^{\prime}\right)\right]\right| \Psi_{0}^{N}\right\rangle
$$

which corresponds to the limit $t_{1}^{\prime}=t_{2}^{\prime+}$ and $t_{2}=t_{1}^{+}$of $g^{4-p t}$.

}
\end{small}
}
\frame
{
\frametitle{Single-particle Green's functions}
\begin{small}
{\scriptsize
As for the case of $g_{\alpha \beta}\left(t, t^{\prime}\right)$, if
the Hamiltonian is time-independent, Eq. (2.40) is a function of the
time difference only. Therefore it has a Lehmann representation
containing the exact spectrum of the $(N \pm 2)$-body systems

$$
\begin{aligned}
g_{\alpha \beta, \gamma \delta}^{I I}(\omega) & =\sum_{n} \frac{\left\langle\Psi_{0}^{N}\left|c_{\beta} c_{\alpha}\right| \Psi_{n}^{N+2}\right\rangle\left\langle\Psi^{N+2}\left|c_{\gamma}^{\dagger} c_{\delta}^{\dagger}\right| \Psi_{0}^{N}\right\rangle}{\omega-\left(E_{n}^{N+2}-E_{0}^{N}\right)+i \eta} \\
& -\sum_{k} \frac{\left\langle\Psi_{0}^{N}\left|c_{\gamma}^{\dagger} c_{\delta}^{\dagger}\right| \Psi_{k}^{N-2}\right\rangle\left\langle\Psi_{k}^{N-2}\left|c_{\beta} c_{\alpha}\right| \Psi_{0}^{N}\right\rangle}{\omega-\left(E_{0}^{N}-E_{k}^{N-2}\right)-i \eta}
\end{aligned}
$$

}
\end{small}
}
\frame
{
\frametitle{Single-particle Green's functions}
\begin{small}
{\scriptsize
Similarly one defines the two-particle and two-hole spectral functions

$$
S_{\alpha \beta}^{I I}, \gamma \delta(\omega)=S_{\alpha \beta, \gamma \delta}^{p p}(\omega)+S_{\alpha \beta, \gamma \delta}^{h h}(\omega)
$$

and

$$
\begin{aligned}
S_{\alpha \beta, \gamma \delta}^{p p}(\omega) & =-\frac{1}{\pi} \operatorname{Im} g_{\alpha \beta, \gamma \delta}^{p p}(\omega) \\
& =\sum_{n}\left\langle\Psi_{0}^{N}\left|c_{\beta} c_{\alpha}\right| \Psi_{n}^{N+2}\right\rangle\left\langle\Psi_{n}^{N+2}\left|c_{\gamma}^{\dagger} c_{\delta}^{\dagger}\right| \Psi_{0}^{N}\right\rangle \delta\left(\hbar \omega-\left(E_{n}^{N+2}-E_{0}^{N}\right)\right), \\
S_{\alpha \beta, \gamma \delta}^{h h}(\omega) & =\frac{1}{\pi} \operatorname{Im} g_{\alpha \beta, \gamma \delta}^{h h}(\omega) \\
& =-\sum_{k}\left\langle\Psi_{0}^{N}\left|c_{\gamma}^{\dagger} c_{\delta}^{\dagger}\right| \Psi_{k}^{N-2}\right\rangle\left\langle\Psi_{k}^{N-2}\left|c_{\beta} c_{\alpha}\right| \Psi_{0}^{N}\right\rangle \delta\left(\hbar \omega-\left(E_{0}^{N}-E_{k}^{N-2}\right)\right) .
\end{aligned}
$$

}
\end{small}
}
\frame
{
\frametitle{Single-particle Green's functions}
\begin{small}
{\scriptsize
Following the demonstration of Sec. 2.3.1, it is immediate to obtain relations for the two-body density matrix (1.44)

$$
\Gamma_{\alpha \beta, \gamma \delta}=\left\langle\Psi^{N}\left|c_{\gamma}^{\dagger} c_{\delta}^{\dagger} c_{\beta} c_{\alpha}\right| \Psi^{N}\right\rangle=-\int d \omega S_{\alpha \beta, \gamma \delta}^{h h}(\omega)
$$

and, hence, for the expectation value of any two-body operator

$$
\begin{aligned}
\left\langle\Psi_{0}^{N}|V| \Psi_{0}^{N}\right\rangle & =-\sum_{\alpha \beta \gamma \delta} \int d \omega v_{\alpha \beta, \gamma \delta} S_{\gamma \delta, \alpha \beta}^{h}(\omega) \\
& =+i \hbar \lim _{t^{\prime} \rightarrow t^{+}} \frac{1}{4} \sum_{\alpha \beta \gamma \delta} v_{\alpha \beta, \gamma \delta} g_{\gamma \delta, \alpha \beta}^{I I}\left(t, t^{\prime}\right) .
\end{aligned}
$$

}
\end{small}
}
\frame
{
\frametitle{Single-particle Green's functions}
\begin{small}
{\scriptsize
The polarization propagator $\Pi_{\alpha \beta, \gamma \delta}$
corresponds to the time ordering of $g^{4-p t}$ in which a
particle-hole excitation is created at one single time. Therefore, no
process involving particle transfer in included. However it describes
transition to the excitations of the system, as long as they can be
reached with a one-body operator. For example, this includes
collective modes of a nucleus. This is defined as

$$
\begin{aligned}
\Pi_{\alpha \beta, \gamma \delta}\left(t, t^{\prime}\right)=- & \frac{i}{\hbar}\left\langle\Psi_{0}^{N}\left|T\left[c_{\beta}^{\dagger}(t) c_{\alpha}(t) c_{\gamma}^{\dagger}\left(t^{\prime}\right) c_{\delta}\left(t^{\prime}\right)\right]\right| \Psi_{0}^{N}\right\rangle \\
& +\frac{i}{\hbar}\left\langle\Psi_{0}^{N}\left|c_{\beta}^{\dagger} c_{\alpha}\right| \Psi_{0}^{N}\right\rangle\left\langle\Psi_{0}^{N}\left|c_{\gamma}^{\dagger} c_{\delta}\right| \Psi_{0}^{N}\right\rangle
\end{aligned}
$$

}
\end{small}
}
\frame
{
\frametitle{Single-particle Green's functions}
\begin{small}
{\scriptsize

  After including a completeness of $\left|\Psi_{n}^{N}\right\rangle$
states in (2.47), the contribution of to the ground states (at zero
energy) is cancelled by the last term in the equation. Thus one can
Fourier transform to the Lehmann representation

$$
\begin{aligned}
\Pi_{\alpha \beta, \gamma \delta}(\omega) & =\sum_{n \neq 0} \frac{\left\langle\Psi_{0}^{N}\left|c_{\beta}^{\dagger} c_{\alpha}\right| \Psi_{n}^{N}\right\rangle\left\langle\Psi_{n}^{N}\left|c_{\gamma}^{\dagger} c_{\delta}\right| \Psi_{0}^{N}\right\rangle}{\omega-\left(E_{n}^{N}-E_{0}^{N}\right)+i \eta} \\
& -\sum_{n \neq 0} \frac{\left\langle\Psi_{0}^{N}\left|c_{\gamma}^{\dagger} c_{\delta}\right| \Psi_{n}^{N}\right\rangle\left\langle\Psi_{n}^{N}\left|c_{\beta}^{\dagger} c_{\alpha}\right| \Psi_{0}^{N}\right\rangle}{\omega+\left(E_{n}^{N}-E_{0}^{N}\right)-i \eta}
\end{aligned}
$$

}
\end{small}
}
\frame
{
\frametitle{Single-particle Green's functions}
\begin{small}
{\scriptsize
Note that $\Pi_{\alpha \beta, \gamma \delta}(\omega)=\Pi_{\delta
  \gamma, \beta \alpha}(-\omega)$ due to time reversal symmetry. Also
the forward and backward parts carry the same information.

Once again, the residues of the propagator (2.48) can be used to
calculate expectation values. In this case, given a one-body operator
(1.30) on obtains the transition matrix elements to any excited state

$$
\left\langle\Psi_{n}^{N}|O| \Psi_{0}^{N}\right\rangle=\sum_{\alpha \beta} o_{\beta \alpha}\left\langle\Psi_{n}^{N}\left|c_{\beta}^{\dagger} c_{\alpha}\right| \Psi_{0}^{N}\right\rangle
$$

}
\end{small}
}
\frame
{
\frametitle{Single-particle Green's functions}
\begin{small}
{\scriptsize

  Here we explore the connection between the information contained in various propagators and experimental data. The focus is on the experimental properties that are probed by the removal of particles. Also, from now on, we will only consider fermionic systems.

An important case is when the spectrum for the $N \pm 1$-particle system near the Fermi energy involves discrete bound states. This happens in finite system like nuclei or molecules. In these cases the main quantity of interest is the overlap wave function, which appears in the residues of Eq. (2.20) and in Eq. (2.24). This is

$$
\begin{aligned}
\psi_{k}^{\text {overlap }}(\mathbf{r})= & \left\langle\Psi_{k}^{N-1}\left|\psi_{s}(\mathbf{r})\right| \Psi_{0}^{N}\right\rangle \\
= & \sqrt{N} \int d \mathbf{r}_{2} \int d \mathbf{r}_{3} \cdots \int d \mathbf{r}_{N} \\
& \quad \times\left[\Psi_{k}^{N-1}\left(\mathbf{r}_{2}, \mathbf{r}_{3}, \ldots \mathbf{r}_{N}\right)\right]^{*} \Psi_{0}^{N}\left(\mathbf{r}, \mathbf{r}_{2}, \mathbf{r}_{3}, \ldots \mathbf{r}_{N}\right) .
\end{aligned}
$$

}
\end{small}
}
\frame
{
\frametitle{Single-particle Green's functions}
\begin{small}
{\scriptsize
The second line in Eq. (3.1) can be proved by using relations (1.19)
and (1.20). This integral comes out in the description of most
particle knock out processes because it represents the matrix element
between the initial and final states, in the case when the emitted
particle is ejected with energy large enough the it interacts only
weakly with the residual system. The quantity of interest here is the
so called spectroscopic factor to the final state $k$,

$$
S_{k}=\int d \mathbf{r}\left|\psi_{k}^{\text {overlap }}(\mathbf{r})\right|^{2}
$$

When the system is made of completely non interacting particles,
$S_{k}$ is unity. In real cases however, correlations among the
constituents reduce this value. The possibility of extracting this
quantity from experimental data gives us information on the spectral
function and therefore on the structure of the correlated system.

}
\end{small}
}
\frame
{
\frametitle{Single-particle Green's functions}
\begin{small}
{\scriptsize
In order to make the connection with experimental data obtained from
knockout reactions, it is useful to consider the response of a system
to a weak probe. The hole spectral function introduced in Eq. (2.24)
can be substantially "observed" these reactions. The general idea is
to transfer a large amount of momentum and energy to a particle of a
bound system in the ground state. This is then ejected from the
system, and one ends up with a fast-moving particle and a bound
$(N-1)$-particle system. By observing the momentum of the ejected
particle it is then possible the reconstruct the spectral function of
the system, provided that the interaction between the ejected particle
and the remainder is sufficiently weak or treated in a controlled
fashion, $e . g$. by constraining this treatment with information from
other experimental data.

We assume that the $N$-particle system is initially in its ground state,

$$
\left|\Psi_{i}\right\rangle=\left|\Psi_{0}^{N}\right\rangle
$$

and makes a transition to a final $N$-particle eigenstate

$$
\left|\Psi_{f}\right\rangle=a_{p}^{\dagger}\left|\Psi_{n}^{N-1}\right\rangle
$$

composed of a bound $(N-1)$-particle eigenstate, $\left|\Psi_{n}^{N-1}\right\rangle$, and a particle with momentum $\boldsymbol{p}$.

}
\end{small}
}
\frame
{
\frametitle{Single-particle Green's functions}
\begin{small}
{\scriptsize
For simplicity we consider the transition matrix elements for a scalar external probe

$$
\rho(\boldsymbol{q})=\sum_{j=1}^{N} \exp \left(i \boldsymbol{q} \cdot \boldsymbol{r}_{j}\right)
$$

which transfers momentum $\boldsymbol{q}$ to a particle. Suppressing other possible sp quantum numbers, like e.g. spin, the second-quantized form of this operator is given by

$$
\hat{\rho}(\boldsymbol{q})=\sum_{\boldsymbol{p}, \boldsymbol{p}^{\prime}}\left\langle\boldsymbol{p}|\exp (i \boldsymbol{q} \cdot \boldsymbol{r})| \boldsymbol{p}^{\prime}\right\rangle a_{\boldsymbol{p}}^{\dagger} a_{\boldsymbol{p}^{\prime}}=\sum_{\boldsymbol{p}} a_{\boldsymbol{p}}^{\dagger} a_{\boldsymbol{p}-\boldsymbol{q}}
$$

}
\end{small}
}
\frame
{
\frametitle{Single-particle Green's functions}
\begin{small}
{\scriptsize
The transition matrix element now becomes

$$
\begin{aligned}
\left\langle\Psi_{f}|\hat{\rho}(\boldsymbol{q})| \Psi_{i}\right\rangle & =\sum_{\boldsymbol{p}^{\prime}}\left\langle\Psi_{n}^{N-1}\left|a_{\boldsymbol{p}} a_{\boldsymbol{p}^{\prime}}^{\dagger} a_{\boldsymbol{p}^{\prime}-\boldsymbol{q}}\right| \Psi_{0}^{N}\right\rangle \\
& =\sum_{\boldsymbol{p}^{\prime}}\left\langle\Psi_{n}^{N-1}\left|\delta_{\boldsymbol{p}^{\prime}, \boldsymbol{p}} a_{\boldsymbol{p}^{\prime}-\boldsymbol{q}}+a_{\boldsymbol{p}^{\prime}}^{\dagger} a_{\boldsymbol{p}^{\prime}-\boldsymbol{q}} a_{\boldsymbol{p}}\right| \Psi_{0}^{N}\right\rangle \\
& \approx\left\langle\Psi_{n}^{N-1}\left|a_{\boldsymbol{p}-\boldsymbol{q}}\right| \Psi_{0}^{N}\right\rangle .
\end{aligned}
$$

}
\end{small}
}
\frame
{
\frametitle{Single-particle Green's functions}
\begin{small}
{\scriptsize
The last line is obtained in the so-called Impulse Approximation (or Sudden Approximation), where it is assumed that the ejected particle is the one that
has absorbed the momentum from the external field. This is a very good approximation whenever the momentum $\boldsymbol{p}$ of the ejectile is much larger than typical momenta for the particles in the bound states; the neglected term in Eq. (3.7) is then very small, as it involves the removal of a particle with momentum $\boldsymbol{p}$ from $\left|\Psi_{0}^{N}\right\rangle$.

There is one other assumption in the derivation: the fact that the final eigenstate of the $N$-particle system was written in the form of Eq. (3.4), i.e. a plane-wave state for the ejectile on top of an $(N-1)$-particle eigenstate. This is again a good approximation if the ejectile momentum is large enough, as can be understood by rewriting the Hamiltonian in the $N$-particle system as

$$
H_{N}=\sum_{i=1}^{N} \frac{\boldsymbol{p}_{i}^{2}}{2 m}+\sum_{i<j=1}^{N} V(i, j)=H_{N-1}+\frac{\boldsymbol{p}_{N}^{2}}{2 m}+\sum_{i=1}^{N-1} V(i, N)
$$


}
\end{small}
}
\frame
{
\frametitle{Single-particle Green's functions}
\begin{small}
{\scriptsize
The last term in Eq. (3.8) represents the Final State Interaction, or
the interaction between the ejected particle $N$ and the other
particles $1 \ldots N-1$. If the relative momentum between particle
$N$ and the others is large enough their mutual interaction can be
neglected, and $H_{N} \approx H_{N-1}+\boldsymbol{p}_{N}^{2} / 2
m$. The result given by Eq. (3.7) is called the Plane Wave Impulse
Approximation or PWIA knock-out amplitude, for obvious reasons, and is
precisely a removal amplitude (in the momentum representation)
appearing in the Lehmann representation of the sp propagator
[Eq. (3.1) and (2.24)].

}
\end{small}
}
\frame
{
\frametitle{Single-particle Green's functions}
\begin{small}
{\scriptsize
The cross section of the knock-out reaction, where the momentum and
energy of the ejected particle and the probe are either measured or
known, is according to Fermi's golden rule proportional to

$$
d \sigma \sim \sum_{n} \delta\left(\omega+E_{i}-E_{f}\right)\left|\left\langle\Psi_{f}|\hat{\rho}(\boldsymbol{q})| \Psi_{i}\right\rangle\right|^{2}
$$

where the energy-conserving $\delta$-function contains the energy transfer $\omega$ of the probe, and the initial and final energies of the system are $E_{i}=E_{0}^{N}$ and $E_{f}=E_{n}^{N-1}+\boldsymbol{p}^{2} / 2 m$, respectively. Note that the internal state of the residual $N-1$ system is not measured, hence the summation over $n$ in Eq. (3.9). Defining the missing momentum $\boldsymbol{p}_{\text {miss }}$ and missing energy $E_{\text {miss }}$ of the knock-out reaction as ${ }^{1}$

$$
\boldsymbol{p}_{m i s s}=\boldsymbol{p}-\boldsymbol{q}
$$

and

}
\end{small}
}
\frame
{
\frametitle{Single-particle Green's functions}
\begin{small}
{\scriptsize
$$
E_{\text {miss }}=\boldsymbol{p}^{2} / 2 m-\omega=E_{0}^{N}-E_{n}^{N-1},
$$
\footnotetext{${ }^{1}$ We will neglect here the recoil of the residual $N-1$ system, i.e. we assume the mass of the $N$ and $N-1$ system to be much heavier than the mass $m$ of the ejected particle.
}
respectively, the PWIA knock-out cross section can be rewritten as

$$
\begin{aligned}
d \sigma & \sim \sum_{n} \delta\left(E_{\text {miss }}-E_{0}^{N}+E_{n}^{N-1}\right)\left|\left\langle\Psi_{n}^{N-1}\left|a_{\boldsymbol{p}_{\text {miss }}}\right| \Psi_{0}^{N}\right\rangle\right|^{2} \\
& =S^{h}\left(\boldsymbol{p}_{\text {miss }}, E_{\text {miss }}\right) .
\end{aligned}
$$

The PWIA cross section is therefore exactly proportional to the diagonal part of the hole spectral function defined in Eq. (2.24). This is of course only true in the PWIA, but when the deviations of the impulse approximation and the effects of the final state interaction are under control, it is possible to obtain precise experimental information on the hole spectral function of the system under study.
}
\end{small}
}

\end{document}


\frame
{
\frametitle{Single-particle Green's functions}
\begin{small}
{\scriptsize

}
\end{small}
}
\frame
{
\frametitle{Single-particle Green's functions}
\begin{small}
{\scriptsize

}
\end{small}
}
\frame
{
\frametitle{Single-particle Green's functions}
\begin{small}
{\scriptsize

}
\end{small}
}



















