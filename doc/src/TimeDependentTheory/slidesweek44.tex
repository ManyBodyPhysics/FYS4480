\frame{\titlepage}
\frame
{
\frametitle{Schr\"odinger picture}
\begin{small}
{\scriptsize
The time-dependent Schr\"odinger equation (or equation of motion) reads
\[
\imath \hbar\frac{\partial }{\partial t}|\Psi_S(t)\rangle = \hat{H}\Psi_S(t)\rangle,
\]
where the subscript $S$ stands for Schr\"odinger here.
A formal solution is given by 
\[
|\Psi_S(t)\rangle = \exp{(-\imath\hat{H}(t-t_0)/\hbar)}|\Psi_S(t_0)\rangle.
\]
The Hamiltonian $\hat{H}$ is hermitian and the exponent represents a unitary 
operator with an operation carried ut on the wave function at a time $t_0$.
}
\end{small}
}
\frame
{
\frametitle{Interaction picture}
\begin{small}
{\scriptsize
Our Hamiltonian is normally written out as the sum of an unperturbed part $\hat{H}_0$ and an interaction part $\hat{H}_I$, that is
\[
\hat{H}=\hat{H}_0+\hat{H}_I.
\]
In general we have $[\hat{H}_0,\hat{H}_I]\ne 0$ since $[\hat{T},\hat{V}]\ne 0$.
We wish now to define a unitary transformation in terms of $\hat{H}_0$ by defining
\[
|\Psi_I(t)\rangle = \exp{(\imath\hat{H}_0t/\hbar)}|\Psi_S(t)\rangle,
\]
which is again a unitary transformation carried out now at the time $t$ on the 
wave function in the Schr\"odinger picture. 
}
\end{small}
}
\frame
{
\frametitle{Interaction picture}
\begin{small}
{\scriptsize
We can easily find the equation of motion by taking the time derivative
\[
\imath \hbar\frac{\partial }{\partial t}|\Psi_I(t)\rangle = -\hat{H}_0\exp{(\imath\hat{H}_0t/\hbar)}\Psi_S(t)\rangle+\exp{(\imath\hat{H}_0t/\hbar)}
\imath \hbar\frac{\partial }{\partial t}\Psi_S(t)\rangle.
\]
}
\end{small}
}
\frame
{
\frametitle{Interaction picture}
\begin{small}
{\scriptsize
Using the definition of the Schr\"odinger equation, we can rewrite the last equation as 
\[
\imath \hbar\frac{\partial }{\partial t}|\Psi_I(t)\rangle = \exp{(\imath\hat{H}_0t/\hbar)}\left[-\hat{H}_0+\hat{H}_0+\hat{H}_I\right]\exp{(-\imath\hat{H}_0t/\hbar)}\Psi_I(t)\rangle,
\]
which gives us
\[
\imath \hbar\frac{\partial }{\partial t}|\Psi_I(t)\rangle = \hat{H}_I(t)\Psi_I(t)\rangle,
\]
 with 
\[
\hat{H}_I(t)=
\exp{(\imath\hat{H}_0t/\hbar)}\hat{H}_I\exp{(-\imath\hat{H}_0t/\hbar)}.
\]
}
\end{small}
}
\frame
{
\frametitle{Interaction picture}
\begin{small}
{\scriptsize
The order of the operators is important since $\hat{H}_0$ and $\hat{H}_I$ do generally not commute.
The expectation value of
an arbitrary operator in the interaction picture can now be written as
\[
\langle \Psi'_S(t)|\hat{O}_S|\Psi_S(t)\rangle = 
\langle \Psi'_I(t) |\exp{(\imath\hat{H}_0t/\hbar)}\hat{O}_I
\exp{(-\imath\hat{H}_0t/\hbar)}|\Psi_I(t)\rangle,
\]
and using the definition
\[
\hat{O}_I(t)=
\exp{(\imath\hat{H}_0t/\hbar)}\hat{O}_I\exp{(-\imath\hat{H}_0t/\hbar)},
\]
we obtain
\[
\langle \Psi'_S(t)|\hat{O}_S|\Psi_S(t)\rangle = 
\langle \Psi'_I(t) |\hat{O}_I(t)|\Psi_I(t)\rangle,
\]
stating that a unitary transformation does not change expectation values!
}
\end{small}
}
\frame
{
\frametitle{Interaction picture}
\begin{small}
{\scriptsize
If the take the time derivative of the operator in the interaction picture we arrive at the following equation of motion
\[
\imath \hbar\frac{\partial }{\partial t}\hat{O}_I(t) = \exp{(\imath\hat{H}_0t/\hbar)}\left[\hat{O}_S\hat{H}_0-\hat{H}_0\hat{O}_S\right]\exp{(-\imath\hat{H}_0t/\hbar)}=\left[\hat{O}_I(t),\hat{H}_0\right].
\]
Here we have used the time-independence of the Schr\"odinger equation
together with the observation that any function of an operator commutes with the operator itself. 
}
\end{small}
}
\frame
{
\frametitle{Interaction picture}
\begin{small}
{\scriptsize
In order to solve the equation of motion equation in the interaction picture, we define a unitary operator
time-development operator $\hat{U}(t,t')$. Later we will derive its
connection with the linked-diagram theorem, which yields a
linked expression for the actual operator. 
The action of the operator on the wave function is
\[
|\Psi_I(t) \rangle = \hat{U}(t,t_0)|\Psi_I(t_0)\rangle,
\]
with the obvious value $\hat{U}(t_0,t_0)=1$.
}
\end{small}
}
\frame
{
\frametitle{Interaction picture}
\begin{small}
{\scriptsize
The time-development operator $U$ has the
properties that
\[
     \hat{U}^{\dagger}(t,t')\hat{U}(t,t')=\hat{U}(t,t')\hat{U}^{\dagger}(t,t')=1,
\]
which implies that $U$ is unitary
\[
     \hat{U}^{\dagger}(t,t')=\hat{U}^{-1}(t,t').
\]
Further,
\[
    \hat{U}(t,t')\hat{U}(t't'')=\hat{U}(t,t'')
\]
and
\[
    \hat{U}(t,t')\hat{U}(t',t)=1,
\]
which leads to
\[
    \hat{U}(t,t')=\hat{U}^{\dagger}(t',t).
\]
}
\end{small}
}
\frame
{
\frametitle{Interaction picture}
\begin{small}
{\scriptsize
Using our definition of Schr\"odinger's equation in the interaction picture, we can then construct the operator $\hat{U}$. We have defined
\[
|\Psi_I(t)\rangle = \exp{(\imath\hat{H}_0t/\hbar)}|\Psi_S(t)\rangle,
\]
which can be rewritten as 
\[
|\Psi_I(t)\rangle = \exp{(\imath\hat{H}_0t/\hbar)}\exp{(-\imath\hat{H}(t-t_0)/\hbar)}|\Psi_S(t_0)\rangle,
\]
or
\[
|\Psi_I(t)\rangle = \exp{(\imath\hat{H}_0t/\hbar)}\exp{(-\imath\hat{H}(t-t_0)/\hbar)}\exp{(-\imath\hat{H}_0t_0/\hbar)}|\Psi_I(t_0)\rangle.
\]
}
\end{small}
}
\frame
{
\frametitle{Interaction picture}
\begin{small}
{\scriptsize
From the last expression we can define
\[
\hat{U}(t,t_0)=\exp{(\imath\hat{H}_0t/\hbar)}\exp{(-\imath\hat{H}(t-t_0)/\hbar)}\exp{(-\imath\hat{H}_0t_0/\hbar)}.
\]
It is then easy to convince oneself that the properties defined above are satisfied by the definition of $\hat{U}$. 
}
\end{small}
}
\frame
{
\frametitle{Interaction picture}
\begin{small}
{\scriptsize
We derive the equation of motion for $\hat{U}$ using the above definition.
This results in
\[
\imath \hbar\frac{\partial }{\partial t}\hat{U}(t,t_0) = \hat{H}_I(t)\hat{U}(t,t_0),
\]
which we integrate from $t_0$ to a time $t$ resulting in
\[
\hat{U}(t,t_0)-\hat{U}(t_0,t_0)=\hat{U}(t,t_0)-1=-\frac{\imath}{\hbar}\int_{t_0}^t dt' \hat{H}_I(t')\hat{U}(t',t_0),
\]
which can be rewritten as
\[
\hat{U}(t,t_0)=1-\frac{\imath}{\hbar}\int_{t_0}^t dt' \hat{H}_I(t')\hat{U}(t',t_0).
\]
}
\end{small}
}
\frame
{
\frametitle{Interaction picture}
\begin{small}
{\scriptsize
We can solve this equation iteratively keeping in mind the time-ordering of the of the operators
\[
\hat{U}(t,t_0)=1-\frac{\imath}{\hbar}\int_{t_0}^t dt' \hat{H}_I(t')+\left(\frac{-\imath}{\hbar}\right)^2\int_{t_0}^t dt'\int_{t_0}^{t'} dt'' \hat{H}_I(t')\hat{H}_I(t'')+\dots
\]
The third term can be written as 
\[
\int_{t_0}^t dt'\int_{t_0}^{t'} dt'' \hat{H}_I(t')\hat{H}_I(t'')=
\frac{1}{2}\int_{t_0}^t dt'\int_{t_0}^{t'} dt'' \hat{H}_I(t')\hat{H}_I(t'')
+\frac{1}{2}\int_{t_0}^t dt''\int_{t''}^{t} dt' \hat{H}_I(t')\hat{H}_I(t'').
\]
}
\end{small}
}
\frame
{
\frametitle{Interaction picture}
\begin{small}
{\scriptsize
We obtain this expression by changing the integration order in the second term
via a change of the integration variables $t'$ and $t''$  in 
\[
\frac{1}{2}\int_{t_0}^t dt'\int_{t_0}^{t'} dt'' \hat{H}_I(t')\hat{H}_I(t'').
\]
We can rewrite the terms which contain the double integral as
\[
\int_{t_0}^t dt'\int_{t_0}^{t'} dt'' \hat{H}_I(t')\hat{H}_I(t'')=\]
\[
\frac{1}{2}\int_{t_0}^t dt'\int_{t_0}^{t'} dt''\left[\hat{H}_I(t')\hat{H}_I(t'')\Theta(t'-t'')
+\hat{H}_I(t')\hat{H}_I(t'')\Theta(t''-t')\right],
\]
with $\Theta(t''-t')$ being the standard Heavyside or step function. The step function allows us to give a specific time-ordering to the above expression.
}
\end{small}
}
\frame
{
\frametitle{Interaction picture}
\begin{small}
{\scriptsize
With the $\Theta$-function we can rewrite the last expression as 
\[
\int_{t_0}^t dt'\int_{t_0}^{t'} dt'' \hat{H}_I(t')\hat{H}_I(t'')=
\frac{1}{2}\int_{t_0}^t dt'\int_{t_0}^{t'} dt''\hat{T}\left[\hat{H}_I(t')\hat{H}_I(t'')\right],
\]
where $\Hat{T}$ is the so-called time-ordering operator. 
}
\end{small}
}
\frame
{
\frametitle{Interaction picture}
\begin{small}
{\scriptsize
With this definition, we can rewrite the expression for $\hat{U}$ as 
\[
\hat{U}(t,t_0)=\sum_{n=0}^{\infty}\left(\frac{-\imath}{\hbar}\right)^n\frac{1}{n!}
\int_{t_0}^t dt_1\dots \int_{t_0}^t dt_N \hat{T}\left[\hat{H}_I(t_1)\dots\hat{H}_I(t_n)\right]=\hat{T}\exp{\left[\frac{-\imath}{\hbar}
\int_{t_0}^t dt' \hat{H}_I(t')\right]}.
\]
The above time-evolution operator in the interaction picture will be used
to derive various contributions to many-body perturbation theory. See also exercise 26 for a discussion of the various time orderings.
}
\end{small}
}
\frame
{
\frametitle{Heisenberg picture}
\begin{small}
{\scriptsize
We wish now to define a unitary transformation in terms of $\hat{H}$ by defining
\[
|\Psi_H(t)\rangle = \exp{(\imath\hat{H}t/\hbar)}|\Psi_S(t)\rangle,
\]
which is again a unitary transformation carried out now at the time $t$ on the 
wave function in the Schr\"odinger picture. If we combine this equation with 
Schr\"odinger's equation we obtain the following equation of motion
\[
\imath \hbar\frac{\partial }{\partial t}|\Psi_H(t)\rangle = 0,
\]
meaning that $|\Psi_H(t)\rangle$ is time independent. An operator in this picture is defined as
\[
\hat{O}_H(t)=
\exp{(\imath\hat{H}t/\hbar)}\hat{O}_S\exp{(-\imath\hat{H}t/\hbar)}.
\]
}
\end{small}
}
\frame
{
\frametitle{Heisenberg picture}
\begin{small}
{\scriptsize
The time dependence is then in the operator itself, and this yields in turn the
following equation of motion
\[
\imath \hbar\frac{\partial }{\partial t}\hat{O}_H(t) = \exp{(\imath\hat{H}t/\hbar)}\left[\hat{O}_H\hat{H}-\hat{H}\hat{O}_H\right]\exp{(-\imath\hat{H}t/\hbar)}=\left[\hat{O}_H(t),\hat{H}\right].
\]
We note that an operator in the Heisenberg picture can be related to the corresponding
operator in the interaction picture as 
\[
\hat{O}_H(t)=
\exp{(\imath\hat{H}t/\hbar)}\hat{O}_S\exp{(-\imath\hat{H}t/\hbar)}=\]
\[
\exp{(\imath\hat{H}_It/\hbar)}\exp{(-\imath\hat{H}_0t/\hbar)}\hat{O}_I\exp{(\imath\hat{H}_0t/\hbar)}\exp{(-\imath\hat{H}_It/\hbar)}.
\]
}
\end{small}
}
\frame
{
\frametitle{Heisenberg picture}
\begin{small}
{\scriptsize
With our definition of the time evolution operator we see that
\[
\hat{O}_H(t)=\hat{U}(0,t)\hat{O}_I\hat{U}(t,0),
\]
which in turn implies that $\hat{O}_S=\hat{O}_I(0)=\hat{O}_H(0)$, all operators are equal at $t=0$. The wave function in the Heisenberg formalism is 
related to the other pictures as 
\[
|\Psi_H\rangle=|\Psi_S(0)\rangle=|\Psi_I(0)\rangle,
\]
since the wave function in the Heisenberg picture is time independent. 
We can relate this wave function to that a given time $t$ via the time evolution operator as
\[
|\Psi_H\rangle=\hat{U}(0,t)|\Psi_I(t)\rangle.
\]
}
\end{small}
}
\frame
{
\frametitle{Adiabatic hypothesis}
\begin{small}
{\scriptsize
We assume that the interaction term is switched on gradually. Our wave function at time $t=-\infty$ and $t=\infty$ is supposed to represent a non-interacting system
given by the solution to the unperturbed part of our Hamiltonian $\hat{H}_0$.
We assume the ground state is given by $|\Phi_0\rangle$, which could be a Slater determinant.
We define our Hamiltonian as
\[
\hat{H}=\hat{H}_0+\exp{(-\varepsilon t/\hbar)}\hat{H}_I,
\]
where $\varepsilon$ is a small number. The way we write the Hamiltonian 
and its interaction term is meant to simulate the switching of the interaction.
}
\end{small}
}
\frame
{
\frametitle{Adiabatic hypothesis}
\begin{small}
{\scriptsize
The time evolution of the wave function in the interaction picture is then
\[
|\Psi_I(t) \rangle = \hat{U}_{\varepsilon}(t,t_0)|\Psi_I(t_0)\rangle,
\]
with 
\[
\hat{U}_{\varepsilon}(t,t_0)=\sum_{n=0}^{\infty}\left(\frac{-\imath}{\hbar}\right)^n\frac{1}{n!}
\int_{t_0}^t dt_1\dots \int_{t_0}^t dt_N \exp{(-\varepsilon(t_1+\dots+t_n)/\hbar)}\hat{T}\left[\hat{H}_I(t_1)\dots\hat{H}_I(t_n)\right]
\]
}
\end{small}
}
\frame
{
\frametitle{Adiabatic hypothesis}
\begin{small}
{\scriptsize
In the limit $t_0\rightarrow -\infty$, the solution ot Schr\"odinger's equation is
$|\Phi_0\rangle$, and the eigenenergies are given by 
\[
\hat{H}_0|\Phi_0\rangle=W_0|\Phi_0\rangle,
\]
meaning that 
\[
|\Psi_S(t_0)\rangle = \exp{(-\imath W_0t_0/\hbar)}|\Phi_0\rangle,
\]
with the corresponding interaction picture wave function given by
\[
|\Psi_I(t_0)\rangle = \exp{(\imath \hat{H}_0t_0/\hbar)}|\Psi_S(t_0)\rangle=|\Phi_0\rangle.
\]
}
\end{small}
}
\frame
{
\frametitle{Adiabatic hypothesis}
\begin{small}
{\scriptsize
The solution becomes time independent in the limit $t_0\rightarrow -\infty$.
The same conclusion can be reached by looking at 
\[
\imath \hbar\frac{\partial }{\partial t}|\Psi_I(t)\rangle =
\exp{(\varepsilon |t|/\hbar)}\hat{H}_I|\Psi_I(t)\rangle 
\]
and taking the limit $t\rightarrow -\infty$.
We can rewrite the equation for the wave function at a time $t=0$ as
\[
|\Psi_I(0) \rangle = \hat{U}_{\varepsilon}(0,-\infty)|\Phi_0\rangle.
\]
}
\end{small}
}
\frame
{
\frametitle{Goldstone's theorem and Gell-Mann and Low theorem on the ground state}
\begin{small}
{\scriptsize
Our wave function for ground state (after Gell-Mann and Low, see Phys.~Rev.~{\bf 84}, 350 (1951)) is then
\[
        \frac{\ket{\Psi_0}}{\left\langle\Phi_0 | \Psi_0 \right\rangle}=
    \lim_{\epsilon \rightarrow 0}
   \lim_{t'\rightarrow -\infty}
   \frac{U(0,-\infty )\ket{\Phi_0} }
   { \bra{\Phi_0} U(0,-\infty )\ket{\Phi_0} },
\]
and we ask whether this quantity exists to all orders in perturbation theory.
Goldstone's theorem states that only linked diagrams enter the expression for the final binding energy. It means that energy difference reads now
\[
\Delta E=\sum_{i=0}^{\infty}\langle \Phi_0|\hat{H}_I\left\{\frac{\hat{Q}}{W_0-\hat{H}_0}\hat{H}_I\right\}^i|\Phi_0\rangle_L,
\]
where the subscript $L$ indicates that only linked diagrams are included. In our Rayleigh-Schr\"odinger expansion, the energy difference included also unlinked diagrams. 
}
\end{small}
}
\frame
{
\frametitle{Goldstone's theorem and Gell-Mann and Low theorem on the ground state}
\begin{small}
{\scriptsize
If it does, Gell-Mann and Low showed that it is an eigenstate of $\hat{H}$ with eigenvalue
\[
 \hat{H}\frac{\ket{\Psi_0}}{\left\langle\Phi_0 | \Psi_0 \right\rangle}= E\frac{\ket{\Psi_0}}{\left\langle\Phi_0 | \Psi_0 \right\rangle}
\]
and multiplying from the left with $\langle \Phi_0|$ we can rewrite the last equation
as
\[
E-W_0=\frac{\langle \Phi_0|\hat{H}_I\ket{\Psi_0}}{\left\langle\Phi_0 | \Psi_0 \right\rangle},
\]
since $\hat{H}_0|\Phi_0\rangle = W_0|\Phi_0\rangle$. The numerator and the denominators of the last equation do not exist separately. The theorem of Gell-Mann and Low asserts that this ratio exists. 
}
\end{small}
}
\frame
{
\frametitle{Goldstone's theorem and Gell-Mann and Low theorem on the ground state}
\begin{small}
{\scriptsize
We wish to link the above expression with the corresponding expression from time-dependent perturbation theory. We write our expression as
\[
E-W_0=\Delta E= \lim_{\epsilon \rightarrow 0^+}
   \frac{\bra{\Phi_0(0)}\hat{H}_IU_{\epsilon }(0,-\infty )\ket{\Phi_0(-\infty)} }
   { \bra{\Phi_0(0)} U_{\epsilon}(0,-\infty )\ket{\Phi_0(-\infty)} },
\]
with a numerator 
\[
N=\lim_{\epsilon \rightarrow 0^+}\bra{\Phi_0(0)}\hat{H}_IU_{\epsilon}(0,-\infty )\ket{\Phi_0(-\infty)}, 
\]
which we rewrite as
\[
 N=\lim_{\epsilon \rightarrow 0^+}\bra{\Phi_0(0)}\hat{H}_I(t=0)\displaystyle\sum_{n=0}^{\infty}\frac{(-i)^n}{n!}
   \int_{-\infty}^{0}dt_1  \int_{-\infty}^{0}dt_2\dots  \int_{-\infty}^{0}dt_n}\exp{(\epsilon/\hbar(t_1+\dots+t_n))} \hat{T}\left[H_1(t_1)H_1(t_2)\dots H_1(t_n)\right] \ket{\Phi_0(-\infty)}. 
\]
}
\end{small}
}
\frame
{
\frametitle{Goldstone's theorem and Gell-Mann and Low theorem on the ground state}
\begin{small}
{\scriptsize
A linked diagram (or connected diagram) is a diagram which is linked to the last interaction vertex at $t=0$.
We divide the diagrams into linked and unlinked. 
In general, the way we can distribute $\mu$ unlinked diagrams among the total of $n$ diagrams is given by the combinatorial factor
\[
\left(\begin{array}{c} n \\ \mu \end{array}\right) = \frac{n!}{\mu!\nu!},
\]
and using the following relation
\[
\sum_{n=0}^{\infty}\frac{1}{n!}\sum_{\mu+\nu=n}^{\infty}\frac{n!}{\mu!\nu!}=\sum_{\mu=0}^{\infty}\frac{1}{\mu!}\sum_{\nu}^{\infty}\frac{1}{\nu!},
\]
we can rewrite the numerator $N$ as 
\[
N=\bra{\Phi_0(0)}\hat{H}_IU_{\epsilon}(0,-\infty )\ket{\Phi_0(-\infty)}_L\bra{\Phi_0(0)}U_{\epsilon}(0,-\infty )\ket{\Phi_0(-\infty)},  
=N_LD,
\]
with $N_L$ now containing only linked terms
\[
 N_L=\lim_{\epsilon \rightarrow 0^+}\bra{\Phi_0(0)}\hat{H}_I(t=0)\displaystyle\sum_{\nu=0}^{\infty}\frac{(-i)^{\nu}}{\nu!}
   \int_{-\infty}^{0}dt_1  \int_{-\infty}^{0}dt_2\dots  \int_{-\infty}^{0}dt_n}\exp{(\epsilon/\hbar(t_1+\dots+t_n))} \hat{T}\left[H_1(t_1)H_1(t_2)\dots H_1(t_n)\right] \ket{\Phi_0(-\infty)}_L, 
\]
with the subscript $L$ indicating that only linked diagrams appear, that is those diagrams which are linked to the last interaction vertex.
}
\end{small}
}
\frame
{
\frametitle{Goldstone's theorem and Gell-Mann and Low theorem on the ground state}
\begin{small}
{\scriptsize
A linked diagram (or connected diagram) is a diagram which is linked to the last interaction vertex at $t=0$.
We divide the diagrams into linked and unlinked. 
In general, the way we can distribute $\mu$ unlinked diagrams among the total of $n$ diagrams is given by the combinatorial factor
\[
\left(\begin{array}{c} n \\ \mu \end{array}\right) = \frac{n!}{\mu!\nu!},
\]
and using the following relation
\[
\sum_{n=0}^{\infty}\frac{1}{n!}\sum_{\mu+\nu=n}^{\infty}\frac{n!}{\mu!\nu!}=\sum_{\mu=0}^{\infty}\frac{1}{\mu!}\sum_{\nu}^{\infty}\frac{1}{\nu!},
\]
we can rewrite the numerator $N$ as 
\[
N=\bra{\Phi_0(0)}\hat{H}_IU_{\epsilon}(0,-\infty )\ket{\Phi_0(-\infty)}_L\bra{\Phi_0(0)}U_{\epsilon}(0,-\infty )\ket{\Phi_0(-\infty)},  
=N_LD,
\]
with $N_L$ now containing only linked terms
\[
 N_L=\lim_{\epsilon \rightarrow 0^+}\bra{\Phi_0(0)}\hat{H}_I(t=0)\displaystyle\sum_{\nu=0}^{\infty}\frac{(-i)^{\nu}}{\nu!}
   \int_{-\infty}^{0}dt_1  \int_{-\infty}^{0}dt_2\dots  \int_{-\infty}^{0}dt_n}\exp{(\epsilon/\hbar(t_1+\dots+t_n))} \hat{T}\left[H_1(t_1)H_1(t_2)\dots H_1(t_n)\right] \ket{\Phi_0(-\infty)}_L, 
\]
with the subscript $L$ indicating that only linked diagrams appear, that is those diagrams which are linked to the last interaction vertex.
}
\end{small}
}
\frame
{
\frametitle{Goldstone's theorem and Gell-Mann and Low theorem on the ground state}
\begin{small}
{\scriptsize
We note that also that the term $D$ is nothing but the denominator of the equation for the energy. We obtain then the following expression for the energy
\[
E-W_0=\Delta E=N_L= \bra{\Phi_0(0)}\hat{H}_IU_{\epsilon}(0,-\infty )\ket{\Phi_0(-\infty)}_L,
\]
and Goldstone's theorem is then proved. 
The corresponding expression from Rayleigh-Schr\"odinger perturbation theory is given by
\[
\Delta E=\langle \Phi_0|\left(\hat{H}_I+\hat{H}_I\frac{\hat{Q}}{W_0-\hat{H}_0}\hat{H}_I+
\hat{H}_I\frac{\hat{Q}}{W_0-\hat{H}_0}\hat{H}_I\frac{\hat{Q}}{W_0-\hat{H}_0}\hat{H}_I+\dots\right)|\Phi_0\rangle_C.
\]
}
\end{small}
}
\frame
{
\frametitle{Goldstone's theorem and Gell-Mann and Low theorem on the ground state}
\begin{small}
{\scriptsize
An important point in the derivation of the Gell-Mann and Low theorem
\[
E-W_0=\frac{\langle \Phi_0|\hat{H}_I\ket{\Psi_0}}{\left\langle\Phi_0 | \Psi_0 \right\rangle},
\]
is that the numerator and the denominators of the last equation do not exist separately. The theorem of Gell-Mann and Low asserts that this ratio exists. To prove it we proceed as follows. Consider the expression
\[
(\hat{H}_0-E)U_{\epsilon }(0,-\infty )\ket{\Phi_0}=\left[\hat{H}_0,U_{\epsilon }(0,-\infty )\right]\ket{\Phi_0}.
\]
}
\end{small}
}
\frame
{
\frametitle{Goldstone's theorem and Gell-Mann and Low theorem on the ground state}
\begin{small}
{\scriptsize
To evaluate the commutator 
\[
(\hat{H}_0-E)U_{\epsilon }(0,-\infty )\ket{\Phi_0}=\left[\hat{H}_0,U_{\epsilon }(0,-\infty )\right]\ket{\Phi_0}.
\]
we write the associate commutator as
\[
\left[\hat{H}_0,\hat{H}_I(t_1)\hat{H}_I(t_2)\dots \hat{H}_I(t_n)\right]=
\left[\hat{H}_0,\hat{H}_I(t_1)\right]\hat{H}_I(t_2)\dots \hat{H}_I(t_n)+
\]
\[
\dots+\hat{H}_I(t_1)\left[\hat{H}_0,\hat{H}_I(t_2)\right]\hat{H}_I(t_3)\dots \hat{H}_I(t_n)+\dots
\]
Using the equation of motion for an operator in the interaction picture we have
\[
\imath \hbar\frac{\partial }{\partial t}\hat{H}_I(t) = \left[\hat{H}_I(t),\hat{H}_0\right].
\]
Each of the above commutators yield then a time derivative!
}
\end{small}
}
\frame
{
\frametitle{Goldstone's theorem and Gell-Mann and Low theorem on the ground state}
\begin{small}
{\scriptsize
We have then
\[
\left[\hat{H}_0,\hat{H}_I(t_1)\hat{H}_I(t_2)\dots \hat{H}_I(t_n)\right]=\imath \hbar\left(\frac{\partial }{\partial t_n}+\frac{\partial }{\partial t_1}+\dots+\frac{\partial }{\partial t_n}\right) \hat{H}_I(t_1)\hat{H}_I(t_2)\dots\hat{H}_I(t_n),
\]
meaning that we can rewrite
\[
(\hat{H}_0-E)U_{\epsilon }(0,-\infty )\ket{\Phi_0}=\left[\hat{H}_0,U_{\epsilon }(0,-\infty )\right]\ket{\Phi_0},
\]
as
\[
(\hat{H}_0-E)U_{\epsilon }(0,-\infty )\ket{\Phi_0}=-\sum_{n=1}^{\infty}\left(\frac{-\imath}{\hbar}\right)^{n-1}\frac{1}{n!}
\int_{t_0}^t dt_1\dots \int_{t_0}^t dt_N \exp{(-\varepsilon(t_1+\dots+t_n)/\hbar)}
\]
\[
\times\sum_{i=1}^n(\frac{\partial }{\partial t_i} )\hat{T}\left[\hat{H}_I(t_1)\dots\hat{H}_I(t_n)\right].
\]
}
\end{small}
}
\frame
{
\frametitle{Goldstone's theorem and Gell-Mann and Low theorem on the ground state}
\begin{small}
{\scriptsize
All the time derivatives in this equation 
\[
(\hat{H}_0-E)U_{\epsilon }(0,-\infty )\ket{\Phi_0}=-\sum_{n=1}^{\infty}\left(\frac{-\imath}{\hbar}\right)^{n-1}\frac{1}{n!}
\int_{t_0}^t dt_1\dots \int_{t_0}^t dt_N \exp{(-\varepsilon(t_1+\dots+t_n)/\hbar)}
\]
\[
\times\sum_{i=1}^n(\frac{\partial }{\partial t_i} )\hat{T}\left[\hat{H}_I(t_1)\dots\hat{H}_I(t_n)\right],
\]
make the same contribution, as can be seen by changing dummy variables. We can therefore retain just one time derivative $\partial/\partial t$ and multiply with $n$. Integrating by parts wrt $t_1$  we obtain two terms. 
}
\end{small}
}
\frame
{
\frametitle{Goldstone's theorem and Gell-Mann and Low theorem on the ground state}
\begin{small}
{\scriptsize
Integrating by parts wrt $t_1$  one can finally show that
\[
        \frac{\ket{\Psi_0}}{\left\langle\Phi_0 | \Psi_0 \right\rangle}=
    \lim_{\epsilon \rightarrow 0}
   \lim_{t'\rightarrow -\infty}
   \frac{U(0,-\infty )\ket{\Phi_0} }
   { \bra{\Phi_0} U(0,-\infty )\ket{\Phi_0} },
\]
For more details about the derivation, see Gell-Mann and Low, Phys.~Rev.~{\bf 84}, 350  (1951). See also chapter 6.2 of Raimes or Fetter and Walecka, chapter 3.6.
}
\end{small}
}
\frame
{
\frametitle{Goldstone's theorem and Gell-Mann and Low theorem on the ground state}
\begin{small}
{\scriptsize
In the present discussion of the time-dependent theory we will make
use of the so-called complex-time approach to describe the time
evolution operator $U$.
This means that we
allow the time $t$ to be rotated by a small angle $\epsilon$
relative to the real time axis. The complex time $t$ is then
related to the real time $\tilde{t}$ by
\[
t=\tilde{t}(1-i\epsilon ).
\]
Let us first study the true eigenvector $\Psi_{\alpha}$ which evolves
from the unperturbed eigenvectors $\Phi_{\alpha}$ through the action of the
time development operator
\[
   U_{\varepsilon}(t,t')=\lim_{\epsilon \rightarrow 0}
   \lim_{t'\rightarrow -\infty}
   {\displaystyle\sum_{n=0}^{\infty}\frac{(-i)^n}{n!}
   \int_{t'}^{t}dt_1  \int_{t'}^{t}dt_2\dots  \int_{t'}^{t}dt_n}
\]
\[
	      \times T\left[H_1(t_1)H_1(t_2)\dots H_1(t_n)\right],
\]
where $T$ stands for the correct time-ordering.
}
\end{small}
}
\frame
{
\frametitle{Goldstone's theorem and Gell-Mann and Low theorem on the ground state}
\begin{small}
{\scriptsize
In time-dependent
perturbation theory we let $\Psi_{\alpha}$ develop from $\Phi_{\alpha}$ in the
remote past to a given time $t$
\[
    \frac{\ket{\Psi_{\alpha}}}
    {\left\langle\psi_{\alpha} | \Psi_{\alpha} \right\rangle}=
    \lim_{\epsilon \rightarrow 0}
   \lim_{t'\rightarrow -\infty}
   \frac{U_{\varepsilon}(t,t' )\ket{\psi_{\alpha}} }
   { \bra{\psi_{\alpha}} U(t,t' )\ket{\Phi_{\alpha}} },
\]
and similarly, we let
$\Psi_{\beta}$ develop from $\Phi_{\beta}$ in the remote future
\[
    \frac{\bra{\Psi_{\beta}}}{\left\langle
    \psi_{\beta} | \Psi_{\beta} \right\rangle}=
    \lim_{\epsilon \rightarrow 0}
    \lim_{t'\rightarrow \infty}
    \frac{\bra{\psi_{\beta}}U_{\varepsilon}(t' ,t) }
    { \bra{\psi_{\beta}} U_{\varepsilon}(t' ,t)\ket{\Phi_{\beta}} }.
\]
}
\end{small}
}
\frame
{
\frametitle{Goldstone's theorem and Gell-Mann and Low theorem on the ground state}
\begin{small}
{\scriptsize
Here we are interested in the expectation value of a given
operator ${\cal O}$ acting at a time $t=0$. This can be achieved
from the two previous equations defining
\[
     \ket{\Psi_{\alpha ,\beta}'}=
     \frac{\ket{\Psi_{\alpha ,\beta}}}
     {\left\langle\Phi_{\alpha ,\beta} | \Psi_{\alpha ,\beta} \right\rangle}
\]
we have
\[
   {\cal O}_{\alpha\beta}
  =\frac{N_{\beta\alpha}}{D_{\beta}D_{\alpha}},
\]
where we have introduced
\[
   N_{\beta\alpha}=
   \bra{\Phi_{\beta}}U_{\varepsilon}(\infty ,0){\cal O}U_{\varepsilon}(0,-\infty )\ket{\Phi_{\alpha}} ,
\]
and 
\[
   D_{\alpha ,\beta}=
   \sqrt{\bra{\psi_{\alpha ,\beta}}
   U_{\varepsilon}(\infty ,0)U_{\varepsilon}(0,-\infty )\ket{\Phi_{\alpha ,\beta}}}. 
\]
}
\end{small}
}
\frame
{
\frametitle{Goldstone's theorem and Gell-Mann and Low theorem on the ground state}
\begin{small}
{\scriptsize
If the operator ${\cal O}$ stands for the hamiltonian $H$ we obtain
\[
    {\displaystyle  \frac{\bra{\Psi_{\lambda}'}H\ket{\Psi_{\lambda}'} }
   { \left\langle\Psi_{\lambda}' | \Psi_{\lambda}' \right\rangle} }
\]
At this stage, {\em it is important to observe} that our 
expression for the expectation value of a given operator ${\cal O}$
{\em is hermitian} insofar ${\cal O}^{\dagger}={\cal O}$. This is readily 
demonstrated. The above equation is of the general form
\[
U(t,t_0){\cal O}U(t_0,-t),
\]
and noting that 
\[
   U^{\dagger}(t,t_0)=
   \left({\displaystyle e^{iH_0t}e^{-iH(t-t_0)}e^{-iH_0t}}\right)^{\dagger}
   =U(t_0,-t),
\]
since $H^{\dagger}=H$ and $H_0^{\dagger}=H_0$, we have that
\[
    \left(U(t,t_0){\cal O}U(t_0,-t)\right)^{\dagger}
    =U(t,t_0){\cal O}U(t_0,-t).
\]
The question we pose now is what happens in the limit $\varepsilon\rightarrow 0$?
Do we get results which are meaningful?
}
\end{small}
}
\frame
{
\frametitle{Goldstone's theorem and Gell-Mann and Low theorem on the ground state}
\begin{small}
{\scriptsize
Our wave function for ground state is then
\[
        \frac{\ket{\Psi_0}}{\left\langle\Phi_0 | \Psi_0 \right\rangle}=
    \lim_{\epsilon \rightarrow 0}
   \lim_{t'\rightarrow -\infty}
   \frac{U(0,-\infty )\ket{\Phi_0} }
   { \bra{\Phi_0} U(0,-\infty )\ket{\Phi_0} },
\]
meaning that the energy difference is given by
\[
E_0-W_0=\Delta E_0= \lim_{\epsilon \rightarrow 0}
   \lim_{t'\rightarrow -\infty}
   \frac{\bra{\Phi_0}\hat{H}_IU_{\varepsilon}(0,-\infty )\ket{\Phi_0} }
   { \bra{\Phi_0} U_{\varepsilon}(0,-\infty )\ket{\Phi_0} },
\]
and we ask whether this quantity exists to all orders in perturbation theory.
}
\end{small}
}
\frame
{
\frametitle{Goldstone's theorem and Gell-Mann and Low theorem on the ground state}
\begin{small}
{\scriptsize
If it does, Gell-Mann and Low showed that it is an eigenstate of $\hat{H}$ with eigenvalue
\[
 \hat{H}\frac{\ket{\Psi_0}}{\left\langle\Phi_0 | \Psi_0 \right\rangle}= E_0\frac{\ket{\Psi_0}}{\left\langle\Phi_0 | \Psi_0 \right\rangle}
\]
and multiplying from the left with $\langle \Phi_0|$ we can rewrite the last equation
as
\[
E_0-W_0=\frac{\langle \Phi_0|\hat{H}_I\ket{\Psi_0}}{\left\langle\Phi_0 | \Psi_0 \right\rangle},
\]
since $\hat{H}_0|\Phi_0\rangle = W_0|\Phi_0\rangle$. The numerator and the denominators of the last equation do not exist separately. The theorem of Gell-Mann and Low asserts that this ratio exists. 
}
\end{small}
}
\frame
{
\frametitle{Goldstone's theorem and Gell-Mann and Low theorem on the ground state}
\begin{small}
{\scriptsize
Goldstone's theorem states that only linked diagrams enter the expression for the final binding energy. It means that energy difference reads now
\[
\Delta E_0=\sum_{i=0}^{\infty}\langle \Phi_0|\hat{H}_I\left\{\frac{\hat{Q}}{W_0-\hat{H}_0}\hat{H}_I\right\}^i|\Phi_0\rangle_L,
\]
where the subscript $L$ indicates that only linked diagrams are included. In our Rayleigh-Schr\"odinger expansion, the energy difference included also unlinked diagrams. 
}
\end{small}
}
\frame
{
\frametitle{Goldstone's theorem and Gell-Mann and Low theorem on the ground state}
\begin{small}
{\scriptsize
We wish to link the above expression with the corresponding expression from time-dependent perturbation theory. We write our expression as
\[
E_0-W_0=\Delta E_0= \lim_{\epsilon \rightarrow 0^+}
   \frac{\bra{\Phi_0(0)}\hat{H}_IU_{\epsilon }(0,-\infty )\ket{\Phi_0(-\infty)} }
   { \bra{\Phi_0(0)} U_{\epsilon}(0,-\infty )\ket{\Phi_0(-\infty)} },
\]
with a numerator 
\[
N=\lim_{\epsilon \rightarrow 0^+}\bra{\Phi_0(0)}\hat{H}_IU_{\epsilon}(0,-\infty )\ket{\Phi_0(-\infty)}, 
\]
which we rewrite as
\[
 N=\lim_{\epsilon \rightarrow 0^+}\bra{\Phi_0(0)}\hat{H}_I(t=0)\displaystyle\sum_{n=0}^{\infty}\frac{(-i)^n}{n!}
   \int_{-\infty}^{0}dt_1  \int_{-\infty}^{0}dt_2\dots  \int_{-\infty}^{0}dt_n}\exp{(\epsilon/\hbar(t_1+\dots+t_n))} \hat{T}\left[H_1(t_1)H_1(t_2)\dots H_1(t_n)\right] \ket{\Phi_0(-\infty)}. 
\]
}
\end{small}
}
\frame
{
\frametitle{Goldstone's theorem and Gell-Mann and Low theorem on the ground state}
\begin{small}
{\scriptsize
From this term we can obtain both linked and unlinked contributions. Goldstone's theorem states that only linked diagrams enter the expression for the final binding energy. 
A linked diagram (or connected diagram) is a diagram which is linked to the last interaction vertex at $t=0$.
We label the number of linked diagrams with the variable $\nu$ and the number of unlinked with $\mu$  with $n=\nu+\mu$.  The number of unlinked diagrams is then $\mu=n-\nu$. 
}
\end{small}
}
\frame
{
\frametitle{Goldstone's theorem and Gell-Mann and Low theorem on the ground state}
\begin{small}
{\scriptsize
In general, the way we can distribute $\mu$ unlinked diagrams among the total of $n$ diagrams is given by the combinatorial factor
\[
\left(\begin{array}{c} n \\ \mu \end{array}\right) = \frac{n!}{\mu!\nu!},
\]
and using the following relation
\[
\sum_{n=0}^{\infty}\frac{1}{n!}\sum_{\mu+\nu=n}^{\infty}\frac{n!}{\mu!\nu!}=\sum_{\mu=0}^{\infty}\frac{1}{\mu!}\sum_{\nu}^{\infty}\frac{1}{\nu!},
\]
we can rewrite the numerator $N$ as 
\[
N=\bra{\Phi_0(0)}\hat{H}_IU_{\epsilon}(0,-\infty )\ket{\Phi_0(-\infty)}_L\bra{\Phi_0(0)}U_{\epsilon}(0,-\infty )\ket{\Phi_0(-\infty)}=N_LD.
\]
}
\end{small}
}
\frame
{
\frametitle{Goldstone's theorem and Gell-Mann and Low theorem on the ground state}
\begin{small}
{\scriptsize
We define  $N_L$ to contain only linked terms
%\[
% N_L=\lim_{\epsilon \rightarrow 0^+}\bra{\Phi_0(0)}\hat{H}_I(t=0)\sum_{\nu=0}^{\infty}\frac{(-i)^{\nu}}{\nu!}\int_{-\infty}^{0}dt_1  \int_{-\infty}^{0}dt_2\dots  \int_{-\infty}^{0}dt_n}\exp{(\epsilon/\hbar(t_1+\dots+t_n))} \hat{T}\left[H_1(t_1)H_1(t_2)\dots H_1(t_n)\right] \ket{\Phi_0(-\infty)}_L, 
%\]
with the subscript $L$ indicating that only linked diagrams appear, that is those diagrams which are linked to the last interaction vertex.
}
\end{small}
}
\frame
{
\frametitle{Goldstone's theorem and Gell-Mann and Low theorem on the ground state}
\begin{small}
{\scriptsize
We note that also that the term $D$ is nothing but the denominator of the equation for the energy. We obtain then the following expression for the energy
\[
E_0-W_0=\Delta E_0=N_L= \bra{\Phi_0(0)}\hat{H}_IU_{\epsilon}(0,-\infty )\ket{\Phi_0(-\infty)}_L,
\]
and Goldstone's theorem is then proved. 
The corresponding expression from Rayleigh-Schr\"odinger perturbation theory is given by
\[
\Delta E_0=\langle \Phi_0|\left(\hat{H}_I+\hat{H}_I\frac{\hat{Q}}{W_0-\hat{H}_0}\hat{H}_I+
\hat{H}_I\frac{\hat{Q}}{W_0-\hat{H}_0}\hat{H}_I\frac{\hat{Q}}{W_0-\hat{H}_0}\hat{H}_I+\dots\right)|\Phi_0\rangle_C.
\]
}
\end{small}
}
\end{document}
\end{document}
