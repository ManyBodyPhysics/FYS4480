\documentclass[10pt,aspectratio=169]{beamer}
\usepackage{amsmath,amssymb,bm}
\usepackage{tikz}
\usepackage{physics}
\usepackage{mathtools}
\usetikzlibrary{decorations.pathmorphing,arrows.meta,positioning}
\usetheme{default}
\usecolortheme{seahorse}
\setbeamertemplate{navigation symbols}{}
\setbeamertemplate{footline}[frame number]

% Packages
\usepackage[T1]{fontenc}
\usepackage[utf8]{inputenc}
\usepackage{lmodern}
\usepackage{physics}
\usepackage{bm}
\usepackage{mathtools, amsmath, amssymb}
\usepackage{cancel}
\usepackage{tikz}
\usetikzlibrary{arrows.meta,calc,decorations.pathmorphing,positioning}

% Handy macros
\newcommand{\NO}[1]{:\!#1\!:}                         % normal ordering
\newcommand{\con}{\;\;\raisebox{0.2ex}{\tiny$\contraction{}{}{}{}$}\;} % visual contraction hint (optional)
\newcommand{\E}{\mathcal{E}}
\newcommand{\Hcal}{\mathcal{H}}
\newcommand{\Hn}{H_0}
\newcommand{\V}{V}
\newcommand{\Pcal}{\mathcal{P}}
\newcommand{\Qcal}{\mathcal{Q}}
\newcommand{\G}{\mathcal{G}}
\newcommand{\Wop}{W}
\newcommand{\Wopb}{\bar{W}}
\newcommand{\Order}{\mathcal{T}}                       % (time) ordering symbol placeholder
\newcommand{\ketPhi}{\ket{\Phi}}                      % reference (Slater) determinant
\newcommand{\braPhi}{\bra{\Phi}}
\newcommand{\Egs}{E}
\newcommand{\Z}{\mathcal{Z}}
\newcommand{\C}{\mathrm{C}}                           % connected
\DeclareMathOperator{\Tr}{Tr}





\title{Diagram Rules in Goldstone Diagrams and Linked Diagram Theorem}
\subtitle{Time-Independent Fermionic Many-Body Perturbation Theory}
\author{FYS4480/9480}
\date{Fall Semester 2025}

\begin{document}
\maketitle

% ================================================================
\begin{frame}{Outline}
\tableofcontents
\end{frame}

\section{Setup and Notation}

\begin{frame}{Operator Algebra and Reference State}
We work in the second-quantized representation:
\[
\{a_p,a_q^\dagger\}=\delta_{pq}, \qquad \{a_p,a_q\}=0, \qquad \{a_p^\dagger,a_q^\dagger\}=0.
\]
The reference determinant $\ket{\Phi_0}$ defines contractions:
\[
\contraction{}{a_p}{}{a_q^\dagger} a_p a_q^\dagger
\equiv \langle \Phi_0 | a_p a_q^\dagger | \Phi_0 \rangle
= \delta_{pq}\, n_p,
\]
where $n_p=1$ for hole states, $0$ for particle states.

A typical term in the MBPT expansion involves
\[
\mathcal{O} = O_1 O_2 \cdots O_{2n}, \quad O_k \in \{a_p,a_q^\dagger\},
\]
where each (O_k) is either (a_r) or (a_s^\dagger). Wick’s theorem writes
[
  \langle \Phi_0 | T[ \mathcal{O} ] |\Phi_0\rangle
  = \sum_{\text{all pairings }P} \operatorname{sgn}(P)\prod_{\text{pairs }(i,j)\in P}\contraction{}{O_i}{}{O_j} O_i O_j,
]
where $(\operatorname{sgn}(P)\in{\pm1})$ is the sign for pairing (P). Our goal is to
compute and interpret $(\operatorname{sgn}(P))$.


Our goal: derive the exact sign of each Wick contraction term.
\end{frame}

% ================================================================
\section{Lemma 1 — Sign from Reordering}

\begin{frame}{Lemma 1: Sign from Reordering}
\textbf{Statement.}  
Let $P$ be a pairing of indices $\{1,\dots,2n\}$ into $n$ pairs $(i,j)$ with $i<j$.  
Then
\[
\langle \Phi_0 | T[\mathcal{O}] |\Phi_0\rangle
= \sum_P (-1)^{N_{\text{swaps}}(P)} 
  \prod_{(i,j)\in P} 
  \contraction{}{O_i}{}{O_j} O_i O_j,
\]
where $N_{\text{swaps}}$ is the number of adjacent anticommutations required
to bring paired operators adjacent.

\textbf{Proof sketch.}
\begin{itemize}
  \item Each exchange of two fermion operators introduces a factor $-1$.
  \item Bringing the operators into pair-adjacent form requires a finite number
        of adjacent swaps.
  \item The total sign is $(-1)^{N_{\text{swaps}}}$.
\end{itemize}
\end{frame}

% ================================================================
\section{Lemma 2 — Permutation Parity and Crossings}

\begin{frame}{Lemma 2: Crossings and Sign Parity}
Represent operator order on a horizontal line and draw arcs between paired operators:
\[
(i,j) \to \text{arc connecting positions } i \text{ and } j.
\]


\textbf{Lemma:}  
The parity of $N_{\text{swaps}}$ equals the parity of the number of pairwise arc intersections:
\[
(-1)^{N_{\text{swaps}}} = (-1)^{N_{\text{crossings}}}.
\]


\textbf{Proof sketch.}
\begin{itemize}
  \item Consider two pairs $(i,j)$ and $(k,\ell)$ with $i<j$ and $k<\ell$.
  \item If $i<k<j<\ell$, the arcs cross.
  \item Each crossing corresponds to one swap of fermionic operators.
\end{itemize}


\textbf{Corollary:} Diagrammatically,
\[
\operatorname{sgn}(P) = (-1)^{N_{\text{crossings}}}.
\]
\end{frame}

% ================================================================
\section{Lemma 3 — Closed Fermion Loops}

\begin{frame}{Lemma 3: Closed Fermion Loops}
\textbf{Statement:}  
Each closed fermion loop contributes an additional factor $(-1)$.

\textbf{Operator-level proof.}
\begin{itemize}
  \item A closed loop corresponds to a cyclic contraction such as
        $a_{p_1}^\dagger a_{q_1} a_{p_2}^\dagger a_{q_2}\cdots a_{p_m}^\dagger a_{q_m}$,
        where $a_{q_m}$ contracts with $a_{p_1}^\dagger$.
  \item To perform this contraction, one operator must be moved
        past all other fermionic operators --- an odd number of swaps.
  \item Hence, one factor of $(-1)$ per closed fermion loop.
\end{itemize}
\end{frame}

% ================================================================
\section{Diagrammatic Interpretation}

\begin{frame}{From Algebra to Diagrams}
\textbf{Phase rule for fermions:}
\[
\operatorname{sgn}(P) = (-1)^{N_{\text{crossings}} + N_{\text{loops}}}.
\]

\textbf{Interpretation:}
\begin{itemize}
  \item Each line crossing $\Rightarrow$ one minus sign.
  \item Each closed fermion loop $\Rightarrow$ one minus sign.
\end{itemize}

\textbf{Example:}
\begin{itemize}
  \item 2p--2h ladder: $N_{\text{crossings}}=0$, $N_{\text{loops}}=0$ $\Rightarrow$ sign $+1$.
  \item Particle-hole ring: $N_{\text{crossings}}=0$, $N_{\text{loops}}=1$ $\Rightarrow$ sign $-1$.
\end{itemize}
\end{frame}

% ================================================================
\section{Symmetry Factors}

\begin{frame}{Counting Equivalent Contractions}
Each perturbative order has:
\[
E^{(n)} \sim \frac{1}{n!}
  \langle \Phi_0 | V \frac{Q}{E_0-H_0} \cdots V | \Phi_0\rangle.
\]

The factor $1/n!$ comes from expansion of $e^{-tV}$ or Dyson series.


\textbf{When several Wick contractions yield the same topology:}
\[
S = |\mathrm{Aut}(\text{diagram})| \;\Rightarrow\; 
\text{include factor } \frac{1}{S}.
\]


\textbf{Final rule:}
\[
\text{Diagram weight} = 
\frac{(-1)^{N_{\text{crossings}} + N_{\text{loops}}}}{S}\,
\text{(product of matrix elements and denominators)}.
\]
\end{frame}

% ================================================================
\section{Grassmann Formulation}

\begin{frame}{Grassmann Derivation (I)}
Consider generating functional for Grassmann fields $\psi_p, \bar{\psi}_p$:
\[
Z[\bar{\eta},\eta]
= \int \mathcal{D}(\bar{\psi},\psi)\;
  e^{-\bar{\psi}G^{-1}\psi + \bar{\eta}\psi + \bar{\psi}\eta}.
\]

Expanding the interaction
\[
e^{-\bar{\psi}V\psi\psi} 
= \sum_n \frac{(-1)^n}{n!} (\bar{\psi}\bar{\psi}\psi\psi)^n
\]
and performing Wick contractions via Gaussian integration generates
all diagrams automatically.
\end{frame}

\begin{frame}{Grassmann Derivation (II)}
A closed fermion loop corresponds to a trace of propagators:
\[
\Tr(GG\cdots G).
\]


Under a cyclic permutation of Grassmann variables,
\[
\psi_1\psi_2\cdots\psi_n = (-1)^{n-1}\psi_2\cdots\psi_n\psi_1.
\]


Thus, each cyclic trace contributes a factor $(-1)$
relative to bosons.


\textbf{Hence:}
\[
\boxed{\text{Each closed fermion loop } \Rightarrow (-1).}
\]

This reproduces the operator-based sign rule
in a manifestly algebraic, field-theoretic way.
\end{frame}

% ================================================================
\section{Summary}

\begin{frame}{Summary of Fermionic Phase Rules}
\begin{block}{Goldstone Diagram Sign Rules}
For any fermionic diagram in time-independent MBPT:
\[
\text{sign} = (-1)^{N_{\text{crossings}} + N_{\text{loops}}}.
\]
\end{block}

\textbf{Each element contributes:}
\begin{itemize}
  \item Crossing of fermion lines: $-1$.
  \item Closed fermion loop: $-1$.
  \item Symmetry factor $S$ from equivalent Wick contractions.
\end{itemize}

\textbf{Final expression:}
\[
E^{(n)}_{\text{diagram}} = 
\frac{(-1)^{N_{\text{crossings}}+N_{\text{loops}}}}{S}\,
\sum_{\text{indices}}
\frac{\prod V_{\text{lines}}}{\Delta E_{\text{denominators}}}.
\]
\end{frame}


\begin{frame}{Hamiltonian Partition and Reference State}
We consider
\[
H=\Hn+\V,\qquad \Hn\ket{\Phi}=E_0\ket{\Phi},
\]
with $\ket{\Phi}$ a Slater determinant (closed-shell, non-degenerate). Define projection operators
\[
\Pcal=\ketPhi\braPhi,\qquad \Qcal=1-\Pcal.
\]
\smallskip
\textbf{Goal:} Compute the exact ground-state energy $\Egs$ as an expansion in $\V$ centered at $\ketPhi$.
\end{frame}

\begin{frame}{RS Energy Expansion (Time-Independent MBPT)}
Let $\ket{\Psi}$ be the exact ground state, normalized as $\braket{\Phi}{\Psi}=1$ (intermediate normalization). Then
\[
\Egs = E_0 + \Delta E, \qquad \Delta E = \sum_{n\ge 1} E^{(n)}.
\]
Standard RS formulas (compactly):
\[
E^{(1)}=\matrixel{\Phi}{\V}{\Phi},\qquad
E^{(2)}=\sum_{\nu\neq 0}\frac{\abs{\matrixel{\Phi}{\V}{\nu}}^2}{E_0-E_\nu^{(0)}},\quad \dots
\]
\textbf{Diagrammatically:} each term $\leftrightarrow$ collection of Goldstone diagrams (closed, vacuum diagrams) built from $\V$ and $\Hn$ lines/propagators.
\end{frame}

\begin{frame}{Normal Ordering and Contractions (Assumed Known)}
Write (w.r.t.\ $\ketPhi$):
\[
\V=\NO{\V} + \underbrace{\wick{\c1 a^\dagger \c1 a}}_{\text{contractions}} + \cdots
\]
Wick's theorem (time-independent version) reduces expectation values to sums of products of contractions.\medskip

\textbf{Diagram rules (Goldstone):} vertices for $\V$, lines for particle/hole propagators, symmetry factors $S$, energy denominators from ordered integrals / resolvents.
\end{frame}

\begin{frame}{Vacuum-to-Vacuum Amplitude and Generating Picture}
Introduce a bookkeeping parameter $\lambda$:
\[
H(\lambda)=\Hn+\lambda \V.
\]
Define the (adiabatic) vacuum-to-vacuum amplitude
\[
\Z(\lambda) \equiv \braPhi \Omega(\lambda) \ketPhi,
\]
where $\Omega(\lambda)$ denotes the M\o ller wave operator that maps $\ketPhi$ to the interacting state in the adiabatic limit.
\smallskip

\textbf{Heuristic:} Many-body \emph{vacuum diagrams} (closed diagrams) contribute to $\Z$. \emph{Connected} vacuum diagrams contribute to $\log\Z$.
\end{frame}

\begin{frame}{Statement: Linked Diagram (Linked-Cluster) Theorem}
\textbf{Theorem.} \emph{For time-independent MBPT with intermediate normalization, the ground-state energy correction $\Delta E$ is given by the sum of \emph{linked} (connected) vacuum diagrams only. Unlinked (disconnected) vacuum diagrams cancel order by order due to normalization and exponentiation.}
\[
\boxed{\;\Delta E = \sum_{\text{connected vacuum diagrams}} \text{(value of diagram)}\;}
\]
Equivalently: if $\Z$ is the sum of all (linked and unlinked) vacuum diagrams,
\[
\log \Z = \sum_{\text{connected vacuum diagrams}},
\]
and the energy shift follows from the $\lambda$-derivative of $\log\Z$ at $\lambda=1$ (or from standard RS expressions).
\end{frame}

\begin{frame}{Connected vs.\ Unlinked: A Visual Reminder}
\centering
\begin{tikzpicture}[scale=1.0, every node/.style={font=\small}]
% Connected bubble
\draw[thick] (0,0) circle (0.7);
\node at (0,-1.1){Connected};

% Unlinked pair
\draw[thick] (4,0.3) circle (0.5);
\draw[thick] (5.2,-0.3) circle (0.5);
\node at (4.6,-1.1){Unlinked (disconnected)};
\end{tikzpicture}

\smallskip
In algebra: products of lower-order \emph{connected} contributions generate \emph{unlinked} composites; these are precisely removed by the logarithm/normalization.
\end{frame}

\begin{frame}{Goldstone Diagrams for Energy: Examples}
\begin{columns}
\column{0.55\textwidth}
\textbf{Second order}
\[
E^{(2)} = \sum_{\substack{ab \\ ij}} 
\frac{\abs{\matrixel{ij}{\V}{ab}}^2}{\varepsilon_i+\varepsilon_j-\varepsilon_a-\varepsilon_b}
\]
(one connected bubble)
\medskip

\textbf{Third order}
\[
E^{(3)} = \sum \frac{\matrixel{\Phi}{\V}{\nu}\matrixel{\nu}{\V}{\mu}\matrixel{\mu}{\V}{\Phi}}
{(E_0-E_\nu^{(0)})(E_0-E_\mu^{(0)})}
\]
(topologies: ring, ladder, crossed-ladder)
\column{0.45\textwidth}
\centering
\begin{tikzpicture}[scale=0.9]
% Simple bubble
\draw[thick] (0,0) circle (0.9);
\draw[thick] (-0.5,0.6) -- (0.5,-0.6);
\draw[thick] (-0.5,-0.6) -- (0.5,0.6);
\node at (0,-1.3){2nd order};
\end{tikzpicture}

\medskip

\begin{tikzpicture}[scale=0.9]
% Ring-like
\draw[thick] (0,0) circle (1.0);
\foreach \a in {30,150,270}
  \fill (\a:1.0) circle (2pt);
\draw[thick] (30:1.0) -- (150:1.0);
\draw[thick] (150:1.0) -- (270:1.0);
\draw[thick] (270:1.0) -- (30:1.0);
\node at (0,-1.4){3rd order (ring)};
\end{tikzpicture}
\end{columns}
\end{frame}

\begin{frame}{Key Takeaways from Lecture 1}
\begin{itemize}
  \item Energies in RS-MBPT are sums over vacuum (closed) diagrams built from $\V$.
  \item \textbf{Linked-Cluster Theorem:} only connected vacuum diagrams contribute to $\Delta E$.
  \item The mechanism is combinatorial: unlinked diagrams exponentiate and cancel against normalization; \(\log \Z\) generates connected clusters.
\end{itemize}
Next: formal derivation via cumulant (connected) expansion.
\end{frame}


\begin{frame}{Derivation}
\begin{enumerate}
  \item Define the vacuum functional $\Z(\lambda)$ and its perturbative expansion
  \item Show: $\log \Z(\lambda)$ collects only connected vacuum diagrams
  \item Extract $\Delta E$ from $\log \Z$
  \item Work through explicit 2nd and 3rd order to see cancellation of unlinked pieces
\end{enumerate}
\end{frame}

\begin{frame}{Vacuum Functional and Expansion}
Introduce an adiabatic regulator (formal):
\[
\Z(\lambda)=\frac{\braPhi \Omega(\lambda)\ketPhi}{\braket{\Phi}{\Phi}}
= \sum_{n=0}^{\infty}\frac{\lambda^n}{n!}\,\braPhi \V^n \ketPhi_{\text{connected+disconnected}},
\]
where Wick reduces $\braPhi \V^n \ketPhi$ to sums of complete contractions.
\smallskip

\textbf{Cluster decomposition:} any complete contraction decomposes uniquely into a product of \emph{connected} contractions (clusters).
\end{frame}

\begin{frame}{Combinatorics of Clusters $\Rightarrow$ Exponentiation}
Let $C_n$ denote the sum of all connected vacuum contractions with $n$ vertices of $\V$. Then every disconnected contraction is a product of connected pieces:
\[
\Z(\lambda)= 1 + \sum_{n\ge 1} \frac{\lambda^n}{n!}\Big(\text{all contractions}\Big)
= \exp\!\left(\sum_{n\ge 1}\frac{\lambda^n}{n!}\, C_n\right).
\]
\textbf{Thus,}
\[
\boxed{\log\Z(\lambda)=\sum_{n\ge 1}\frac{\lambda^n}{n!}\, C_n \equiv \sum_{\text{connected vacuum diagrams}}.}
\]
This is the \emph{linked-cluster (connected) expansion}.
\end{frame}

\begin{frame}{Extracting the Energy from $\log \Z$}
With intermediate normalization ($\braket{\Phi}{\Psi}=1$) one can show
\[
\Delta E = \left.\frac{d}{d\lambda}\log\Z(\lambda)\right|_{\lambda=1}.
\]
Expanded:
\[
\Delta E = \sum_{n\ge 1}\frac{1}{n!}\, C_n,
\]
where $C_n$ are the $n$-th order \emph{connected} (linked) vacuum diagrams evaluated with the usual Goldstone rules (including symmetry factors, energy denominators).
\end{frame}

\begin{frame}{Sketch of the Proof (1): Wick and Cumulants}
\textbf{Idea:} Write $\Z=\exp W$ with $W=\log\Z$, and interpret $W$ as the \emph{cumulant} generating functional of vacuum contractions.
\begin{itemize}
  \item Introduce sources $J$ linearly coupled to $\V$ or to field bilinears, expand $\Z[J]$.
  \item Cumulants (derivatives of $W[J]$ at $J=0$) pick out \emph{connected} correlation functions only.
  \item Setting $J$-structure to reproduce insertions of $\V$ yields $C_n$ as the $n$-point connected vacuum objects.
\end{itemize}
This establishes $\log \Z = \sum \text{connected}$ non-constructively but generally.
\end{frame}

\begin{frame}{Sketch of the Proof (2): Direct Counting}
Alternatively, count how many times a product of connected pieces appears in $\Z$ vs.\ $\log\Z$:
\[
\Z = \sum_{\{m_k\}}\prod_{k\ge 1}\frac{1}{m_k!}\Big(\frac{C_k}{k!}\Big)^{m_k},
\]
where $m_k$ is the multiplicity of $k$-vertex connected components and $\sum_k k\,m_k=n$ at $n$-th order.
\[
\log\Z = \sum_{k\ge 1} \frac{C_k}{k!},
\]
which follows from the exponential formula in combinatorics (the set-partition theorem). Hence only connected contributions survive in $\log\Z$.
\end{frame}

\begin{frame}{Cancellation of Unlinked Diagrams in Energy}
Energy from RS can also be written as
\[
\Egs = \frac{\braPhi H \ket{\Psi}}{\braket{\Phi}{\Psi}}
= E_0 + \frac{\braPhi \V \ket{\Psi}}{\braket{\Phi}{\Psi}}.
\]
Expanding numerator and denominator in $\lambda$:
\[
\frac{N(\lambda)}{D(\lambda)}=\frac{\sum_n \lambda^n N_n}{1+\sum_{m\ge 1}\lambda^m D_m}
= \sum_{r\ge 0} \lambda^r \Big(N_r - \sum_{m=1}^r D_m N_{r-m} + \cdots\Big),
\]
and one finds precisely that terms factorizing into products of lower-order vacuum pieces cancel against the denominator. The survivors are the \emph{linked} contributions, reproducing $\dv{\lambda}\log\Z$.
\end{frame}

\begin{frame}{Goldstone Rules Refresher (for Energy Diagrams)}
\begin{itemize}
  \item Place $n$ interaction vertices $\V$; connect lines respecting fermionic statistics.
  \item Assign particle/hole propagators; each closed fermion loop $\Rightarrow$ a factor $(-1)$.
  \item Symmetry factor $S$: divide by automorphisms that leave the diagram invariant.
  \item Energy denominator: product over intermediate-state energy differences (or via resolvent method).
  \item Sum over all internal indices (spin, orbitals); overall factor $1/n!$ from perturbative expansion cancels overcountings.
\end{itemize}
\end{frame}

\begin{frame}{Explicit 2nd Order: Linked Only}
For a two-body $\V=\tfrac{1}{4}\sum \bar{v}_{pqrs} a_p^\dagger a_q^\dagger a_s a_r$,
\[
E^{(2)} = \frac{1}{4}\sum_{ijab} \frac{\abs{\bar{v}_{ijab}}^2}{\varepsilon_i+\varepsilon_j-\varepsilon_a-\varepsilon_b},
\]
which corresponds to the single \emph{connected} bubble diagram.\medskip

Any attempt to form products of first-order pieces is null because $E^{(1)}$ for normal-ordered $V$ vanishes in a Hartree–Fock reference (or is absorbed in $\Hn$), illustrating the absence/cancellation of unlinked composites at this order.
\end{frame}

\begin{frame}{Explicit 3rd Order: Topologies and Cancellations}
At third order, connected topologies include (schematically): ring, ladder, and crossed-ladder. Their analytical expressions (one example):
\[
E^{(3)}_{\text{lad}}=\frac{1}{8}\sum_{\substack{ijab\\kc}}
\frac{\bar{v}_{ijab}\,\bar{v}_{bk cj}\,\bar{v}_{ci ka}}
{(\varepsilon_i+\varepsilon_j-\varepsilon_a-\varepsilon_b)\,(\varepsilon_i+\varepsilon_k-\varepsilon_a-\varepsilon_c)}.
\]
Unlinked structures (products of a connected 2nd-order bubble with a disconnected 1st-order tadpole, etc.) cancel once the denominator normalization (or $\log\Z$) is accounted for.
\end{frame}

\begin{frame}{Diagram Placeholders You Can Extend}
\centering
\begin{tikzpicture}[x=1.2cm,y=1.2cm]
% Ladder-like sketch
\draw[thick] (-1,1) -- (1,1);
\draw[thick] (-1,-1) -- (1,-1);
\draw[thick] (-0.6,1) -- (-0.6,-1);
\draw[thick] (0.6,1) -- (0.6,-1);
\node at (0,-1.5){Ladder (connected)};
\end{tikzpicture}
\hspace{1cm}
\begin{tikzpicture}[x=1.2cm,y=1.2cm]
% Unlinked: two bubbles
\draw[thick] (-0.6,0) circle (0.5);
\draw[thick] (0.6,0) circle (0.5);
\node at (0,-1.1){Unlinked (cancels)};
\end{tikzpicture}
\end{frame}

\begin{frame}{Alternative Derivation Route: Bloch Equation}
Let $\Omega$ be the wave operator, $\ket{\Psi}=\Omega\ketPhi$, with the Bloch equation
\[
[\Omega,\Hn]\Pcal = \Qcal(\V\Omega - \Omega \Wop)\Pcal,\qquad
\Wop \equiv \Pcal H \Omega \Pcal.
\]
Expanding $\Omega=\sum_{n\ge 0}\Omega^{(n)}$ and $\Wop$ order by order, one can show algebraically that the $\Wop$ (energy) receives only \emph{connected} contributions, because the disconnected pieces generated in $\Omega$ cancel in $\Wop$ through $\Pcal$-projection and the commutator structure. This is equivalent to the cumulant proof.
\end{frame}

\begin{frame}{Summary of the Derivation}
\begin{itemize}
  \item Write the vacuum functional $\Z$ as a sum over all vacuum diagrams.
  \item Use cluster decomposition $\Rightarrow$ $\Z=\exp\big(\sum \text{connected}\big)$.
  \item Take $\log$ and differentiate: $\Delta E = \eval{\dv{\lambda}\log \Z(\lambda)}_{\lambda=1}$.
  \item Hence $\Delta E$ equals the sum of values of \emph{connected} vacuum diagrams only.
\end{itemize}
This holds to all orders and underpins size-extensivity and additivity for noninteracting fragments.
\end{frame}

\begin{frame}{Practical Notes for Calculations}
\begin{itemize}
  \item Prefer normal-ordered Hamiltonians; $E^{(1)}$ often vanishes for HF reference.
  \item Automate diagram generation: enumerate topologies, compute symmetry factors, energy denominators, and index sums.
  \item Check size-extensivity: only linked diagrams ensure correct scaling with particle number.
  \item Cross-validate: numerical MBPT vs.\ coupled-cluster (which sums linked \emph{connected} diagrams to infinite order via exponentiation of the cluster operator).
\end{itemize}
\end{frame}

\begin{frame}{Exercise}
\textbf{Task:} For a two-body interaction in an HF basis, derive $E^{(3)}_{\text{ring}}$ explicitly:
\begin{enumerate}
  \item Draw the connected ring topology and assign indices.
  \item Write the algebraic expression using antisymmetrized matrix elements $\bar{v}_{pqrs}$.
  \item Derive the energy denominators from intermediate-state energies.
  \item Verify that any product of a 2nd-order connected and a 1st-order tadpole cancels in the normalized expression.
\end{enumerate}
\end{frame}

\begin{frame}{Take-Home Messages}
\begin{itemize}
  \item The linked (connected) nature of contributing diagrams to $\Delta E$ follows from general combinatorics (cumulants).
  \item Unlinked diagrams exponentiate and cancel via normalization $\Rightarrow$ size-extensive energies.
  \item The theorem guides practical many-body methods (MBPT, CC, MBPT-derived effective interactions).
\end{itemize}
\end{frame}
  \end{itemize}
\end{frame}



\end{document}


