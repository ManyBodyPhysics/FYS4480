% Slides for FYS4480


\documentclass[compress]{beamer}


% Try the class options [notes], [notes=only], [trans], [handout],
% [red], [compress], [draft], [class=article] and see what happens!

% For a green structure color use:
%\colorlet{structure}{green!50!black}

\mode<article> % only for the article version
{
  \usepackage{beamerbasearticle}
  \usepackage{fullpage}
  \usepackage{hyperref}
}

\beamertemplateshadingbackground{red!10}{blue!10}

%\beamertemplateshadingbackground{red!10}{blue!10}
\beamertemplatetransparentcovereddynamic
%\usetheme{Hannover}

\setbeamertemplate{footline}[page number]


%\usepackage{beamerthemeshadow}

%\usepackage{beamerthemeshadow}
\usepackage{ucs}


\usepackage{pgf,pgfarrows,pgfnodes,pgfautomata,pgfheaps,pgfshade}
\usepackage{graphicx}
\usepackage{simplewick}
\usepackage{amsmath,amssymb}
\usepackage[latin1]{inputenc}
\usepackage{colortbl}
\usepackage[english]{babel}
\usepackage{listings}
\usepackage{shadow}
\lstset{language=c++}
\lstset{alsolanguage=[90]Fortran}
\lstset{basicstyle=\small}
%\lstset{backgroundcolor=\color{white}}
%\lstset{frame=single}
\lstset{stringstyle=\ttfamily}
%\lstset{keywordstyle=\color{red}\bfseries}
%\lstset{commentstyle=\itshape\color{blue}}
\lstset{showspaces=false}
\lstset{showstringspaces=false}
\lstset{showtabs=false}
\lstset{breaklines}
\usepackage{times}

% Use some nice templates
\beamertemplatetransparentcovereddynamic

% own commands
\newcommand*{\cre}[1]{a^{\dagger}_{#1}}
\newcommand*{\an}[1]{a_{#1}}
\newcommand*{\crequasi}[1]{b^{\dagger}_{#1}}
\newcommand*{\anquasi}[1]{b_{#1}}
\newcommand*{\for}[3]{\langle#1|#2|#3\rangle}
\newcommand{\be}{\begin{equation}}
\newcommand{\ee}{\end{equation}}
\newcommand*{\kpr}[1]{\left\{#1\right\}}
\newcommand*{\ket}[1]{|#1\rangle}
\newcommand*{\bra}[1]{\langle#1|}
%\newcommand{\bra}[1]{\left\langle #1 \right|}
%\newcommand{\ket}[1]{\left| # \right\rangle}
\newcommand{\braket}[2]{\left\langle #1 \right| #2 \right\rangle}
\newcommand{\OP}[1]{{\bf\widehat{#1}}}
\newcommand{\matr}[1]{{\bf \cal{#1}}}
\newcommand{\beN}{\begin{equation*}}
\newcommand{\bea}{\begin{eqnarray}}
\newcommand{\beaN}{\begin{eqnarray*}}
\newcommand{\eeN}{\end{equation*}}
\newcommand{\eea}{\end{eqnarray}}
\newcommand{\eeaN}{\end{eqnarray*}}
\newcommand{\bdm}{\begin{displaymath}}
\newcommand{\edm}{\end{displaymath}}
\newcommand{\bsubeqs}{\begin{subequations}}
\newcommand*{\fpr}[1]{\left[#1\right]}
\newcommand{\esubeqs}{\end{subequations}}
\newcommand*{\pr}[1]{\left(#1\right)}
\newcommand{\element}[3]
        {\bra{#1}#2\ket{#3}}
\newcommand{\md}{\mathrm{d}}
\newcommand{\e}[1]{\times 10^{#1}}
\renewcommand{\vec}[1]{\mathbf{#1}}
\newcommand{\gvec}[1]{\boldsymbol{#1}}
\newcommand{\dr}{\, \mathrm d^3 \vec r}
\newcommand{\dk}{\, \mathrm d^3 \vec k}
\def\psii{\psi_{i}}
\def\psij{\psi_{j}}
\def\psiij{\psi_{ij}}
\def\psisq{\psi^2}
\def\psisqex{\langle \psi^2 \rangle}
\def\psiR{\psi({\bf R})}
\def\psiRk{\psi({\bf R}_k)}
\def\psiiRk{\psi_{i}(\Rveck)}
\def\psijRk{\psi_{j}(\Rveck)}
\def\psiijRk{\psi_{ij}(\Rveck)}
\def\ranglep{\rangle_{\psisq}}
\def\Hpsibypsi{{H \psi \over \psi}}
\def\Hpsiibypsi{{H \psii \over \psi}}
\def\HmEpsibypsi{{(H-E) \psi \over \psi}}
\def\HmEpsiibypsi{{(H-E) \psii \over \psi}}
\def\HmEpsijbypsi{{(H-E) \psij \over \psi}}
\def\psiibypsi{{\psii \over \psi}}
\def\psijbypsi{{\psij \over \psi}}
\def\psiijbypsi{{\psiij \over \psi}}
\def\psiibypsiRk{{\psii(\Rveck) \over \psi(\Rveck)}}
\def\psijbypsiRk{{\psij(\Rveck) \over \psi(\Rveck)}}
\def\psiijbypsiRk{{\psiij(\Rveck) \over \psi(\Rveck)}}
\def\EL{E_{\rm L}}
\def\ELi{E_{{\rm L},i}}
\def\ELj{E_{{\rm L},j}}
\def\ELRk{E_{\rm L}(\Rveck)}
\def\ELiRk{E_{{\rm L},i}(\Rveck)}
\def\ELjRk{E_{{\rm L},j}(\Rveck)}
\def\Ebar{\bar{E}}
\def\Ei{\Ebar_{i}}
\def\Ej{\Ebar_{j}}
\def\Ebar{\bar{E}}
\def\Rvec{{\bf R}}
\def\Rveck{{\bf R}_k}
\def\Rvecl{{\bf R}_l}
\def\NMC{N_{\rm MC}}
\def\sumMC{\sum_{k=1}^{\NMC}}
\def\MC{Monte Carlo}
\def\adiag{a_{\rm diag}}
\def\tcorr{T_{\rm corr}}
\def\intR{{\int {\rm d}^{3N}\!\!R\;}}
\def\ul{\underline}
\def\beq{\begin{eqnarray}}
\def\eeq{\end{eqnarray}}
\newcommand{\eqbrace}[4]{\left\{
\begin{array}{ll}
#1 & #2 \\[0.5cm]
#3 & #4
\end{array}\right.}
\newcommand{\eqbraced}[4]{\left\{
\begin{array}{ll}
#1 & #2 \\[0.5cm]
#3 & #4
\end{array}\right\}}
\newcommand{\eqbracetriple}[6]{\left\{
\begin{array}{ll}
#1 & #2 \\
#3 & #4 \\
#5 & #6
\end{array}\right.}
\newcommand{\eqbracedtriple}[6]{\left\{
\begin{array}{ll}
#1 & #2 \\
#3 & #4 \\
#5 & #6
\end{array}\right\}}
\newcommand{\mybox}[3]{\mbox{\makebox[#1][#2]{$#3$}}}
\newcommand{\myframedbox}[3]{\mbox{\framebox[#1][#2]{$#3$}}}
%% Infinitesimal (and double infinitesimal), useful at end of integrals
%\newcommand{\ud}[1]{\mathrm d#1}
\newcommand{\ud}[1]{d#1}
\newcommand{\udd}[1]{d^2\!#1}
%% Operators, algebraic matrices, algebraic vectors
%% Operator (hat, bold or bold symbol, whichever you like best):
\newcommand{\op}[1]{\widehat{#1}}
%\newcommand{\op}[1]{\mathbf{#1}}
%\newcommand{\op}[1]{\boldsymbol{#1}}
%% Vector:
\renewcommand{\vec}[1]{\boldsymbol{#1}}
%% Matrix symbol:
%\newcommand{\matr}[1]{\boldsymbol{#1}}
%\newcommand{\bb}[1]{\mathbb{#1}}
%% Determinant symbol:
\renewcommand{\det}[1]{|#1|}
%% Means (expectation values) of varius sizes
\newcommand{\mean}[1]{\langle #1 \rangle}
\newcommand{\meanb}[1]{\big\langle #1 \big\rangle}
\newcommand{\meanbb}[1]{\Big\langle #1 \Big\rangle}
\newcommand{\meanbbb}[1]{\bigg\langle #1 \bigg\rangle}
\newcommand{\meanbbbb}[1]{\Bigg\langle #1 \Bigg\rangle}
%% Shorthands for text set in roman font
\newcommand{\prob}[0]{\mathrm{Prob}} %probability
\newcommand{\cov}[0]{\mathrm{Cov}}   %covariance
\newcommand{\var}[0]{\mathrm{Var}}   %variancd
%% Big-O (typically for specifying the speed scaling of an algorithm)
\newcommand{\bigO}{\mathcal{O}}
%% Real value of a complex number
\newcommand{\real}[1]{\mathrm{Re}\!\left\{#1\right\}}
%% Quantum mechanical state vectors and matrix elements (of different sizes)
%\newcommand{\bra}[1]{\langle #1 |}
\newcommand{\brab}[1]{\big\langle #1 \big|}
\newcommand{\brabb}[1]{\Big\langle #1 \Big|}
\newcommand{\brabbb}[1]{\bigg\langle #1 \bigg|}
\newcommand{\brabbbb}[1]{\Bigg\langle #1 \Bigg|}
%\newcommand{\ket}[1]{| #1 \rangle}
\newcommand{\ketb}[1]{\big| #1 \big\rangle}
\newcommand{\ketbb}[1]{\Big| #1 \Big\rangle}
\newcommand{\ketbbb}[1]{\bigg| #1 \bigg\rangle}
\newcommand{\ketbbbb}[1]{\Bigg| #1 \Bigg\rangle}
%\newcommand{\overlap}[2]{\langle #1 | #2 \rangle}
\newcommand{\overlapb}[2]{\big\langle #1 \big| #2 \big\rangle}
\newcommand{\overlapbb}[2]{\Big\langle #1 \Big| #2 \Big\rangle}
\newcommand{\overlapbbb}[2]{\bigg\langle #1 \bigg| #2 \bigg\rangle}
\newcommand{\overlapbbbb}[2]{\Bigg\langle #1 \Bigg| #2 \Bigg\rangle}
\newcommand{\bracket}[3]{\langle #1 | #2 | #3 \rangle}
\newcommand{\bracketb}[3]{\big\langle #1 \big| #2 \big| #3 \big\rangle}
\newcommand{\bracketbb}[3]{\Big\langle #1 \Big| #2 \Big| #3 \Big\rangle}
\newcommand{\bracketbbb}[3]{\bigg\langle #1 \bigg| #2 \bigg| #3 \bigg\rangle}
\newcommand{\bracketbbbb}[3]{\Bigg\langle #1 \Bigg| #2 \Bigg| #3 \Bigg\rangle}
\newcommand{\projection}[2]
{| #1 \rangle \langle  #2 |}
\newcommand{\projectionb}[2]
{\big| #1 \big\rangle \big\langle #2 \big|}
\newcommand{\projectionbb}[2]
{ \Big| #1 \Big\rangle \Big\langle #2 \Big|}
\newcommand{\projectionbbb}[2]
{ \bigg| #1 \bigg\rangle \bigg\langle #2 \bigg|}
\newcommand{\projectionbbbb}[2]
{ \Bigg| #1 \Bigg\rangle \Bigg\langle #2 \Bigg|}
%% If you run out of greek symbols, here's another one you haven't
%% thought of:
\newcommand{\Feta}{\hspace{0.6ex}\begin{turn}{180}
        {\raisebox{-\height}{\parbox[c]{1mm}{F}}}\end{turn}}
\newcommand{\feta}{\hspace{-1.6ex}\begin{turn}{180}
        {\raisebox{-\height}{\parbox[b]{4mm}{f}}}\end{turn}}
\title[FYS4480]{Slides from FYS4480/9480 Lectures}
\author[Quantum mechanics for many-particle systems]{%
  Morten Hjorth-Jensen}
\institute[ORNL, University of Oslo and MSU]{
  Department of Physics and Center for Computing in Science Education\\
  University of Oslo, N-0316 Oslo, Norway and\\
  Department of Physics and Astronomy and Facility for Rare Isotope Beams, Michigan State University, East Lansing, MI 48824, USA }
  
\date[UiO]{Fall  2022}
\subject{FYS4480/9480 Quantum mechanics for  many-particle systems}
\begin{document}
\newcommand{\braket}[2]
    {\langle #1 | #2 \rangle}
\newcommand{\ket}[1]
    {|#1\rangle}
\newcommand{\bra}[1]
    {\langle #1 |}
\newcommand{\element}[3]
    {\bra{#1}#2\ket{#3}}
\newcommand{\vbraket}[2]
    {\langle \mathbf{#1} | \mathbf{#2} \rangle}
\newcommand{\vket}[1]
    {|\mathbf{#1}\rangle}
\newcommand{\vbra}[1]
    {\langle \mathbf{#1} |}
\newcommand{\velement}[3]
    {\vbra{#1}\mathbf{#2}\vket{#3}}
\newcommand{\ud}{\mathrm{d}}
\newcommand{\nn}{\notag \\}

\newcommand{\vect}[1]
    {\mathbf{#1}}
\newcommand{\op}[1]
    {\hat{\mathrm{#1}}}

\newcommand{\SE}{Schr\"{o}dinger equation }
\newcommand{\SED}{Schr\"{o}dinger equation. }
\newcommand{\barh}{\bar{\mathrm{H}}}

\newcommand{\normord}[1]{
    \left\{#1\right\}
}
\newcommand{\BCH}{Baker-Campbell-Hausdorff formula}

% Box for sketching algorithms
\newsavebox{\fmbox}
\newenvironment{fmpage}[1]
    {\begin{lrbox}{\fmbox}\begin{minipage}{#1}}
    {\end{minipage}\end{lrbox}\fbox{\usebox{\fmbox}}}

\numberwithin{equation}{subsection}
\numberwithin{figure}{section}
\renewcommand{\theequation}{\arabic{section}.\arabic{subsection}.\arabic{equation}}
\renewcommand{\thefigure}{\arabic{section}.\arabic{figure}}
\renewcommand{\thempfootnote}{\arabic{mpfootnote}}
%\renewcommand{\thesubfigure}{\alph{subfigure}}
\newcommand{\sphelaplop}{
    \frac{1}{r^2} \frac{\partial}{\partial r}
    \left(r^2 \frac{\partial}{\partial r} \right) + \frac{1}{r^2 \sin{\theta}}
    \frac{\partial}{\partial \theta} \left( \sin{\theta} \frac{\partial}
    {\partial \theta} \right) + \frac{1}{r^2 \sin^2\theta} \left(
    \frac{\partial^2}{\partial \phi^2}\right)}

\newcommand{\sphelapl}[1]{
    \frac{1}{r^2} \frac{\partial}{\partial r}
    \left(r^2 \frac{\partial #1}{\partial r} \right) + \frac{1}{r^2 \sin{\theta}}
    \frac{\partial}{\partial \theta} \left( \sin{\theta} \frac{\partial #1}
    {\partial \theta} \right) + \frac{1}{r^2 \sin^2\theta} \left(
    \frac{\partial^2 #1}{\partial \phi^2}\right)}


%\pagenumbering{plain}
\frame{\titlepage}


\frame
{
\frametitle{Adiabatic hypothesis}
\begin{small}
{\scriptsize
We assume that the interaction term is switched on gradually. Our wave function at time $t=-\infty$ and $t=\infty$ is supposed to represent a non-interacting system
given by the solution to the unperturbed part of our Hamiltonian $\hat{H}_0$.
We assume the ground state is given by $|\Phi_0\rangle$, which could be a Slater determinant.
We define our Hamiltonian as
\[
\hat{H}=\hat{H}_0+\exp{(-\varepsilon t/\hbar)}\hat{H}_I,
\]
where $\varepsilon$ is a small number. The way we write the Hamiltonian 
and its interaction term is meant to simulate the switching of the interaction.
}
\end{small}
}
\frame
{
\frametitle{Adiabatic hypothesis}
\begin{small}
{\scriptsize
The time evolution of the wave function in the interaction picture is then
\[
|\Psi_I(t) \rangle = \hat{U}_{\varepsilon}(t,t_0)|\Psi_I(t_0)\rangle,
\]
with 
\[
\hat{U}_{\varepsilon}(t,t_0)=\sum_{n=0}^{\infty}\left(\frac{-\imath}{\hbar}\right)^n\frac{1}{n!}
\int_{t_0}^t dt_1\dots \int_{t_0}^t dt_N \exp{(-\varepsilon(t_1+\dots+t_n)/\hbar)}\hat{T}\left[\hat{H}_I(t_1)\dots\hat{H}_I(t_n)\right]
\]
}
\end{small}
}
\frame
{
\frametitle{Adiabatic hypothesis}
\begin{small}
{\scriptsize
In the limit $t_0\rightarrow -\infty$, the solution ot Schr\"odinger's equation is
$|\Phi_0\rangle$, and the eigenenergies are given by 
\[
\hat{H}_0|\Phi_0\rangle=W_0|\Phi_0\rangle,
\]
meaning that 
\[
|\Psi_S(t_0)\rangle = \exp{(-\imath W_0t_0/\hbar)}|\Phi_0\rangle,
\]
with the corresponding interaction picture wave function given by
\[
|\Psi_I(t_0)\rangle = \exp{(\imath \hat{H}_0t_0/\hbar)}|\Psi_S(t_0)\rangle=|\Phi_0\rangle.
\]
}
\end{small}
}
\frame
{
\frametitle{Adiabatic hypothesis}
\begin{small}
{\scriptsize
The solution becomes time independent in the limit $t_0\rightarrow -\infty$.
The same conclusion can be reached by looking at 
\[
\imath \hbar\frac{\partial }{\partial t}|\Psi_I(t)\rangle =
\exp{(\varepsilon |t|/\hbar)}\hat{H}_I|\Psi_I(t)\rangle 
\]
and taking the limit $t\rightarrow -\infty$.
We can rewrite the equation for the wave function at a time $t=0$ as
\[
|\Psi_I(0) \rangle = \hat{U}_{\varepsilon}(0,-\infty)|\Phi_0\rangle.
\]
}
\end{small}
}
\frame
{
\frametitle{Goldstone's theorem and Gell-Mann and Low theorem on the ground state}
\begin{small}
{\scriptsize
Our wave function for ground state (after Gell-Mann and Low, see Phys.~Rev.~{\bf 84}, 350 (1951)) is then
\[
        \frac{\ket{\Psi_0}}{\left\langle\Phi_0 | \Psi_0 \right\rangle}=
    \lim_{\epsilon \rightarrow 0}
   \lim_{t'\rightarrow -\infty}
   \frac{U(0,-\infty )\ket{\Phi_0} }
   { \bra{\Phi_0} U(0,-\infty )\ket{\Phi_0} },
\]
and we ask whether this quantity exists to all orders in perturbation theory.
Goldstone's theorem states that only linked diagrams enter the expression for the final binding energy. It means that energy difference reads now
\[
\Delta E=\sum_{i=0}^{\infty}\langle \Phi_0|\hat{H}_I\left\{\frac{\hat{Q}}{W_0-\hat{H}_0}\hat{H}_I\right\}^i|\Phi_0\rangle_L,
\]
where the subscript $L$ indicates that only linked diagrams are included. In our Rayleigh-Schr\"odinger expansion, the energy difference included also unlinked diagrams. 
}
\end{small}
}
\frame
{
\frametitle{Goldstone's theorem and Gell-Mann and Low theorem on the ground state}
\begin{small}
{\scriptsize
If it does, Gell-Mann and Low showed that it is an eigenstate of $\hat{H}$ with eigenvalue
\[
 \hat{H}\frac{\ket{\Psi_0}}{\left\langle\Phi_0 | \Psi_0 \right\rangle}= E\frac{\ket{\Psi_0}}{\left\langle\Phi_0 | \Psi_0 \right\rangle}
\]
and multiplying from the left with $\langle \Phi_0|$ we can rewrite the last equation
as
\[
E-W_0=\frac{\langle \Phi_0|\hat{H}_I\ket{\Psi_0}}{\left\langle\Phi_0 | \Psi_0 \right\rangle},
\]
since $\hat{H}_0|\Phi_0\rangle = W_0|\Phi_0\rangle$. The numerator and the denominators of the last equation do not exist separately. The theorem of Gell-Mann and Low asserts that this ratio exists. 
}
\end{small}
}
\frame
{
\frametitle{Goldstone's theorem and Gell-Mann and Low theorem on the ground state}
\begin{small}
{\scriptsize
We wish to link the above expression with the corresponding expression from time-dependent perturbation theory. We write our expression as
\[
E-W_0=\Delta E= \lim_{\epsilon \rightarrow 0^+}
   \frac{\bra{\Phi_0(0)}\hat{H}_IU_{\epsilon }(0,-\infty )\ket{\Phi_0(-\infty)} }
   { \bra{\Phi_0(0)} U_{\epsilon}(0,-\infty )\ket{\Phi_0(-\infty)} },
\]
with a numerator 
\[
N=\lim_{\epsilon \rightarrow 0^+}\bra{\Phi_0(0)}\hat{H}_IU_{\epsilon}(0,-\infty )\ket{\Phi_0(-\infty)}, 
\]
which we rewrite as
\[
 N=\lim_{\epsilon \rightarrow 0^+}\bra{\Phi_0(0)}\hat{H}_I(t=0)\displaystyle\sum_{n=0}^{\infty}\frac{(-i)^n}{n!}
   \int_{-\infty}^{0}dt_1  \int_{-\infty}^{0}dt_2\dots  \int_{-\infty}^{0}dt_n}\exp{(\epsilon/\hbar(t_1+\dots+t_n))} \hat{T}\left[H_1(t_1)H_1(t_2)\dots H_1(t_n)\right] \ket{\Phi_0(-\infty)}. 
\]
}
\end{small}
}
\frame
{
\frametitle{Goldstone's theorem and Gell-Mann and Low theorem on the ground state}
\begin{small}
{\scriptsize
A linked diagram (or connected diagram) is a diagram which is linked to the last interaction vertex at $t=0$.
We divide the diagrams into linked and unlinked. 
In general, the way we can distribute $\mu$ unlinked diagrams among the total of $n$ diagrams is given by the combinatorial factor
\[
\left(\begin{array}{c} n \\ \mu \end{array}\right) = \frac{n!}{\mu!\nu!},
\]
and using the following relation
\[
\sum_{n=0}^{\infty}\frac{1}{n!}\sum_{\mu+\nu=n}^{\infty}\frac{n!}{\mu!\nu!}=\sum_{\mu=0}^{\infty}\frac{1}{\mu!}\sum_{\nu}^{\infty}\frac{1}{\nu!},
\]
we can rewrite the numerator $N$ as 
\[
N=\bra{\Phi_0(0)}\hat{H}_IU_{\epsilon}(0,-\infty )\ket{\Phi_0(-\infty)}_L\bra{\Phi_0(0)}U_{\epsilon}(0,-\infty )\ket{\Phi_0(-\infty)},  
=N_LD,
\]
with $N_L$ now containing only linked terms
\[
 N_L=\lim_{\epsilon \rightarrow 0^+}\bra{\Phi_0(0)}\hat{H}_I(t=0)\displaystyle\sum_{\nu=0}^{\infty}\frac{(-i)^{\nu}}{\nu!}
   \int_{-\infty}^{0}dt_1  \int_{-\infty}^{0}dt_2\dots  \int_{-\infty}^{0}dt_n}\exp{(\epsilon/\hbar(t_1+\dots+t_n))} \hat{T}\left[H_1(t_1)H_1(t_2)\dots H_1(t_n)\right] \ket{\Phi_0(-\infty)}_L, 
\]
with the subscript $L$ indicating that only linked diagrams appear, that is those diagrams which are linked to the last interaction vertex.
}
\end{small}
}
\frame
{
\frametitle{Goldstone's theorem and Gell-Mann and Low theorem on the ground state}
\begin{small}
{\scriptsize
A linked diagram (or connected diagram) is a diagram which is linked to the last interaction vertex at $t=0$.
We divide the diagrams into linked and unlinked. 
In general, the way we can distribute $\mu$ unlinked diagrams among the total of $n$ diagrams is given by the combinatorial factor
\[
\left(\begin{array}{c} n \\ \mu \end{array}\right) = \frac{n!}{\mu!\nu!},
\]
and using the following relation
\[
\sum_{n=0}^{\infty}\frac{1}{n!}\sum_{\mu+\nu=n}^{\infty}\frac{n!}{\mu!\nu!}=\sum_{\mu=0}^{\infty}\frac{1}{\mu!}\sum_{\nu}^{\infty}\frac{1}{\nu!},
\]
we can rewrite the numerator $N$ as 
\[
N=\bra{\Phi_0(0)}\hat{H}_IU_{\epsilon}(0,-\infty )\ket{\Phi_0(-\infty)}_L\bra{\Phi_0(0)}U_{\epsilon}(0,-\infty )\ket{\Phi_0(-\infty)},  
=N_LD,
\]
with $N_L$ now containing only linked terms
\[
 N_L=\lim_{\epsilon \rightarrow 0^+}\bra{\Phi_0(0)}\hat{H}_I(t=0)\displaystyle\sum_{\nu=0}^{\infty}\frac{(-i)^{\nu}}{\nu!}
   \int_{-\infty}^{0}dt_1  \int_{-\infty}^{0}dt_2\dots  \int_{-\infty}^{0}dt_n}\exp{(\epsilon/\hbar(t_1+\dots+t_n))} \hat{T}\left[H_1(t_1)H_1(t_2)\dots H_1(t_n)\right] \ket{\Phi_0(-\infty)}_L, 
\]
with the subscript $L$ indicating that only linked diagrams appear, that is those diagrams which are linked to the last interaction vertex.
}
\end{small}
}
\frame
{
\frametitle{Goldstone's theorem and Gell-Mann and Low theorem on the ground state}
\begin{small}
{\scriptsize
We note that also that the term $D$ is nothing but the denominator of the equation for the energy. We obtain then the following expression for the energy
\[
E-W_0=\Delta E=N_L= \bra{\Phi_0(0)}\hat{H}_IU_{\epsilon}(0,-\infty )\ket{\Phi_0(-\infty)}_L,
\]
and Goldstone's theorem is then proved. 
The corresponding expression from Rayleigh-Schr\"odinger perturbation theory is given by
\[
\Delta E=\langle \Phi_0|\left(\hat{H}_I+\hat{H}_I\frac{\hat{Q}}{W_0-\hat{H}_0}\hat{H}_I+
\hat{H}_I\frac{\hat{Q}}{W_0-\hat{H}_0}\hat{H}_I\frac{\hat{Q}}{W_0-\hat{H}_0}\hat{H}_I+\dots\right)|\Phi_0\rangle_C.
\]
}
\end{small}
}
\frame
{
\frametitle{Goldstone's theorem and Gell-Mann and Low theorem on the ground state}
\begin{small}
{\scriptsize
An important point in the derivation of the Gell-Mann and Low theorem
\[
E-W_0=\frac{\langle \Phi_0|\hat{H}_I\ket{\Psi_0}}{\left\langle\Phi_0 | \Psi_0 \right\rangle},
\]
is that the numerator and the denominators of the last equation do not exist separately. The theorem of Gell-Mann and Low asserts that this ratio exists. To prove it we proceed as follows. Consider the expression
\[
(\hat{H}_0-E)U_{\epsilon }(0,-\infty )\ket{\Phi_0}=\left[\hat{H}_0,U_{\epsilon }(0,-\infty )\right]\ket{\Phi_0}.
\]
}
\end{small}
}
\frame
{
\frametitle{Goldstone's theorem and Gell-Mann and Low theorem on the ground state}
\begin{small}
{\scriptsize
To evaluate the commutator 
\[
(\hat{H}_0-E)U_{\epsilon }(0,-\infty )\ket{\Phi_0}=\left[\hat{H}_0,U_{\epsilon }(0,-\infty )\right]\ket{\Phi_0}.
\]
we write the associate commutator as
\[
\left[\hat{H}_0,\hat{H}_I(t_1)\hat{H}_I(t_2)\dots \hat{H}_I(t_n)\right]=
\left[\hat{H}_0,\hat{H}_I(t_1)\right]\hat{H}_I(t_2)\dots \hat{H}_I(t_n)+
\]
\[
\dots+\hat{H}_I(t_1)\left[\hat{H}_0,\hat{H}_I(t_2)\right]\hat{H}_I(t_3)\dots \hat{H}_I(t_n)+\dots
\]
Using the equation of motion for an operator in the interaction picture we have
\[
\imath \hbar\frac{\partial }{\partial t}\hat{H}_I(t) = \left[\hat{H}_I(t),\hat{H}_0\right].
\]
Each of the above commutators yield then a time derivative!
}
\end{small}
}
\frame
{
\frametitle{Goldstone's theorem and Gell-Mann and Low theorem on the ground state}
\begin{small}
{\scriptsize
We have then
\[
\left[\hat{H}_0,\hat{H}_I(t_1)\hat{H}_I(t_2)\dots \hat{H}_I(t_n)\right]=\imath \hbar\left(\frac{\partial }{\partial t_n}+\frac{\partial }{\partial t_1}+\dots+\frac{\partial }{\partial t_n}\right) \hat{H}_I(t_1)\hat{H}_I(t_2)\dots\hat{H}_I(t_n),
\]
meaning that we can rewrite
\[
(\hat{H}_0-E)U_{\epsilon }(0,-\infty )\ket{\Phi_0}=\left[\hat{H}_0,U_{\epsilon }(0,-\infty )\right]\ket{\Phi_0},
\]
as
\[
(\hat{H}_0-E)U_{\epsilon }(0,-\infty )\ket{\Phi_0}=-\sum_{n=1}^{\infty}\left(\frac{-\imath}{\hbar}\right)^{n-1}\frac{1}{n!}
\int_{t_0}^t dt_1\dots \int_{t_0}^t dt_N \exp{(-\varepsilon(t_1+\dots+t_n)/\hbar)}
\]
\[
\times\sum_{i=1}^n(\frac{\partial }{\partial t_i} )\hat{T}\left[\hat{H}_I(t_1)\dots\hat{H}_I(t_n)\right].
\]
}
\end{small}
}
\frame
{
\frametitle{Goldstone's theorem and Gell-Mann and Low theorem on the ground state}
\begin{small}
{\scriptsize
All the time derivatives in this equation 
\[
(\hat{H}_0-E)U_{\epsilon }(0,-\infty )\ket{\Phi_0}=-\sum_{n=1}^{\infty}\left(\frac{-\imath}{\hbar}\right)^{n-1}\frac{1}{n!}
\int_{t_0}^t dt_1\dots \int_{t_0}^t dt_N \exp{(-\varepsilon(t_1+\dots+t_n)/\hbar)}
\]
\[
\times\sum_{i=1}^n(\frac{\partial }{\partial t_i} )\hat{T}\left[\hat{H}_I(t_1)\dots\hat{H}_I(t_n)\right],
\]
make the same contribution, as can be seen by changing dummy variables. We can therefore retain just one time derivative $\partial/\partial t$ and multiply with $n$. Integrating by parts wrt $t_1$  we obtain two terms. 
}
\end{small}
}
\frame
{
\frametitle{Goldstone's theorem and Gell-Mann and Low theorem on the ground state}
\begin{small}
{\scriptsize
Integrating by parts wrt $t_1$  one can finally show that
\[
        \frac{\ket{\Psi_0}}{\left\langle\Phi_0 | \Psi_0 \right\rangle}=
    \lim_{\epsilon \rightarrow 0}
   \lim_{t'\rightarrow -\infty}
   \frac{U(0,-\infty )\ket{\Phi_0} }
   { \bra{\Phi_0} U(0,-\infty )\ket{\Phi_0} },
\]
For more details about the derivation, see Gell-Mann and Low, Phys.~Rev.~{\bf 84}, 350  (1951). See also chapter 6.2 of Raimes or Fetter and Walecka, chapter 3.6.
}
\end{small}
}
\frame
{
\frametitle{Goldstone's theorem and Gell-Mann and Low theorem on the ground state}
\begin{small}
{\scriptsize
In the present discussion of the time-dependent theory we will make
use of the so-called complex-time approach to describe the time
evolution operator $U$.
This means that we
allow the time $t$ to be rotated by a small angle $\epsilon$
relative to the real time axis. The complex time $t$ is then
related to the real time $\tilde{t}$ by
\[
t=\tilde{t}(1-i\epsilon ).
\]
Let us first study the true eigenvector $\Psi_{\alpha}$ which evolves
from the unperturbed eigenvectors $\Phi_{\alpha}$ through the action of the
time development operator
\[
   U_{\varepsilon}(t,t')=\lim_{\epsilon \rightarrow 0}
   \lim_{t'\rightarrow -\infty}
   {\displaystyle\sum_{n=0}^{\infty}\frac{(-i)^n}{n!}
   \int_{t'}^{t}dt_1  \int_{t'}^{t}dt_2\dots  \int_{t'}^{t}dt_n}
\]
\[
	      \times T\left[H_1(t_1)H_1(t_2)\dots H_1(t_n)\right],
\]
where $T$ stands for the correct time-ordering.
}
\end{small}
}
\frame
{
\frametitle{Goldstone's theorem and Gell-Mann and Low theorem on the ground state}
\begin{small}
{\scriptsize
In time-dependent
perturbation theory we let $\Psi_{\alpha}$ develop from $\Phi_{\alpha}$ in the
remote past to a given time $t$
\[
    \frac{\ket{\Psi_{\alpha}}}
    {\left\langle\psi_{\alpha} | \Psi_{\alpha} \right\rangle}=
    \lim_{\epsilon \rightarrow 0}
   \lim_{t'\rightarrow -\infty}
   \frac{U_{\varepsilon}(t,t' )\ket{\psi_{\alpha}} }
   { \bra{\psi_{\alpha}} U(t,t' )\ket{\Phi_{\alpha}} },
\]
and similarly, we let
$\Psi_{\beta}$ develop from $\Phi_{\beta}$ in the remote future
\[
    \frac{\bra{\Psi_{\beta}}}{\left\langle
    \psi_{\beta} | \Psi_{\beta} \right\rangle}=
    \lim_{\epsilon \rightarrow 0}
    \lim_{t'\rightarrow \infty}
    \frac{\bra{\psi_{\beta}}U_{\varepsilon}(t' ,t) }
    { \bra{\psi_{\beta}} U_{\varepsilon}(t' ,t)\ket{\Phi_{\beta}} }.
\]
}
\end{small}
}
\frame
{
\frametitle{Goldstone's theorem and Gell-Mann and Low theorem on the ground state}
\begin{small}
{\scriptsize
Here we are interested in the expectation value of a given
operator ${\cal O}$ acting at a time $t=0$. This can be achieved
from the two previous equations defining
\[
     \ket{\Psi_{\alpha ,\beta}'}=
     \frac{\ket{\Psi_{\alpha ,\beta}}}
     {\left\langle\Phi_{\alpha ,\beta} | \Psi_{\alpha ,\beta} \right\rangle}
\]
we have
\[
   {\cal O}_{\alpha\beta}
  =\frac{N_{\beta\alpha}}{D_{\beta}D_{\alpha}},
\]
where we have introduced
\[
   N_{\beta\alpha}=
   \bra{\Phi_{\beta}}U_{\varepsilon}(\infty ,0){\cal O}U_{\varepsilon}(0,-\infty )\ket{\Phi_{\alpha}} ,
\]
and 
\[
   D_{\alpha ,\beta}=
   \sqrt{\bra{\psi_{\alpha ,\beta}}
   U_{\varepsilon}(\infty ,0)U_{\varepsilon}(0,-\infty )\ket{\Phi_{\alpha ,\beta}}}. 
\]
}
\end{small}
}
\frame
{
\frametitle{Goldstone's theorem and Gell-Mann and Low theorem on the ground state}
\begin{small}
{\scriptsize
If the operator ${\cal O}$ stands for the hamiltonian $H$ we obtain
\[
    {\displaystyle  \frac{\bra{\Psi_{\lambda}'}H\ket{\Psi_{\lambda}'} }
   { \left\langle\Psi_{\lambda}' | \Psi_{\lambda}' \right\rangle} }
\]
At this stage, {\em it is important to observe} that our 
expression for the expectation value of a given operator ${\cal O}$
{\em is hermitian} insofar ${\cal O}^{\dagger}={\cal O}$. This is readily 
demonstrated. The above equation is of the general form
\[
U(t,t_0){\cal O}U(t_0,-t),
\]
and noting that 
\[
   U^{\dagger}(t,t_0)=
   \left({\displaystyle e^{iH_0t}e^{-iH(t-t_0)}e^{-iH_0t}}\right)^{\dagger}
   =U(t_0,-t),
\]
since $H^{\dagger}=H$ and $H_0^{\dagger}=H_0$, we have that
\[
    \left(U(t,t_0){\cal O}U(t_0,-t)\right)^{\dagger}
    =U(t,t_0){\cal O}U(t_0,-t).
\]
The question we pose now is what happens in the limit $\varepsilon\rightarrow 0$?
Do we get results which are meaningful?
}
\end{small}
}
\frame
{
\frametitle{Goldstone's theorem and Gell-Mann and Low theorem on the ground state}
\begin{small}
{\scriptsize
Our wave function for ground state is then
\[
        \frac{\ket{\Psi_0}}{\left\langle\Phi_0 | \Psi_0 \right\rangle}=
    \lim_{\epsilon \rightarrow 0}
   \lim_{t'\rightarrow -\infty}
   \frac{U(0,-\infty )\ket{\Phi_0} }
   { \bra{\Phi_0} U(0,-\infty )\ket{\Phi_0} },
\]
meaning that the energy difference is given by
\[
E_0-W_0=\Delta E_0= \lim_{\epsilon \rightarrow 0}
   \lim_{t'\rightarrow -\infty}
   \frac{\bra{\Phi_0}\hat{H}_IU_{\varepsilon}(0,-\infty )\ket{\Phi_0} }
   { \bra{\Phi_0} U_{\varepsilon}(0,-\infty )\ket{\Phi_0} },
\]
and we ask whether this quantity exists to all orders in perturbation theory.
}
\end{small}
}
\frame
{
\frametitle{Goldstone's theorem and Gell-Mann and Low theorem on the ground state}
\begin{small}
{\scriptsize
If it does, Gell-Mann and Low showed that it is an eigenstate of $\hat{H}$ with eigenvalue
\[
 \hat{H}\frac{\ket{\Psi_0}}{\left\langle\Phi_0 | \Psi_0 \right\rangle}= E_0\frac{\ket{\Psi_0}}{\left\langle\Phi_0 | \Psi_0 \right\rangle}
\]
and multiplying from the left with $\langle \Phi_0|$ we can rewrite the last equation
as
\[
E_0-W_0=\frac{\langle \Phi_0|\hat{H}_I\ket{\Psi_0}}{\left\langle\Phi_0 | \Psi_0 \right\rangle},
\]
since $\hat{H}_0|\Phi_0\rangle = W_0|\Phi_0\rangle$. The numerator and the denominators of the last equation do not exist separately. The theorem of Gell-Mann and Low asserts that this ratio exists. 
}
\end{small}
}
\frame
{
\frametitle{Goldstone's theorem and Gell-Mann and Low theorem on the ground state}
\begin{small}
{\scriptsize
Goldstone's theorem states that only linked diagrams enter the expression for the final binding energy. It means that energy difference reads now
\[
\Delta E_0=\sum_{i=0}^{\infty}\langle \Phi_0|\hat{H}_I\left\{\frac{\hat{Q}}{W_0-\hat{H}_0}\hat{H}_I\right\}^i|\Phi_0\rangle_L,
\]
where the subscript $L$ indicates that only linked diagrams are included. In our Rayleigh-Schr\"odinger expansion, the energy difference included also unlinked diagrams. 
}
\end{small}
}
\frame
{
\frametitle{Goldstone's theorem and Gell-Mann and Low theorem on the ground state}
\begin{small}
{\scriptsize
We wish to link the above expression with the corresponding expression from time-dependent perturbation theory. We write our expression as
\[
E_0-W_0=\Delta E_0= \lim_{\epsilon \rightarrow 0^+}
   \frac{\bra{\Phi_0(0)}\hat{H}_IU_{\epsilon }(0,-\infty )\ket{\Phi_0(-\infty)} }
   { \bra{\Phi_0(0)} U_{\epsilon}(0,-\infty )\ket{\Phi_0(-\infty)} },
\]
with a numerator 
\[
N=\lim_{\epsilon \rightarrow 0^+}\bra{\Phi_0(0)}\hat{H}_IU_{\epsilon}(0,-\infty )\ket{\Phi_0(-\infty)}, 
\]
which we rewrite as
\[
 N=\lim_{\epsilon \rightarrow 0^+}\bra{\Phi_0(0)}\hat{H}_I(t=0)\displaystyle\sum_{n=0}^{\infty}\frac{(-i)^n}{n!}
   \int_{-\infty}^{0}dt_1  \int_{-\infty}^{0}dt_2\dots  \int_{-\infty}^{0}dt_n}\exp{(\epsilon/\hbar(t_1+\dots+t_n))} \hat{T}\left[H_1(t_1)H_1(t_2)\dots H_1(t_n)\right] \ket{\Phi_0(-\infty)}. 
\]
}
\end{small}
}
\frame
{
\frametitle{Goldstone's theorem and Gell-Mann and Low theorem on the ground state}
\begin{small}
{\scriptsize
From this term we can obtain both linked and unlinked contributions. Goldstone's theorem states that only linked diagrams enter the expression for the final binding energy. 
A linked diagram (or connected diagram) is a diagram which is linked to the last interaction vertex at $t=0$.
We label the number of linked diagrams with the variable $\nu$ and the number of unlinked with $\mu$  with $n=\nu+\mu$.  The number of unlinked diagrams is then $\mu=n-\nu$. 
}
\end{small}
}
\frame
{
\frametitle{Goldstone's theorem and Gell-Mann and Low theorem on the ground state}
\begin{small}
{\scriptsize
In general, the way we can distribute $\mu$ unlinked diagrams among the total of $n$ diagrams is given by the combinatorial factor
\[
\left(\begin{array}{c} n \\ \mu \end{array}\right) = \frac{n!}{\mu!\nu!},
\]
and using the following relation
\[
\sum_{n=0}^{\infty}\frac{1}{n!}\sum_{\mu+\nu=n}^{\infty}\frac{n!}{\mu!\nu!}=\sum_{\mu=0}^{\infty}\frac{1}{\mu!}\sum_{\nu}^{\infty}\frac{1}{\nu!},
\]
we can rewrite the numerator $N$ as 
\[
N=\bra{\Phi_0(0)}\hat{H}_IU_{\epsilon}(0,-\infty )\ket{\Phi_0(-\infty)}_L\bra{\Phi_0(0)}U_{\epsilon}(0,-\infty )\ket{\Phi_0(-\infty)}=N_LD.
\]
}
\end{small}
}
\frame
{
\frametitle{Goldstone's theorem and Gell-Mann and Low theorem on the ground state}
\begin{small}
{\scriptsize
We define  $N_L$ to contain only linked terms
%\[
% N_L=\lim_{\epsilon \rightarrow 0^+}\bra{\Phi_0(0)}\hat{H}_I(t=0)\sum_{\nu=0}^{\infty}\frac{(-i)^{\nu}}{\nu!}\int_{-\infty}^{0}dt_1  \int_{-\infty}^{0}dt_2\dots  \int_{-\infty}^{0}dt_n}\exp{(\epsilon/\hbar(t_1+\dots+t_n))} \hat{T}\left[H_1(t_1)H_1(t_2)\dots H_1(t_n)\right] \ket{\Phi_0(-\infty)}_L, 
%\]
with the subscript $L$ indicating that only linked diagrams appear, that is those diagrams which are linked to the last interaction vertex.
}
\end{small}
}
\frame
{
\frametitle{Goldstone's theorem and Gell-Mann and Low theorem on the ground state}
\begin{small}
{\scriptsize
We note that also that the term $D$ is nothing but the denominator of the equation for the energy. We obtain then the following expression for the energy
\[
E_0-W_0=\Delta E_0=N_L= \bra{\Phi_0(0)}\hat{H}_IU_{\epsilon}(0,-\infty )\ket{\Phi_0(-\infty)}_L,
\]
and Goldstone's theorem is then proved. 
The corresponding expression from Rayleigh-Schr\"odinger perturbation theory is given by
\[
\Delta E_0=\langle \Phi_0|\left(\hat{H}_I+\hat{H}_I\frac{\hat{Q}}{W_0-\hat{H}_0}\hat{H}_I+
\hat{H}_I\frac{\hat{Q}}{W_0-\hat{H}_0}\hat{H}_I\frac{\hat{Q}}{W_0-\hat{H}_0}\hat{H}_I+\dots\right)|\Phi_0\rangle_C.
\]
}
\end{small}
}
\end{document}
\end{document}
