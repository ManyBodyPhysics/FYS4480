\documentclass[compress]{beamer}


% Try the class options [notes], [notes=only], [trans], [handout],
% [red], [compress], [draft], [class=article] and see what happens!

% For a green structure color use:
%\colorlet{structure}{green!50!black}

\mode<article> % only for the article version
{
  \usepackage{beamerbasearticle}
  \usepackage{fullpage}
  \usepackage{hyperref}
}

\beamertemplateshadingbackground{red!10}{blue!10}

%\beamertemplateshadingbackground{red!10}{blue!10}
\beamertemplatetransparentcovereddynamic
%\usetheme{Hannover}

\setbeamertemplate{footline}[page number]


%\usepackage{beamerthemeshadow}

%\usepackage{beamerthemeshadow}
\usepackage{ucs}


\usepackage{pgf,pgfarrows,pgfnodes,pgfautomata,pgfheaps,pgfshade}
\usepackage{graphicx}
\usepackage{simplewick}
\usepackage{amsmath,amssymb}
\usepackage[latin1]{inputenc}
\usepackage{colortbl}
\usepackage[english]{babel}
\usepackage{listings}
\usepackage{shadow}
\lstset{language=c++}
\lstset{alsolanguage=[90]Fortran}
\lstset{basicstyle=\small}
%\lstset{backgroundcolor=\color{white}}
%\lstset{frame=single}
\lstset{stringstyle=\ttfamily}
%\lstset{keywordstyle=\color{red}\bfseries}
%\lstset{commentstyle=\itshape\color{blue}}
\lstset{showspaces=false}
\lstset{showstringspaces=false}
\lstset{showtabs=false}
\lstset{breaklines}
\usepackage{times}

% Use some nice templates
\beamertemplatetransparentcovereddynamic

% own commands
\newcommand*{\cre}[1]{a^{\dagger}_{#1}}
\newcommand*{\an}[1]{a_{#1}}
\newcommand*{\crequasi}[1]{b^{\dagger}_{#1}}
\newcommand*{\anquasi}[1]{b_{#1}}
\newcommand*{\for}[3]{\langle#1|#2|#3\rangle}
\newcommand{\be}{\begin{equation}}
\newcommand{\ee}{\end{equation}}
\newcommand*{\kpr}[1]{\left\{#1\right\}}
\newcommand*{\ket}[1]{|#1\rangle}
\newcommand*{\bra}[1]{\langle#1|}
%\newcommand{\bra}[1]{\left\langle #1 \right|}
%\newcommand{\ket}[1]{\left| # \right\rangle}
\newcommand{\braket}[2]{\left\langle #1 \right| #2 \right\rangle}
\newcommand{\OP}[1]{{\bf\widehat{#1}}}
\newcommand{\matr}[1]{{\bf \cal{#1}}}
\newcommand{\beN}{\begin{equation*}}
\newcommand{\bea}{\begin{eqnarray}}
\newcommand{\beaN}{\begin{eqnarray*}}
\newcommand{\eeN}{\end{equation*}}
\newcommand{\eea}{\end{eqnarray}}
\newcommand{\eeaN}{\end{eqnarray*}}
\newcommand{\bdm}{\begin{displaymath}}
\newcommand{\edm}{\end{displaymath}}
\newcommand{\bsubeqs}{\begin{subequations}}
\newcommand*{\fpr}[1]{\left[#1\right]}
\newcommand{\esubeqs}{\end{subequations}}
\newcommand*{\pr}[1]{\left(#1\right)}
\newcommand{\element}[3]
        {\bra{#1}#2\ket{#3}}

\newcommand{\md}{\mathrm{d}}
\newcommand{\e}[1]{\times 10^{#1}}
\renewcommand{\vec}[1]{\mathbf{#1}}
\newcommand{\gvec}[1]{\boldsymbol{#1}}
\newcommand{\dr}{\, \mathrm d^3 \vec r}
\newcommand{\dk}{\, \mathrm d^3 \vec k}
\def\psii{\psi_{i}}
\def\psij{\psi_{j}}
\def\psiij{\psi_{ij}}
\def\psisq{\psi^2}
\def\psisqex{\langle \psi^2 \rangle}
\def\psiR{\psi({\bf R})}
\def\psiRk{\psi({\bf R}_k)}
\def\psiiRk{\psi_{i}(\Rveck)}
\def\psijRk{\psi_{j}(\Rveck)}
\def\psiijRk{\psi_{ij}(\Rveck)}
\def\ranglep{\rangle_{\psisq}}
\def\Hpsibypsi{{H \psi \over \psi}}
\def\Hpsiibypsi{{H \psii \over \psi}}
\def\HmEpsibypsi{{(H-E) \psi \over \psi}}
\def\HmEpsiibypsi{{(H-E) \psii \over \psi}}
\def\HmEpsijbypsi{{(H-E) \psij \over \psi}}
\def\psiibypsi{{\psii \over \psi}}
\def\psijbypsi{{\psij \over \psi}}
\def\psiijbypsi{{\psiij \over \psi}}
\def\psiibypsiRk{{\psii(\Rveck) \over \psi(\Rveck)}}
\def\psijbypsiRk{{\psij(\Rveck) \over \psi(\Rveck)}}
\def\psiijbypsiRk{{\psiij(\Rveck) \over \psi(\Rveck)}}
\def\EL{E_{\rm L}}
\def\ELi{E_{{\rm L},i}}
\def\ELj{E_{{\rm L},j}}
\def\ELRk{E_{\rm L}(\Rveck)}
\def\ELiRk{E_{{\rm L},i}(\Rveck)}
\def\ELjRk{E_{{\rm L},j}(\Rveck)}
\def\Ebar{\bar{E}}
\def\Ei{\Ebar_{i}}
\def\Ej{\Ebar_{j}}
\def\Ebar{\bar{E}}
\def\Rvec{{\bf R}}
\def\Rveck{{\bf R}_k}
\def\Rvecl{{\bf R}_l}
\def\NMC{N_{\rm MC}}
\def\sumMC{\sum_{k=1}^{\NMC}}
\def\MC{Monte Carlo}
\def\adiag{a_{\rm diag}}
\def\tcorr{T_{\rm corr}}
\def\intR{{\int {\rm d}^{3N}\!\!R\;}}

\def\ul{\underline}
\def\beq{\begin{eqnarray}}
\def\eeq{\end{eqnarray}}

\newcommand{\eqbrace}[4]{\left\{
\begin{array}{ll}
#1 & #2 \\[0.5cm]
#3 & #4
\end{array}\right.}
\newcommand{\eqbraced}[4]{\left\{
\begin{array}{ll}
#1 & #2 \\[0.5cm]
#3 & #4
\end{array}\right\}}
\newcommand{\eqbracetriple}[6]{\left\{
\begin{array}{ll}
#1 & #2 \\
#3 & #4 \\
#5 & #6
\end{array}\right.}
\newcommand{\eqbracedtriple}[6]{\left\{
\begin{array}{ll}
#1 & #2 \\
#3 & #4 \\
#5 & #6
\end{array}\right\}}

\newcommand{\mybox}[3]{\mbox{\makebox[#1][#2]{$#3$}}}
\newcommand{\myframedbox}[3]{\mbox{\framebox[#1][#2]{$#3$}}}

%% Infinitesimal (and double infinitesimal), useful at end of integrals
%\newcommand{\ud}[1]{\mathrm d#1}
\newcommand{\ud}[1]{d#1}
\newcommand{\udd}[1]{d^2\!#1}

%% Operators, algebraic matrices, algebraic vectors

%% Operator (hat, bold or bold symbol, whichever you like best):
\newcommand{\op}[1]{\widehat{#1}}
%\newcommand{\op}[1]{\mathbf{#1}}
%\newcommand{\op}[1]{\boldsymbol{#1}}

%% Vector:
\renewcommand{\vec}[1]{\boldsymbol{#1}}

%% Matrix symbol:
%\newcommand{\matr}[1]{\boldsymbol{#1}}
%\newcommand{\bb}[1]{\mathbb{#1}}

%% Determinant symbol:
\renewcommand{\det}[1]{|#1|}

%% Means (expectation values) of varius sizes
\newcommand{\mean}[1]{\langle #1 \rangle}
\newcommand{\meanb}[1]{\big\langle #1 \big\rangle}
\newcommand{\meanbb}[1]{\Big\langle #1 \Big\rangle}
\newcommand{\meanbbb}[1]{\bigg\langle #1 \bigg\rangle}
\newcommand{\meanbbbb}[1]{\Bigg\langle #1 \Bigg\rangle}

%% Shorthands for text set in roman font
\newcommand{\prob}[0]{\mathrm{Prob}} %probability
\newcommand{\cov}[0]{\mathrm{Cov}}   %covariance
\newcommand{\var}[0]{\mathrm{Var}}   %variancd

%% Big-O (typically for specifying the speed scaling of an algorithm)
\newcommand{\bigO}{\mathcal{O}}

%% Real value of a complex number
\newcommand{\real}[1]{\mathrm{Re}\!\left\{#1\right\}}

%% Quantum mechanical state vectors and matrix elements (of different sizes)
%\newcommand{\bra}[1]{\langle #1 |}
\newcommand{\brab}[1]{\big\langle #1 \big|}
\newcommand{\brabb}[1]{\Big\langle #1 \Big|}
\newcommand{\brabbb}[1]{\bigg\langle #1 \bigg|}
\newcommand{\brabbbb}[1]{\Bigg\langle #1 \Bigg|}
%\newcommand{\ket}[1]{| #1 \rangle}
\newcommand{\ketb}[1]{\big| #1 \big\rangle}
\newcommand{\ketbb}[1]{\Big| #1 \Big\rangle}
\newcommand{\ketbbb}[1]{\bigg| #1 \bigg\rangle}
\newcommand{\ketbbbb}[1]{\Bigg| #1 \Bigg\rangle}
%\newcommand{\overlap}[2]{\langle #1 | #2 \rangle}
\newcommand{\overlapb}[2]{\big\langle #1 \big| #2 \big\rangle}
\newcommand{\overlapbb}[2]{\Big\langle #1 \Big| #2 \Big\rangle}
\newcommand{\overlapbbb}[2]{\bigg\langle #1 \bigg| #2 \bigg\rangle}
\newcommand{\overlapbbbb}[2]{\Bigg\langle #1 \Bigg| #2 \Bigg\rangle}
\newcommand{\bracket}[3]{\langle #1 | #2 | #3 \rangle}
\newcommand{\bracketb}[3]{\big\langle #1 \big| #2 \big| #3 \big\rangle}
\newcommand{\bracketbb}[3]{\Big\langle #1 \Big| #2 \Big| #3 \Big\rangle}
\newcommand{\bracketbbb}[3]{\bigg\langle #1 \bigg| #2 \bigg| #3 \bigg\rangle}
\newcommand{\bracketbbbb}[3]{\Bigg\langle #1 \Bigg| #2 \Bigg| #3 \Bigg\rangle}
\newcommand{\projection}[2]
{| #1 \rangle \langle  #2 |}
\newcommand{\projectionb}[2]
{\big| #1 \big\rangle \big\langle #2 \big|}
\newcommand{\projectionbb}[2]
{ \Big| #1 \Big\rangle \Big\langle #2 \Big|}
\newcommand{\projectionbbb}[2]
{ \bigg| #1 \bigg\rangle \bigg\langle #2 \bigg|}
\newcommand{\projectionbbbb}[2]
{ \Bigg| #1 \Bigg\rangle \Bigg\langle #2 \Bigg|}


%% If you run out of greek symbols, here's another one you haven't
%% thought of:
\newcommand{\Feta}{\hspace{0.6ex}\begin{turn}{180}
        {\raisebox{-\height}{\parbox[c]{1mm}{F}}}\end{turn}}
\newcommand{\feta}{\hspace{-1.6ex}\begin{turn}{180}
        {\raisebox{-\height}{\parbox[b]{4mm}{f}}}\end{turn}}




\title[FYS4480]{Density functional theory}
\author[DFT]{%
  Morten Hjorth-Jensen}
\institute[University of Oslo]{
  Department of Physics and Center for Computing in Science Education\\
  University of Oslo, N-0316 Oslo, Norway}
  
\date[UiO]{October 23 and 24, 2025}
\subject{Density Functional Theory}

\begin{document}
\newcommand{\braket}[2]
    {\langle #1 | #2 \rangle}
\newcommand{\ket}[1]
    {|#1\rangle}
\newcommand{\bra}[1]
    {\langle #1 |}
\newcommand{\element}[3]
    {\bra{#1}#2\ket{#3}}
\newcommand{\vbraket}[2]
    {\langle \mathbf{#1} | \mathbf{#2} \rangle}
\newcommand{\vket}[1]
    {|\mathbf{#1}\rangle}
\newcommand{\vbra}[1]
    {\langle \mathbf{#1} |}
\newcommand{\velement}[3]
    {\vbra{#1}\mathbf{#2}\vket{#3}}
\newcommand{\ud}{\mathrm{d}}
\newcommand{\nn}{\notag \\}

\newcommand{\vect}[1]
    {\mathbf{#1}}
\newcommand{\op}[1]
    {\hat{\mathrm{#1}}}

\newcommand{\SE}{Schr\"{o}dinger equation }
\newcommand{\SED}{Schr\"{o}dinger equation. }
\newcommand{\barh}{\bar{\mathrm{H}}}

\newcommand{\normord}[1]{
    \left\{#1\right\}
}
\newcommand{\BCH}{Baker-Campbell-Hausdorff formula}

% Box for sketching algorithms
\newsavebox{\fmbox}
\newenvironment{fmpage}[1]
    {\begin{lrbox}{\fmbox}\begin{minipage}{#1}}
    {\end{minipage}\end{lrbox}\fbox{\usebox{\fmbox}}}

\numberwithin{equation}{subsection}
\numberwithin{figure}{section}
\renewcommand{\theequation}{\arabic{section}.\arabic{subsection}.\arabic{equation}}
\renewcommand{\thefigure}{\arabic{section}.\arabic{figure}}
\renewcommand{\thempfootnote}{\arabic{mpfootnote}}
%\renewcommand{\thesubfigure}{\alph{subfigure}}
\newcommand{\sphelaplop}{
    \frac{1}{r^2} \frac{\partial}{\partial r}
    \left(r^2 \frac{\partial}{\partial r} \right) + \frac{1}{r^2 \sin{\theta}}
    \frac{\partial}{\partial \theta} \left( \sin{\theta} \frac{\partial}
    {\partial \theta} \right) + \frac{1}{r^2 \sin^2\theta} \left(
    \frac{\partial^2}{\partial \phi^2}\right)}

\newcommand{\sphelapl}[1]{
    \frac{1}{r^2} \frac{\partial}{\partial r}
    \left(r^2 \frac{\partial #1}{\partial r} \right) + \frac{1}{r^2 \sin{\theta}}
    \frac{\partial}{\partial \theta} \left( \sin{\theta} \frac{\partial #1}
    {\partial \theta} \right) + \frac{1}{r^2 \sin^2\theta} \left(
    \frac{\partial^2 #1}{\partial \phi^2}\right)}




%\pagenumbering{plain}

\frame{\titlepage}

\frame
{
  \begin{small}
    {\scriptsize
      \frametitle{Litterature I}
      \begin{itemize}
      \item R. van Leeuwen: \emph{Density functional approach to the many-body problem: key concepts and exact functionals}, Adv. Quant. Chem. \textbf{43}, 25 (2003). (Mathematical foundations of DFT)\\
        %$\quad \quad $ Article on seminar web page. \alert{Description}: Mathematical foundations of \\
        %$\quad \quad $ DFT for chemists and physicists.  
      \item R. M. Dreizler and E. K. U. Gross: \emph{Density functional theory: An approach to the quantum many-body problem}. (Introductory book)\\
        %$\quad \quad $ These lectures based to a large extent on this book. \alert{Description}: Much of\\
        %$\quad \quad $ the same things as in article by van Leeuwen, but less mathematically rigorous. 
      \item W. Koch and M. C. Holthausen: \emph{A chemist's guide to density functional theory}. (Introductory book, less formal than Dreizler/Gross)
      \item E. H. Lieb: Density functionals for Coulomb systems, Int. J. Quant. Chem. \textbf{24}, 243-277 (1983). (Mathematical analysis of DFT) 
      \end{itemize}

    }
  \end{small}
}

\frame
{
  \begin{small}
    {\scriptsize
      \frametitle{Litterature II}
      \begin{itemize}
      \item J. P. Perdew and S. Kurth: In \emph{A Primer in Density Functional Theory: Density Functionals for Non-relativistic Coulomb Systems in the New Century}, ed. C. Fiolhais \emph{et al}. (Introductory course, partly difficult, but interesting points of view) \\
       \item E. Engel: In \emph{A Primer in Density Functional Theory: Orbital-Dependent Functionals for the Exchange-Correlation Energy}, ed. C. Fiolhais \emph{et al}. (Introductory lectures, only about orbital-dependent functionals) \\ 
        
      \end{itemize}

    }
  \end{small}
}



\frame[containsverbatim]
{
  \frametitle{Density Functional Theory}
\begin{small}
{\scriptsize
The electronic energy $E$ is said to be a \emph{functional} of the
electronic density, $E[n]$, in the sense that for a given function
$n(r)$, there is a single corresponding energy. The  
\emph{Hohenberg-Kohn theorem} confirms that such
a functional exists, but does not tell us the form of the
functional. As shown by Kohn and Sham, the exact ground-state energy
$E$ of an $N$-electron system can be written as

\begin{equation*}
  E[n] = -\frac{1}{2} \sum_{i=1}^N\int
  \Psi_i^*(\mathbf{r_1})\nabla_1^2 \Psi_i(\mathbf{r_1}) d\mathbf{r_1}
  - \int \frac{Z}{r_1} n(\mathbf{r_1}) d\mathbf{r_1} +
  \frac{1}{2} \int\frac{n(\mathbf{r_1})n(\mathbf{r_2})}{r_{12}}
  d\mathbf{r_1}d\mathbf{r_2} + E_{XC}[n]
\end{equation*}

with $\Psi_i$ the \emph{Kohn-Sham} (KS) \emph{orbitals}.
Note that we have limited ourselves to atomic physics here.

How do we arrive at the above equation? 
 }
 \end{small}
 }



\frame[containsverbatim]
{
  \frametitle{Density Functional Theory}
\begin{small}
{\scriptsize
 The ground-state charge density is given by
\begin{equation*}
  n(\mathbf{r}) = \sum_{i=1}^N|\Psi_i(\mathbf{r})|^2, 
  %\label{}
\end{equation*}

where the sum is over the occupied Kohn-Sham orbitals. The last term,
$E_{XC}[n]$, is the \emph{exchange-correlation energy} which in
theory takes into account all non-classical electron-electron
interaction. However, we do not know how to obtain this term exactly,
and are forced to approximate it. The KS orbitals are found by solving
the \emph{Kohn-Sham equations}, which can be found by applying a
variational principle to the electronic energy $E[n]$. This approach
is similar to the one used for obtaining the HF equation.
 }
 \end{small}
 }




\frame
{
  \frametitle{The Hohenberg-Kohn theorems}
%\begin{small}
{\scriptsize
%\begin{center}
%\begin{equation}
%\begin{center}
Assume we have a given Hamiltonian for a  many-fermion system
\[
\hat{H} = \hat{T}+\hat{V}_{\mathrm{ext}}+\hat{V}.
\]
}
%\end{small}
}


\frame
{
  \frametitle{Theorem I}
\begin{small}
{\scriptsize
We assume that there is a
$\mathcal{V}_{\mathrm{ext}} =$ set of external single-particle \alert{potentials} $v$ so that
\begin{equation}
\hat{H}\ket{\phi} = \left(\hat{T}+\hat{V}_{\mathrm{ext}}+\hat{V}\right)=E\ket{\phi},\qquad \hat{V}_{\mathrm{ext}}\in \mathcal{V}_{\mathrm{ext}},\nonumber
\end{equation} 
gives a \alert{non-degenerate} N-particle ground state $\ket{\Psi }_0$.
For any system of interacting particles in an external potential 
$\mathcal{V}_{\mathrm{ext}}$, the potential $\mathcal{V}_{\mathrm{ext}}$ is uniquely 
determined (by a near constant) by the ground state density $n_0$.
There is a corollary to this statement which states that since $\hat{H}$ is determined, the many-body functions for all states are also determined. All properties of the system are determined via $n_0$.
}
\end{small}
}


\frame
{
  \frametitle{Theorem II}
\begin{small}
{\scriptsize
The density (assuming normalized state vectors)
\begin{equation}
  n(\mathbf{r})=\sum_{i}\int dx_{2}\dots \int dx_{N}\vert \Psi(\mathbf{r},x_{2},\dots ,x_{N})\vert^{2} \nonumber
\end{equation}
Theorem II states that a universal functional for the energy $E[n]$ (function of $n$) can be defined for every external potential
$\mathcal{U}_{\mathrm{ext}}$. For a given external potential, the exact ground state energy of the system is a global minimum of this functional. The density which minimizes this functional is $n_0$.
}
\end{small}
}




\end{equation*}
 }
 \end{small}
 }



%\frame[containsverbatim]
\frame
 {
   \frametitle{}
 \begin{small}
 {\scriptsize
\begin{proof}[Proof I]
Let us prove $C:\mathcal{V}(C)\longrightarrow \Psi$  injective:
%\vspace{0.5cm}

\begin{equation}
  \hat{V}\neq \hat{V}'+\text{constant} \qquad \stackrel{{\LARGE ?}}{\Longrightarrow } \qquad \ket{\Psi}\neq \ket{\Psi'}, \nonumber
\end{equation}
where $\hat{V}, \hat{V}' \in \mathcal{V}$
\vspace{0.5cm}
%\pause

\textbf{\emph{Reductio ad absurdum}}: 
\vspace{1mm}

Assume $\ket{\Psi}=\ket{\Psi'}$ for some $\hat{V}\neq \hat{V}'+\text{const}$, $\hat{V}, \hat{V}' \in \mathcal{V}$ 
%\pause

$\hat{T}\neq \hat{T}[V]$, $\hat{W}\neq \hat{W}[V] \quad \Longrightarrow $\footnote{Unique continuation theorem: $\ket{\Psi}\neq 0$ on a set of positive measure}
\begin{equation}
  \left(\hat{V}-\hat{V}'\right)\ket{\Psi }=\left(E_{gs}-E_{gs}'\right)\ket{\Psi }.\nonumber  
\end{equation}
%\pause
\begin{align}
  \Longrightarrow \qquad & \hat{V}-\hat{V}'=E_{gs}-E_{gs}' \nonumber \\
  \Longrightarrow \qquad & \hat{V}=\hat{V}'+\text{constant} \qquad \alert{\text{Contradiction!}}\nonumber
\end{align}
\end{proof}
%\stackrel{{\LARGE ?}}{\Longrightarrow }
%\genfrac{}{}{0pt}{}{?}{\Longrightarrow }

% \begin{verbatim}
% // random numbers with gaussian distribution
% double gaussian_deviate(long * idum)
% {
%   static int iset = 0;
%   static double gset;
%   double fac, rsq, v1, v2;
%   if ( idum < 0) iset =0;
%   if (iset == 0) {
%     do {
%       v1 = 2.*ran0(idum) -1.0;
%       v2 = 2.*ran0(idum) -1.0;
%       rsq = v1*v1+v2*v2;
%     } while (rsq >= 1.0 || rsq == 0.);
%     fac = sqrt(-2.*log(rsq)/rsq);
%     gset = v1*fac;
%     iset = 1;
%     return v2*fac;
%   } else {
%     iset =0;
%     return gset;
%   }
% \end{verbatim}
 }
 \end{small}
 }



\frame
 {
   \frametitle{}
 \begin{small}
 {\scriptsize
\begin{proof}[Proof II]
  Let us prove $D:\Psi \longrightarrow \mathcal{N}$ injective:

\begin{equation}
  \ket{\Psi}\neq \ket{\Psi'} \qquad \stackrel{{\LARGE ?}}{\Longrightarrow } \qquad n(\mathbf{r})\neq n'(\mathbf{r}) \nonumber
\end{equation}
%\vspace{0.1cm}
%\pause
\textbf{\emph{Reductio ad absurdum}}: 
\vspace{1mm}

Assume $n(\mathbf{r})=n'(\mathbf{r})$ for some $\ket{\Psi}\neq \ket{\Psi'}$ 
%\pause

Ritz principle $\quad \Longrightarrow $
\begin{equation}
  E_{gs}=\bra{\Psi}\hat{H}\ket{\Psi}<\bra{\Psi'}\hat{H}\ket{\Psi'} \nonumber   
\end{equation}
%\pause 
\begin{equation}
  \bra{\Psi'}\hat{H}\ket{\Psi'}=\bra{\Psi'}\hat{H}'+\hat{V}-\hat{V}'\ket{\Psi'}=E_{gs}'+\int n'(\mathbf{r})[v(\mathbf{r})-v'(\mathbf{r})]d^{3}r \nonumber
\end{equation}
%\pause
\begin{equation}\label{eq:ineq1}
  \Longrightarrow \qquad E_{gs}'<E_{gs}+\int n'(\mathbf{r})[v(\mathbf{r})-v'(\mathbf{r})]d^{3}r %\nonumber
\end{equation}
By symmetry
\begin{equation}\label{eq:ineq2}
  \Longrightarrow \qquad E_{gs}<E_{gs}'+\int n'(\mathbf{r})[v'(\mathbf{r})-v(\mathbf{r})]d^{3}r %\nonumber
\end{equation}
%\pause
(\ref{eq:ineq1}) \& (\ref{eq:ineq2}) $\qquad \Longrightarrow $
\begin{equation}
  E_{gs}+E_{gs}'<E_{gs}+E_{gs}' \qquad \text{\alert{Contradiction!}} \nonumber
\end{equation}
\end{proof}
 }
 \end{small}
 }




\frame
{ 
\frametitle{}
\begin{small}
{\scriptsize
    
Define
\begin{equation}
  E_{v_{0}}[n]:=\bra{\Psi[n]}\hat{T}+\hat{W}+\hat{V_{0}}\ket{\Psi[n]} \nonumber
\end{equation}
$\hat{V_{0}} =$ external potential, $n_{0}(\mathbf{r}) =$ corresponding GS density, $E_{0} =$ GS energy
%\pause

\vspace{2mm}
Rayleigh-Ritz principle $\quad \Longrightarrow \quad $ \alert{second statement of H-K theorem}:
\begin{equation}
  E_{0}=\min_{n\in \mathcal{N}}E_{v_{0}}[n] \nonumber
\end{equation}
%\pause
\alert{Last satement of H-K theorem}:
\begin{equation}
  F_{HK}[n]\equiv \bra{\Psi[n]}\hat{T}+\hat{W}\ket{\Psi[n]} \nonumber
\end{equation}
is \emph{universal} ($F_{HK}\neq F_{HK}[\hat{V_{0}}]$)
} 
\end{small}
}



\frame
{ 
  \frametitle{The Kohn-Sham scheme }
  \begin{small}
    {\scriptsize
      The \alert{classic Kohn-Sham} scheme:
      \begin{equation}
         \left(-\frac{\hbar^{2}}{2m}\nabla^{2}+v_{s,0}(\mathbf{r})\right)\phi_{i,0}(\mathbf{r})=\varepsilon_{i}\phi_{i,0}(\mathbf{r}), \qquad \varepsilon_{1}\geq \varepsilon_{2} \geq \dots \;, \nonumber
      \end{equation}
      where
      \begin{equation}
        v_{s,0}(\mathbf{r})=v_{0}(\mathbf{r})+\int d^{3}r' w(\mathbf{r},\mathbf{r}')n_{0}(\mathbf{r}')+v_{\mathrm{XC}}([n_{0}];\mathbf{r}) \nonumber 
      \end{equation}
      The density calculated as 
      \begin{equation}
        n_{0}(\mathbf{r})=\sum_{i=1}^{N}\vert \phi_{i,0}(\mathbf{r})\vert^{2}, \nonumber
      \end{equation}
      Equation \alert{solved selfconsistently} 

      Total energy: 
      \begin{equation}
        E=\sum_{i=1}^{N}\varepsilon_{i}-\frac{1}{2}\int d^{3}r d^{3}r'n(\mathbf{r})w(\mathbf{r},\mathbf{r}')n(\mathbf{r}')+E_{\mathrm{XC}}[n]-\int d^{3}r v_{\mathrm{XC}}([n];\mathbf{r})n(\mathbf{r}) \nonumber
      \end{equation}
    }
  \end{small}
}

\frame
{ 
  \frametitle{Exchange Energy and Correlation Energy}
  \begin{small}
    {\scriptsize

      \structure{Hartree-Fock} equation:
      \begin{align}
        \left(-\frac{\hbar^{2}}{2m}\nabla^{2}+v_{0}(\mathbf{r})+\int d^{3}r'w(\mathbf{r},\mathbf{r}')n(\mathbf{r}')\right)\phi_{k}(\mathbf{r})& \nonumber \\
        \underbrace{-\sum_{l=1}^{N}\int d^{3}r'\phi_{l}^{*}(\mathbf{r}')w(\mathbf{r},\mathbf{r}')\phi_{k}(\mathbf{r}')\phi_{l}(\mathbf{r})}_{\text{exchange term}}
        &=\varepsilon_{k}\phi_{k}(\mathbf{r}), \nonumber
      \end{align}
      
      %\begin{itemize}
      $\qquad \qquad $ \alert{Non-local} exchange term (Pauli exclusion principle)
      %\end{itemize}

      \vspace{5mm}     
      \structure{Kohn-Sham} equation:

      \begin{equation}
         \left(-\frac{\hbar^{2}}{2m}\nabla^{2}+v_{0}(\mathbf{r})
           +\int d^{3}r'w(\mathbf{r},\mathbf{r}')n(\mathbf{r}')
           +\underbrace{v_{\mathrm{XC}}([n];\mathbf{r})}_{\text{exchange + correlation}}
         \right)\phi_{k}(\mathbf{r})=\varepsilon_{k}\phi_{k}(\mathbf{r}), \nonumber
       \end{equation}
       %\begin{itemize}
       $\qquad \qquad $  \alert{Local} exchange-correlation term
       %\end{itemize}
      
    }
  \end{small}
}

\frame
{ 
  \frametitle{}
  \begin{small}
    {\scriptsize
      Exchange-correlation energy $=$ \structure{Exchange energy} $+$ \alert{Correlation energy}
      \begin{equation}
        E_{\mathrm{XC}}[n]=E_{x}[n]+E_{c}[n] \nonumber
      \end{equation}
      From earlier: 
      \begin{equation}
        E_{\mathrm{XC}}[n]=F_{L}[n]-T_{s}[n]-\frac{1}{2}\iint d^{3}r d^{3}r' n(\mathbf{r})w(\mathbf{r},\mathbf{r}')n(\mathbf{r}') \nonumber
      \end{equation}
      \vspace{10mm}
      \begin{center}
        We want to show: $\quad $$E_{c}[n]\leq 0$
      \end{center}
    }
  \end{small}
}

\frame
{ 
  \frametitle{}
  \begin{small}
    {\scriptsize
      Here we have (assume $F_{L}[n]=F_{LL}[n]$)
      \begin{align}
        F_{L}[n]&\equiv \inf_{\Psi \rightarrow n}\bra{\Psi }\hat{T}+\hat{W}\ket{\Psi } \nonumber \\
        &=\bra{\Psi_{n}^{min}}\hat{T}+\hat{W}\ket{\Psi_{n}^{min}}, \nonumber 
      \end{align}
      and 
      \begin{align}
        T_{s}[n]&\equiv \inf_{\Psi \rightarrow n}\bra{\Psi }\hat{T}\ket{\Psi }=\bra{\Phi_{n}^{min}}\hat{T}\ket{\Phi_{n}^{min}}, \nonumber 
      \end{align}
      $\Psi = $ normalized, antisymm. $N$-particle wavefunction, \\
      $\Phi_{n}^{min}$ lin. komb. of Slater determinants of \\
      single-particle orbitals $\psi_{i}(r_{j})$ 
    }
  \end{small}
}

\frame
{ 
  \frametitle{}
  \begin{small}
    {\scriptsize
      Eq. (4.35) in J. M. Thijssen: \emph{Computational Physics}:
      \begin{align}
        \bra{\Phi_{n}^{min}}\hat{W}\ket{\Phi_{n}^{min}}&=\frac{1}{2}\sum_{k,l}\bigg[ \iint d^{3}r d^{3}r' n(\mathbf{r})w(\mathbf{r},\mathbf{r}')n(\mathbf{r}') \nonumber \\
        &-\iint d^{3}r d^{3}r' \psi_{l}^{*}(\mathbf{r})\psi_{l}(\mathbf{r}')w(\mathbf{r},\mathbf{r}')\psi_{k}^{*}(\mathbf{r}')\psi_{k}(\mathbf{r})\bigg] \nonumber
      \end{align}
      By definition,
      \begin{equation}
        E_{x}[n]\equiv -\frac{1}{2}\sum_{k,l}\iint d^{3}r d^{3}r' \psi_{l}^{*}(\mathbf{r})\psi_{l}(\mathbf{r}')w(\mathbf{r},\mathbf{r}')\psi_{k}^{*}(\mathbf{r}')\psi_{k}(\mathbf{r})  \nonumber
      \end{equation}
    }
  \end{small}
}

\frame
{ 
  \frametitle{}
  \begin{small}
    {\scriptsize
      \begin{align}
        E_{c}[n]&=E_{\mathrm{XC}}[n]-E_{x}[n] \nonumber \\
        &=F_{L}[n]-T_{s}[n]-\frac{1}{2}\iint d^{3}r d^{3}r' n(\mathbf{r})w(\mathbf{r},\mathbf{r}')n(\mathbf{r}') \nonumber \\
        &+\frac{1}{2}\sum_{k,l}\iint d^{3}r d^{3}r' \psi_{l}^{*}(\mathbf{r})\psi_{l}(\mathbf{r}')w(\mathbf{r},\mathbf{r}')\psi_{k}^{*}(\mathbf{r}')\psi_{k}(\mathbf{r}) \nonumber \\
        &=\bra{\Psi_{n}^{min}}\hat{T}+\hat{W}\ket{\Psi_{n}^{min}}-\bra{\Phi_{n}^{min}}\hat{T}+\hat{W}\ket{\Phi_{n}^{min}} \nonumber
      \end{align}

      Since
      \begin{equation}
        \bra{\Psi_{n}^{min}}\hat{T}+\hat{W}\ket{\Psi_{n}^{min}}=\inf_{\Psi \rightarrow n}\bra{\Psi }\hat{T}+\hat{W}\ket{\Psi }, \nonumber
      \end{equation}
      we see that  
      \begin{equation}
        E_{c}[n]\leq 0 \nonumber
      \end{equation}
    }
  \end{small}
}

\frame[containsverbatim]
{
  \frametitle{Computing $E_{XC}$ from {\em ab initio} calculations}
\begin{small}
{\scriptsize
Question: can we compute the 'exact' $E_{XC}$ that enters DFT calculations? Yes! 

Let us define a continuous variable $\lambda$ and a Hamiltonian which depends on this variable
\[
\hat{H}_{\lambda}=\hat{T}+\lambda \hat{V}+\hat{v}_{\mathrm{ext}},
\]
where $\hat{T}$ is the kinetic energy, $\hat{V}$ is in our case the Coulomb interaction between two electrons an $\hat{v}_{\mathrm{ext}}$ is our external potential, here the two-dimensional
harmonic oscillator potential. 

For $\lambda=0$ we have the non-interacting system, whose solution in our case is a single
Slater determinant for the ground state (non-degenerate case). For $\lambda=1$ we
have the full interacting case. 
 }
 \end{small}
 }



\frame[containsverbatim]
{
  \frametitle{Computing $E_{XC}$ from {\em ab initio} calculations}
\begin{small}
{\scriptsize
The standard variational principle is to find the minimum of 
\[
E_{\lambda}[\hat{v}_{\mathrm{ext}}]=\inf_{\Psi\rightarrown}\langle \Psi_{\lambda}|\hat{H}_{\lambda}|\Psi_{\lambda}\rangle,
\]
with respect to the wave function $\Psi_{\lambda}$.  If a maximizing potential 
$\hat{v}_{\mathrm{ext}}^{\lambda}$ exists, then according  to the Hohenberg and Kohn, it is the one which has the density $n$ as the ground state density and we have a functional
\[
F_{\lambda}[n] = E_{\lambda}[\hat{v}_{\mathrm{ext}}^{\lambda}]
-\int d{\bf r}n({\bf r}) \hat{v}_{\mathrm{ext}}^{\lambda}({\bf r}).
\]
 }
 \end{small}
 }


\frame[containsverbatim]
{
  \frametitle{Computing $E_{XC}$ from {\em ab initio} calculations}
\begin{small}
{\scriptsize
Which leads to the Lieb variational principle 
\[
F_{\lambda}[n] = \sup_{\hat{v}_{\mathrm{ext}}}\left(E_{\lambda}[\hat{v}_{\mathrm{ext}}^{\lambda}]
-\int d{\bf r}n({\bf r}) \hat{v}_{\mathrm{ext}}^{\lambda}({\bf r})\right).
\]
We define
\[
F_{\lambda}[n]= \langle \Psi_{\lambda}|\hat{T}+\lambda \hat{V}|\Psi_{\lambda}\rangle,
\]
which we rewrite as
\[
F_{\lambda}[n]= \langle \Psi_{\lambda}|\hat{T}|\Psi_{\lambda}\rangle+\lambda J[n]+E_{XC}[n],
\]
with the standard Hartree term
\[
J= \frac{1}{2}\int d{\bf r}_1d{\bf r}_2n({\bf r}_1)n({\bf r}_2)V(r_{12}).
\]
 }
 \end{small}
 }



\frame[containsverbatim]
{
  \frametitle{Computing $E_{XC}$ from {\em ab initio} calculations}
\begin{small}
{\scriptsize
We want to find $E_{XC}[n]$ in
\[
F_{\lambda}[n]= \langle \Psi_{\lambda}|\hat{T}|\Psi_{\lambda}\rangle+\lambda J[n]+E_{XC}[n].
\]
To do this, since we use a variational method, we can employ the Hellmann-Feynman theorem,
which states that
\[
\Delta E = \int_{\lambda_1}^{\lambda_2}d\lambda \frac{\partial E_{\lambda}}{\partial \lambda} =
\int_{\lambda_1}^{\lambda_2}d\lambda \langle \Psi_{\lambda}|\frac{\partial\hat{H}_{\lambda}}{\partial \lambda}|\Psi_{\lambda}\rangle.
\]
Setting $\lambda_1=0$ and $\lambda_2=1$ we arrive at 
\[
\Delta E = \int_{0}^{1}d\lambda \langle \Psi_{\lambda}|\hat{V}|\Psi_{\lambda}\rangle,
\] 
where the wave function at $\lambda =0$ is our single Slater determinant for the reference state. In the case of a VMC caclulation there would be no Jastrow factor. For $\lambda=1$ we can use our best variational Monte Carlo function.  Note that $\hat{V}$
is the full interaction at $\lambda=1$!
}
 \end{small}
 }




\frame[containsverbatim]
{
  \frametitle{Computing $E_{XC}$ from {\em ab initio} calculations}
\begin{small}
{\scriptsize
We wish to relate 
\[
\Delta E = \int_{0}^{1}d\lambda\langle \Psi_{\lambda}|\hat{V}|\Psi_{\lambda}\rangle,
\] 
to $E_{XC}$. 
Recalling that we defined 
\[
\langle \Psi_{\lambda}|\lambda \hat{V}|\Psi_{\lambda}\rangle=\lambda J[n]+E_{XC}[n],
\]
we rewrite our equation as
\[
E_{XC}= \int_{0}^{1}d\lambda\langle \Psi_{\lambda}|\hat{W}_{\lambda}|\Psi_{\lambda}\rangle,
\] 
where 
\[
W_{\lambda}=\langle \Psi_{\lambda}|\lambda \hat{V}|\Psi_{\lambda}\rangle-J.
\]
}
 \end{small}
 }



\frame[containsverbatim]
{
  \frametitle{Computing $E_{XC}$ from {\em ab initio} calculations}
\begin{small}
{\scriptsize
Using the fundamental theorem of calculus we have then
\[
E_{XC}= \langle\Psi_{1}|\hat{V}|\Psi_{1}\rangle-\langle\Psi_{0}|\hat{V}|\Psi_{0}\rangle.
\]
We need thus simply to compute the expectation value of $\hat{V}$ for the single Slater
determinant $\lambda=0$ and the fully correlated wave 
function with, if we do a VMC calculation,  the Jastrow factor as well
for the $\lambda=1$ case.  

The total correlation energy, including kinetic energy is then (computed at a fixed density)
equal to
\[
E_C=\langle\Psi_{1}|\hat{T}+\hat{V}|\Psi_{1}\rangle-\langle\Psi_{0}|\hat{T}+\hat{V}|\Psi_{0}\rangle.
\]
}
 \end{small}
 }




\frame
{ 
\frametitle{}
\begin{small}
{\scriptsize
    
Define
\begin{equation}
  E_{v_{0}}[n]:=\bra{\Psi[n]}\hat{T}+\hat{W}+\hat{V_{0}}\ket{\Psi[n]} \nonumber
\end{equation}
$\hat{V_{0}} =$ external potential, $n_{0}(\mathbf{r}) =$ corresponding GS density, $E_{0} =$ GS energy
%\pause

\vspace{2mm}
Rayleigh-Ritz principle $\quad \Longrightarrow \quad $ \alert{second statement of H-K theorem}:
\begin{equation}
  E_{0}=\min_{n\in \mathcal{N}}E_{v_{0}}[n] \nonumber
\end{equation}
%\pause
\alert{Last satement of H-K theorem}:
\begin{equation}
  F_{HK}[n]\equiv \bra{\Psi[n]}\hat{T}+\hat{W}\ket{\Psi[n]} \nonumber
\end{equation}
is \emph{universal} ($F_{HK}\neq F_{HK}[\hat{V_{0}}]$)
} 
\end{small}
}


\frame
{ 
  \frametitle{The Basic Kohn-Sham Equations}
  \begin{small}
    {\scriptsize
      \begin{itemize}
      \item So far: \\
        $\quad \quad $ H-K \alert{variational principle} $\quad \Longrightarrow \quad $ \\
        $\quad \quad $ exact GS density of many-particle system \\
        $\quad \quad $ \alert{Practically intractable !!}  
      %\item So far: The exact GS density of a many-particle system can in principle be determined by \alert{the variational principle} of Hohenberg and Kohn. This is, however, a \alert{practically intractable} problem.  
      \item Next step: \\
        $\quad \quad $ Kohn and Sham (1965): \alert{single-particle picture} \\
 $\quad \quad $ $\longrightarrow \quad $ equations solved \alert{selfconsistently} (iterative scheme) 
      \end{itemize}
    }
  \end{small}
}

\frame
{ 
  \frametitle{}
  \begin{small}
    {\scriptsize
      Hamiltonian of $N$ \emph{non-interacting} particles:
      \begin{equation}
        \hat{H}_{s}=\hat{T}+\hat{V}_{s} \nonumber
      \end{equation}
      Hohenberg and Kohn $\quad \Longrightarrow \quad $ $\exists \;$ unique energy functional 
      \begin{equation}
        E_{s}[n]=T_{s}[n]+\int v_{s}(\mathbf{r})n(\mathbf{r})d^{3}r \nonumber
      \end{equation}
      s. t. $\delta E_{s}[n]=0$ gives GS density $n_{s}(\mathbf{r})$ corresp. to $\hat{H}_{s}$


    }
  \end{small}
}

\frame
{ 
  \frametitle{}
  \begin{small}
    {\scriptsize


      \begin{theorem}
        Let 
        \begin{align}
          v_{s}(\mathbf{r})\quad &= \quad \text{local single-particle pot.}, \nonumber \\
          n(\mathbf{r})\quad &=\quad \text{GS density of interacting system}, \nonumber \\
          n_{s}(\mathbf{r})\quad &=\quad \text{GS density of non-interacting system} \nonumber 
        \end{align}
        $\Longrightarrow \quad $ for \alert{any interacting system},
        \begin{equation}
          \exists \;\; \text{ a }\;\; v_{s}(\mathbf{r}) \;\; \text{s. t.} \;\; n_{s}(\mathbf{r})=n(\mathbf{r})  \nonumber
        \end{equation}
      \end{theorem}
      %\pause
      Proof in book by Dreizler/Gross, Sec. 4.2 \\
%      In proof: $F_{HK}[n]$ replaced by $F_{L}[n]$ 
%      rove this important theorem, but in the proof it is necessary to replace the H-K  functional $F_{HK}[n]$ by the Lieb functional $F_{L}[n]$ in the expression of $v_{s}(\mathbf{r})$.
    }
  \end{small}
}

\frame
{ 
  \frametitle{}
  \begin{small}
    {\scriptsize
      Assume \structure{nondegenerate GS}. Then
      \begin{equation}
        n(\mathbf{r})=n_{s}(\mathbf{r})=\sum_{i=1}^{N}\left\vert \phi_{i}(\mathbf{r})\right\vert^{2}, \nonumber
      \end{equation}
      where $\phi_{i}(\mathbf{r})$ are determined by
      \begin{equation}
        \left( -\frac{\hbar^{2}}{2m}\nabla^{2}+v_{s}(\mathbf{r})\right)\phi_{i}(\mathbf{r})=\varepsilon_{i}\phi_{i}(\mathbf{r}), \qquad \varepsilon_{1}\leq \varepsilon_{2}\leq \dots \;. \nonumber
      \end{equation}
      %\pause

      \vspace{12mm}
      If $\exists $ $v_{s}(\mathbf{r})$, then H-K theorem gives \emph{uniqueness} of $v_{s}(\mathbf{r})$ \\
      Consequently, we may write
      \begin{equation}
        \phi_{i}(\mathbf{r})=\phi_{i}([n(\mathbf{r})]) \qquad \alert{\textbf{!!}} \nonumber
      \end{equation}
      
    }
  \end{small}
}

\frame
{ 
  \frametitle{}
  \begin{small}
    {\scriptsize
      Assume \\
      $\qquad \qquad \qquad \qquad \qquad $ $v_{0}(\mathbf{r}) =$ ext. potential \\
      $\qquad \qquad \qquad \qquad \qquad $ $n_{0}(\mathbf{r}) =$ GS density \\
      of \alert{interacting} system 
      \begin{itemize}
      \item Wanted: \structure{single-particle potential} $v_{s}(\mathbf{r})$ of \alert{non-interacting} system
      \end{itemize}
    }
  \end{small}
}

\frame
{ 
  \frametitle{Exchange-correlation functional}
  \begin{small}
    {\scriptsize
      Many-particle energy functional:
      \begin{align}
        E_{v_{0}}[n]&=F_{L}[n]+\int d^{3}v_{0}(\mathbf{r})n(\mathbf{r}) \nonumber \\
        &=\left(T_{s}[n]+\frac{1}{2}\iint d^{3}r d^{3}r' n(\mathbf{r})w(\mathbf{r},\mathbf{r}')n(\mathbf{r}')+E_{\mathrm{XC}}[n]\right) + \int d^{3}r v_{0}(\mathbf{r})n(\mathbf{r}) \nonumber 
      \end{align}
      Here \alert{exchange-correlation functional} defined: 
      \begin{equation}
        E_{\mathrm{XC}}[n]=F_{L}[n]-\frac{1}{2}\iint d^{3}r d^{3}r' n(\mathbf{r})w(\mathbf{r},\mathbf{r}')n(\mathbf{r}')-T_{s}[n] \nonumber
      \end{equation}
    }
  \end{small}
}

\frame
{ 
  \frametitle{}
  \begin{small}
    {\scriptsize
      The exchange-correlation functional defined: 
      \begin{equation}
        E_{\mathrm{XC}}[n]=F_{L}[n]-\frac{1}{2}\iint d^{3}r d^{3}r' n(\mathbf{r})w(\mathbf{r},\mathbf{r}')n(\mathbf{r}')-T_{s}[n] \nonumber
      \end{equation}
      \vspace{5mm}
      \alert{Explicit form} of $F_{L}[n]$ as functional of $n$ \alert{unknown}
      \begin{itemize}
        \item $E_{\mathrm{XC}}[n]$ unknown functional, must be approximated \\
          Otherwise, Kohn-Sham scheme exact 
      \end{itemize}
    }
  \end{small}
}

\frame
{ 
  \frametitle{}
  \begin{small}
    {\scriptsize
      %Side-remark:
      
      \begin{definition}
        Let $F:B\rightarrow \mathbb{R}$ be a \emph{functional} from  normed function space $B$ to real numbers $\mathbb{R}$. \\
        
        \vspace{2mm}
        The \alert{functional derivative} (G{\^a}teaux derivative) \\
        $\delta F[n]\equiv \delta F[n]/\delta n(\mathbf{r})$ is defined as
        \begin{equation}
          \frac{\delta F}{\delta n}[\varphi ]=\lim_{\varepsilon \rightarrow 0}\frac{F[n+\varepsilon \varphi ]-F[n]}{\varepsilon } \nonumber
        \end{equation}
        Another useful definition of $\delta F[n]$:
        \begin{equation}
          \left< \delta F[n],\varphi \right>=\frac{d}{d\varepsilon }F[n+\varepsilon \phi ]\Bigg\vert_{\varepsilon =0}, \nonumber
        \end{equation}
        where
        \begin{equation}
          \left< \delta F[n],\varphi \right>\equiv \int d\mathbf{r}(\delta F[n(\mathbf{r})])\varphi(\mathbf{r}), \nonumber
        \end{equation}
        $\varphi =$ test function 
      \end{definition}
    }
  \end{small}
}


\frame
{ 
  \frametitle{Gradient expansion}
  \begin{small}
    {\scriptsize
      The \alert{gradient expansion approximation (GEA)} -- a natural extension of LDA ??
      
      \vspace{2mm}
      \structure{Taylor expansion} of $E_{\mathrm{XC}}[n]$ \\
      around homogeneous electron gas (HEG) \\
      density $n_{0}$ $\quad $ ($(n-n_{0})/n_{0}\ll 1$):
      
     \begin{equation}
       E_{\mathrm{XC}}[n]=E_{\mathrm{XC}}[n_{0}]+\sum_{m=1}^{\infty }\frac{1}{m!}\int d^{3m}r \frac{\delta^{m}E_{\mathrm{XC}}}{\delta n(\mathbf{r}_{1})\dots \delta n(\mathbf{r}_{m})}\Bigg\vert_{n=n_{0}}\delta n(\mathbf{r}_{1})\dots \delta n(\mathbf{r}_{m}) \nonumber
     \end{equation}
    }
  \end{small}
}

\frame
{ 
  \frametitle{Gradient expansion}
  \begin{small}
    {\scriptsize
      Shown in article by van Leeuwen: \\
      \vspace{2mm}
      Expansion can be written
      \begin{align}
        E_{\mathrm{XC}}[n]&=E_{\mathrm{XC}}^{LDA}[n]+\int d^{3}r g_{1}(n(\mathbf{r}))(\nabla n(\mathbf{r}))^{2} \nonumber \\
        &+\int d^{3}r g_{2}(n(\mathbf{r}))(\nabla^{2}n(\mathbf{r}))^{2}+\dots , \nonumber
      \end{align}
      $g_{i}(n)$ uniquely determined by the density response functions of a HEG
    }
  \end{small}
}

\frame
{ 
  \frametitle{}
  \begin{small}
    {\scriptsize
      \begin{center}
        Gradient expansion in principle exact, \alert{provided series converges}
        
        \vspace{10mm}
        \begin{tabular}{l l}
          \hline
          Metallic systems: & good convergence \\
          Insulators: & bad convergence \\
          Finite systems: & bad convergece \\
          \hline
        \end{tabular}
      \end{center}
    }
  \end{small}
}

\frame
{ 
  \frametitle{}
  \begin{small}
    {\scriptsize
      \alert{\textbf{Caution!}}
      
      \vspace{5mm}
      \begin{center}
        Numerical tests show: \\
        Inclusion of second-order gradient term \\
        may give a considerably worse $E_{\mathrm{XC}}[n]$ than $E_{\mathrm{XC}}^{LDA}[n]$
      \end{center}

      %%\pause
      \vspace{5mm}
      $\qquad \qquad \qquad \qquad \qquad \qquad \qquad \qquad \qquad \qquad \qquad \qquad \qquad \qquad \quad $ \structure{Why?}
    }
  \end{small}
}

\frame
{ 
  \frametitle{}
  \begin{small}
    {\scriptsize
      %\begin{center}
      $E_{\mathrm{XC}}^{LDA}[n]$ $\quad $ provides rather realistic results $\quad $ for atoms, molecules, and solids

      \vspace{7mm}
      But: second-order term (\alert{next systematic correction} \\
      $\qquad \qquad \qquad \qquad \qquad \qquad \qquad $ for slowly-varying densities) \alert{makes} $E_{\mathrm{XC}}$ \alert{worse}
      %\end{center}
    }
  \end{small}
}

\frame
{ 
  \frametitle{}
  \begin{small}
    {\scriptsize
      Why does gradient expansion fail?
      \begin{enumerate}
        \item Realistic electron densities not very close to slowly-varying limit
        \item LDA: $\quad $ xc hole is the hole of \structure{a possible physical system} \\
          $\qquad \qquad $ $\Longrightarrow $ $\quad $ satisfies exact constraints \\
          GEA: $\quad $ xc hole \structure{not physical} \\
          $\qquad \qquad $ $\Longrightarrow $ $\quad $ does not satisfy constraints
          
      \end{enumerate}
    }
  \end{small}
}

\frame
{ 
  \frametitle{}
  \begin{small}
    {\scriptsize
      Example of constraints:
      
      %\vspace{10mm}
      \begin{center}
        \begin{tabular}{c c c}
          Physical constraint & LDA & GEA \\
          \hline
          $E_{c}<0$ & $<0$ & $>0$ \\
          $E_{x}<0$ & $<0$ & not restricted \\
          $\int h_{\mathrm{XC}}(\mathbf{r}_{1};\mathbf{r}_{2})d\mathbf{r}_{2}=-1$ & $-1$ & not restricted \\
          \hline
        \end{tabular}
      \end{center}
      
      \vspace{5mm}
      $\qquad \qquad \qquad \qquad \qquad \qquad \qquad \qquad $ $\Longrightarrow $ $\quad $ \alert{Wrong behaviour} of GEA
    }
  \end{small}
}



\frame
{ 
  \frametitle{The Generalized Gradient Approximation}
  \begin{small}
    {\scriptsize
      Method: \alert{Enforce} physical \alert{restrictions} for the xc hole\\
      $\qquad \qquad $ $\Longrightarrow $ $\quad $ Generalized gradient approximation (GGA):
      \begin{equation}
        E_{\mathrm{XC}}^{GGA}[n_{\uparrow },n_{\downarrow }]=\int d^{3}r f(n_{\uparrow },n_{\downarrow },\nabla n_{\uparrow },\nabla n_{\downarrow }) \nonumber
      \end{equation}
      
      
      \begin{itemize}
      \item $f(n_{\uparrow },n_{\downarrow },\nabla n_{\uparrow },\nabla n_{\downarrow })$ not unique, \\ 
        but formal features of LDA $\quad \Longrightarrow \quad $ constraints
      \item GGA-functionals with/without semiempirical parameters
      \item Successful in quantum chemistry 
      \item No systematic approach to improve GGA-functionals
      \end{itemize}
    }
  \end{small}
}

\frame
{ 
  \frametitle{}
  \begin{small}
    {\scriptsize
      \alert{Typical errors} for atoms, molecules, and solids (Perdew/Kurth):
      \begin{center}
        \begin{tabular}{c c c}
          \hline
          Property & LDA & GGA \\
          \hline
          $E_{x}$ & $5\% $ (not negative enough) & $0.5\% $\\
          $E_{c}$ & $100\% $ (too negative) & $5\% $\\
          bond length & $1\% $ (too short) & $1\% $ (too long) \\
          structure & overly favours close packing & more correct \\
          energy barrier & $100\% $ (too low) & $30\% $ (too low) \\
          \hline
        \end{tabular}
      \end{center}
      \vspace{2mm}
      \begin{itemize}
      \item GGA in most cases better than LDA  
      \item Typically cancellation of errors between $E_{x}$ and $E_{c}$
      \item "Energy barrier" = barrier to a chemical reaction
      \end{itemize}
    }
  \end{small}
}

\frame
{ 
  \frametitle{}
  \begin{small}
    {\scriptsize
      Situations where GGA fails:
      %\begin{itemize}
      
      \vspace{3mm}
      $\qquad \qquad $ Unaccurate results for \textbf{heavy elements} \\
            
      \vspace{2mm}
      $\qquad \qquad $ Does not predict existence of \textbf{negative ions} \\
            
      \vspace{2mm}
      $\qquad \qquad $ Fails to reproduce \textbf{dispersion forces} ($\approx $ van der Waals forces) \\
            
      \vspace{2mm}
      $\qquad \qquad $ Can not describe properly \textbf{strongly correlated systems} \\
      %\end{itemize}
    }
  \end{small}
}

\frame
{ 
  \frametitle{}
  \begin{small}
    {\scriptsize
      GGA gives unaccurate results for \alert{\textbf{heavy elements}}:
      
      \vspace{3mm}
      Gold (Au):

      \begin{center}
        \begin{tabular}{c c c}
          \hline
          $E_{\mathrm{XC}}[n]$ & Equilibrium & Cohesive \\
                    & lattice constant & energy \\
          \hline
           LDA & \structure{\textbf{7.68}} & \textbf{4.12} \\
           relativistic LDA & 7.68 & 4.09 \\
           GGA & \alert{\textbf{7.87}} & \textbf{2.91} \\
           relativistic GGA & 7.88 & 2.89 \\
           experiment & \structure{\textbf{7.67}} & \textbf{3.78}\\
          \hline
        \end{tabular}
      \end{center}
      
      \begin{itemize}
      \item Here: LDA better than GGA
      \item Problem not due to relativistic effects
      \item GGA: problems with high angular momenta \\
        (higher ion charge $\quad \Longrightarrow \quad $ higher electron angular momentum)
      \end{itemize}
    }
  \end{small}
}

\frame
{ 
  \frametitle{}
  \begin{small}
    {\scriptsize
      GGA does not predict existence of \alert{\textbf{negative ions}}:
      
      \vspace{3mm}
      For neutral atoms exactly:
      \begin{equation}
        v_{s}(\mathbf{r})\xrightarrow[r \rightarrow \infty ]{} -\frac{1}{r} \nonumber
      \end{equation}
      $\qquad \qquad \Longrightarrow \quad $ additional electron feels a Coulomb-like potential \\
      $\qquad \qquad \Longrightarrow \quad $ Rydberg series of excited states \\
      $\qquad \qquad \Longrightarrow \quad $ necessary criterion for negative ion state fulfilled
    
      \vspace{3mm}
      In LDA:
      \begin{equation}
        v_{s}(\mathbf{r})\xrightarrow[r \rightarrow \infty ]{} \exp(-\alpha r) \nonumber
      \end{equation}
      $\qquad \qquad \Longrightarrow \quad $ not able to bind additional electron (negative ion) \\
      Same problem with GGA
      }
  \end{small}
}



\end{document}












\frame
{ 
  \frametitle{The Basic Kohn-Sham Equations}
  \begin{small}
    {\scriptsize
      \begin{itemize}
      \item So far: \\
        $\quad \quad $ H-K \alert{variational principle} $\quad \Longrightarrow \quad $ \\
        $\quad \quad $ exact GS density of many-particle system \\
        $\quad \quad $ \alert{Practically intractable !!}  
      %\item So far: The exact GS density of a many-particle system can in principle be determined by \alert{the variational principle} of Hohenberg and Kohn. This is, however, a \alert{practically intractable} problem.  
      \item Next step: \\
        $\quad \quad $ Kohn and Sham (1965): \alert{single-particle picture} \\
 $\quad \quad $ $\longrightarrow \quad $ equations solved \alert{selfconsistently} (iterative scheme) 
      \end{itemize}
    }
  \end{small}
}

\frame
{ 
  \frametitle{}
  \begin{small}
    {\scriptsize
      Hamiltonian of $N$ \emph{non-interacting} particles:
      \begin{equation}
        \hat{H}_{s}=\hat{T}+\hat{V}_{s} \nonumber
      \end{equation}
      Hohenberg and Kohn $\quad \Longrightarrow \quad $ $\exists \;$ unique energy functional 
      \begin{equation}
        E_{s}[n]=T_{s}[n]+\int v_{s}(\mathbf{r})n(\mathbf{r})d^{3}r \nonumber
      \end{equation}
      s. t. $\delta E_{s}[n]=0$ gives GS density $n_{s}(\mathbf{r})$ corresp. to $\hat{H}_{s}$


    }
  \end{small}
}

\frame
{ 
  \frametitle{}
  \begin{small}
    {\scriptsize


      \begin{theorem}
        Let 
        \begin{align}
          v_{s}(\mathbf{r})\quad &= \quad \text{local single-particle pot.}, \nonumber \\
          n(\mathbf{r})\quad &=\quad \text{GS density of interacting system}, \nonumber \\
          n_{s}(\mathbf{r})\quad &=\quad \text{GS density of non-interacting system} \nonumber 
        \end{align}
        $\Longrightarrow \quad $ for \alert{any interacting system},
        \begin{equation}
          \exists \;\; \text{ a }\;\; v_{s}(\mathbf{r}) \;\; \text{s. t.} \;\; n_{s}(\mathbf{r})=n(\mathbf{r})  \nonumber
        \end{equation}
      \end{theorem}
      %\pause
      Proof in book by Dreizler/Gross, Sec. 4.2 \\
%      In proof: $F_{HK}[n]$ replaced by $F_{L}[n]$ 
%      rove this important theorem, but in the proof it is necessary to replace the H-K  functional $F_{HK}[n]$ by the Lieb functional $F_{L}[n]$ in the expression of $v_{s}(\mathbf{r})$.
    }
  \end{small}
}

\frame
{ 
  \frametitle{}
  \begin{small}
    {\scriptsize
      Assume \structure{nondegenerate GS}. Then
      \begin{equation}
        n(\mathbf{r})=n_{s}(\mathbf{r})=\sum_{i=1}^{N}\left\vert \phi_{i}(\mathbf{r})\right\vert^{2}, \nonumber
      \end{equation}
      where $\phi_{i}(\mathbf{r})$ are determined by
      \begin{equation}
        \left( -\frac{\hbar^{2}}{2m}\nabla^{2}+v_{s}(\mathbf{r})\right)\phi_{i}(\mathbf{r})=\varepsilon_{i}\phi_{i}(\mathbf{r}), \qquad \varepsilon_{1}\leq \varepsilon_{2}\leq \dots \;. \nonumber
      \end{equation}
      %\pause

      \vspace{12mm}
      If $\exists $ $v_{s}(\mathbf{r})$, then H-K theorem gives \emph{uniqueness} of $v_{s}(\mathbf{r})$ \\
      Consequently, we may write
      \begin{equation}
        \phi_{i}(\mathbf{r})=\phi_{i}([n(\mathbf{r})]) \qquad \alert{\textbf{!!}} \nonumber
      \end{equation}
      
    }
  \end{small}
}

\frame
{ 
  \frametitle{}
  \begin{small}
    {\scriptsize
      Assume \\
      $\qquad \qquad \qquad \qquad \qquad $ $v_{0}(\mathbf{r}) =$ ext. potential \\
      $\qquad \qquad \qquad \qquad \qquad $ $n_{0}(\mathbf{r}) =$ GS density \\
      of \alert{interacting} system 
      \begin{itemize}
      \item Wanted: \structure{single-particle potential} $v_{s}(\mathbf{r})$ of \alert{non-interacting} system
      \end{itemize}
    }
  \end{small}
}

\frame
{ 
  \frametitle{Exchange-correlation functional}
  \begin{small}
    {\scriptsize
      Many-particle energy functional:
      \begin{align}
        E_{v_{0}}[n]&=F_{L}[n]+\int d^{3}v_{0}(\mathbf{r})n(\mathbf{r}) \nonumber \\
        &=\left(T_{s}[n]+\frac{1}{2}\iint d^{3}r d^{3}r' n(\mathbf{r})w(\mathbf{r},\mathbf{r}')n(\mathbf{r}')+E_{\mathrm{exc}}[n]\right) + \int d^{3}r v_{0}(\mathbf{r})n(\mathbf{r}) \nonumber 
      \end{align}
      Here \alert{exchange-correlation functional} defined: 
      \begin{equation}
        E_{\mathrm{exc}}[n]=F_{L}[n]-\frac{1}{2}\iint d^{3}r d^{3}r' n(\mathbf{r})w(\mathbf{r},\mathbf{r}')n(\mathbf{r}')-T_{s}[n] \nonumber
      \end{equation}
    }
  \end{small}
}

\frame
{ 
  \frametitle{}
  \begin{small}
    {\scriptsize
      The exchange-correlation functional defined: 
      \begin{equation}
        E_{\mathrm{exc}}[n]=F_{L}[n]-\frac{1}{2}\iint d^{3}r d^{3}r' n(\mathbf{r})w(\mathbf{r},\mathbf{r}')n(\mathbf{r}')-T_{s}[n] \nonumber
      \end{equation}
      \vspace{5mm}
      \alert{Explicit form} of $F_{L}[n]$ as functional of $n$ \alert{unknown}
      \begin{itemize}
        \item $E_{\mathrm{exc}}[n]$ unknown functional, must be approximated \\
          Otherwise, Kohn-Sham scheme exact 
      \end{itemize}
    }
  \end{small}
}

\frame
{ 
  \frametitle{}
  \begin{small}
    {\scriptsize
      %Side-remark:
      
      \begin{definition}
        Let $F:B\rightarrow \mathbb{R}$ be a \emph{functional} from  normed function space $B$ to real numbers $\mathbb{R}$. \\
        
        \vspace{2mm}
        The \alert{functional derivative} (G{\^a}teaux derivative) \\
        $\delta F[n]\equiv \delta F[n]/\delta n(\mathbf{r})$ is defined as
        \begin{equation}
          \frac{\delta F}{\delta n}[\varphi ]=\lim_{\varepsilon \rightarrow 0}\frac{F[n+\varepsilon \varphi ]-F[n]}{\varepsilon } \nonumber
        \end{equation}
        Another useful definition of $\delta F[n]$:
        \begin{equation}
          \left< \delta F[n],\varphi \right>=\frac{d}{d\varepsilon }F[n+\varepsilon \phi ]\Bigg\vert_{\varepsilon =0}, \nonumber
        \end{equation}
        where
        \begin{equation}
          \left< \delta F[n],\varphi \right>\equiv \int d\mathbf{r}(\delta F[n(\mathbf{r})])\varphi(\mathbf{r}), \nonumber
        \end{equation}
        $\varphi =$ test function 
      \end{definition}
    }
  \end{small}
}

\frame
{ 
  \frametitle{}
  \begin{small}
    {\scriptsize
      Let us \alert{derive} expression for \alert{single-particle potential} $v_{s}(\mathbf{r})$ of non-interacting system:
      
      \vspace{10mm}
      H-K variational principle:
      \begin{align} \label{eq:hkVar}
        0&=\delta E_{v_{0}}=E_{v_{0}}[n_{0}+\delta n]-E_{v_{0}}[n_{0}] \nonumber \\
        &=\delta T_{s}+\int d^{3}r\delta n(\mathbf{r})\left[v_{0}(\mathbf{r})+\int w(\mathbf{r},\mathbf{r}')d^{3}r'+v_{\mathrm{exc}}([n_{0}];\mathbf{r})\right],   
      \end{align}
      where exchange-coorelation potential 
      \begin{equation}
        v_{\mathrm{exc}}([n_{0}];\mathbf{r})=\frac{\delta E_{\mathrm{exc}}[n]}{\delta n(\mathbf{r})}\Bigg\vert_{n_{0}}, \nonumber
      \end{equation}
      $n_{0}(\mathbf{r}) =$ GS density
    }
  \end{small}
}

\frame
{ 
  \frametitle{}
  \begin{small}
    {\scriptsize
      $n_{0}(\mathbf{r})+\delta n(\mathbf{r})$ non-interacting $v$-representable $\quad \Longrightarrow \quad $ unique representation $\phi_{i,0}(\mathbf{r})+\delta \phi_{i}(\mathbf{r})$ 
      \vspace{10mm}
      
      \begin{align} \label{eq:dT}
        \delta T_{s}&=\sum_{i}^{N}\int d^{3}r\left[\delta \phi_{i}^{*}(\mathbf{r})\left(-\frac{\hbar^{2}}{2m}\nabla^{2}\right)\phi_{i,0}(\mathbf{r})+\phi_{i,0}^{*}(\mathbf{r})\left(-\frac{\hbar^{2}}{2m}\nabla^{2}\right)\delta \phi_{i}(\mathbf{r})\right] \nonumber \\
        &=\sum_{i}^{N}\int d^{3}r\left[\delta \phi_{i}^{*}(\mathbf{r})\left(-\frac{\hbar^{2}}{2m}\nabla^{2}\right)\phi_{i,0}(\mathbf{r})+\delta \phi_{i,0}^{*}(\mathbf{r})\left(-\frac{\hbar^{2}}{2m}\nabla^{2}\right)\phi_{i}(\mathbf{r})\right]
      \end{align}
      \put(38,60){\vector(0,1){20}}
      \put(30,45){Green's first identity}
      

    }
  \end{small}
}

\frame
{ 
  \frametitle{}
  \begin{small}
    {\scriptsize
      Green's first identity:
      \begin{equation}
        \int_{V}f\:\nabla^{2}g\:dV=\oint_{S}f(\nabla g\cdot n)\:dS-\int_{V}\nabla f\cdot \nabla g\:dV, \nonumber
      \end{equation}
      where $V\in \mathbb{R}^{3}$, $S\equiv \partial V\in \mathbb{R}^{2}$ and $f$, $g =$ arb. real scalar functions

      \vspace{10mm}
      Let surface $\partial V$ approach infinity w.r.t. origin, \\
      $\quad \quad $ assume $f,g \longrightarrow 0$ on $\partial V$, \\
      $\quad \quad $ Apply Green's first identity twice $\quad \Longrightarrow $
      \begin{align}
        \int_{V}f\:\nabla^{2}g\:dV&=0-\int_{V}\nabla f\cdot \nabla g\:dV \nonumber \\
        &=-\left(0-\int_{V}\nabla f\cdot \nabla g\:dV\right) \nonumber \\
        &=\int_{V}g\:\nabla^{2}f\:dV \nonumber
      \end{align}

    }
  \end{small}
}


\frame
{ 
  \frametitle{}
  \begin{small}
    {\scriptsize
      The orbitals $\phi_{i,0}(\mathbf{r})$ in Eq. (\ref{eq:dT}) satisfy
      \begin{equation}
        \left(-\frac{\hbar^{2}}{2m}\nabla^{2}+v_{s,0}(\mathbf{r})\right)\phi_{i,0}(\mathbf{r})=\varepsilon_{i}\phi_{i,0}(\mathbf{r}), \qquad \varepsilon_{1}\geq \varepsilon_{2} \geq \dots \;.
      \end{equation}
      Using this relation, we may rewrite Eq. (\ref{eq:dT}) as
      \begin{align} \label{eq:dT2}
        \delta T_{s}&=\sum_{i}^{N}\int d^{3}r\left[\delta \phi_{i}^{*}(\mathbf{r})\left(\varepsilon_{i}-v_{s,0}(\mathbf{r})\right)\phi_{i,0}(\mathbf{r})+\delta \phi_{i}(\mathbf{r})\left(\varepsilon_{i}-v_{s,0}(\mathbf{r})\right)\phi_{i}^{*}(\mathbf{r})\right] \nonumber \\
        &=\sum_{i=1}^{N}\varepsilon_{i}\int d^{3}r \delta \vert \phi_{i}(\mathbf{r})\vert^{2}-\sum_{i=1}^{N}\int d^{3}r v_{s,0}(\mathbf{r})\delta \vert \phi_{i}(\mathbf{r})\vert^{2}.
      \end{align}
   
    }
  \end{small}
}

\frame
{ 
  \frametitle{}
  \begin{small}
    {\scriptsize
      Since
      \begin{align}
        \int d^{3}r \delta\vert \phi_{i}(\mathbf{r})\vert^{2}&=\int d^{3}r \left[\vert \phi_{i,0}(\mathbf{r})+\delta \phi_{i,0}(\mathbf{r})\vert^{2}-\vert \phi_{i,0}(\mathbf{r})\vert^{2}\right] \nonumber \\
        &=1-1=0,
      \end{align}
      the first term of Eq. (\ref{eq:dT2}) vanishes, and we get
      \begin{equation} \label{eq:dT3}
        \delta T_{s}=-\int d^{3}r v_{s,0}(\mathbf{r})\delta n(\mathbf{r}).
      \end{equation}
      Combine Eqs. (\ref{eq:hkVar}) and (\ref{eq:dT3}): $\quad \Longrightarrow \quad $ total single-particle potential: 
      \begin{equation}
        v_{s,0}(\mathbf{r})=v_{0}(\mathbf{r})+\int d^{3}r' w(\mathbf{r},\mathbf{r}')n_{0}(\mathbf{r}')+v_{\mathrm{exc}}([n_{0}];\mathbf{r})
      \end{equation}

    }
  \end{small}
}

\frame
{ 
  \frametitle{The Kohn-Sham scheme I}
  \begin{small}
    {\scriptsize
      The \alert{classic Kohn-Sham} scheme:
      \begin{equation}
         \left(-\frac{\hbar^{2}}{2m}\nabla^{2}+v_{s,0}(\mathbf{r})\right)\phi_{i,0}(\mathbf{r})=\varepsilon_{i}\phi_{i,0}(\mathbf{r}), \qquad \varepsilon_{1}\geq \varepsilon_{2} \geq \dots \;, \nonumber
      \end{equation}
      where
      \begin{equation}
        v_{s,0}(\mathbf{r})=v_{0}(\mathbf{r})+\int d^{3}r' w(\mathbf{r},\mathbf{r}')n_{0}(\mathbf{r}')+v_{\mathrm{exc}}([n_{0}];\mathbf{r}) \nonumber 
      \end{equation}
      The density calculated as 
      \begin{equation}
        n_{0}(\mathbf{r})=\sum_{i=1}^{N}\vert \phi_{i,0}(\mathbf{r})\vert^{2}, \nonumber
      \end{equation}
      Equation \alert{solved selfconsistently} 
      %\pause
      
      Total energy: 
      \begin{equation}
        E=\sum_{i=1}^{N}\varepsilon_{i}-\frac{1}{2}\int d^{3}r d^{3}r'n(\mathbf{r})w(\mathbf{r},\mathbf{r}')n(\mathbf{r}')+E_{\mathrm{exc}}[n]-\int d^{3}r v_{\mathrm{exc}}([n];\mathbf{r})n(\mathbf{r}) \nonumber
      \end{equation}
    }
  \end{small}
}


\frame
{ 
  \frametitle{The Kohn-Sham scheme II}
  \begin{small}
    {\scriptsize
      \alert{Kohn-Sham} scheme for systems with \alert{degenerate} GS:
            \begin{equation}
         \left(-\frac{\hbar^{2}}{2m}\nabla^{2}+v_{s,0}(\mathbf{r})\right)\phi_{i,0}(\mathbf{r})=\varepsilon_{i}\phi_{i,0}(\mathbf{r}), \qquad \varepsilon_{1}\geq \varepsilon_{2} \geq \dots \;, \nonumber
      \end{equation}
      where
      \begin{equation}
        v_{s,0}(\mathbf{r})=v_{0}(\mathbf{r})+\int d^{3}r' w(\mathbf{r},\mathbf{r}')n_{0}(\mathbf{r}')+v_{\mathrm{exc}}([n_{0}];\mathbf{r}) \nonumber 
      \end{equation}
      and
      \begin{align}
        v_{\mathrm{exc}}([n];\mathbf{r})&=\frac{\delta E_{\mathrm{exc}}[n]}{\delta n(\mathbf{r})} \nonumber \\
        &=\frac{\delta }{\delta n(\mathbf{r})}\left(F_{L}[n]-\frac{1}{2}\iint d^{3}r d^{3}r' n(\mathbf{r})w(\mathbf{r},\mathbf{r}')n(\mathbf{r}')-T_{L}[n]\right) \nonumber 
      \end{align}

    }
  \end{small}
}

\frame
{ 
  \frametitle{The Kohn-Sham scheme II}
  \begin{small}
    {\scriptsize
      Density of degen. K-S scheme:
      \begin{equation}
        n_{0}(\mathbf{r})=\sum_{i=1}^{N}\gamma_{i}\vert \phi_{i,0}(\mathbf{r})\vert^{2}, \nonumber
      \end{equation}
      occupation numbers $\gamma_{i}$ satisfy 
      \begin{align}
        \gamma_{i}=1:\; &\varepsilon_{i}<\mu \nonumber \\
        0\leq \gamma_{i}\leq 1:\; &\varepsilon_{i}=\mu \nonumber \\
        \gamma_{i}=0:\; &\varepsilon_{i}>\mu \nonumber
      \end{align}
      and
      \begin{equation}
        \sum_{i=1}^{N}\gamma_{i}=N \nonumber
      \end{equation}
    }
  \end{small}
}

\frame
{ 
  \frametitle{Exchange Energy and Correlation Energy}
  \begin{small}
    {\scriptsize

      \structure{Hartree-Fock} equation:
      \begin{align}
        \left(-\frac{\hbar^{2}}{2m}\nabla^{2}+v_{0}(\mathbf{r})+\int d^{3}r'w(\mathbf{r},\mathbf{r}')n(\mathbf{r}')\right)\phi_{k}(\mathbf{r})& \nonumber \\
        \underbrace{-\sum_{l=1}^{N}\int d^{3}r'\phi_{l}^{*}(\mathbf{r}')w(\mathbf{r},\mathbf{r}')\phi_{k}(\mathbf{r}')\phi_{l}(\mathbf{r})}_{\text{exchange term}}
        &=\varepsilon_{k}\phi_{k}(\mathbf{r}), \nonumber
      \end{align}
      
      %\begin{itemize}
      $\qquad \qquad $ \alert{Non-local} exchange term (Pauli exclusion principle)
      %\end{itemize}

      \vspace{5mm}     
      \structure{Kohn-Sham} equation:

      \begin{equation}
         \left(-\frac{\hbar^{2}}{2m}\nabla^{2}+v_{0}(\mathbf{r})
           +\int d^{3}r'w(\mathbf{r},\mathbf{r}')n(\mathbf{r}')
           +\underbrace{v_{\mathrm{exc}}([n];\mathbf{r})}_{\text{exchange + correlation}}
         \right)\phi_{k}(\mathbf{r})=\varepsilon_{k}\phi_{k}(\mathbf{r}), \nonumber
       \end{equation}
       %\begin{itemize}
       $\qquad \qquad $  \alert{Local} exchange-correlation term
       %\end{itemize}
      
    }
  \end{small}
}

\frame
{ 
  \frametitle{}
  \begin{small}
    {\scriptsize
      Exchange-correlation energy $=$ \structure{Exchange energy} $+$ \alert{Correlation energy}
      \begin{equation}
        E_{\mathrm{exc}}[n]=E_{x}[n]+E_{c}[n] \nonumber
      \end{equation}
      From earlier: 
      \begin{equation}
        E_{\mathrm{exc}}[n]=F_{L}[n]-T_{s}[n]-\frac{1}{2}\iint d^{3}r d^{3}r' n(\mathbf{r})w(\mathbf{r},\mathbf{r}')n(\mathbf{r}') \nonumber
      \end{equation}
      \vspace{10mm}
      \begin{center}
        We want to show: $\quad $$E_{c}[n]\leq 0$
      \end{center}
    }
  \end{small}
}

\frame
{ 
  \frametitle{}
  \begin{small}
    {\scriptsize
      Here we have (assume $F_{L}[n]=F_{LL}[n]$)
      \begin{align}
        F_{L}[n]&\equiv \inf_{\Psi \rightarrow n}\bra{\Psi }\hat{T}+\hat{W}\ket{\Psi } \nonumber \\
        &=\bra{\Psi_{n}^{min}}\hat{T}+\hat{W}\ket{\Psi_{n}^{min}}, \nonumber 
      \end{align}
      and 
      \begin{align}
        T_{s}[n]&\equiv \inf_{\Psi \rightarrow n}\bra{\Psi }\hat{T}\ket{\Psi }=\bra{\Phi_{n}^{min}}\hat{T}\ket{\Phi_{n}^{min}}, \nonumber 
      \end{align}
      $\Psi = $ normalized, antisymm. $N$-particle wavefunction, \\
      $\Phi_{n}^{min}$ lin. komb. of Slater determinants of \\
      single-particle orbitals $\psi_{i}(r_{j})$ 
    }
  \end{small}
}

\frame
{ 
  \frametitle{}
  \begin{small}
    {\scriptsize
      Eq. (4.35) in J. M. Thijssen: \emph{Computational Physics}:
      \begin{align}
        \bra{\Phi_{n}^{min}}\hat{W}\ket{\Phi_{n}^{min}}&=\frac{1}{2}\sum_{k,l}\bigg[ \iint d^{3}r d^{3}r' n(\mathbf{r})w(\mathbf{r},\mathbf{r}')n(\mathbf{r}') \nonumber \\
        &-\iint d^{3}r d^{3}r' \psi_{l}^{*}(\mathbf{r})\psi_{l}(\mathbf{r}')w(\mathbf{r},\mathbf{r}')\psi_{k}^{*}(\mathbf{r}')\psi_{k}(\mathbf{r})\bigg] \nonumber
      \end{align}
      By definition,
      \begin{equation}
        E_{x}[n]\equiv -\frac{1}{2}\sum_{k,l}\iint d^{3}r d^{3}r' \psi_{l}^{*}(\mathbf{r})\psi_{l}(\mathbf{r}')w(\mathbf{r},\mathbf{r}')\psi_{k}^{*}(\mathbf{r}')\psi_{k}(\mathbf{r})  \nonumber
      \end{equation}
    }
  \end{small}
}

\frame
{ 
  \frametitle{}
  \begin{small}
    {\scriptsize
      Using expressions from previous pages gives
      \begin{align}
        E_{c}[n]&=E_{\mathrm{exc}}[n]-E_{x}[n] \nonumber \\
        &=F_{L}[n]-T_{s}[n]-\frac{1}{2}\iint d^{3}r d^{3}r' n(\mathbf{r})w(\mathbf{r},\mathbf{r}')n(\mathbf{r}') \nonumber \\
        &+\frac{1}{2}\sum_{k,l}\iint d^{3}r d^{3}r' \psi_{l}^{*}(\mathbf{r})\psi_{l}(\mathbf{r}')w(\mathbf{r},\mathbf{r}')\psi_{k}^{*}(\mathbf{r}')\psi_{k}(\mathbf{r}) \nonumber \\
        &=\bra{\Psi_{n}^{min}}\hat{T}+\hat{W}\ket{\Psi_{n}^{min}}-\bra{\Phi_{n}^{min}}\hat{T}+\hat{W}\ket{\Phi_{n}^{min}} \nonumber
      \end{align}

      Since
      \begin{equation}
        \bra{\Psi_{n}^{min}}\hat{T}+\hat{W}\ket{\Psi_{n}^{min}}=\inf_{\Psi \rightarrow n}\bra{\Psi }\hat{T}+\hat{W}\ket{\Psi }, \nonumber
      \end{equation}
      we see that  
      \begin{equation}
        E_{c}[n]\leq 0 \nonumber
      \end{equation}
    }
  \end{small}
}


\frame
{ 
  \frametitle{Gradient expansion}
  \begin{small}
    {\scriptsize
      The \alert{gradient expansion approximation (GEA)} -- a natural extension of LDA ??
      
      \vspace{2mm}
      \structure{Taylor expansion} of $E_{\mathrm{exc}}[n]$ \\
      around homogeneous electron gas (HEG) \\
      density $n_{0}$ $\quad $ ($(n-n_{0})/n_{0}\ll 1$):
      
     \begin{equation}
       E_{\mathrm{exc}}[n]=E_{\mathrm{exc}}[n_{0}]+\sum_{m=1}^{\infty }\frac{1}{m!}\int d^{3m}r \frac{\delta^{m}E_{\mathrm{exc}}}{\delta n(\mathbf{r}_{1})\dots \delta n(\mathbf{r}_{m})}\Bigg\vert_{n=n_{0}}\delta n(\mathbf{r}_{1})\dots \delta n(\mathbf{r}_{m}) \nonumber
     \end{equation}
    }
  \end{small}
}

\frame
{ 
  \frametitle{Gradient expansion}
  \begin{small}
    {\scriptsize
      Shown in article by van Leeuwen: \\
      \vspace{2mm}
      Expansion can be written
      \begin{align}
        E_{\mathrm{exc}}[n]&=E_{\mathrm{exc}}^{LDA}[n]+\int d^{3}r g_{1}(n(\mathbf{r}))(\nabla n(\mathbf{r}))^{2} \nonumber \\
        &+\int d^{3}r g_{2}(n(\mathbf{r}))(\nabla^{2}n(\mathbf{r}))^{2}+\dots , \nonumber
      \end{align}
      $g_{i}(n)$ uniquely determined by the density response functions of a HEG
    }
  \end{small}
}

\frame
{ 
  \frametitle{}
  \begin{small}
    {\scriptsize
      \begin{center}
        Gradient expansion in principle exact, \alert{provided series converges}
        
        \vspace{10mm}
        \begin{tabular}{l l}
          \hline
          Metallic systems: & good convergence \\
          Insulators: & bad convergence \\
          Finite systems: & bad convergece \\
          \hline
        \end{tabular}
      \end{center}
    }
  \end{small}
}

\frame
{ 
  \frametitle{}
  \begin{small}
    {\scriptsize
      \alert{\textbf{Caution!}}
      
      \vspace{5mm}
      \begin{center}
        Numerical tests show: \\
        Inclusion of second-order gradient term \\
        may give a considerably worse $E_{\mathrm{exc}}[n]$ than $E_{\mathrm{exc}}^{LDA}[n]$
      \end{center}

      %%\pause
      \vspace{5mm}
      $\qquad \qquad \qquad \qquad \qquad \qquad \qquad \qquad \qquad \qquad \qquad \qquad \qquad \qquad \quad $ \structure{Why?}
    }
  \end{small}
}

\frame
{ 
  \frametitle{}
  \begin{small}
    {\scriptsize
      %\begin{center}
      $E_{\mathrm{exc}}^{LDA}[n]$ $\quad $ provides rather realistic results $\quad $ for atoms, molecules, and solids

      \vspace{7mm}
      But: second-order term (\alert{next systematic correction} \\
      $\qquad \qquad \qquad \qquad \qquad \qquad \qquad $ for slowly-varying densities) \alert{makes} $E_{\mathrm{exc}}$ \alert{worse}
      %\end{center}
    }
  \end{small}
}

\frame
{ 
  \frametitle{}
  \begin{small}
    {\scriptsize
      Why does gradient expansion fail?
      \begin{enumerate}
        \item Realistic electron densities not very close to slowly-varying limit
        \item LDA: $\quad $ xc hole is the hole of \structure{a possible physical system} \\
          $\qquad \qquad $ $\Longrightarrow $ $\quad $ satisfies exact constraints \\
          GEA: $\quad $ xc hole \structure{not physical} \\
          $\qquad \qquad $ $\Longrightarrow $ $\quad $ does not satisfy constraints
          
      \end{enumerate}
    }
  \end{small}
}

\frame
{ 
  \frametitle{}
  \begin{small}
    {\scriptsize
      Example of constraints:
      
      %\vspace{10mm}
      \begin{center}
        \begin{tabular}{c c c}
          Physical constraint & LDA & GEA \\
          \hline
          $E_{c}<0$ & $<0$ & $>0$ \\
          $E_{x}<0$ & $<0$ & not restricted \\
          $\int h_{\mathrm{exc}}(\mathbf{r}_{1};\mathbf{r}_{2})d\mathbf{r}_{2}=-1$ & $-1$ & not restricted \\
          \hline
        \end{tabular}
      \end{center}
      
      \vspace{5mm}
      $\qquad \qquad \qquad \qquad \qquad \qquad \qquad \qquad $ $\Longrightarrow $ $\quad $ \alert{Wrong behaviour} of GEA
    }
  \end{small}
}



\frame
{ 
  \frametitle{The Generalized Gradient Approximation}
  \begin{small}
    {\scriptsize
      Method: \alert{Enforce} physical \alert{restrictions} for the xc hole\\
      $\qquad \qquad $ $\Longrightarrow $ $\quad $ Generalized gradient approximation (GGA):
      \begin{equation}
        E_{\mathrm{exc}}^{GGA}[n_{\uparrow },n_{\downarrow }]=\int d^{3}r f(n_{\uparrow },n_{\downarrow },\nabla n_{\uparrow },\nabla n_{\downarrow }) \nonumber
      \end{equation}
      
      
      \begin{itemize}
      \item $f(n_{\uparrow },n_{\downarrow },\nabla n_{\uparrow },\nabla n_{\downarrow })$ not unique, \\ 
        but formal features of LDA $\quad \Longrightarrow \quad $ constraints
      \item GGA-functionals with/without semiempirical parameters
      \item Successful in quantum chemistry 
      \item No systematic approach to improve GGA-functionals
      \end{itemize}
    }
  \end{small}
}

\frame
{ 
  \frametitle{}
  \begin{small}
    {\scriptsize
      \alert{Typical errors} for atoms, molecules, and solids (Perdew/Kurth):
      \begin{center}
        \begin{tabular}{c c c}
          \hline
          Property & LDA & GGA \\
          \hline
          $E_{x}$ & $5\% $ (not negative enough) & $0.5\% $\\
          $E_{c}$ & $100\% $ (too negative) & $5\% $\\
          bond length & $1\% $ (too short) & $1\% $ (too long) \\
          structure & overly favours close packing & more correct \\
          energy barrier & $100\% $ (too low) & $30\% $ (too low) \\
          \hline
        \end{tabular}
      \end{center}
      \vspace{2mm}
      \begin{itemize}
      \item GGA in most cases better than LDA  
      \item Typically cancellation of errors between $E_{x}$ and $E_{c}$
      \item "Energy barrier" = barrier to a chemical reaction
      \end{itemize}
    }
  \end{small}
}

\frame
{ 
  \frametitle{}
  \begin{small}
    {\scriptsize
      Situations where GGA fails:
      %\begin{itemize}
      
      \vspace{3mm}
      $\qquad \qquad $ Unaccurate results for \textbf{heavy elements} \\
            
      \vspace{2mm}
      $\qquad \qquad $ Does not predict existence of \textbf{negative ions} \\
            
      \vspace{2mm}
      $\qquad \qquad $ Fails to reproduce \textbf{dispersion forces} ($\approx $ van der Waals forces) \\
            
      \vspace{2mm}
      $\qquad \qquad $ Can not describe properly \textbf{strongly correlated systems} \\
      %\end{itemize}
    }
  \end{small}
}

\frame
{ 
  \frametitle{}
  \begin{small}
    {\scriptsize
      GGA gives unaccurate results for \alert{\textbf{heavy elements}}:
      
      \vspace{3mm}
      Gold (Au):

      \begin{center}
        \begin{tabular}{c c c}
          \hline
          $E_{\mathrm{exc}}[n]$ & Equilibrium & Cohesive \\
                    & lattice constant & energy \\
          \hline
           LDA & \structure{\textbf{7.68}} & \textbf{4.12} \\
           relativistic LDA & 7.68 & 4.09 \\
           GGA & \alert{\textbf{7.87}} & \textbf{2.91} \\
           relativistic GGA & 7.88 & 2.89 \\
           experiment & \structure{\textbf{7.67}} & \textbf{3.78}\\
          \hline
        \end{tabular}
      \end{center}
      
      \begin{itemize}
      \item Here: LDA better than GGA
      \item Problem not due to relativistic effects
      \item GGA: problems with high angular momenta \\
        (higher ion charge $\quad \Longrightarrow \quad $ higher electron angular momentum)
      \end{itemize}
    }
  \end{small}
}

\frame
{ 
  \frametitle{}
  \begin{small}
    {\scriptsize
      GGA does not predict existence of \alert{\textbf{negative ions}}:
      
      \vspace{3mm}
      For neutral atoms exactly:
      \begin{equation}
        v_{s}(\mathbf{r})\xrightarrow[r \rightarrow \infty ]{} -\frac{1}{r} \nonumber
      \end{equation}
      $\qquad \qquad \Longrightarrow \quad $ additional electron feels a Coulomb-like potential \\
      $\qquad \qquad \Longrightarrow \quad $ Rydberg series of excited states \\
      $\qquad \qquad \Longrightarrow \quad $ necessary criterion for negative ion state fulfilled
    
      \vspace{3mm}
      In LDA:
      \begin{equation}
        v_{s}(\mathbf{r})\xrightarrow[r \rightarrow \infty ]{} \exp(-\alpha r) \nonumber
      \end{equation}
      $\qquad \qquad \Longrightarrow \quad $ not able to bind additional electron (negative ion) \\
      Same problem with GGA
      }
  \end{small}
}


\end{document}












