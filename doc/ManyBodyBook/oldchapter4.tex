
\chapter{Hartree-Fock theory}

\section{Introduction}

\section{Calculus of variations}


The calculus of variations involves 
problems where the quantity to be minimized or maximized is an integral. 

In the general case we have an integral of the type
\[ E[\Phi]= \int_a^b f(\Phi(x),\frac{\partial \Phi}{\partial x},x)dx,\]
where $E$ is the quantity which is sought minimized or maximized.
The problem is that although $f$ is a function of the variables $\Phi$, $\partial \Phi/\partial x$ and $x$, the exact dependence of
$\Phi$ on $x$ is not known.  This means again that even though the integral has fixed limits $a$ and $b$, the path of integration is
not known. In our case the unknown quantities are the single-particle wave functions and we wish to choose an integration path which makes
the functional $E[\Phi]$ stationary. This means that we want to find minima, or maxima or saddle points. In physics we search normally for minima.
Our task is therefore to find the minimum of $E[\Phi]$ so that its variation $\delta E$ is zero  subject to specific
constraints. In our case the constraints appear as the integral which expresses the orthogonality of the  single-particle wave functions.
The constraints can be treated via the technique of Lagrangian multipliers

We assume the existence of an optimum path, that is a path for which $E[\Phi]$ is stationary. There are infinitely many such paths.
The difference between two paths $\delta\Phi$ is called the variation of $\Phi$.

We call the variation $\eta(x)$  and it is scaled by a factor $\alpha$.  The function $\eta(x)$ is arbitrary except
for 
\[ 
\eta(a)=\eta(b)=0,
\]
and we assume that we can model the change in $\Phi$ as 
\[
\Phi(x,\alpha) = \Phi(x,0)+\alpha\eta(x),
\]
and 
\[
\delta\Phi = \Phi(x,\alpha) -\Phi(x,0)=\alpha\eta(x).
\]

We choose $\Phi(x,\alpha=0)$ as the unkonwn path  that will minimize $E$.  The value 
$\Phi(x,\alpha\ne 0)$  describes a neighbouring path. 
 
We have
\[ E[\Phi(\alpha)]= \int_a^b f(\Phi(x,\alpha),\frac{\partial \Phi(x,\alpha)}{\partial x},x)dx.\] 
In the slides I will use  the shorthand
\[
\Phi_x(x,\alpha) = \frac{\partial \Phi(x,\alpha)}{\partial x}.
\]
In our case $a=0$ and $b=\infty$ and we know the value of the wave function.

The condition for an extreme of
\[ E[\Phi(\alpha)]= \int_a^b f(\Phi(x,\alpha),\Phi_x(x,\alpha),x)dx,\]
is
\[
\left[\frac{\partial  E[\Phi(\alpha)]}{\partial x}\right]_{\alpha=0} =0.\]
The $\alpha$ dependence is contained in $\Phi(x,\alpha)$ and $\Phi_x(x,\alpha)$ meaning that
\[
\left[\frac{\partial  E[\Phi(\alpha)]}{\partial \alpha}\right]=\int_a^b \left( \frac{\partial f}{\partial \Phi}\frac{\partial \Phi}{\partial \alpha}+\frac{\partial f}{\partial \Phi_x}\frac{\partial \Phi_x}{\partial \alpha}        \right)dx.\]
We have defined 
\[
\frac{\partial \Phi(x,\alpha)}{\partial \alpha}=\eta(x)
\]
and thereby 
\[
\frac{\partial \Phi_x(x,\alpha)}{\partial \alpha}=\frac{d(\eta(x))}{dx}.
\]
Using
\[
\frac{\partial \Phi(x,\alpha)}{\partial \alpha}=\eta(x),
\]
and 
\[
\frac{\partial \Phi_x(x,\alpha)}{\partial \alpha}=\frac{d(\eta(x))}{dx},
\]
in the integral gives 
\[
\left[\frac{\partial  E[\Phi(\alpha)]}{\partial \alpha}\right]=\int_a^b \left( \frac{\partial f}{\partial \Phi}\eta(x)+\frac{\partial f}{\partial \Phi_x}\frac{d(\eta(x))}{dx}        \right)dx.\]
Integrate the second term by parts

\[
\int_a^b \frac{\partial f}{\partial \Phi_x}\frac{d(\eta(x))}{dx}dx =\eta(x)\frac{\partial f}{\partial \Phi_x}|_a^b-
\int_a^b \eta(x)\frac{d}{dx}\frac{\partial f}{\partial \Phi_x}dx, 
\]
and since the first term dissappears due to $\eta(a)=\eta(b)=0$, we obtain
\[
\left[\frac{\partial  E[\Phi(\alpha)]}{\partial \alpha}\right]=\int_a^b \left( \frac{\partial f}{\partial \Phi}-\frac{d}{dx}\frac{\partial f}{\partial \Phi_x}
\right)\eta(x)dx=0.\]
\[
\left[\frac{\partial  E[\Phi(\alpha)]}{\partial \alpha}\right]=\int_a^b \left( \frac{\partial f}{\partial \Phi}-\frac{d}{dx}\frac{\partial f}{\partial \Phi_x}
\right)\eta(x)dx=0,\]
can also be written as 
\[
\alpha\left[\frac{\partial  E[\Phi(\alpha)]}{\partial \alpha}\right]_{\alpha=0}=\int_a^b \left( \frac{\partial f}{\partial \Phi}-\frac{d}{dx}\frac{\partial f}{\partial \Phi_x}
\right)\delta\Phi(x)dx=\delta E = 0.\]

The condition for a stationary value is thus a partial differential equation
\[
\frac{\partial f}{\partial \Phi}-\frac{d}{dx}\frac{\partial f}{\partial \Phi_x}=0,\]
known as Euler's equation.
Can easily be generalized to more variables.


Consider a function of three independent variables $f(x,y,z)$ . For the function $f$ to be an 
extreme we have
\[
df=0.
\]
A necessary and sufficient condition is
\[
\frac{\partial f}{\partial x} =\frac{\partial f}{\partial y}=\frac{\partial f}{\partial z}=0,
\]
due to 
\[
df = \frac{\partial f}{\partial x}dx+\frac{\partial f}{\partial y}dy+\frac{\partial f}{\partial z}dz.
\]
In physical problems the variables $x,y,z$ are often subject to constraints (in our case $\Phi$ and the orthogonality constraint)
so that they are no longer all independent. It is possible at least in principle to use each constraint to eliminate one variable
and to proceed with a new and smaller set of independent varables.

The use of so-called Lagrangian  multipliers is an alternative technique  when the elimination of
of variables is incovenient or undesirable.  Assume that we have an equation of constraint on the variables $x,y,z$
\[
\phi(x,y,z) = 0,
\]
 resulting in
\[
d\phi = \frac{\partial \phi}{\partial x}dx+\frac{\partial \phi}{\partial y}dy+\frac{\partial \phi}{\partial z}dz =0.
\]
Now we cannot set anymore 
\[
\frac{\partial f}{\partial x} =\frac{\partial f}{\partial y}=\frac{\partial f}{\partial z}=0,
\]
if $df=0$ is wanted 
because there are now only two independent variables!  Assume $x$ and $y$ are the independent variables.
Then $dz$ is no longer arbitrary. 

However, we can add to
\[
df = \frac{\partial f}{\partial x}dx+\frac{\partial f}{\partial y}dy+\frac{\partial f}{\partial z}dz,
\]
a multiplum of $d\phi$, viz. $\lambda d\phi$, resulting  in

\[
df+\lambda d\phi = (\frac{\partial f}{\partial z}+\lambda\frac{\partial \phi}{\partial x})dx+(\frac{\partial f}{\partial y}+\lambda\frac{\partial \phi}{\partial y})dy+
(\frac{\partial f}{\partial z}+\lambda\frac{\partial \phi}{\partial z})dz =0.
\]
Our multiplier is chosen so that
\[
\frac{\partial f}{\partial z}+\lambda\frac{\partial \phi}{\partial z} =0.
\]

However, we took $dx$ and $dy$ as to be arbitrary and thus we must have
\[
\frac{\partial f}{\partial x}+\lambda\frac{\partial \phi}{\partial x} =0,
\]
and
\[
\frac{\partial f}{\partial y}+\lambda\frac{\partial \phi}{\partial y} =0.
\]
When all these equations are satisfied, $df=0$.  We have four unknowns, $x,y,z$ and
$\lambda$. Actually we want only $x,y,z$, $\lambda$ need not to be determined, it is therefore often called
Lagrange's undetermined multiplier. 
If we have a set of constraints $\phi_k$ we have the equations
\[
\frac{\partial f}{\partial x_i}+\sum_k\lambda_k\frac{\partial \phi_k}{\partial x_i} =0.
\]

Let us specialize to the expectation value of the energy for one particle in three-dimensions.
This expectation value reads
\[
  E=\int dxdydz \psi^*(x,y,z) \hat{H} \psi(x,y,z),
\]
with the constraint
\[
 \int dxdydz \psi^*(x,y,z) \psi(x,y,z)=1,
\]
and a Hamiltonian
\[
\hat{H}=-\frac{1}{2}\nabla^2+V(x,y,z).
\]
I will skip the variables $x,y,z$ below, and write for example $V(x,y,z)=V$.

The integral involving the kinetic energy can be written as, if we assume periodic boundary conditions or that the function $\psi$ vanishes
strongly for large values of $x,y,z$, 
 \[
  \int dxdydz \psi^* \left(-\frac{1}{2}\nabla^2\right) \psi dxdydz = \psi^*\nabla\psi|+\int dxdydz\frac{1}{2}\nabla\psi^*\nabla\psi.
\]
Inserting this expression into the expectation value for the energy and taking the variational minimum  we obtain
\[
\delta E = \delta \left\{\int dxdydz\left( \frac{1}{2}\nabla\psi^*\nabla\psi+V\psi^*\psi\right)\right\} = 0.
\]

The constraint appears in integral form as 
\[
 \int dxdydz \psi^* \psi=\mathrm{constant},
\]
and multiplying with a Lagrangian multiplier $\lambda$ and taking the variational minimum we obtain the final variational equation
\[
\delta \left\{\int dxdydz\left( \frac{1}{2}\nabla\psi^*\nabla\psi+V\psi^*\psi-\lambda\psi^*\psi\right)\right\} = 0.
\]
Introducing the function  $f$
\[
  f =  \frac{1}{2}\nabla\psi^*\nabla\psi+V\psi^*\psi-\lambda\psi^*\psi=
\frac{1}{2}(\psi^*_x\psi_x+\psi^*_y\psi_y+\psi^*_z\psi_z)+V\psi^*\psi-\lambda\psi^*\psi,
\]
where we have skipped the dependence on $x,y,z$ and introduced the shorthand $\psi_x$, $\psi_y$ and $\psi_z$  for the various derivatives.

For $\psi^*$ the Euler  equation results in
\[
\frac{\partial f}{\partial \psi^*}- \frac{\partial }{\partial x}\frac{\partial f}{\partial \psi^*_x}-\frac{\partial }{\partial y}\frac{\partial f}{\partial \psi^*_y}-\frac{\partial }{\partial z}\frac{\partial f}{\partial \psi^*_z}=0,
\] 
which yields 
\[
    -\frac{1}{2}(\psi_{xx}+\psi_{yy}+\psi_{zz})+V\psi=\lambda \psi.
\]
We can then identify the  Lagrangian multiplier as the energy of the system. Then the last equation is 
nothing but the standard 
Schr\"odinger equation and the variational  approach discussed here provides 
a powerful method for obtaining approximate solutions of the wave function.

\section{Finding the Hartree-Fock functional $E[\Phi]$}

We rewrite our Hamiltonian 
\[
  \hat{H} = -\sum_{i=1}^N \frac{1}{2} \nabla^2_i 
  - \sum_{i=1}^N \frac{Z}{r_i} + \sum_{i<j}^N \frac{1}{r_{ij}},
\]
as
\[
    \hat{H} = \hat{H_0} + \hat{H_I} 
    = \sum_{i=1}^N\hat{h_i} + \sum_{i<j=1}^N\frac{1}{r_{ij}},
\]

\[
  \hat{h}_0(x_i) = - \frac{1}{2} \nabla^2_i - \frac{Z}{r_i}.
\]

Let us denote the ground state energy by $E_0$. According to the
variational principle we have
\begin{equation*}
  E_0 \le E[\Phi] = \int \Phi^*\hat{H}\Phi d\mathbf{\tau}
\end{equation*}
where $\Phi$ is a trial function which we assume to be normalized
\begin{equation*}
  \int \Phi^*\Phi d\mathbf{\tau} = 1,
\end{equation*}
where we have used the shorthand $d\mathbf{\tau}=dx_1dx_2\dots dx_N$.

In the Hartree-Fock method the trial function is the Slater
determinant which can be rewritten as 
\[
  \Psi(x_1,x_2,\dots,x_N,\alpha,\beta,\dots,\nu) = \frac{1}{\sqrt{N!}}\sum_{P} (-)^PP\psi_{\alpha}(x_1)
    \psi_{\beta}(x_2)\dots\psi_{\nu}(x_N)=\sqrt{N!}{\cal A}\Phi_H,
\]
where we have introduced the anti-symmetrization operator ${\cal A}$ defined by the 
summation over all possible permutations of two eletrons.
It is defined as
\[
  {\cal A} = \frac{1}{N!}\sum_{P} (-)^PP,
\]
with the the Hartree-function given by the simple product of all possible single-particle function (two for helium, four for beryllium and ten for
neon)
\[
  \Phi_H(x_1,x_2,\dots,x_N,\alpha,\beta,\dots,\nu) =
  \psi_{\alpha}(x_1)
    \psi_{\beta}(x_2)\dots\psi_{\nu}(x_N).
\]


Both $\hat{H}_0$ and $\hat{H}_I$ are invariant under electron
permutations, and hence commute with ${\cal A}$
\[
  [H_0,{\cal A}] = [H_I,{\cal A}] = 0.
\]
Furthermore, ${\cal A}$ satisfies
\[
  {\cal A}^2 = {\cal A},
\]
since every permutation of the Slater
determinant reproduces it.

Our functional is written as 
\[
  E[\Phi] = \sum_{\mu=1}^N \int \psi_{\mu}^*(x_i)\hat{h}_0(x_i)\psi_{\mu}(x_i) dx_i 
  + \frac{1}{2}\sum_{\mu=1}^N\sum_{\nu=1}^N
   \left[ \int \psi_{\mu}^*(x_i)\psi_{\nu}^*(x_j)\frac{1} 
    {r_{ij}}\psi_{\mu}(x_i)\psi_{\nu}(x_j)
    dx_ix_j \right.
\]
\[ \left.
  - \int \psi_{\mu}^*(x_i)\psi_{\nu}^*(x_j)
  \frac{1}{r_{ij}}\psi_{\nu}(x_i)\psi_{\mu}(x_j)
  dx_ix_j\right]
\]
The more compact version is
\[
  E[\Phi] 
  = \sum_{\mu=1}^N \langle \mu | \hat{h}_0 | \mu\rangle+ \frac{1}{2}\sum_{\mu=1}^N\sum_{\nu=1}^N\left[\langle \mu\nu |\frac{1}{r_{ij}}|\mu\nu\rangle-\langle \mu\nu |\frac{1}{r_{ij}}|\nu\mu\rangle\right].
\]

If we generalize the Euler-Lagrange equations to more variables 
and introduce $N^2$ Lagrange multipliers which we denote by 
$\epsilon_{\mu\nu}$, we can write the variational equation for the functional of $E$
\[
  \delta E - \sum_{{\mu}=1}^N\sum_{{\nu}=1}^N \epsilon_{\mu\nu} \delta
  \int \psi_{\mu}^* \psi_{\nu} = 0.
\]
For the orthogonal wave functions $\psi_{\mu}$ this reduces to
\[
  \delta E - \sum_{{\mu}=1}^N \epsilon_{\mu} \delta
  \int \psi_{\mu}^* \psi_{\mu} = 0.
\]

Variation with respect to the single-particle wave functions $\psi_{\mu}$ yields then

\begin{equation*}
\begin{split}
  \sum_{\mu=1}^N \int \delta\psi_{\mu}^*\hat{h_i}\psi_{\mu}
  dx_i  
  + \frac{1}{2}\sum_{{\mu}=1}^N\sum_{{\nu}=1}^N \left[ \int
  \delta\psi_{\mu}^*\psi_{\nu}^*\frac{1} 
  {r_{ij}}\psi_{\mu}\psi_{\nu} dx_idx_j- \int
  \delta\psi_{\mu}^*\psi_{\nu}^*\frac{1}{r_{ij}}\psi_{\nu}\psi_{\mu}
  dx_idx_j \right] & \\
  + \sum_{\mu=1}^N \int \psi_{\mu}^*\hat{h_i}\delta\psi_{\mu}
  dx_i 
  + \frac{1}{2}\sum_{{\mu}=1}^N\sum_{{\nu}=1}^N \left[ \int
  \psi_{\mu}^*\psi_{\nu}^*\frac{1} 
  {r_{ij}}\delta\psi_{\mu}\psi_{\nu} dx_idx_j- \int
  \psi_{\mu}^*\psi_{\nu}^*\frac{1}{r_{ij}}\psi_{\nu}\delta\psi_{\mu}
  dx_idx_j \right] & \\
  -  \sum_{{\mu}=1}^N E_{\mu} \int \delta\psi_{\mu}^*
  \psi_{\mu}dx_i
  -  \sum_{{\mu}=1}^N E_{\mu} \int \psi_{\mu}^*
  \delta\psi_{\mu}dx_i & = 0.
\end{split}
\end{equation*}


Although the variations $\delta\psi$ and $\delta\psi^*$ are not
independent, they may in fact be treated as such, so that the 
terms dependent on either $\delta\psi$ and $\delta\psi^*$ individually 
may be set equal to zero. To see this, simply 
replace the arbitrary variation $\delta\psi$ by $i\delta\psi$, so that
$\delta\psi^*$ is replaced by $-i\delta\psi^*$, and combine the two
equations. We thus arrive at the Hartree-Fock equations
\[
  \begin{split}
    \left[ -\frac{1}{2}\nabla_i^2-\frac{Z}{r_i} + \sum_{{\nu}=1}^N
      \int \psi_{\nu}^*(x_j)\frac{1}{r_{ij}}
      \psi_{\nu}(x_j)dx_j \right]
    \psi_{\mu}(x_i)  & \\
    - \left[ \sum_{{\nu}=1}^N \int
      \psi_{\nu}^*(x_j) 
      \frac{1}{r_{ij}}\psi_{\mu}(x_j) dx_j
      \right] \psi_{\nu}(x_i)  & 
  = \epsilon_{\mu} \psi_{\mu}(x_i).
  \end{split}
\]
Notice that the integration $\int dx_j$ implies an
integration over the spatial coordinates $\mathbf{r_j}$ and a summation
over the spin-coordinate of electron $j$.

The two first terms are the one-body kinetic energy and the
electron-nucleus potential. The third or
\emph{direct} term is the averaged electronic repulsion of the other
electrons. This term is identical to the Coulomb integral introduced in
the simple perturbative approach to the helium atom. As written, the
term includes the 'self-interaction' of 
electrons when $i=j$. The self-interaction is cancelled in the fourth
term, or the \emph{exchange} term. The exchange term results from our
inclusion of the Pauli principle and the assumed determinantal form of
the wave-function. The effect of exchange is for electrons of
like-spin to avoid each other.

  A theoretically convenient form of the
Hartree-Fock equation is to regard the direct and exchange operator
defined through 
\begin{equation*}
  V_{\mu}^{d}(x_i) = \int \psi_{\mu}^*(x_j) 
  \frac{1}{r_{ij}}\psi_{\mu}(x_j) dx_j
\end{equation*}
and
\begin{equation*}
  V_{\mu}^{ex}(x_i) g(x_i) 
  = \left(\int \psi_{\mu}^*(x_j) 
  \frac{1}{r_{ij}}g(x_j) dx_j
  \right)\psi_{\mu}(x_i),
\end{equation*}
respectively. 

The function $g(x_i)$ is an arbitrary function,
and by the substitution $g(x_i) = \psi_{\nu}(x_i)$
we get
\begin{equation*}
  V_{\mu}^{ex}(x_i) \psi_{\nu}(x_i) 
  = \left(\int \psi_{\mu}^*(x_j) 
  \frac{1}{r_{ij}}\psi_{\nu}(x_j)
  dx_j\right)\psi_{\mu}(x_i).
\end{equation*}

We may then rewrite the Hartree-Fock equations as
\[
  H_i^{HF} \psi_{\nu}(x_i) = \epsilon_{\nu}\psi_{\nu}(x_i),
\]
with
\[
  H_i^{HF}= h_0(i) + \sum_{\mu=1}^NV_{\mu}^{d}(x_i) -
  \sum_{\mu=1}^NV_{\mu}^{ex}(x_i),
\]

and where $h_0(i)$ is the one-body part

Another possibility is to expand the single-particle functions in a known basis  and vary the coefficients, 
that is, the new single-particle wave function is written as a linear expansion
in terms of a fixed chosen orthogonal basis (for example harmonic oscillator, Laguerre polynomials etc)
\be
\psi_a  = \sum_{\lambda} C_{a\lambda}\psi_{\lambda}.
\label{eq:newbasis}
\ee
In this case we vary the coefficients $C_{a\lambda}$. If the basis has infinitely many solutions, we need
to truncate the above sum.  In all our equations we assume a truncation has been made.

The single-particle wave functions $\psi_{\lambda}({\bf r})$, defined by the quantum numbers $\lambda$ and ${\bf r}$
are defined as the overlap 
\[
   \psi_{\lambda}({\bf r})  = \langle {\bf r} | \lambda \rangle .
\]

We will omit the radial dependence of the wave functions and 
introduce first the following shorthands for the Hartree and Fock integrals
\[
\langle \mu\nu|V|\mu\nu\rangle =  \int \psi_{\mu}^*(\mathbf{r}_i)\psi_{\nu}^*(\mathbf{r}_j)V(r_{ij})\psi_{\mu}(\mathbf{r}_i)\psi_{\nu}(\mathbf{r}_j)
    d\mathbf{r}_i\mathbf{r}_j,
\]
and 
\[
\langle \mu\nu|V|\nu\mu\rangle = \int \psi_{\mu}^*(\mathbf{r}_i)\psi_{\nu}^*(\mathbf{r}_j)
  V(r_{ij})\psi_{\nu}(\mathbf{r}_i)\psi_{\mu}(\mathbf{r}_i)
  d\mathbf{r}_i\mathbf{r}_j.  
\]

Since the interaction is invariant under the interchange of two particles it means for example that we have
\[
\langle \mu\nu|V|\mu\nu\rangle =  \langle \nu\mu|V|\nu\mu\rangle,  
\]
or in the more general case
\[
\langle \mu\nu|V|\sigma\tau\rangle =  \langle \nu\mu|V|\tau\sigma\rangle.  
\]

The direct and exchange matrix elements can be  brought together if we define the antisymmetrized matrix element
\[
\langle \mu\nu|V|\mu\nu\rangle_{AS}= \langle \mu\nu|V|\mu\nu\rangle-\langle \mu\nu|V|\nu\mu\rangle,
\]
or for a general matrix element  
\[
\langle \mu\nu|V|\sigma\tau\rangle_{AS}= \langle \mu\nu|V|\sigma\tau\rangle-\langle \mu\nu|V|\tau\sigma\rangle.
\]
It has the symmetry property
\[
\langle \mu\nu|V|\sigma\tau\rangle_{AS}= -\langle \mu\nu|V|\tau\sigma\rangle_{AS}=-\langle \nu\mu|V|\sigma\tau\rangle_{AS}.
\]
The antisymmetric matrix element is also hermitian, implying 
\[
\langle \mu\nu|V|\sigma\tau\rangle_{AS}= \langle \sigma\tau|V|\mu\nu\rangle_{AS}.
\]

With these notations we rewrite the Hartree-Fock functional as
\begin{equation}
  \int \Phi^*\hat{H_1}\Phi d\mathbf{\tau} 
  = \frac{1}{2}\sum_{\mu=1}^A\sum_{\nu=1}^A \langle \mu\nu|V|\mu\nu\rangle_{AS}.
\label{H2Expectation2}
\end{equation}


Combining Eqs.~(\ref{H1Expectation1}) and
(\ref{H2Expectation2}) we obtain the energy functional 
\begin{equation}
  E[\Phi] 
  = \sum_{\mu=1}^N \langle \mu | h | \mu \rangle +
  \frac{1}{2}\sum_{{\mu}=1}^N\sum_{{\nu}=1}^N \langle \mu\nu|V|\mu\nu\rangle_{AS}.
\label{FunctionalEPhi}
\end{equation}

If we vary the above energy functional with respect to the basis functions $|\mu \rangle$, this corresponds to 
what was done in the previous case. We are however interested in defining a new basis defined in terms of
a chosen basis as defined in Eq.~(\ref{eq:newbasis}). We can then rewrite the energy functional as
\begin{equation}
  E[\Psi] 
  = \sum_{a=1}^N \langle a | h | a \rangle +
  \frac{1}{2}\sum_{ab=1}^N\langle ab|V|ab\rangle_{AS},
\label{FunctionalEPhi2}
\end{equation}
where $\Psi$ is the new Slater determinant defined by the new basis of Eq.~(\ref{eq:newbasis}). 

Using Eq.~(\ref{eq:newbasis}) we can rewrite Eq.~(\ref{FunctionalEPhi2}) as 
\begin{equation}
  E[\Psi] 
  = \sum_{a=1}^N \sum_{\alpha\beta} C^*_{a\alpha}C_{a\beta}\langle \alpha | h | \beta \rangle +
  \frac{1}{2}\sum_{ab=1}^N\sum_{{\alpha\beta\gamma\delta}} C^*_{a\alpha}C^*_{b\beta}C_{a\gamma}C_{b\delta}\langle \alpha\beta|V|\gamma\delta\rangle_{AS}.
\label{FunctionalEPhi3}
\end{equation}

We wish now to minimize the above functional. We introduce again a set of Lagrange multipliers, noting that
since $\langle a | b \rangle = \delta_{a,b}$ and $\langle \alpha | \beta \rangle = \delta_{\alpha,\beta}$, 
the coefficients $C_{a\gamma}$ obey the relation
\[
 \langle a | b \rangle=\delta_{a,b}=\sum_{\alpha\beta} C^*_{a\alpha}C_{a\beta}\langle \alpha | \beta \rangle=
\sum_{\alpha} C^*_{a\alpha}C_{a\alpha},
\]
which allows us to define a functional to be minimized that reads
\begin{equation}
  E[\Psi] - \sum_{a=1}^N\epsilon_a\sum_{\alpha} C^*_{a\alpha}C_{a\alpha}.
\end{equation}
Minimizing with respect to $C^*_{k\alpha}$, remembering that $C^*_{k\alpha}$ and $C_{k\alpha}$
are independent, we obtain
\be
\frac{d}{dC^*_{k\alpha}}\left[  E[\Psi] - \sum_{a}\epsilon_a\sum_{\alpha} C^*_{a\alpha}C_{a\alpha}\right]=0,
\ee
which yields for every single-particle state $k$ the following Hartree-Fock equations
\be
\sum_{\gamma} C_{k\gamma}\langle \alpha | h | \gamma \rangle+
\sum_{a=1}^N\sum_{\beta\gamma\delta} C^*_{a\beta}C_{a\delta}C_{k\gamma}\langle \alpha\beta|V|\gamma\delta\rangle_{AS}=\epsilon_kC_{k\alpha}.
\ee

We can rewrite this equation as 
\be
\sum_{\gamma} \left\{\langle \alpha | h | \gamma \rangle+
\sum_{a}^N\sum_{\beta\delta} C^*_{a\beta}C_{a\delta}\langle \alpha\beta|V|\gamma\delta\rangle_{AS}\right\}C_{k\gamma}=\epsilon_kC_{k\alpha}.
\ee
Note that the sums over greek indices run over the number of basis set functions (in principle an infinite number).

Defining 
\[
h_{\alpha\gamma}^{HF}=\langle \alpha | h | \gamma \rangle+
\sum_{a=1}^N\sum_{\beta\delta} C^*_{a\beta}C_{a\delta}\langle \alpha\beta|V|\gamma\delta\rangle_{AS},
\]
we can rewrite the new equations as 
\be
\sum_{\gamma}h_{\alpha\gamma}^{HF}C_{k\gamma}=\epsilon_kC_{k\alpha}.
\label{eq:newhf}
\ee
Note again that the sums over greek indices run over the number of basis set functions (in principle an infinite number).

\section{Stability and interpretation of the Hartree-Fock equations}
HF theory is an algorithm for a finding an approximative expression for the ground state of a given
Hamiltonian. The basic ingredients are
\begin{itemize}
\item Define a single-particle basis $\{\psi_{\alpha}\}$ so that
\[ \hat{h}^{\mathrm{HF}}\psi_{\alpha} = \varepsilon_{\alpha}\psi_{\alpha}\]
with \[\hat{h}^{\mathrm{HF}}=\hat{t}+\hat{u}_{\matrhm{ext}}+\hat{u}^{\mathrm{HF}}\]
\item where $\hat{u}^{\mathrm{HF}}$ is a single-particle potential to be determined by the HF algorithm.
\item The HF algorithm means to choose $\hat{u}^{\mathrm{HF}}$ in order to have 
\[ \langle \hat{H} \rangle = E^{\mathrm{HF}}= \langle \Phi_0 | \hat{H}|\Phi_0 \rangle\]
a local minimum with $\Phi_0$ being the SD ansatz for the ground state. 
\item The variational principle ensures that $E^{\mathrm{HF}} \ge \tilde{E}_0$, $\tilde{E}_0$ the exact ground state energy.
\end{itemize}

Last week we computed the Hamiltonian matrix for a system consisting of a Slater determinant for the ground state 
$|\Phi_0 \rangle$ and two 1p1h SDs $|\Phi_i^a \rangle$ and $|\Phi_j^b \rangle$. This can obviously be generalized to many more 1p1h SDs.  Using diagrammatic as well as algebraic representations we found the following 
expectation values
\[ \langle \Phi_0 | \hat{H}|\Phi_0 \rangle = E_0, \]
\[ \langle \Phi_i^a | \hat{H}|\Phi_0 \rangle = \langle a | \hat{f} | i \rangle,\]
\[ \langle \Phi_j^b | \hat{H}|\Phi_0 \rangle = \langle b | \hat{f} | j \rangle,\]
\[\langle \Phi_i^a | \hat{H}|\Phi_j^b \rangle = \langle aj | \hat{v} | ib \rangle,\]
and the diagonal elements
\[ \langle \Phi_i^a | \hat{H}|\Phi_i^a \rangle = E_0+\varepsilon_{a}-\varepsilon_{i}+\langle ai | \hat{v} | ia \rangle,\]
and 
\[ \langle \Phi_j^b | \hat{H}|\Phi_j^b \rangle =E_0+\varepsilon_{b}-\varepsilon_{j}+\langle bj | \hat{v} | jb \rangle.\]
We can then set up a Hamiltonian matrix to be diagonalized
\[
 \left( \begin{array}{ccc} 
               E_0  & \langle i | \hat{f} | a \rangle &  \langle j | \hat{f} | b\rangle\\
               \langle a | \hat{f} | i \rangle  &E_0+\varepsilon_{a}-\varepsilon_{i}+\langle ai | \hat{v} | ia \rangle  & \langle aj | \hat{v} | ib \rangle      \\
               \langle b | \hat{f} | j \rangle  & \langle bi | \hat{v} | ja \rangle &E_0+\varepsilon_{b}-\varepsilon_{j}+\langle bj | \hat{v} | jb \rangle         \\
             \end{array} \right) .
\]
The HF method corresponds to finding a similarity transformation where the non-diagonal matrix elements
\[\langle i | \hat{f} | a \rangle=0\].   We will link this expectation value with the HF method, meaning that
we want to find
\[\langle i | \hat{h}^{\mathrm{HF}}| a \rangle=0\]
We wish now to derive the Hartree-Fock equations using our second-quantized formalism and study the stability of the equations. 
Our SD ansatz for the ground state of the system is approximated as
\[   |\Phi_0\rangle = |c\rangle = \cre{i} \cre{j}\dots\cre{l}|0\rangle.\]
We wish to determine $\hat{u}^{HF}$ so that 
$E_0^{HF}= \langle c|\hat{H}| c\rangle$ becomes a local minimum. 


An arbitrary Slater determinant $\ket{c'}$ which is not orthogonal to a determinant
$\ket{c}={\displaystyle\prod_{i=1}^{n}}
a_{\alpha_{i}}^{\dagger}\ket{0}$, can be written as
\[
\ket{c'}=exp\left\{\sum_{a>F}\sum_{i\le F}C_{ai}a_{a}^{\dagger}
a_{i}\right\}\ket{c}
\]

The variational condition for deriving the Hartree-Fock equations guarantees only that the expectation value
$\langle c | \hat{H} | c \rangle$ has an extreme value, not necessarily a minimum. To figure out whether the extreme value we have found  is a minimum, we can use second quantization to analyze our results and find a criterion 
for the above expectation value to a local minimum. We will use Thouless' theorem and show that
\[
\frac{\langle c' |\hat{H} | c'\rangle}{\langle c' |c'\rangle} \ge \langle c |\hat{H} | c\rangle= E_0,
\]
with 
\[
|c'\rangle = |c\rangle + |\delta c\rangle.
\]
Using Thouless' theorem we can write out $|c'\rangle$ as
\[
\ket{c'}=exp\left\{\sum_{a>F}
\sum_{i\le F}\delta C_{ai}a_{a}^{\dagger}
a_{i}\right\}\ket{c}=
\]
\[
\left\{1+\sum_{a>F}\sum_{i\le F}\delta C_{ai}a_{a}^{\dagger}
a_{i}+\frac{1}{2!}\sum_{ab>F}
\sum_{ij\le F}\delta C_{ai}\delta C_{bj}a_{a}^{\dagger}a_{i}a_{b}^{\dagger}a_{j}+\dots\right\}
\]
where the amplitudes $\delta C$ are small.

The norm of $|c'\rangle$ is given by (using the intermediate normalization condition $\langle c' |c\rangle=1$) 
\[
\langle c' | c'\rangle = 1+\sum_{a>F}
\sum_{i\le F}|\delta C_{ai}|^2+O(\delta C_{ai}^3).
\]
The expectation value for the energy is now given by (using the Hartree-Fock condition)
\[
\langle c' |\hat{H} | c'\rangle=\langle c |\hat{H} | c\rangle +
\sum_{ab>F}
\sum_{ij\le F}\delta C_{ai}^*\delta C_{bj}\langle c |a_{i}^{\dagger}a_{a}\hat{H}a_{b}^{\dagger}a_{j}|c\rangle+
\]
\[
\frac{1}{2!}\sum_{ab>F}
\sum_{ij\le F}\delta C_{ai}\delta C_{bj}\langle c |\hat{H}a_{a}^{\dagger}a_{i}a_{b}^{\dagger}a_{j}|c\rangle+\frac{1}{2!}\sum_{ab>F}
\sum_{ij\le F}\delta C_{ai}^*\delta C_{bj}^*\langle c|a_{j}^{\dagger}a_{b}a_{i}^{\dagger}a_{a}\hat{H}|c\rangle
+\dots
\] 
We will skip higher-order terms later.

We have already calculated the second term on the rhs of the previous equation
\[
\bra{c} \left(\normord{a^\dagger_i a_a} \op{H} \normord{a^\dagger_b a_j} \right) \ket{c} =
\]
\[
\sum_{pq} \sum_{ijab}\delta C_{ai}^*\delta C_{bj} \bra{p}\hat{h}_0\ket{q} 
            \bra{c} \left(\normord{a^\dagger_i a_a}\normord{a^\dagger_pa_q} 
             \normord{a^\dagger_b a_j} \right)\ket{c}+ 
\]
\[
\frac{1}{4} \sum_{pqrs} \sum_{ijab}\delta C_{ai}^*\delta C_{bj} \bra{pq}\hat{v}\ket{rs} 
            \bra{c} \left(\normord{a^\dagger_i a_a}\normord{a^\dagger_p a^\dagger_q a_s  a_r} \normord{a^\dagger_b a_j} \right)\ket{c},
\]
resulting in
\[
E_0\sum_{ai}|\delta C_{ai}|^2+\sum_{ai}|\delta C_{ai}|^2(\varepsilon_a-\varepsilon_i)-\sum_{ijab} \bra{aj}\hat{v}\ket{bi}\delta C_{ai}^*\delta C_{bj}.
\]

The third term in the rhs of the last equation can then be written out (where is the reference energy and why do we only consider the two-particle interaction $\hat{V}_N$?)
    \begin{align*}
        & \frac{1}{2!}\bra{c}  \left( \op{V}_N \normord{a^\dagger_a a_i} \normord{a^\dagger_b a_j} \right) \ket{c} = \\
            & \quad \frac{1}{8} \sum_{pqrs} \sum_{ijab}\delta C_{ai}\delta C_{bj} \bra{pq}\hat{v}\ket{rs} 
            \bra{c} \left(\normord{a^\dagger_p a^\dagger_q a_s  a_r} 
            \normord{a^\dagger_a a_i} \normord{a^\dagger_b a_j} \right)\ket{c} \\
        &= \frac{1}{8} \sum_{pqrs} \sum_{ijab} \bra{pq}\hat{v}\ket{rs} \delta C_{ai}\delta C_{bj} \bra{c} \\
        & \quad \Bigl( 
        \left\{
        \contraction{a^\dagger_p a^\dagger_q a_s}{a}{{}_r}{a}
        \contraction[1.25ex]{a^\dagger_p}{a}{{}^\dagger_q a_s a_r a^\dagger_a}{a}
        \contraction[1.5ex]{a^\dagger_p a^\dagger_q }{a}{{}_s a_r a^\dagger_a a_i}{a}
        \contraction[1.75ex]{}{a}{{}^\dagger_p a^\dagger_q a_s a_r a^\dagger_a a_i a^\dagger_b}{a}
        a^\dagger_p a^\dagger_q a_s  a_r a^\dagger_a a_i a^\dagger_b a_j \right\}
        +\left\{
        \contraction{a^\dagger_p a^\dagger_q}{a}{{}_s a_r}{a}
        \contraction[1.25ex]{a^\dagger_p}{a}{{}^\dagger_q a_s a_r a^\dagger_a}{a}
        \contraction[1.5ex]{a^\dagger_p a^\dagger_q a_s}{a}{{}_r a^\dagger_a a_i}{a}
        \contraction[1.75ex]{}{a}{{}^\dagger_p a^\dagger_q a_s a_r a^\dagger_a a_i a^\dagger_b}{a}
        a^\dagger_p a^\dagger_q a_s  a_r a^\dagger_a a_i a^\dagger_b a_j \right\}
        + \left\{
        \contraction{a^\dagger_p a^\dagger_q a_s}{a}{{}_r}{a}
        \contraction[1.25ex]{}{a}{{}^\dagger_p q^\dagger_q a_s a_r a^\dagger_a}{a}
        \contraction[1.5ex]{a^\dagger_p a^\dagger_q }{a}{{}_s a_r a^\dagger_a a_i}{a}
        \contraction[1.75ex]{a^\dagger_p}{a}{{}^\dagger_q a_s a_r a^\dagger_a a_i a^\dagger_b}{a}
        a^\dagger_p a^\dagger_q a_s  a_r a^\dagger_a a_i a^\dagger_b a_j \right\} \\
        & \quad +\left\{
        \contraction{a^\dagger_p a^\dagger_q}{a}{{}_s a_r}{a}
        \contraction[1.25ex]{}{a}{{}^\dagger_p q^\dagger_q a_s a_r a^\dagger_a}{a}
        \contraction[1.5ex]{a^\dagger_p a^\dagger_q a_s}{a}{{}_r a^\dagger_a a_i}{a}
        \contraction[1.75ex]{a^\dagger_p}{a}{{}^\dagger_q a_s a_r a^\dagger_a a_i a^\dagger_b}{a}
        a^\dagger_p a^\dagger_q a_s  a_r a^\dagger_a a_i a^\dagger_b a_j \right\}
        \Bigr) \ket{c} \\
        &= \frac{1}{2} \sum_{ijab} \bra{ij}\hat{v}\ket{ab}\delta C_{ai}\delta C_{bj}
    \end{align*}
The final term in the rhs of the last equation can then be written out as
    \[
\frac{1}{2!}\bra{c}\left(\normord{a^\dagger_j a_b} \normord{a^\dagger_i a_a} \op{V}_N  \right) \ket{c} = 
\frac{1}{2!}\bra{c}\left( \op{V}_N \normord{a^\dagger_a a_i} \normord{a^\dagger_b a_j} \right)^{\dagger} \ket{c}
\]
which is nothing but
\[
\frac{1}{2!}\bra{c}  \left( \op{V}_N \normord{a^\dagger_a a_i} \normord{a^\dagger_b a_j} \right) \ket{c}^*
=\frac{1}{2} \sum_{ijab} (\bra{ij}\hat{v}\ket{ab})^*\delta C_{ai}^*\delta C_{bj}^*
\]
or 
\[
\frac{1}{2} \sum_{ijab} (\bra{ab}\hat{v}\ket{ij})\delta C_{ai}^*\delta C_{bj}^*
\]
where we have used the relation
\[ \langle a |\hat{A} | b\rangle =  (\langle b |\hat{A}^{\dagger} | a\rangle)^*
\]
due to the hermiticity of $\hat{H}$ and $\hat{V}$.
We define two matrix elements
\[
A_{ai,bj}=-\bra{aj}\hat{v}\ket{bi}
\]
\[
B_{ai,bj}=\bra{ab}\hat{v}\ket{ij}
\]
both being anti-symmetrized.
We can then write out the energy
\[
\langle c'|H|c'\rangle = \left(1+\sum_{ai}|\delta C_{ai}|^2\right)\langle c |H|c\rangle+
\]
\[
\sum_{ai}|\delta C_{ai}|^2(\varepsilon_a^{HF}-\varepsilon_i^{HF})+\sum_{ijab}A_{ai,bj}\delta C_{ai}^*\delta C_{bj}+
\]
\[
\frac{1}{2} \sum_{ijab} B_{ai,bj}^*\delta C_{ai}\delta C_{bj}+\frac{1}{2} \sum_{ijab} B_{ai,bj}\delta C_{ai}^*\delta C_{bj}^*
+O(\delta C_{ai}^3),\]
which allows us to rewrite it as 
\[
\langle c'|H|c'\rangle = \left(1+\sum_{ai}|\delta C_{ai}|^2\right)\langle c |H|c\rangle+\Delta E+O(\delta C_{ai}^3),
\]
and skipping higher-order terms we have
\[
\frac{\langle c' |\hat{H} | c'\rangle}{\langle c' |c'\rangle} =E_0+\frac{\Delta E}{\left(1+\sum_{ai}|\delta C_{ai}|^2\right)}.
\]

We have defined 
\[
\Delta E = \frac{1}{2} \langle \chi | \hat{M}| \chi \rangle
\]
with the vectors 
\[ \chi = \left[ \delta C\hspace{0.2cm} \delta C^*\right]^T
\]
and the matrix 
\[
\hat{M}=\left(\begin{array}{cc} \Delta + A & B \\ B^* & \Delta + A^*\end{array}\right),
\]
with $\Delta_{ai,bj} = (\varepsilon_a-\varepsilon_i)\delta_{ab}\delta_{ij}$.
The condition
\[
\Delta E = \frac{1}{2} \langle \chi | \hat{M}| \chi \rangle \ge 0
\]
for an arbitrary  vector 
\[ \chi = \left[ \delta C\hspace{0.2cm} \delta C^*\right]^T\]
means that all eigenvalues of the matrix have to be larger than or equal zero. 
A necessary (but no sufficient) condition is that the matrix elements (for all $ai$ )
\[
(\varepsilon_a-\varepsilon_i)\delta_{ab}\delta_{ij}+A_{ai,bj} \ge 0.
\]
This equation can be used as a first test of the stability of the Hartree-Fock equation.


\section{Diagrammatic representation of the Hartree-Fock equations}

\section{Koopman's theorem}

\section{Applications}

\section{Exercises}


\begin{prob}
In this exercise  we will develop two simple models for studying the 
helium atom (with two electrons) and the beryllium atom with four electrons.

After having introduced the  Born-Oppenheimer approximation which effectively freezes out the nucleonic degrees
of freedom, the Hamiltonian for $N$ electrons takes the following form 
\[
  \hat{H} = \sum_{i=1}^{N} t(x_i) 
  - \sum_{i=1}^{N} k\frac{Ze^2}{r_i} + \sum_{i<j}^{N} \frac{ke^2}{r_{ij}},
\]
with $k=1.44$ eVnm. We will use atomic units, this means
that $\hbar=c=e=m_e=1$. The constant $k$ becomes also equal 1. 
The resulting energies have to be multiplied by $2\times 13.6$ eV
in order to obtain energies in eletronvolts.

 We can rewrite our Hamiltonians as
\begin{equation}
    \hat{H} = \hat{H_0} + \hat{H_I} 
    = \sum_{i=1}^{N}\hat{h}_0(x_i) + \sum_{i<j}^{N}\frac{1}{r_{ij}},
\label{H1H2}
\end{equation}
where  we have defined $r_{ij}=| {\bf r}_i-{\bf r}_j|$ and
$\hat{h}_0(x_i) =  \hat{t}(x_i) - \frac{Z}{r_i}$
The variable $x$ contains both the spatial coordinates and the spin values.
The first term of Eq.~(\ref{H1H2}), $H_0$, is the sum of the $N$
\emph{one-body} Hamiltonians $\hat{h}_0$. Each individual
Hamiltonian $\hat{h}_0$ contains the kinetic energy operator of an
electron and its potential energy due to the attraction of the
nucleus. The second term, $H_I$, is the sum of the $N(N-1)/2$
two-body interactions between each pair of electrons. Note that the double sum carries a restriction $i<j$.

As basis functions for our calculations we will use hydrogen-like single-particle functions.  This means the onebody operator is diagonal in this basis for states $i,j$ with quantum numbers $nlm_lsm_s$ with  
energies 
\[\langle i|\hat{h}_0| j\rangle =  -Z^2/2n^2\delta_{i,j}.\]  
The quantum number $n$ refers to the number of nodes 
of the wave function.  Observe that this expectation value is independent of spin.

We will in all calculations here restrict ourselves to only so-called $s$ -waves,
that is the orbital momentum $l$ is zero. We will also limit the quantum number $n$ to $n\le 3$.  It means that every $ns$ state can accomodate two electrons due to the spin degeneracy. This is illustrated in Fig.~\ref{fig:fighelium} here.
\begin{figure}[hbtp]
\vspace{1.0cm}
 \setlength{\unitlength}{1cm}
 \begin{picture}(15,4)
 \thicklines
\put(-0.6,1){\makebox(0,0){$1s$}}
\put(-0.6,2){\makebox(0,0){$2s$}}
\put(-0.6,3){\makebox(0,0){$3s$}}
% first 2-particle state
\put(0.8,1){\circle*{0.3}}
\put(1.7,1){\circle*{0.3}}
% second 2-particle state
\put(5.0,2){\circle*{0.3}}
\put(5.9,2){\circle*{0.3}}
% third 2-particle state
\put(9.2,1){\circle*{0.3}}
\put(10.1,3){\circle*{0.3}}
% fourth 2-particle state
\put(13.4,1){\circle*{0.3}}
\put(14.3,2){\circle*{0.3}}
\dashline[+1]{2.5}(0,1)(15,1)
\dashline[+1]{2.5}(0,2)(15,2)
\dashline[+1]{2.5}(0,3)(15,3)
 \end{picture}
\caption{Schematic plot of the possible single-particle levels with double degeneracy.
The filled circles indicate occupied particle states.
We show some possible two-particle states which can describe states in the helium atom. \label{fig:fighelium}}
\end{figure}

In the calculations you 
will need the Coulomb interaction with matrix elements
involving single-particle wave functions with $l=0$ only, the so-called $s$-waves.
We need only the radial part since the 
spherical harmonics for the $s$-waves are rather simple. We omit single-particle states with $l> 0$.
Our radial wave functions are
\[
R_{n0}(r)=\left(\frac{2Z}{n}\right)^{3/2}\sqrt{\frac{(n-1)!}{2n\times n!}}L_{n-1}^1(\frac{2Zr}{n})\exp{(-\frac{Zr}{n})},
\]
where $L_{n-1}^1(r)$ are the so-called Laguerre polynomials.
These wave functions can then be used to compute the direct part of the
Coulomb interaction
\[
\langle \alpha\beta| V| \gamma\delta\rangle = \int r_1^2dr_1 \int r_2^2dr_2R_{n_{\alpha}0}^*(r_1) R_{n_{\beta}0}^*(r_2) 
  \frac{1}{| {\bf r}_1-{\bf r}_2|}R_{n_{\gamma}0}(r_1)R_{n_{\delta}0}(r_2)
\]
Observe that this is only the radial integral and that the labels $\alpha\beta\gamma\delta$ refer only to the quantum numbers $nlm_l$, with $m_l$ the projection of the orbital momentum $l$. 
A similar expression can be found for the exchange part. Since we have restricted ourselves to only $s$-waves, these integrals are straightforward but tedious to calculate. As an addendum to this exercise we list all closed-form expressions for the relevant matrix elements. Note well that these matrix elements do not include spin. When setting up the final antisymmetrized matrix elements you need to consider the spin degrees of freedom as well. Please pay in particular special attention to the exchange part and the pertinent spin values of the single-particle states.  


We will also, for both helium and beryllium assume that the many-particle states we construct have always the same total spin projection $M_S=0$. This means that if we excite one or two particles from the ground state, the spins of the various single-particle states should always sum up to zero. 

\begin{enumerate}
\item[a)] We start with the helium atom and define our single-particle Hilbert space to consist of the single-particle orbits $1s$, $2s$ and $3s$, with their corresponding spin degeneracies, see Fig.~\ref{fig:fighelium}. 

Set up the ansatz for the ground state $|c\rangle = |\Phi_0\rangle$ in second 
quantization and define a table of single-particle states. Construct thereafter
all possible one-particle-one-hole excitations  $|\Phi_i^a\rangle$ where $i$ refer to levels below the Fermi level (define this level) and $a$ refers to particle states. Define particles and holes. The Slater determinants have to be written in terms of the respective creation and annihilation operators.
The states you construct should all have total spin projection $M_S=0$. 
Construct also all possible two-particle-two-hole states $|\Phi_{ij}^{ab}\rangle$  in a second quantization representation. 


\item[b)]  Define the Hamiltonian in a second-quantized form and use this to
compute the expectation value of the ground state (defining the so-called reference energy of the helium atom. 
Show that it is given by
\[
  E[\Phi_0] = \langle c | \hat{H}| c \rangle 
  = \sum_{i} \langle i | \hat{h}_0 | i\rangle+ \frac{1}{2}\sum_{ij}\left[\langle ij |\frac{1}{r}|ij\rangle-\langle ij |\frac{1}{r}|ji\rangle\right].
\]
Define properly the sums keeping in mind that the states $ij$ refer to all
quantum numbers $nlm_lsm_s$.
Use the values for the various matrix elements listed at the end of the exercise to find the value of $E$
as function of $Z=2$.
Be careful when you set up the matrix elements. Pay in particular attention to the spin values.
\item[c)]
Hereafter we will limit ourselves to a system which now contains only one-particle-one-hole
excitations beyond the chosen state $|c\rangle$.
Using the possible Slater determinants from exercise a) for the helium atom,   
compute also the expectation values (without inserting the explicit values for the matrix elements first) of 
\[
\langle c | \hat{H}| \Phi_i^a \rangle,
\] 
and 
\[
\langle \Phi_i^a | \hat{H}| \Phi_j^b \rangle.
\]
Represent these expectation values in a diagrammatic form, both for the onebody part and the two-body part of the Hamiltonian. 
 
Insert then the explicit values for the various matrix elements and 
set up the final Hamiltonian matrix and diagonalize it using for example
Octave, Matlab, Python, C++ or Fortran as programming tools.

Compare your results from those of exercise b) and comment your results. 
The exact energy with our Hamiltonian is $-2.9037$ atomic units for helium. This value is also close to the experimental energy.
\item[d)] We repeat exercises b) and c) but now for the beryllium atom.
Define the ansatz for $|c\rangle$ and limit yourself again to one-particle-one-hole excitations.   Compute the reference energy 
$\langle c | \hat{H}| c \rangle $ for $Z=4$.  
Thereafter you will need to set up the appropriate Hamiltonian matrix
which involves also one-particle-one-hole excitations. Diagonalize this matrix
and compare your eigenvalues with $\langle c | \hat{H}| c \rangle$ for $Z=4$ and comment your results. 
The exact energy with our Hamiltonian is $-14.6674$ atomic units for beryllium. This value is again close to the experimental energy.
\end{enumerate}

We conclude by listing in Table \ref{tab:mtxlisting} the matrix elements for the radial integrals to be used for the direct part and the exchange part. Note again that these integrals do not include spin. 
\begin{table}[htbp]
\caption{Closed form expressions for the Coulomb matrix elements. The nomenclature is $1=1s$, $2=2s$ and $3=3s$, with no
spin degrees of freedom. \label{tab:mtxlisting}}
\begin{tabular} {|cr|cr|} \hline 
$\langle 11|V|11\rangle  =$& $ (5Z)/8 $ &$\langle 11|V|12\rangle  =$& $ (4096\sqrt{2}Z)/64827$ \\
$\langle 11|V|13\rangle  =$& $ (1269\sqrt{3}Z)/50000$ &$\langle 11|V|21\rangle  =$& $ (4096\sqrt{2}Z)/64827$ \\
$\langle 11|V|22\rangle  =$& $ (16Z)/729$ & $\langle 11|V|23\rangle  =$& $ (110592\sqrt{6}Z)/24137569$ \\
$\langle 11|V|31\rangle  =$& $ (1269\sqrt{3}Z)/50000$ &$\langle 11|V|32\rangle  =$& $ (110592\sqrt{6}Z)/24137569$ \\
$\langle 11|V|33\rangle  =$& $ (189Z)/32768$ & $\langle 12|V|11\rangle  =$& $ (4096\sqrt{2}Z)/64827$ \\
$\langle 12|V|12\rangle  =$& $ (17Z)/81$ &$\langle 12|V|13\rangle  =$& $ (1555918848\sqrt{6}Z)/75429903125$ \\
$\langle 12|V|21\rangle  =$& $ (16Z)/729$ &$\langle 12|V|22\rangle  =$& $ (512\sqrt{2}Z)/84375$ \\
$\langle 12|V|23\rangle  =$& $ (2160\sqrt{3}Z)/823543$ &$\langle 12|V|31\rangle  =$& $ (110592\sqrt{6}Z)/24137569$ \\
$\langle 12|V|32\rangle  =$& $ (29943\sqrt{3}Z)/13176688$ &$\langle 12|V|33\rangle  =$& $ (1216512\sqrt{2}Z)/815730721$ \\
$\langle 13|V|11\rangle  =$& $ (1269\sqrt{3}Z)/50000$ &$\langle 13|V|12\rangle  =$& $ (1555918848\sqrt{6}Z)/75429903125$ \\
$\langle 13|V|13\rangle  =$& $ (815Z)/8192$ & $\langle 13|V|21\rangle  =$& $ (110592\sqrt{6}Z)/24137569$ \\
$\langle 13|V|22\rangle  =$& $ (2160\sqrt{3}Z)/823543$ & $\langle 13|V|23\rangle  =$& $ (37826560\sqrt{2}Z)/22024729467$ \\
$\langle 13|V|31\rangle  =$& $ (189Z)/32768$ & $\langle 13|V|32\rangle  =$& $ (1216512\sqrt{2}Z)/815730721$ \\
$\langle 13|V|33\rangle  =$& $ (617Z)/(314928\sqrt{3})$ &$\langle 21|V|11\rangle  =$& $ (4096\sqrt{2}Z)/64827$ \\
$\langle 21|V|12\rangle  =$& $ (16Z)/729$ & $\langle 21|V|13\rangle  =$& $ (110592\sqrt{6}Z)/24137569$ \\
$\langle 21|V|21\rangle  =$& $ (17Z)/81$ & $\langle 21|V|22\rangle  =$& $ (512\sqrt{2}Z)/84375$ \\
$\langle 21|V|23\rangle  =$& $ (29943\sqrt{3}Z)/13176688$ & $\langle 21|V|31\rangle  =$& $ (1555918848\sqrt{6}Z)/75429903125$ \\
$\langle 21|V|32\rangle  =$& $ (2160\sqrt{3}Z)/823543$ & $\langle 21|V|33\rangle  =$& $ (1216512\sqrt{2}Z)/815730721$ \\
$\langle 22|V|11\rangle  =$& $ (16Z)/729$ & $\langle 22|V|12\rangle  =$& $ (512\sqrt{2}Z)/84375$ \\
$\langle 22|V|13\rangle  =$& $ (2160\sqrt{3}Z)/823543$ & $\langle 22|V|21\rangle  =$& $ (512\sqrt{2}Z)/84375$ \\
$\langle 22|V|22\rangle  =$& $ (77Z)/512$ & $\langle 22|V|23\rangle  =$& $ (5870679552\sqrt{6}Z)/669871503125$ \\
$\langle 22|V|31\rangle  =$& $ (2160\sqrt{3}Z)/823543$ & $\langle 22|V|32\rangle  =$& $ (5870679552\sqrt{6}Z)/669871503125$ \\
$\langle 22|V|33\rangle  =$& $ (73008Z)/9765625$ & $\langle 23|V|11\rangle  =$& $ (110592\sqrt{6}Z)/24137569$ \\
$\langle 23|V|12\rangle  =$& $ (2160\sqrt{3}Z)/823543$ & $\langle 23|V|13\rangle  =$& $ (37826560\sqrt{2}Z)/22024729467$ \\
$\langle 23|V|21\rangle  =$& $ (29943\sqrt{3}Z)/13176688$ & $\langle 23|V|22\rangle  =$& $ (5870679552\sqrt{6}Z)/669871503125$ \\
$\langle 23|V|23\rangle  =$& $ (32857Z)/390625$ & $\langle 23|V|31\rangle  =$& $ (1216512\sqrt{2}Z)/815730721$ \\
$\langle 23|V|32\rangle  =$& $ (73008Z)/9765625$ & $\langle 23|V|33\rangle  =$& $ (6890942464\sqrt{2/3}Z)/1210689028125$ \\
$\langle 31|V|11\rangle  =$& $ (1269\sqrt{3}Z)/50000$ & $\langle 31|V|12\rangle  =$& $ (110592\sqrt{6}Z)/24137569$ \\
$\langle 31|V|13\rangle  =$& $ (189Z)/32768$ & $\langle 31|V|21\rangle  =$& $ (1555918848\sqrt{6}Z)/75429903125$ \\
$\langle 31|V|22\rangle  =$& $ (2160\sqrt{3}Z)/823543$ & $\langle 31|V|23\rangle  =$& $ (1216512\sqrt{2}Z)/815730721$ \\
$\langle 31|V|31\rangle  =$& $ (815Z)/8192$ & $\langle 31|V|32\rangle  =$& $ (37826560\sqrt{2}Z)/22024729467$ \\
$\langle 31|V|33\rangle  =$& $ (617Z)/(314928\sqrt{3})$ & $\langle 32|V|11\rangle  =$& $ (110592\sqrt{6}Z)/24137569$ \\
$\langle 32|V|12\rangle  =$& $ (29943\sqrt{3}Z)/13176688$ & $\langle 32|V|13\rangle  =$& $ (1216512\sqrt{2}Z)/815730721$ \\
$\langle 32|V|21\rangle  =$& $ (2160\sqrt{3}Z)/823543$ & $\langle 32|V|22\rangle  =$& $ (5870679552\sqrt{6}Z)/669871503125$ \\
$\langle 32|V|23\rangle  =$& $ (73008Z)/9765625$ & $\langle 32|V|31\rangle  =$& $ (37826560\sqrt{2}Z)/22024729467$ \\
$\langle 32|V|32\rangle  =$& $ (32857Z)/390625$ & $\langle 32|V|33\rangle  =$& $ (6890942464\sqrt{2/3}Z)/1210689028125$ \\
$\langle 33|V|11\rangle  =$& $ (189Z)/32768$ & $\langle 33|V|12\rangle  =$& $ (1216512\sqrt{2}Z)/815730721$ \\
$\langle 33|V|13\rangle  =$& $ (617Z)/(314928\sqrt{3})$ & $\langle 33|V|21\rangle  =$& $ (1216512\sqrt{2}Z)/815730721$ \\
$\langle 33|V|22\rangle  =$& $ (73008Z)/9765625$ & $\langle 33|V|23\rangle  =$& $ (6890942464\sqrt{2/3}Z)/1210689028125$ \\
$\langle 33|V|31\rangle  =$& $ (617Z)/(314928\sqrt{3})$ &$\langle 33|V|32\rangle  =$& $ (6890942464\sqrt{2/3}Z)/1210689028125$ \\
$\langle 33|V|33\rangle  =$& $ (17Z)/256$ & & \\ \hline
\end{tabular}
\end{table}




\end{prob}

\begin{prob}
Consider a Slater determinant built up of single-particle orbitals $\psi_{\lambda}$, 
with $\lambda = 1,2,\dots,N$.

The unitary transformation
\[
\psi_a  = \sum_{\lambda} C_{a\lambda}\phi_{\lambda},
\]
brings us into the new basis.  
The new basis has quantum numbers $a=1,2,\dots,N$.
Show that the new basis is orthonormal.
Show that the new Slater determinant constructed from the new single-particle wave functions can be
written as the determinant based on the previous basis and the determinant of the matrix $C$.
Show that the old and the new Slater determinants are equal up to a complex constant with absolute value unity.
(Hint, $C$ is a unitary matrix). 

\end{prob}

\begin{prob}
Consider the  Slater  determinant
\[
\Phi_{0}=\frac{1}{\sqrt{n!}}\sum_{p}(-)^{p}P
\prod_{i=1}^{n}\psi_{\alpha_{i}}(x_{i}).
\]
A small variation in this function is given by
\[
\delta\Phi_{0}=\frac{1}{\sqrt{n!}}\sum_{p}(-)^{p}P
\psi_{\alpha_{1}}(x_{1})\psi_{\alpha_{2}}(x_{2})\dots
\psi_{\alpha_{i-1}}(x_{i-1})(\delta\psi_{\alpha_{i}}(x_{i}))
\psi_{\alpha_{i+1}}(x_{i+1})\dots\psi_{\alpha_{n}}(x_{n}).
\]
Show that
\[
\bra{\delta\Phi_{0}}\sum_{i=1}^{n}\left\{t(x_{i})+u(x_{i})
\right\}+\frac{1}{2}
\sum_{i\neq j=1}^{n}v(x_{i},x_{j})\ket{\Phi_{0}}=
\]
\[
\sum_{i=1}^{n}\bra{\delta\psi_{\alpha_{i}}}t+u
\ket{\phi_{\alpha_{i}}}
+\sum_{i\neq j=1}^{n}\left\{\bra{\delta\psi_{\alpha_{i}}
\psi_{\alpha_{j}}}v\ket{\psi_{\alpha_{i}}\psi_{\alpha_{j}}}-
\bra{\delta\psi_{\alpha_{i}}\psi_{\alpha_{j}}}v
\ket{\psi_{\alpha_{j}}\psi_{\alpha_{i}}}\right\}
\]
\end{prob}

\begin{prob}
What is the diagrammatic representation of the HF equation?
\[
-\bra{\alpha_{k}}u^{HF}\ket{\alpha_{i}}+\sum_{j=1}^{n}
\left[\bra{\alpha_{k}\alpha_{j}}v\ket{\alpha_{i}\alpha_{j}}-
\bra{\alpha_{k}\alpha_{j}}v\ket{\alpha_{j}\alpha_{i}}\right]=0
\hspace{0.5cm}?
\]
(Represent $(-u^{HF})$ by the symbol $---$X .)
\end{prob}
\begin{prob}
\subsection*{Exercise 17}

Consider the ground state $\ket{\Phi}$ 
of a bound many-particle system of fermions. Assume that we remove one particle
from the single-particle state $\lambda$ and that our system ends in a new state
$\ket{\Phi_{n}}$. 
Define the energy needed to remove this particle as
\[
{\cal E}_{\lambda}=\sum_{n}\vert
\bra{\Phi_{n}}a_{\lambda}\ket{\Phi}\vert^{2}(E_{0}-E_{n}),
\]
where $E_{0}$ and $E_{n}$  are the ground state energies of the states
$\ket{\Phi}$  and  $\ket{\Phi_{n}}$, respectively.
\newline
a) Show that
\[
{\cal E}_{\lambda}=\bra{\Phi}a_{\lambda}^{\dagger}\left[
a_{\lambda},H \right]\ket{\Phi},
\]
where $H$ is the Hamiltonian of this system.
\newline
b) If we assume that $\Phi$ is the  Hartree-Fock result, find the 
relation
between $\cal{E}_{\lambda}$ and the single-particle energy
$\varepsilon_{\lambda}$
for states $\lambda \leq F$ and $\lambda >F$, with
\[
\varepsilon_{\lambda}=\bra{\lambda}(t+u)\ket{\lambda}
\]
and
\[
\bra{\lambda}u\ket{\lambda}={\displaystyle \sum_{\beta \leq F}}
\bra{\lambda\beta}v\ket{\lambda\beta}.
\]
We have assumed an antisymmetrized matrix element here.
Discuss the result.

The Hamiltonian operator is defined as
\[
H={\displaystyle
\sum_{\alpha\beta}}\bra{\alpha}t\ket{\beta}a_{\alpha}^
{\dagger}a_{\beta}+
\frac{1}{2}{\displaystyle
\sum_{\alpha\beta\gamma\delta}}\bra{\alpha\beta}
v\ket{\gamma\delta}a_{\alpha}^{\dagger}a_{\beta}^{\dagger}
a_{\delta}a_{\gamma}.
\]
\end{prob}
\begin{prob}

The electron gas model allows closed form solutions for quantities like the 
single-particle Hartree-Fock energy.  The latter quantity is given by the following expression
\[
\varepsilon_{k}^{HF}=\frac{\hbar^{2}k^{2}}{2m}-\frac{e^{2}}
{V^{2}}\sum_{k'\leq
k_{F}}\int d\vec{r}e^{i(\vec{k'}-\vec{k})\vec{r}}\int
d\vec{r'}\frac{e^{i(\vec{k}-\vec{k'})\vec{r'}}}
{\vert\vec{r}-\vec{r'}\vert}
\]
a) Show that
\[
\varepsilon_{k}^{HF}=\frac{\hbar^{2}k^{2}}{2m}-\frac{e^{2}
k_{F}}{2\pi}
\left[
2+\frac{k_{F}^{2}-k^{2}}{kk_{F}}ln\left\vert\frac{k+k_{F}}
{k-k_{F}}\right\vert
\right]
\]
(Hint: Introduce the convergence factor 
$e^{-\mu\vert\vec{r}-\vec{r'}\vert}$
in the potential and use  $\sum_{\vec{k}}\rightarrow
\frac{V}{(2\pi)^{3}}\int d\vec{k}$ )
\newline
b) Rewrite the above result as a function of the density
\[
n= \frac{k_F^3}{3\pi^2}=\frac{3}{4\pi r_s^3},
\]
where $n=N/V$, $N$ being the number of particles, and $r_s$ is the radius of a sphere which represents the volum per conducting electron.  
It can be convenient to use the Bohr radius $a_0=\hbar^2/e^2m$.

For most metals we have a relation $r_s/a_0\sim 2-6$.

Make a plot of the free electron energy and the Hartree-Fock energy
and discuss the behavior around the Fermi surface. Extract also  
the Hartree-Fock band width $\Delta\varepsilon^{HF}$ defined as
\[ \Delta\varepsilon^{HF}=\varepsilon_{k_{F}}^{HF}-
\varepsilon_{0}^{HF}.\]
Compare this results with the corresponding one for a free electron and comment
your results. How large is the contribution due to the exchange term in the Hartree-Fock equation?\newline
c) We will now define a quantity called the effective mass.
For $\vert\vec{k}\vert$ near $k_{F}$, we can Taylor expand the Hartree-Fock energy as  
\[
\varepsilon_{k}^{HF}=\varepsilon_{k_{F}}^{HF}+
\left(\frac{\partial
\varepsilon_{k}^{HF}}{\partial k}\right)_{k_{F}}(k-k_{F})+\dots
\]
If we compare the latter with the corresponding expressiyon for the non-interacting system
\[
\varepsilon_{k}^{(0)}=\frac{\hbar^{2}k^{2}_{F}}{2m}+
\frac{\hbar^{2}k_{F}}{m}\left(k-k_{F}\right)+\dots ,
\]
we can define the so-called effective Hartree-Fock mass as
\[
m_{HF}^{*}\equiv\hbar^{2}k_{F}\left(
\frac{\partial\varepsilon_{k}^{HF}}
{\partial k}\right)_{k_{F}}^{-1}
\]
Compute $m_{HF}^{*}$ and comment your results after you have done 
point d). \newline
d) Show that the level density (the number of single-electron states
per unit energy) can be written as
\[
n(\varepsilon)=\frac{Vk^{2}}{2\pi^{2}}\left(
\frac{\partial\varepsilon}{\partial k}\right)^{-1}
\]
Calculate $n(\varepsilon_{F}^{HF})$ and comment the results from c) and d).
\end{prob}
\begin{prob}
We consider a system of electrons in infinite matter, the so-called electron gas. This is a homogeneous system and the one-particle states are given by plane wave function normalized to a volume $\Omega$ 
for a box with length $L$ (the limit $L\rightarrow \infty$ is to be taken after we have computed various expectation values)
\[
\psi_{{\bf k}\sigma}({\bf r})= \frac{1}{\sqrt{\Omega}}\exp{(i{\bf kr})}\xi_{\sigma}
\]
where ${\bf k}$ is the wave number and  $\xi_{\sigma}$ is a spin function for either spin up or down
\[ 
\xi_{\sigma=+1/2}=\left(\begin{array}{c} 1 \\ 0 \end{array}\right) \hspace{0.5cm}
\xi_{\sigma=-1/2}=\left(\begin{array}{c} 0 \\ 1 \end{array}\right).\]

We assume that we have periodic boundary conditions which limit the allowed wave numbers to
\[
k_i=\frac{2\pi n_i}{L}\hspace{0.5cm} i=x,y,z \hspace{0.5cm} n_i=0,\pm 1,\pm 2, \dots
\]
We assume first that the particles interact via a central, symmetric and translationally invariant
interaction  $V(r_{12})$ with
$r_{12}=|{\bf r}_1-{\bf r}_2|$.  The interaction is spin independent.

The total Hamiltonian consists then of kinetic and potential energy
\[
\hat{H} = \hat{T}+\hat{V}.
\]
\begin{enumerate}
\item[a)] Show that the operator for the kinetic energy can be written as
\[
\hat{T}=\sum_{{\bf k}\sigma}\frac{\hbar^2k^2}{2m}a_{{\bf k}\sigma}^{\dagger}a_{{\bf k}\sigma}.
\]
Find also the number operator $\hat{N}$ and find a corresponding expression for the interaction
$\hat{V}$ expressed with creation and annihilation operators.   The expression for the interaction
has to be written in  $k$ space, even though $V$ depends only on the relative distance. It means that you ned to set up the Fourier transform $\langle {\bf k}_i{\bf k}_j| V | {\bf k}_m{\bf k}_n\rangle$.
\item[b)] We assume that  $V(r_{12}) < 0$ kand that the integral 
$\int |V(x)| d^3x < \infty$.

Use the operator form for $\hat{H}$ from the previous exercise and calculate
$E_0=\bra{\Phi_{0}}H\ket{\Phi_{0}}$ for this system to first order in perturbation theory
and express the result as a function of the density $\rho=N/\Omega$. 
The state $\ket{\Phi_{0}}$  is a Slater determinant determined by filling all states up to Fermi level.
Show that the system will collapse (you wil not be able to find an energy minimum). Comment your results.
\item[c)] We will now study the electron gas. The Hamilton operator is given by
\[
\hat{H}=\hat{H}_{el}+\hat{H}_{b}+\hat{H}_{el-b},
\]
with the electronic part
\[
\hat{H}_{el}=\sum_{i=1}^N\frac{p_i^2}{2m}+\frac{e^2}{2}\sum_{i\ne j}\frac{e^{-\mu |{\bf r}_i-{\bf r}_j|}}{|{\bf r}_i-{\bf r}_j|},
\]
where we have introduced an explicit convergence factor
(the limit $\mu\rightarrow 0$ is performed after having calculated the various integrals).
Correspondingly, we have
\[
\hat{H}_{b}=\frac{e^2}{2}\int\int d{\bf r}d{\bf r}'\frac{n({\bf r})n({\bf r}')e^{-\mu |{\bf r}-{\bf r}'|}}{|{\bf r}-{\bf r}'|},
\]
which is the energy contribution from the positive background charge with density
$n({\bf r})=N/\Omega$. Finally,
\[
\hat{H}_{el-b}=-\frac{e^2}{2}\sum_{i=1}^N\int d{\bf r}\frac{n({\bf r})e^{-\mu |{\bf r}-{\bf x}_i|}}{|{\bf r}-{\bf x}_i|},
\]
is the interaction between the electrons and the positive background.

Show that
\[
\hat{H}_{b}=\frac{e^2}{2}\frac{N^2}{\Omega}\frac{4\pi}{\mu^2},
\]
and
\[
\hat{H}_{el-b}=-e^2\frac{N^2}{\Omega}\frac{4\pi}{\mu^2}.
\]
Show thereafter that the final Hamiltonian can be written as 
\[
H=H_{0}+H_{I},
\]
with
\[
H_{0}={\displaystyle\sum_{{\bf k}\sigma}}
\frac{\hbar^{2}k^{2}}{2m}a_{{\bf k}\sigma}^{\dagger}
a_{{\bf k}\sigma},
\]
and
\[
H_{I}=\frac{e^{2}}{2\Omega}{\displaystyle\sum_{\sigma_{1}
\sigma_{2}}}{\displaystyle
\sum_{{\bf q}\neq 0,{\bf k},{\bf p}}}\frac{4\pi}{q^{2}}
a_{{\bf k}+{\bf q},\sigma_{1}}^{\dagger}
a_{{\bf p}-{\bf q},\sigma_{2}}^{\dagger}
a_{{\bf p}\sigma_{2}}a_{{\bf k}\sigma_{1}}.
\] 
\item[d)] Calculate
$E_0/N=\bra{\Phi_{0}}H\ket{\Phi_{0}}/N$ for
for this system to first order in the interaction.
Show that, by using
\[
\rho= \frac{k_F^3}{3\pi^2}=\frac{3}{4\pi r_0^3},
\]
with $\rho=N/\Omega$, $r_0$
being the radius of a sphere representing the volume an electron occupies 
and the Bohr radius $a_0=\hbar^2/e^2m$, 
that the energy per electron can be written as 
\[
E_0/N=\frac{e^2}{2a_0}\left[\frac{2.21}{r_s^2}-\frac{0.916}{r_s}\right].
\]
Here we have defined
$r_s=r_0/a_0$ to be a dimensionless quantity.

Plot your results and link your discussion to the result in exercise b). Why is this system stable?
\item[e)] 
Calculate thermodynamical quantities like the pressure, given by
\[
P=-\left(\frac{\partial E}{\partial \Omega}\right)_N,\]
and the bulk modulus
\[
B=-\Omega\left(\frac{\partial P}{\partial \Omega}\right)_N,\]
and comment your results.
\item[f)] 
The single-particle Hartree-Fock energies are given by the expression 
\[
\varepsilon_{k}^{HF}=\frac{\hbar^{2}k^{2}}{2m}-\frac{e^{2}
k_{F}}{2\pi}
\left[
2+\frac{k_{F}^{2}-k^{2}}{kk_{F}}ln\left\vert\frac{k+k_{F}}
{k-k_{F}}\right\vert
\right].
\]
(You don't need to calculate this quantity).  How can you use the Hartree-Fock energy to 
find the ground state energy? Are there differences between the Hartree-Fock results and those you 
found in exercise d)?  Comment your results.
\end{enumerate}
\end{prob}
\begin{prob}
\subsection*{Exercise 20}
Show Thouless' theorem: An arbitrary Slater determinant $\ket{c'}$ which is not orthogonal to a determinant
$\ket{c}={\displaystyle\prod_{i=1}^{n}}
a_{\alpha_{i}}^{\dagger}\ket{0}$, can be written as
\[
\ket{c'}=exp\left\{\sum_{p=\alpha_{n+1}}^{\infty}
\sum_{h=\alpha_{1}}^{\alpha_{n}}C_{ph}a_{p}^{\dagger}
a_{h}\right\}\ket{c}
\]
\end{prob}
\begin{prob}
We have
\[
\ket{c}=\prod_{k>0}(u_{k}+v_{k}a_{k}^{\dagger}a_{-
k}^{\dagger})\ket{0}
\]
with $u_{k}^{2}+v_{k}^{2}=1$,
$\hat{N}={\displaystyle \sum_{\nu}}a_{\nu}^{\dagger}a_{\nu}$ and
$\hat{H}=-|G|{\displaystyle
\sum_{\nu,\nu'>0}}a_{\nu}^{\dagger}a_{-
\nu}^{\dagger}a_{-\nu'}a_{\nu'}$.
Show that
\begin{enumerate}
\item[a)]
\[
\left \langle c|c\right \rangle=\prod_{k>0}(u_{k}^{2}+v_{k}^{2})
\]
\item[b)]
\[
\bra{c}\hat{N}\ket{c}=2\sum_{\nu >0}v_{\nu}^{2}
\]
\item[c)]
\[
\bra{c}\hat{N}^{2}\ket{c}=4\sum_{\nu>0}
v_{\nu}^{2}+4\sum_{\nu\neq\nu '>0}v_{\nu}^{2}v_{\nu '}^{2}
\]
\item[d)]
\[
(\Delta
N)^{2}=\bra{c}\hat{N}^{2}\ket{c}-(\bra{c}\hat{N}\ket{c})^{2}
	   =4{\displaystyle \sum_{\nu >0}}u_{\nu}^{2}v_{\nu}^{2}
\]
\item[e)]
\[
\bra{c}\hat{H}\ket{c}=-|G|\left[\sum_{\nu
>0}u_{\nu}v_{\nu}\right]^{2}
		-|G|\sum_{\nu >0}v_{\nu}^{4}
\]
\end{enumerate}
\end{prob}
