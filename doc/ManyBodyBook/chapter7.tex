\chapter{Large-scale diagonalization methods and the nuclear shell model}

\section{Introduction}


\section{Basics of the shell model}


\subsection{Two-body Hamiltonian and their applications to closed-shell systems}

\subsection{Definition of model spaces and configuration mixing}
\subsection{Configurations mixing and discussion of specific two-valence nucleon systems like $^{18}$O and $^{42}$Ca}

\section{Slater determinants as basis states,}

The simplest possible choice for many-body wavefunctions are
\textit{product} wavefunctions.
\[ 
\Psi(x_1, x_2, x_3, \ldots, x_A) \approx \phi_1(x_1) \phi_2(x_2) \phi_3(x_3) \ldots
\]
because we are really only good  at thinking about one particle at a time. Such 
product wavefunctions, without correlations (e.g. Jastrow-type functions), are easy to 
work with; for example, if the single-particle states $\{ \phi_i(x)\}$ are orthonormal, then 
the product wavefunctions are easy to orthonormalize.   
Similarly, computing matrix elements of operators are relatively easy, because the 
integrals factorize.
The price we pay is the lack of correlations, which we must build up by using many, many product 
wavefunctions. (Thus we have a trade-off: compact representation of correlations but 
difficult integrals versus easy integrals but many states required.) 

Because we have fermions, we are required to have antisymmetric wavefunctions, e.g.
\[
\Psi(x_1, x_2, x_3, \ldots, x_A) = - \Psi(x_2, x_1, x_3, \ldots, x_A)
\]
etc. This is accomplished formally by using the determinantal formalism:
\[
\Psi(x_1, x_2, \ldots, x_N) 
= \frac{1}{\sqrt{N!}} 
\det \left | 
\begin{array}{cccc}
\phi_1(x_1) & \phi_1(x_2) & \ldots & \phi_1(x_N) \\
\phi_2(x_1) & \phi_2(x_2) & \ldots & \phi_2(x_N) \\
 \vdots & & &  \\
\phi_N(x_1) & \phi_N(x_2) & \ldots & \phi_N(x_N) 
\end{array}
\right |
\]
Product wavefunction + antisymmetry = Slater determinant. 
\[
\Psi(x_1, x_2, \ldots, x_N) 
= \frac{1}{\sqrt{N!}} 
\det \left | 
\begin{array}{cccc}
\phi_1(x_1) & \phi_1(x_2) & \ldots & \phi_1(x_N) \\
\phi_2(x_1) & \phi_2(x_2) & \ldots & \phi_2(x_N) \\
 \vdots & & &  \\
\phi_N(x_1) & \phi_N(x_2) & \ldots & \phi_N(x_N) 
\end{array}
\right |
\]
Properties of the determinant (interchange of any two rows or 
any two columns yields a change in sign; thus no two rows and no 
two columns can be the same) lead to the Pauli principle:

As a practical matter, however, Slater determinants beyond $N=4$ quickly become 
unwieldy. 
The occupation representation, using fermion \textit{creation} and \textit{annihilation} 
operators, is compact and efficient. It is also abstract and, at first encounter, not easy to 
internalize. It is inspired by other operator formalism, such as the ladder operators for 
the harmonic oscillator or for angular momentum, but unlike those cases, the operators 
\textit{do not have coordinate space representations}.

Instead, one can think of fermion creation/annihilation operators as a game of symbols that 
compactly reproduces what one would do, albeit clumsily, with full coordinate-space Slater 
determinants. 

We start with a set of orthonormal single-particle states $\{ \phi_i(x) \}$. 
(Note: this requirement, and others, can be relaxed, but leads to a 
more involved formalism.) \textit{Any} orthonormal set will do. 
To each single-particle state $\phi_i(x)$ we associate a creation operator 
$\hat{a}^\dagger_i$ and an annihilation operator $\hat{a}_i$. 
When acting on the vacuum state $| 0 \rangle$, the creation operator $\hat{a}^\dagger_i$ causes 
a particle to occupy the single-particle state $\phi_i(x)$:
\[
\phi_i(x) \rightarrow \hat{a}^\dagger_i |0 \rangle
\]

But with multiple creation operators we can occupy multiple states:
\[
\phi_i(x) \phi_j(x^\prime) \phi_k(x^{\prime \prime}) 
\rightarrow \hat{a}^\dagger_i \hat{a}^\dagger_j \hat{a}^\dagger_k |0 \rangle.
\]

Now we impose antisymmetry, by having the fermion operators satisfy 
\textit{anticommutation relations}:
\[
\hat{a}^\dagger_i \hat{a}^\dagger_j + \hat{a}^\dagger_j \hat{a}^\dagger_i
= [ \hat{a}^\dagger_i ,\hat{a}^\dagger_j ]_+ 
= \{ \hat{a}^\dagger_i ,\hat{a}^\dagger_j \} = 0
\]
so that 
\[
\hat{a}^\dagger_i \hat{a}^\dagger_j = - \hat{a}^\dagger_j \hat{a}^\dagger_i
\]
Because of this property, automatically $\hat{a}^\dagger_i \hat{a}^\dagger_i = 0$, 
enforcing the Pauli exclusion principle.  Thus when writing a Slater determinant 
using creation operators, 
\[
\hat{a}^\dagger_i \hat{a}^\dagger_j \hat{a}^\dagger_k \ldots |0 \rangle
\]
each index $i,j,k, \ldots$ must be unique.
The annihilation operators also anticommute $\{ \hat{a}_i, \hat{a}_j \} = 0$ so that
\[
\hat{a}_i \hat{a}_j = - \hat{a}_j \hat{a}_i
\]
Furthermore, in case it is not obvious, the adjoint of a Slater determinant is 
\[
\left(\hat{a}^\dagger_i \hat{a}^\dagger_j \hat{a}^\dagger_k \ldots |0 \rangle\right)^\dagger
= \langle 0 | \ldots \hat{a}_k \hat{a}_j \hat{a}_i.
\]

We need two more rules. The first is that an annihilation operator 
$\hat{a}_i$ acting on a ket vacuum $| 0 \rangle$...\textit{annihilates} it entirely:
\[
\hat{a}_i | 0 \rangle = 0
\]
(also the adjoint: $\langle 0 | \hat{a}^\dagger  = 0$); 
\smallskip

furthermore the creation and annihilation operators with the same index $i$ have a 
special anticommutator:
\[
\{ \hat{a}_i , \hat{a}^\dagger_j \} = \delta_{ij}
\]
so that if $i,j$ are different the operators always anticommute.

For computational work, we can go further and represent Slater determinants one of 
two ways:

\begin{itemize}

\item We can simply write down a list of occupied states, e.g. $ | 1,2 \rangle$
$ | 2,3 \rangle$, $| 4, 5 \rangle$, etc.; or

\item We can use bit notation: a one {\tt 1} for occupied and a zero {\tt 0} 
for unoccupied: $| 11000 \rangle$, $|01100 \rangle$, $| 00011 \rangle$ etc. 
\end{itemize}

Both have their advantages and disadvantages; for small number of particles the list of 
occupied states is more compact, but for large number of particles storing Slater determinants 
in bit form can be advantageous and is commonly used. This can form the topic for the final presentation for some of you.

We have defined the ansatz for the ground state as 
\[
|\Phi_0\rangle = \left(\prod_{i=1}^n}\hat{a}_{i}^{\dagger}\right)|0\rangle,
\]
where the $i$ define different single-particle states up to the Fermi level. We have assumed that we have $n$ fermions. 
A given one-particle-one-hole ($1p1h$) state can be written as
\[
|\Phi_i^a\rangle = \hat{a}_{a}^{\dagger}\hat{a}_i|\Phi_0\rangle,
\]
while a $2p2h$ state can be written as
\[
|\Phi_{ij}^{ab}\rangle = \hat{a}_{a}^{\dagger}\hat{a}_{b}^{\dagger}\hat{a}_j\hat{a}_i|\Phi_0\rangle,
\]
and a general $npnh$ state as 
\[
|\Phi_{ijk\dots}^{abc\dots}\rangle = \hat{a}_{a}^{\dagger}\hat{a}_{b}^{\dagger}\hat{a}_{c}^{\dagger}\dots\hat{a}_k\hat{a}_j\hat{a}_i|\Phi_0\rangle.
\]

We can then expand our exact state function for the ground state 
as
\[
|\Psi_0\rangle=C_0|\Phi_0\rangle+\sum_{ai}C_i^a|\Phi_i^a\rangle+\sum_{abij}C_{ij}^{ab}|\Phi_{ij}^{ab}\rangle+\dots
=(C_0+\hat{C})|\Phi_0\rangle,
\]
where we have introduced the so-called correlation operator 
\[
\hat{C}=\sum_{ai}C_i^a\hat{a}_{a}^{\dagger}\hat{a}_i  +\sum_{abij}C_{ij}^{ab}\hat{a}_{a}^{\dagger}\hat{a}_{b}^{\dagger}\hat{a}_j\hat{a}_i+\dots
\]
Since the normalization of $\Psi_0$ is at our disposal and since $C_0$ is by hypothesis non-zero, we may arbitrarily set $C_0=1$ with 
corresponding proportional changes in all other coefficients. Using this so-called intermediate normalization we have
\[
\langle \Psi_0 | \Phi_0 \rangle = \langle \Phi_0 | \Phi_0 \rangle = 1, 
\]
resulting in 
\[
|\Psi_0\rangle=(1+\hat{C})|\Phi_0\rangle.
\]
We rewrite 
\[
|\Psi_0\rangle=C_0|\Phi_0\rangle+\sum_{ai}C_i^a|\Phi_i^a\rangle+\sum_{abij}C_{ij}^{ab}|\Phi_{ij}^{ab}\rangle+\dots,
\]
in a more compact form as 
\[
|\Psi_0\rangle=\sum_{PH}C_H^P\Phi_H^P=\left(\sum_{PH}C_H^P\hat{A}_H^P\right)|\Phi_0\rangle,
\]
where $H$ stands for $0,1,\dots,n$ hole states and $P$ for $0,1,\dots,n$ particle states. 
Our requirement of unit normalization gives
\[
\langle \Psi_0 | \Phi_0 \rangle = \sum_{PH}|C_H^P|^2= 1,
\]
and the energy can be written as 
\[
E= \langle \Psi_0 | \hat{H} |\Phi_0 \rangle= \sum_{PP'HH'}C_H^{*P}\langle \Phi_H^P | \hat{H} |\Phi_{H'}^{P'} \rangle C_{H'}^{P'}.
\]

Normally 
\[
E= \langle \Psi_0 | \hat{H} |\Phi_0 \rangle= \sum_{PP'HH'}C_H^{*P}\langle \Phi_H^P | \hat{H} |\Phi_{H'}^{P'} \rangle C_{H'}^{P'},
\]
is solved by diagonalization setting up the Hamiltonian matrix defined by the basis of all possible Slater determinants. A diagonalization
is equivalent to finding the variational minimum  of 
\[
 \langle \Psi_0 | \hat{H} |\Phi_0 \rangle-\lambda \langle \Psi_0 |\Phi_0 \rangle,
\]
where $\lambda$ is a variational multiplier to be identified with the energy of the system.
The minimization process results in 
\[
\delta\left[ \langle \Psi_0 | \hat{H} |\Phi_0 \rangle-\lambda \langle \Psi_0 |\Phi_0 \rangle\right]=
\sum_{P'H'}\left\{\delta[C_H^{*P}]\langle \Phi_H^P | \hat{H} |\Phi_{H'}^{P'} \rangle C_{H'}^{P'}+
C_H^{*P}\langle \Phi_H^P | \hat{H} |\Phi_{H'}^{P'} \rangle \delta[C_{H'}^{P'}]  .\right
\]
\[
.\left -
\lambda( \delta[C_H^{*P}]C_{H'}^{P'}+C_H^{*P}\delta[C_{H'}^{P'}]\right\} = 0.
\]
Since the coefficients $\delta[C_H^{*P}]$ and $\delta[C_{H'}^{P'}]$ are complex conjugates it is necessary and sufficient to require the quantities that multiply with $\delta[C_H^{*P}]$ to vanish.  

This leads to 
\[
\sum_{P'H'}\langle \Phi_H^P | \hat{H} |\Phi_{H'}^{P'} \rangle C_{H'}^{P'}-\lambda C_H^{P}=0,
\]
for all sets of $P$ and $H$.

If we then multiply by the corresponding $C_H^{*P}$ and sum over $PH$ we obtain
\[ 
\sum_{PP'HH'}C_H^{*P}\langle \Phi_H^P | \hat{H} |\Phi_{H'}^{P'} \rangle C_{H'}^{P'}-\lambda\sum_{PH}|C_H^P|^2=0,
\]
leading to the identification $\lambda = E$. This means that we have for all $PH$ sets
\begin{equation}
\sum_{P'H'}\langle \Phi_H^P | \hat{H} -E|\Phi_{H'}^{P'} \rangle = 0.\label{eq:fullci}
\end{equation}
An alternative way to derive the last equation is to start from 
\[
(\hat{H} -E)|\Psi_0\rangle = (\hat{H} -E)\sum_{P'H'}C_{H'}^{P'}|\Phi_{H'}^{P'} \rangle=0, 
\]
and if this equation is successively projected against all $\Phi_H^P$ in the expansion of $\Psi$, then the last equation on the previous slide
results.   As stated previously, one solves this equation normally by diagonalization. If we are able to solve this equation exactly (that is
numerically exactly) in a large Hilbert space (it will be truncated in terms of the number of single-particle states included in the definition
of Slater determinants), it can then serve as a benchmark for other many-body methods which approximate the correlation operator
$\hat{C}$.  

Let us then become more practical again.

The first step in such codes--and in your project--is to construct the many-body basis.  
While the formalism is independent of the choice of basis, the \textit{effectiveness} of a calculation 
will certainly be basis dependent. 
Furthermore there are common conventions useful to know.

First, the single-particle basis has angular momentum as a good quantum number.  You can 
imagine the single-particle wavefunctions being generated by a one-body Hamiltonian, 
for example a harmonic oscillator.  Modifications include harmonic oscillator plus 
spin-orbit splitting, or self-consistent mean-field potentials, or the Woods-Saxon potential which mocks 
up the self-consistent mean-field. 

For nuclei, the harmonic oscillator, modified by spin-orbit splitting, provides a useful language 
for describing single-particle states.

Each single-particle state is labeled by the following quantum numbers: 
\begin{itemize}
\item Orbital angular momentum $l$

\item Intrinsic spin $s$ = 1/2 for protons and neutrons

\item Angular momentum $j = l \pm 1/2$

\item $z$-component $j_z$ (or $m$)

\item Some labeling of the radial wavefunction, typically $n$ the number of nodes in 
the radial wavefunction, but in the case of harmonic oscillator one can also use 
the principal quantum number $N$, where the harmonic oscillator energy is $(N+3/2)\hbar \omega$. 

\end{itemize}

In this format one labels states by $n(l)_j$, with $(l)$ replaced by a letter:
$s$ for $l=0$, $p$ for $l=1$, $d$ for $l=2$, $f$ for $l=3$, and thenceforth alphabetical.

 In practice the single-particle space has to be severely truncated.  This truncation is 
typically based upon the \textit{single-particle energies}, which is the effective energy 
from a mean-field potential. 

Sometimes we freeze the core and only consider a valence space. For example, one 
may assume a frozen $^{4}$He core, with 2 protons and 2 neutrons in the $0s_{1/2}$ 
shell, and then only allow active particles in the $0p_{1/2}$ and $0p_{3/2}$ orbits. 

Another example is a frozen $^{16}$O core, with 8 protons and 8 neutrons filling the 
$0s_{1/2}$,  $0p_{1/2}$ and $0p_{3/2}$ orbits, with valence particles in the 
$0d_{5/2}, 1s_{1/2}$ and $0d_{3/2}$ orbits.

Sometimes we refer to nuclei by the valence space where their last nucleons go.  
So, for example, we call $^{12}$C a $p$-shell nucleus, while $^{26}$Al is an 
$sd$-shell nucleus and $^{56}$Fe is a $pf$-shell nucleus.

There are different kinds of truncations.


For example, one can start with `filled' orbits (almost always the lowest), and then 
allow one, two, three... particles excited out of those filled orbits. These are called 
1p-1h, 2p-2h, 3p-3h excitations. 

Alternately, one can state a maximal orbit and allow all possible configurations with 
particles occupying states up to that maximum. This is called \textit{full configuration}.

Finally, for particular use in nuclear physics, there is the \textit{energy} truncation, also 
called the $N\hbar\Omega$ or $N_{max}$ truncation. 


 Here one works in a harmonic oscillator basis, with each major oscillator shell assigned 
a principal quantum number $N=0,1,2,3,...$. 

The $N\hbar\Omega$ or $N_{max}$ truncation:
Any configuration is given an noninteracting energy, which is the sum 
of the single-particle harmonic oscillator energies. (Thus this ignores 
spin-orbit splitting.)

The $N\hbar\Omega$ or $N_{max}$ truncation:
Excited state are labeled relative to the lowest configuration by the 
number of harmonic oscillator quanta
A case of $N=2$.
The $N\hbar\Omega$ or $N_{max}$ truncation:
Excited state are labeled relative to the lowest configuration by the 
number of harmonic oscillator quanta
Another case of $N=2$.

This truncation is useful because: if one includes \textit{all} configuration up to 
some $N_{max}$, and has a translationally invariant interaction, then the intrinsic 
motion and the center-of-mass motion factor. In other words, we can know exactly 
the center-of-mass wavefunction. 

In almost all cases, the many-body Hamiltonian is rotationally invariant. This means 
it commutes with the operators $\hat{J}^2, \hat{J}_z$ and so eigenstates will have 
good $J,M$. Furthermore, the eigenenergies do not depend upon the orientation $M$. 

Alternately, one can construct a many-body basis which has fixed $J$, or a $J$-scheme 
basis. 

The Hamiltonian matrix will have smaller dimensions (a factor of 10 or more)
 in the $J$-scheme than in the $M$-scheme. 
On the other hand, as we'll show in the next slide, the $M$-scheme is very easy to 
construct with Slater determinants, while the $J$-scheme basis states, and thus the 
matrix elements, are more complicated, almost always being linear combinations of 
$M$-scheme states. $J$-scheme bases are important and useful, but we'll focus on the 
simpler $M$-scheme.

The quantum number $m$ is additive (because the underlying group is Abelian): 
if a Slater determinant $\hat{a}_i^\dagger \hat{a}^\dagger_j \hat{a}^\dagger_k \ldots 
| 0 \rangle$ is built from single-particle states all with good $m$, then 
the total 
\[
M = m_i + m_j + m_k + \ldots
\]
This is \textit{not} true of $J$, because the angular momentum group SU(2) is not Abelian.

The upshot is that 
\begin{itemize}

\item It is easy to construct a Slater determinant with good total $M$;

\item It is trivial to calculate $M$ for each Slater determinant;

\item So it is easy to construct an $M$-scheme basis with fixed total $M$.

\end{itemize}

Note that the individual $M$-scheme basis states will \textit{not}, in general, 
have good total $J$. 
Because the Hamiltonian is rotationally invariant, however, the eigenstates will 
have good $J$. (The situation is muddied when one has states of different $J$ that are 
nonetheless degenerate.) 

Example: two $j=1/2$ orbits:
\begin{center}

\begin{tabular}{|c|c|c|c|c|} \hline
Index & $n$ & $l$  & $j$ & $m$ \\ \hline 
1 & 0 & 0 & 1/2 & -1/2 \\
2 & 0 & 0 & 1/2 &  1/2 \\
3 & 1 & 0 & 1/2 & -1/2 \\
4 & 1 & 0 & 1/2 & 1/2 \\ \hline
\end{tabular}
%
Note: the order is arbitrary
\end{center}
There are $\left ( \begin{array}{c} 4 \\ 2 \end{array} \right) = 6$ two-particle states, 
which we list with the total $M$:

\begin{center}
\begin{tabular}{|c|c|}\hline
Occupied & $M$  \\ \hline
1,2 & 0    \\
1,3 & -1              \\
1,4 & 0             \\
2,3 & 0           \\
2,4 & 1 \\
3,4 &  0     \\ \hline
\end{tabular}
There are 4 states with $M= 0$, 
and 1 each with $M = \pm 1$.
\end{center}

Example: consider using only single particle states from the $0d_{5/2}$ space. 
They have the following quantum numbers

\begin{center}

\begin{tabular}{|c|c|c|c|c|} \hline
Index & $n$ & $l$  & $j$ & $m$ \\ \hline 
1 & 0 & 2 & 5/2 & -5/2 \\
2 & 0 & 2 & 5/2 & -3/2 \\
3 & 0 & 2 & 5/2 & -1/2 \\
4 & 0 & 2 & 5/2 & 1/2 \\
5 & 0 & 2 & 5/2 & 3/2 \\
6 & 0 & 2 & 5/2 &  5/2 \\ \hline
\end{tabular}
%
\end{center}
There are $\left ( \begin{array}{c} 6 \\ 2 \end{array} \right) = 15$ two-particle states, 
which we list with the total $M$:

\begin{center}
\begin{tabular}{|c|c||c|c||c|c|}\hline
Occupied & $M$ & Occupied & $M$ &Occupied & $M$ \\ \hline
1,2 & -4   &  2,3 & -2  &  3,5 & 1 \\
1,3 & -3  &  2,4 & -1 &    3,6 & 2             \\
1,4 & -2  &  2,5 & 0 &     4,5 & 2            \\
1,5 & -1  &  2,6 & 1 &     4,6 & 3          \\
1,6 & 0  &  3,4 & 0 &      5,6 & 4         \\ \hline
\end{tabular}

\end{center}
There are 3 states with $M= 0$, 2 with $M = 1$, and so on.

The first step  is to construct the $M$-scheme basis of Slater determinants.
Here $M$-scheme means the total $J_z$ of the many-body states is fixed.

The steps could be:

\begin{itemize}

\item  Read in a user-supplied file of single-particle states (examples can be given) or just code these internally;

\item Ask for the total $M$ of the system and the number of particles $N$;

\item Construct all the $N$-particle states with given $M$.  You will validate the code by 
comparing both the number of states and specific states.

\end{itemize}
The format of a possible input  file could be

\begin{verbatim}
12         ! number of single-particle states
1     1     0      1     -1      !   index   nrnodes   l    2 x j      2 x jz
2     1     0      1      1
3     0     2      3     -3
4     0     2      3     -1
5     0     2      3      1
6     0     2      3      3
7     0     2      5     -5
8     0     2      5     -3
9     0     2      5     -1
10    0     2      5      1
11    0     2      5      3
12    0     2      5      5
\end{verbatim}

This represents the $1s_{1/2}$-$0d_{3/2}$-$0d_{5/2}$ valence space, or the $sd$-space.  There are 
twelve single-particle states, labeled by an overall index, and which have associated quantum 
numbers the number of radial nodes, the orbital angular momentum $l$, and the 
angular momentum $j$ and third component $j_z$.  To keep everything as integers, we could store $2 \times j$ and 
$2 \times j_z$. 
To read in the single-particle states you need to:
\medskip

\begin{itemize}


\item Open the file 

\item  Read the number of single-particle states (in the above example, 12);  allocate memory; all you need is a single array storing $2\times j_z$ for each state, labeled by 
the index;

\item Read in the quantum numbers and store $2 \times j_z$ (and anything else you happen to want).

\end{itemize}
The next step is to read in the number of particles $N$ and the fixed total $M$ (or, actually, $2 \times M$). 
For this project we assume only a single species of particles, say neutrons, although this can be 
relaxed. \textit{Note}: Although it is often a good idea to try to write a more general code, given the 
short time alloted we would suggest you keep your ambition in check, at least in the initial phases of the 
project.  


You should probably write an error trap to make sure $N$ and $M$ are congruent; if $N$ is even, then 
$2 \times M$ should be even, and if $N$ is odd then $2\times M$ should be odd. 

The final step is to generate the set of $N$-particle Slater determinants with fixed $M$. 
The Slater determinants will be stored in occupation representation.  Although in many codes
this representation is done compactly in bit notation ({\tt 1}s and {\tt 0}s), but for 
greater transparency and simplicity we will list the occupied single particle states.
 Hence we can 
store the Slater determinant basis states as {\tt sd(i,j)}
an array of dimension $N_{SD}$, the number of Slater determinants, by $N$, the number of occupied 
state. So if for the 7th Slater determinant the 2nd, 3rd, and 9th single-particle states are occupied, 
then {\tt sd(7,1)} = 2, {\tt sd(7,2)} = 3, and {\tt sd(7,3)} = 9.

We can construct an occupation representation of Slater determinants by the \textit{odometer}
method.  Consider $N_{sp} = 12$ and $N=4$. 
Start with the first 4 states occupied, that is:
\noindent {\tt sd(1,:)= 1,2,3,4} (also written as $|1,2,3,4 \rangle$

Now increase the last occupance recursively:
\noindent {\tt sd(2,:)= 1,2,3,5}

\noindent {\tt sd(3,:)= 1,2,3,6}

\noindent {\tt sd(4,:)= 1,2,3,7}


$\ldots$

\noindent {\tt sd(9,:)= 1,2,3,12}

\smallskip

Then start over with 

\smallskip

\noindent {\tt sd(10,:)= 1,2,4,5}
and again increase the rightmost digit

\noindent {\tt sd(11,:)= 1,2,4,6}

\noindent {\tt sd(12,:)= 1,2,4,7}

$\ldots$

\noindent {\tt sd(17,:)= 1,2,4,12}

To help you, here is the algorithm in Fortran90.  Here {\tt N} is the number 
of particles, {\tt Nsp} is the number of single-particle states, and 
{\tt occ} is an array of dimension {\tt N}.  The following routine takes {\tt occ} and moves it 
to the next `odometer' reading:

\begin{verbatim}
   do j = N,1,-1
      if(occ(j) < Nsp-N+j )then
         l = occ(j)
         do k = j,N
             occ(k) = l+ 1+k-j  !  sequential order
         end do
         return
      end if
   end do
  occ(:) = 0  ! flag 

\end{verbatim}
 You need to confirm that you get all possible states. You might try a smaller value that is easily 
confirmed by hand. 
If there are $N_{sp}$ single-particle states and $N$ particles, the maximal number of Slater determinants
is $\left( \begin{array}{c} N_{sp} \\ N \end{array} \right )$.  You should confirm you get this.

When we restrict ourselves to an $M$-scheme basis, however, there will be fewer 
states.  You have two options. The first is simplest (and simplest is often best, at 
least in the first draft of a code): generate \textit{all} possible Slater determinants, 
and then extract from this initial list a list of those Slater determinants with a given 
$M$. (You will need to write a short function or routine that computes $M$ for any 
given occupation.)  
Alternately, and not too difficult, is to run the odometer routine twice: each time, as 
as a Slater determinant is calculated, compute $M$, but do not store the Slater determinants 
except the current one. You can then count up the number of Slater determinants with a 
chosen $M$.  Then allocated storage for the Slater determinants, and run the odometer 
algorithm again, this time storing Slater determinants with the desired $M$ (this can be 
done with a simple logical flag). 
\textit{Some example solutions}:  Let's begin with a simple case, the $0d_{5/2}$ space:
\begin{verbatim}
6         ! number of single-particle states
1     0     2      5     -5
2     0     2      5     -3
3     0     2      5     -1
4     0     2      5      1
5     0     2      5      3
6     0     2      5      5
\end{verbatim}
For two particles, there are a total of 15 states, which we list here with the total $M$:
\noindent $| 1,2 \rangle$, $M= -4$; 
 $| 1,3 \rangle$, $M= -3$

\noindent  $| 1,4 \rangle$, $M= -2$; 
 $| 1,5 \rangle$, $M= -1$

\noindent $| 1,5 \rangle$, $M= 0$; 
 $| 2,3 \rangle$, $M= -2$

\noindent $| 2,4 \rangle$, $M= -1$;
 $| 2,5 \rangle$, $M= 0$

\noindent $| 2,6 \rangle$, $M= 1$;
 $| 3,4 \rangle$, $M= 0$

\noindent $| 3,5 \rangle$, $M= 1$;
 $| 3,6 \rangle$, $M= 2$

\noindent $| 4,5 \rangle$, $M= 2$;
 $ | 4,6 \rangle$, $M= 3$

\noindent $| 5,6 \rangle$, $M= 4$

Of these, there are only 3 states with $M=0$. 

\textit{You should try} by hand to show that in this same single-particle space, that for 
$N=3$ there are 3 states with $M=1/2$ and for $N= 4$ there are also only 3 states with $M=0$. 

\textit{To test your code}, confirm the above. 

Also, 
for the $sd$-space given above, for $N=2$ there are 14 states with $M=0$, for $N=3$ there are 37 
states with $M=1/2$, for $N=4$ there are 81 states with $M=0$.

\frametitle{Project 2, step 1}
For our project, we will only consider the pairing model.
A simple space is the $(1/2)^2$ space
\begin{verbatim}
4         ! number of single-particle states
1     0     0      1     -1
2     0     0      1      1
3     1     0      1     -1
4     1     0      1      1
\end{verbatim}
For $N=2$ there are 4 states with $M=0$; show this by hand and confirm your code reproduces it. 
Another, slightly more challenging space is the $(1/2)^4$ space, that is, 
\begin{verbatim}
8         ! number of single-particle states
1     0     0      1     -1
2     0     0      1      1
3     1     0      1     -1
4     1     0      1      1
5     2     0      1     -1
6     2     0      1      1
7     3     0      1     -1
8     3     0      1      1
\end{verbatim}

For $N=2$ there are 16 states with $M=0$; for $N=3$ there are 24 states with $M=1/2$, and for 
$N=4$ there are 36 states with $M=0$. 

In the shell-model context we can interpret this as 4 $s_{1/2}$ levels, with $m = \pm 1/2$, we can also think of these are simple 4 pairs,  $\pm k, k = 1,2,3,4$. Later on we will 
assign single-particle energies,  depending on the radial quantum number $n$, that is, 
$\epsilon_k = |k| \delta$ so that they are equally spaced.
For application in the pairing model we can go further and consider only states with 
no ``broken pairs,'' that is, if $+k$ is filled (or $m = +1/2$, so is $-k$ ($m=-1/2$). 
If you want, you can write your code to accept only these, and obtain the following 
6 states:

$|           1,           2 ,          3         ,       4  \rangle , $

$|            1      ,     2        ,        5         ,       6 \rangle , $

$|            1         ,       2     ,           7         ,       8  \rangle , $

$|            3        ,        4      ,          5          ,      6  \rangle , $

$|            3        ,        4      ,          7         ,       8  \rangle , $

$|            5        ,        6     ,           7     ,           8  \rangle $

\begin{itemize}

\item Write small modules (routines/functions) ; avoid big subroutines 
that do everything. (But not too small.)

\item Write lots of error traps, even for things that are `obvious.'

\item Document as you go along.  For each subroutine/function write
a header that includes: (A) Main purpose of routine; (B) names and 
brief explanation of input 
variables, if any; (C) names and brief explanation of output variables, 
if any; (D) subroutines and functions called by this routine; (E)
called by which subroutines and functions

\end{itemize}

Hints for coding

\begin{itemize}

\item When debugging, print out intermediate values. It's almost impossible to debug a 
code by looking at it--the code will almost always win a `staring contest.'

\item Validate code with SIMPLE CASES. Validate early and often.   

\end{itemize}

The number one mistake is using a too complex a system to test. \textit{For example} 
If you are computing particles in a potential in a box, try removing the potential--you should get 
particles in a box. And start with one particle, then two, then three... Don't start with 
eight particles.

Our recommended occuptation representation, e.g. $| 1,2,4,8 \rangle$, is 
easy to code, but numerically inefficient when one has hundreds of 
millions of Slater determinants.
In state-of-the-art shell-model codes, one generally uses \textit{bit 
representation}, i.e. $|1101000100... \rangle$ where one stores 
the Slater determinant as a single (or a small number of) integer.

This is much more compact, but more intricate to code with considerable 
more overhead. There exist 
bit-manipulation functions (e.g. {\tt iand, ishft}, etc., in Fortran).  
This is left as a challenge for those of you who would like to study this topic further for the final project.


Going from an occupation listing to a bit representation:


\begin{verbatim}
integer :: Np          ! # of particles
integer :: occ(Np)  ! occupation listing
integer :: imask     ! a "mask"
integer :: bitSD      ! bit representation of a Slater Determinant
integer :: prevocc  ! the previous occupation
integer :: nshift    ! number of positions to shift

bitSD = 0     ! initialize

imask = 1   ! putting a bit in the first location
prevocc = 1
do i = 1,Np
        nshift = occ(Np) - prevocc
        imask = ishft(imask,nshift)   ! ishft shifts a binary word 
                                                      !  by nshift bits
        bitSD = bitSD+ imask
end do  ! loop over i
\end{verbatim}

Step by step...
\medskip

\begin{small}
{\scriptsize

Converting the occupation {\tt 1,2,4,8} to a bit:

Initially:

\noindent {\tt bitSD = 0}

Occupation of first particle is 1, {\tt imask = 1}

\noindent {\tt bitSD = 1}

\bigskip

Occupation of second particle is 2, {\tt imask = 2}

\noindent {\tt bitSD = 1+2 = 3};

\bigskip

Occupation of third particle is 4, {\tt imask = 8}

\noindent {\tt bitSD = 3+8 = 11};

\bigskip

Occupation of fourth particle is 8, {\tt imask = 128}

\noindent {\tt bitSD = 11+128 = 139}
Hence the occupation {\tt 1,2,4,8} is written as a single integer 139.

Given a bit representation of a Slater determinant, to determine if 
a single-particle state is occupied, use {\tt imask} and the 
{\tt iand} (in Fortran) function.
One can then create and destroy bits.
While more compact and thus reduces the memory requirement, 
you can see this procedure adds significant 
computing overhead.

We consider a space with $2\Omega$ single-particle states, with each 
state labeled by 
$k = 1, 2, 3, \Omega$ and $m = \pm 1/2$. The convention is that 
the state with $k>0$ has $m = + 1/2$ while $-k$ has $m = -1/2$.

\smallskip

The Hamiltonian we consider is 
\[
\hat{H} = -G \hat{P}_+ \hat{P}_-,
\]
where
\[
\hat{P}_+ = \sum_{k > 0} \hat{a}^\dagger_k \hat{a}^\dagger_{-{k}}.
\]
and $\hat{P}_- = ( \hat{P}_+)^\dagger$.

\bigskip

We will first solve this using a trick, the quasi-spin formalism, to obtain the 
exact results. Then we will try again using the explicit Slater determinant formalism.

One can show (and this is a good exercise!) that
\[
\left [ \hat{P}_+, \hat{P}_- \right ] = \sum_{k> 0} \left( \hat{a}^\dagger_k \hat{a}_k 
+ \hat{a}^\dagger_{-{k}} \hat{a}_{-{k}} - 1 \right) = \hat{N} - \Omega.
\]
Now define 
\[
\hat{P}_z = \frac{1}{2} ( \hat{N} -\Omega).
\]
Finally you can show
\[
\left [ \hat{P}_z , \hat{P}_\pm \right ] = \pm \hat{P}_\pm.
\]
This means the operators $\hat{P}_\pm, \hat{P}_z$ form an SU(2) algebra, and we can 
bring to bear all our intution about angular momentum, even though there is no actual 
angular momentum involved. 

\smallskip
So we rewrite the Hamiltonian to make this explicit:
\[
\hat{H} = -G \hat{P}_+ \hat{P}_- 
= -G \left( \hat{P}^2 - \hat{P}_z^2 + \hat{P}_z\right)
\]
Because of the SU(2) algebra, we \textit{know} that the eigenvalues of 
$\hat{P}^2$ must be of the form $p(p+1)$, with $p$ either integer or half-integer, and the eigenvalues of $\hat{P}_z$ 
are $m_p$ with $p \geq | m_p|$, with $m_p$ also integer or half-integer. 

\bigskip

But because $\hat{P}_z = (1/2)(\hat{N}-\Omega)$, we know that for $N$ particles 
the value $m_p = (N-\Omega)/2$. Furthermore, the values of $m_p$ range from 
$-\Omega/2$ (for $N=0$) to $+\Omega/2$ (for $N=2\Omega$, with all states filled). 

\smallskip

We deduce the maximal $p = \Omega/2$ and for a given $n$ the 
values range of $p$ range from $|N-\Omega|/2$ to $\Omega/2$ in steps of 1 
(for an even number of particles) 

\smallskip

Following Racah we introduce the notation
$p = (\Omega - v)/2$
where $v = 0, 2, 4,..., \Omega - |N-\Omega|$ 
With this it is easy to deduce that the eigenvalues of the pairing Hamiltonian are
\[
-G(N-v)(2\Omega +2-N-v)/4
\]
This also works for $N$ odd, with $v= 1,3,5, \dots$.
Let's take a specific example: $\Omega = 3$ so there are 6 single-particle states, 
and $N = 3$, with $v= 1,3$. Therefore there are two distinct eigenvalues, 
\[
E = -2G, 0
\]
Now let's work this out explicitly. The single particle space we write as 
\begin{center}
\begin{tabular}{|c| c| c|} \hline
Index & $k$ & $m$ \\ \hline
1 & 1 & -1/2 \\
2 & -1 & 1/2 \\
3 & 2 & -1/2 \\
4 & -2 & 1/2 \\
5 & 3 & -1/2 \\
6 & -3 & 1/2 \\ \hline
\end{tabular}
\end{center}


 There are  $\left( 
\begin{array}{c}6 \\ 3 \end{array} \right) = 20$ three-particle states, but there 
are 9 states with $M = +1/2$:

 $| 1,2,3 \rangle, |1,2,5\rangle, | 1,4,6 \rangle, | 2,3,4 \rangle, |2,3,6 \rangle, 
| 2,4,5 \rangle, | 2, 5, 6 \rangle, |3,4,6 \rangle, | 4,5,6 \rangle$.
In this basis, the operator 
\[
\hat{P}_+
= \hat{a}^\dagger_1 \hat{a}^\dagger_2 + \hat{a}^\dagger_3 \hat{a}^\dagger_4 +
\hat{a}^\dagger_5 \hat{a}^\dagger_6 
\]
From this we can determine that 
\[
\hat{P}_- | 1, 4, 6 \rangle = \hat{P}_- | 2, 3, 6 \rangle
= \hat{P}_- | 2, 4, 5 \rangle = 0
\]
so those states all have eigenvalue 0.

Now for further example, 
\[
\hat{P}_- | 1,2,3 \rangle = | 3 \rangle
\]
so
\[
\hat{P}_+ \hat{P}_- | 1,2,3\rangle = | 1,2,3\rangle+ | 3,4,3\rangle + | 5,6,3\rangle
\]
The second term vanishes because state 3 is occupied twice, and reordering the last 
term we
get
\[
\hat{P}_+ \hat{P}_- | 1,2,3\rangle = | 1,2,3\rangle+ |3, 5,6\rangle
\]
without picking up a phase.

 
