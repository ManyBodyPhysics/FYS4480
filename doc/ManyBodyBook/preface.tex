
Nearly all of physics is many-body physics at the most  
microscopic level of understanding.
A theoretical understanding of the behavior of quantum mechanical systems with many interacting particles,
accompanied with experiments and simulations,   
is a  great challenge and provides fundamental insights into systems governed by quantum mechanical principles.

Many-body physics  applications range from condensed matter physics
(for example antiferromagnets, quantum liquids and solids, plasmas, metals,
superconductors), nuclear and high energy physics (for example nuclear matter, finite
nuclei, quark-gluon plasmas), dense matter astrophysics (for example  neutron
stars, white dwarfs), atoms and molecules (for example electron correlations,
reactions, quantum chemistry), and elementary particles
(for example quantum field theory, lattice gauge field models). 

Microscopic many-body methodologies have attained a high degree of
sophistication in the last decades, from both theoretical and
applicative  points of view. These developments have allowed
researchers to adapt these technologies to a much wider range of physical
systems, and to obtain quantitative descriptions of many 
observables. 

Several of these methods have been developed independently of each other.
In some  cases the same methods have been applied and studied in different
fields of research with little overlap and exchange of knowledge. 
A classic example is the development of coupled-cluster approaches,
which originated from the nuclear many-body problem in the late fifties ({\bf give refs here later}). 
These methods were adapted and refined 
to extremely high precision and predicability by quantum chemists over 
the next four decades. The developments comprised, and still do,  
both methodological innovations and the usage of advances in high-performance computing.    
As time elapsed,   
the overlap with the original nuclear many-body problem vanished gradually, 
dwindling almost to a non-vanishing exchange of knowledge and methodologies. 
These methods were however revived in nuclear physics towards the turn of the last century and offer now a viable approach
to {\em ab initio}\footnote{We would like to come with a warning here, since it is our feeling that the 
labelling {\em ab initio} attached to many titles of scientific articles nowadays, has reached an inflationary stage. 
With {\em ab initio} we will mean solving Schr\"odinger's or Dirac's many-particle equations starting with the presumably best 
inter-particle Hamiltonian, making no approximations. The particles that make up the many-body problem will be approximated 
as the constituent ones. 
In the nuclear many-body problem
these are various baryons such as protons, neutrons, isobars and hyperons and low-mass mesons. All many-body approaches discussed in this text involve
some approximations, and only few of them, if any,  can reach the glorious stage of being truely {\em ab initio}.
Truely {\em ab initio} in nuclear physics means to solve the many-body problem starting with quarks and gluons.}
nuclear structure  studies of nuclei,   spanning the whole range of stable closed-shell and several open-shell nuclei. 


Many of the developments are also not 
always readily accessible to a larger community of
researchers and especially to the young and uninitiated ones. A notorious exception is 
again the quantum chemistry community, where highly effecient codes and methods are made available.  
Numerical packages like
Gamess, Dalton and Gaussian ({\bf give refs here later}) provide a reliable computational 
platform for many  chemical applications and studies.
In other fields this is not always the case.     

Nuclear physics, where we have our basic research experience, is such a field.
This book aims therefore at giving you an overview 
of widely used quantum mechanical  many-body methods. We want also to provide you with computational tools in order to enable you
to perform studies of realistic systems  in nuclear physics. 
The methods we discuss are however not limited to applications
in nuclear physics, rather, with  minor  changes one can apply them to other fields, from quantum chemistry, via studies of atoms and
molecules to materials science and solid state devices such as quantum dots.  

Typical methods we will discuss are 
configuration interaction and large-scale diagonalization.methods (shell-model in nuclear physics), perturbative many-body methods, 
coupled cluster theory, Green's function approaches, various Monte Carlo methods, extensions to weakly  bound systems and links
from {\em ab initio} methods to  density functional and mean field theories. 
The methods we have chosen are those which enjoy a great degree of popularity in low and intermediate energy nuclear physics, 
but due to both our limited knowledge
and due to space considerations, some of these will be treated in rather cavalier way.

Focussing on methods and tools only is however not necessarily a recipe for success, as we all know a 
a bad workman quarrels with his tools. Our aim is to provide 
you with a didactical description
of some of these  powerful methods and tools with the purpose of allowing you to
gain further insights and a possible working experience in modern many-body methods.
There are several textbooks on the market which cover specific 
many-body methods and/or applications, but almost none of them
presents an updated and comparative description of various many-body methods. The reason for choosing such an approach
is that we want you, the reader, to gain the experience and form your own opinion on the suitability and applicability
of specific methods. Too often you may have met many-body practitioners
claiming 'my method is better than your method because...'.  We wish to avoid such a pitfal, although ours, as any other approach,
is tainted by our biases, preferences and obviously ignorance about all possible details. 

We also  want  to shed light, in a hopefully unbiased way, on possible interplays and
differences between the various many-body methods with a focus
on the physics content. 
We have therefore selected several physical systems,
from simple models to realistic cases,  whose properties can be
described within the theoretical frameworks presented here.
In this text we will expose you to problems which tour many nuclear physics applications, 
from the basic forces which bind nucleons together to compact objects such as neutron stars.  
Our hope is that the material we have provided   allows an eventual reader to  assess 
by himself/herself the pro and cons of the methods discussed.

Moreover,  the text can be used and read as a large project. Each chapter ends with a specific project. The code you end up developing 
for that particular project, can in turn be  reused in the next chapter and its pertaining project.  To give you an example,
in chapter xx you will end up constructing your own nucleon-nucleon interaction and solve its Schr\"odinger equation using the Lippman-Schwinger
equation.  This solution results in the so-called $T$-matrix  which in turn can be related to the experimental phase shifts.
The Lippman-Schwinger equation can easily be modified to account for a given nuclear medium, resulting in the 
so-called $G$-matrix.  In chapter xx you end up computing such an effective interaction, using similarity transformations as well.
With these codes, you are in turn in the position where you can define effective interactions for say coupled-cluster calculations
(the topic of chapter xx) or shell-model effective interactions (topics for chapter xx and xx).

Our hope is that this will enable  you to assess at a much deeper level the pros and cons of the various methods.  
We believe firmly that knowledge of the strengths and weaknesses of a given method allows you to realize where
improvements can be made.   The text has also a strong focus on computational issues, from how to build up large many-body codes 
to parallelization and high-performance computing topics.



This texts spans some four hundred pages, and with the promised wide scope we aim at, 
there have to be topics which will not be dealt with properly.
In particular we will not cover  reaction theories.  
For reaction theories we provide all the inputs needed to compute
onebody densities, spectroscopic factors and optical potentials. 
For density functional theories we discuss how one can constrain the exchange correlations based on {\em ab initio} methods.
Furthermore, when it comes to the underlying nuclear forces, we will assume that various baryons and mesons are the 
effective degrees of freedom, limiting ourselves to low and intermediate energy nuclear physics.  A discussion of recent 
advances in lattice quantumchromodynamics (QCD) and its link to effective field theories is also outside the scope
of this book, although we will refer to advances in these fields as well and link the construction of nuclear forces to
undergoing research in effective field theory and lattice QCD.  
The material on infinite matter can be extended to finite temperatures, but we do not 
adventure into the realm of heavy-ion collisions and the search of the holy grail of quark-gluon plasma. 
We will throughout pay loyalty to a physical world governed by the degrees of freedom of baryons and mesons.
We apologize for these shortcomings, which are mainly due to
our lack of detailed research knowledge in the above fields. To be a jack of all trades leads normally to mastering none.

A final warning however.
This text is not a text on many-body theories, rather it is
a text on applications and implementations of many-body theories. This applies also to many of the algoritms  we discuss.  We do not go into
details about for example Gaussian quadrature or the mathematical foundations of random walks and the Metropolis algorithm.  We will often simply state results, but add 
enough references to tutorial texts so that you can look up the missing wonders of numerical mathematics yourself.
Similarly,  handling of angular momenta and their recouplings via $3j$, $6j$ or $9j$ symbols should  not  come as a bolt out from the blue. But again, don't despair,
we'll guide you safely to the appropriate literature.

We have therefore taken the liberty to have  certain expectations about you, the potential reader.  
We assume that you are at least a graduate student who is embarking on studies in 
nuclear physics and that
you have some familiarity with basic nuclear physics, angular momentum theory and many-body theory, 
typically at the level of texts like those of Talmi, Fetter and Walecka, Blaizot and Ripka, Dickhoff and Van Neck or similar monographs (add refs later). 
We will obviously state and repeat the basic rules and theorems, but will not go into derivations.  
Some of the derivations will however be left to you via various exercises interspersed througout the text.


\subsection*{Acknowledgements}
Friends/colleagues bla bla  and sponsors bla bla. also add that errors, misunderstandings etc are mainly due to ourselves!!







 






