\documentclass[a4paper,12pt]{report}

\usepackage{amsmath,amssymb}
\usepackage[english]{babel}
\usepackage{verbatim}
\usepackage{latexsym,amsfonts,mathrsfs,subfigure}
\usepackage{amsthm,amssymb}
\usepackage{graphicx}
\usepackage{graphics}
\usepackage{bbold}
\usepackage{minted} % packages needed for verbatim environments
\usepackage{algpseudocode}
\usepackage{algorithm}
\usepackage{cite}
\usepackage{fancyhdr}
\usepackage{url}

%%%%%%%%%%%%%%%%%%%%%%%%%%%%%%%%%%%%%%%%%%%%%%%%%%%%%%%%%%%%%%%%%%%%%%%%%%%%
%%%%%%%%%%%%%% Tikz %%%%%%%%%%%%%%%%%%%%%%%%%%%%%%%%%%%%%%%%%%%%%%%%%%%%%%%%
%%%%%%%%%%%%%%%%%%%%%%%%%%%%%%%%%%%%%%%%%%%%%%%%%%%%%%%%%%%%%%%%%%%%%%%%%%%%
\usepackage{tikz}
\usetikzlibrary{trees}
\usetikzlibrary{through}
\usetikzlibrary{topaths}
\usetikzlibrary{decorations.pathmorphing}
\usetikzlibrary{decorations.markings}
\usetikzlibrary{decorations.shapes}
\usetikzlibrary{decorations}
\usetikzlibrary{positioning,shadows,arrows}
\usetikzlibrary{fit}
\usetikzlibrary{shapes,snakes}
\usetikzlibrary{scopes}
%%%%%%%%%%%%%%%%%%%%%%%%%%%%%%%%%%%%%%%%%%%%%%%%%%%%%%%%%%%%%%%%%%%%%%%%%%%%
%%%%%%%%%%%%%%%%%%%%%%%%%%%%%%%%%%%%%%%%%%%%%%%%%%%%%%%%%%%%%%%%%%%%%%%%%%%%

\DeclareMathOperator\erf{erf}
\DeclareMathOperator\erfc{erfc}

\begin{document}

%%%%%%%%%%%%%%%%%%%%%%%%%%%%%%%%%%%%%%%%%%%%%%%%%%%%%%%%%%%%%%%%%%%%%%%%%
%%%%%%%%%%%%%%% Tikz %%%%%%%%%%%%%%%%%%%%%%%%%%%%%%%%%%%%%%%%%%%%%%%%%%%%
%%%%%%%%%%%%%%%%%%%%%%%%%%%%%%%%%%%%%%%%%%%%%%%%%%%%%%%%%%%%%%%%%%%%%%%%%
% Define styles for the different kind of edges in a Feynman diagram
\tikzset{
  line1/.style={draw=black,solid,line width=0.3mm}
}
%%%%%%%%%%%%%%%%%%%%%%%%%%%%%%%%%%%%%%%%%%%%%%%%%%%%%%%%%%%%%%%%%%%%%%%%%


\title{Coupled-cluster theory for infinite matter}
%\subtitle{Applications for nuclear matter and the electron gas}

\author{Gustav Baardsen}
\date{\today }

\maketitle

\chapter*{Acknowledgments}
%\addcontentsline{toc}{chapter}{Acknowledgments}

\chapter*{List of papers}
%\addcontentsline{toc}{chapter}{List of papers}

\tableofcontents

\chapter{Introduction}

\begin{itemize}
\item Why infinite nuclear matter?
\end{itemize}
Nucleosynthesis. \\
Core-collapse supernovae and colliding neutron stars are 
candidates for $r$-processes,
which is believed to be the process that has created a 
large portion of the heaviest elements 
\cite{arnould_2007,goriely_2012}. Neutron star
equation of state important here (including symmetry 
energy and stiffness of EOS). \cite{janka_2012} \\


\begin{itemize}
\item Why the electron gas?
\item Why coupled-cluster theory?
\item What I have done
\item An outline of the thesis
\item Nuclear physics long range plan: \verb+http://science.energy.gov/np/nsac/+
\end{itemize}

Computational physics, a third way beside experiments and
theory. High-performance computing plays a crucial role.


\chapter{Background}

The subject of this thesis is related to fields as different 
as nuclear astrophysics and nanotechnology. On a general level, we
are interested in questions considering, for example, the origin of 
the chemical elements, the stability of nuclear matter 
\cite{nap_2013}, and
physical properties of technological devices on a nanoscale
\cite{barsan_2010}.
To get a deeper understanding of these problems, we can study 
complex systems such as neutron stars, supernovae, atomic 
nuclei, and nanoscale transistors. The way to approach complicated 
systems in physics is by using simplified models. Hopefully,
the models can tell us something essential about 
the behaviour of the real physical objects. In this thesis,
we study theoretical approaches for one such class of models:
infinite matter. Papers I-III consider the nuclear interaction 
and nuclear-structure theory, and the main emphasis of the
thesis is therefore on applications for nuclear physics. 
As we define it, nuclear matter is a many-fermion system. 
Another well-known extended fermion system is the homogeneous 
electron gas. Due to the formal similarity, these two 
systems can often be studied using the same theoretical methods.  
We therefore use the homogeneous electron gas as a benchmark 
system, for comparisons with other established methods, and
to illuminate different aspects of the theoretical methods. 
In the following sections, we define infinite nuclear matter 
and the homogeneous electron gas, explain why the systems are 
important, and briefly review some of the related research. 
 

\section{Inifinite nuclear matter}


As far as we know, there are four different fundamental 
types of interactions in nature. These are the strong, weak, 
electromagnetic, and gravitational interactions. At high 
energies, the weak and electromagnetic interactions are shown
to converge towards the same interaction; the electroweak 
interaction \cite{mandl_shaw,martin_2009}. All these fundamental
interactions are present in nucleonic matter, which 
we define as matter with nucleons as building blocks.
As predicted initially by Gell-Mann and Zweig and verified 
later experimentally \cite{riordan_1992,martin_2009}, nucleons have
internal structure. Hadrons, including nucleons, consist 
of quarks and gluons, where the latter mediate the strong
interaction between the quarks. According to the successful 
standard model of elementary particles, strong 
interactions are described by a theory 
called Quantum Chromodynamics (QCD) 
\cite{martin_2009,braun_2009}. 
In addition to being the building blocks of atomic nuclei, 
nucleons are important basic constituents in stellar matter.    

Similarly as described in, for example, the review
article by Day \cite{day1967}, we define infinite nuclear matter
as an infinitely large system of homogeneously distributed 
nucleons. In particular, we study two different 
nuclear-matter systems: symmetric nuclear matter, which consists of 
an equal amount of protons and neutrons, and neutron matter, 
which contains only neutrons. In symmetric nuclear matter,
the electrical charge is by definition switched off. In 
finite nuclei, the gravitational force 
is vanishingly small \cite{wong_book} compared to the other 
fundamental interactions, whereas in macroscopic 
nuclear objects, such as supernovae \cite{janka_2012} 
and neutron stars \cite{heiselberg,lattimer2012}, the 
gravitational force plays an important role. However, the 
gravitational interaction only affects the density of the
system. When studying infinite nuclear matter, we neglect 
the weak interaction, which is responsible for $\beta $
decays \cite{martin_2009}, and concentrate only on residual 
effects of the strong interaction.    


\subsection{Connection to astrophysics} 
%% Nucleons are building blocks of atoms, planets, and stars, 
%% and contribute together to most of the mass in the visible 
%% universe.

To understand more about the origin of the 
chemical elements, the evolution of stars, and the inner 
mechanisms of the building blocks of matter, we need more
knowledge about the structure of nucleons and how they 
interact. In this thesis, we are particularly interested 
in the nuclear-matter equation of state. This equation of 
state describes strongly interacting matter containing
only protons and neutrons. As is reviewed in, for example, 
Ref.~\cite{braun_2009}, strongly interacting matter is 
predicted to appear in several different phases depending 
on the temperature and the baryochemical potential 
\footnote{The chemical potential is defined as the 
derivative of the total energy density with respect to the
particle density related to a given particle type 
\cite{heiselberg}. The baryochemical potential therefore 
tells how much energy is required to add one baryon,
such as for example a nucleon, to the medium.}.      
For example, at sufficiently high temperature and 
baryochemical potential, quarks and gluons become much 
less confined and form a so-called quark-gluon plasma. 
\footnote{At the European Organization for Nuclear Research (CERN) 
\cite{cern_qgp} and at Brookhaven National Laboratory 
\cite{rhic_qgp}, formation of quark-gluon plasma 
is studied using heavy-ion collisions.} In our work,
we assume that the temperature and baryochemical
potential (or density) are sufficiently low such that 
the quarks and gluons are confined as hadrons. 
According to Ref.~\cite{heiselberg}, phases containing 
deconfined quarks occur at densities around two to 
three times the saturation density of nuclear matter,
which is approximately $2.8 \cdot 10^{14}$ g $\cdot $ 
cm$^{-3}$. In this thesis, we study infinite nuclear 
matter at densities up to at most 
$10^{15}$ g $\cdot $ cm$^{-3}$. One should keep
in mind that at the highest densities we study, hadrons
may be mixed with deconfined quark-gluon plasma. 
At densities above the saturation point, other hadrons, 
such as pions, kaons, and hyperons may also coexist 
with nucleons \cite{heiselberg,lattimer2004}. 

It is still an open question how a large part of the heavy 
elements beyond iron have been formed. One of the 
currently best explanations for the formation of
heavy neutron-rich elements is a mechanism called
the rapid-neutron-capture process (r-process) 
\cite{arnould_2007,boyd_2008}. Presently, there is no 
agreement of where in the universe the necessary 
conditions for r-prosesses are fulfilled, but core-collapse 
supernovae and merging 
neutron stars are among the suggested locations.
According to Arnould \cite{arnould_2007}, accurate
r-prosess simulations require very precise data for 
a large number of stable and unstable nuclei. 
It is costly and difficult to measure all the required
nuclei experimentally, and therefore predictive 
theoretical methods will be essential to understand
the r-prosess\cite{arnould_2007}. Ideally, a predictive
microscopic model for many-nucleon systems should 
reproduce the experimental saturation point
of nuclear matter. In that way, nuclear-matter 
calculations is a good test-bed for microsopic theories.   
As part of the larger effort, such calculations are
necessary to accurately model r-prosesses. The nuclear-matter 
equation of state, as well as the closely related
symmetry energy, are also important components, for example, 
when predicting theoretically how much energy is released 
in a supernova type II explosion \cite{bethe_1990}. Likewise,
the same equation of state is an essential input to 
neutron-star models \cite{heiselberg,lattimer2012}.
 
We define a Type II supernova following Boyd 
\cite{boyd_2008}: Consider a heavy star with a mass 
larger than ten times the solar mass. When the star has burnt 
up its fuel, the core contains elements lighter than or equal 
to iron. When no more energy is released from fusion processes, 
the core of this big star may start to collapse due to the 
strong gravitational field. The stellar matter of the core 
gets compressed until the density is a few times the 
saturation value. The mechanisms of the following stages
are still poorly understood, but the result is a 
giant explosion where matter is thrown out and eventually
the core is left. In this so-called Type II supernova, 
the dense core may become a neutron star or even a black 
hole. 

There are several reasons why neutron stars are
scientifically interesting objects.
Firstly, neutron stars are part of several different
hypotheses for where the r-prosess could occur 
\cite{arnould_2007,goriely_2012}. Secondly, we can learn 
more about how strongly interacting matter behaves at high 
densities by comparing different 
theoretical neutron-star models with astrophysical 
observations \cite{lattimer2004}. According to current
observational data and theory \cite{lattimer2012}, neutron stars 
have masses between one and three times the solar mass 
inside a radius that is roughly 12 kilometers. This makes 
neutron stars to be among the densest known objects in the 
universe. General relativity, causality, and rotational 
arguments give upper and lower limits for the mass, as 
well as maximum radii corresponding to different masses 
\cite{lattimer2004}. In addition, different theoretical
approaches predict different relationships between the 
the neutron-star radius and mass 
\cite{heiselberg,lattimer2001,lattimer2004}. 


\begin{table}
  \begin{center}
    
    \begin{tabular}{c|c|c}
      \hline 
      \hline
      Layer & Density (g $\cdot $ cm$^{-3}$) & Composition \\
      \hline
      Inner core & $\approx 10^{15} $ & Nucleons and possibly \\
      && hyperons, Bose condensate \\
      && of kaons and pions, \\
      && and deconfined quark \\
      && matter \\
      \hline
      Outer core & $2 \cdot 10^{14} - 10^{15}$ & Superfluid neutrons, \\
      && superconducting protons, \\
      && electrons and muons \\
      \hline
      Inner crust & $4 \cdot 10^{11} - 2 \cdot 10^{14}$ & Neutron-rich nuclei, \\
      && superfluid neutrons, \\
      && and electrons \\
      \hline 
      Outer crust & $10^{6} - 4 \cdot 10^{11}$ & Heavy nuclei and \\
      && electrons \\
      \hline
      \hline
    \end{tabular}
  \end{center}
  \caption{The composition of a neutron star, as described in 
    Refs.~\cite{heiselberg,lattimer2004}. For comparison, the 
    saturation density of nuclear matter is approximately 
    $2.8 \cdot 10^{14}$ g $\cdot $ cm$^{-3}$.}
  \label{tab:comp_neutron_star}
\end{table}

Neutron stars contain strongly interacting matter in a number 
of different phases. In our description of the composition of
neutron stars, we follow Refs.~\cite{heiselberg,lattimer2004}. 
The phases are believed to exist in layers, forming an 
onion-shell-like structure. The main body of the star 
contains inner and outer cores,
constituting around 99 $\%$ of the total mass, as well as 
inner and outer crusts. Outside the crust, there is an 
envelope, and the neutron star is surrounded by an 
athmosphere. The neutron-star composition is illustrated, 
for example, in Fig.~3 of 
Ref.~\cite{lattimer2004}. In Table \ref{tab:comp_neutron_star},
we list the compositions and estimated densities of the core 
and crust. It is still uncertain exactly what 
kind of phases exist in the inner core, but possibly there is 
a mix of nucleons, hyperons, Bose condensates of kaons and 
pions, and maybe deconfined quark matter. 
The outer core contains mostly superfluid neutrons, but
also a fraction of superconducting protons. This layer is
neutralized by electrons and muons. In the crust, neutron-rich 
and heavy nuclei appear in lattice structures. The crust
also contains electrons and superfluid neutrons. 

Neutron-star models have many different sources of 
uncertainties. As described in, for example, 
Refs.~\cite{akmal1998,heiselberg}, the nuclear-matter 
equation of state is a central ingredient in neutron-star
models. Despite more than half a century of research,
most theoretical approaches are unable to predict the
correct saturation properties of symmetric nuclear
matter \cite{akmal1998,heiselberg,frick2003,li2006,
gandolfi2007,soma2008,dalen2009,hebeler2011,baldo2012,
sammarruca2012,carbone2013,hagen2014}. The 
symmetric nuclear-matter equation of state is therefore
still poorly understood. Another source of uncertainty 
in neutron-star models is nuclear forces including hyperons 
\cite{heiselberg,beane2012}. In neutron stars, the 
neutron-proton ratios are typically highly asymmetric,
and it is therefore necessary to have good models for 
asymmetric nuclear matter \cite{heiselberg}. As we explain 
in the next paragraph, the symmetry energy is also important in
neutron-star models \cite{horowitz2014}. The symmetry 
energy is defined as the difference between the energy per 
nucleon of pure neutron matter and symmetric nuclear matter. 
Because of the uncertainties related to the nuclear-matter 
equation of state, there are large differences in theoretical 
predictions of the symmetry energy \cite{muther1987,
engvik1997,lee1998,li2006,vidana2009,carbone2012}. 


Let us give a few examples that illustrate the importance
of the symmetry energy. There has been suggested two 
important mechanisms for cooling of 
neutron stars by neutrino emission: the modified Urca 
process, which is slow, and the fast so-called direct Urca 
process. It is still and open question whether neutron stars 
may cool by the direct Urca 
process or not \cite{lattimer2004}. The neutron-proton ratio
in neutron stars is directly related to the symmetry energy,
and the symmetry energy determines therefore if Urca processes 
are possible in a neutron star \cite{heiselberg,lattimer2004}. 
More reliable predictions of the symmetry energy would therefore 
help us to understand more about the cooling mechanisms of 
neutron stars. Also other neutron-star observables, such as,
for example, the radius-mass relation, are strongly dependent
on the neutron-proton ratio, and thereby the symmetry energy 
\cite{lattimer2000}. It is well-known that the composition of 
both neutron stars \cite{lattimer2004} and supernova cores 
\cite{botvina_2004} are sensitive to the symmetry energy. 
Better theoretical predictions for the symmetry energy will 
therefore enable more reliable astrophysical models.

The nuclear-matter equation of state has been studied 
experimentally using heavy-ion collisions, giving 
constraints to the equation of state at densities above 
the saturation point \cite{danielewicz2002,lynch2009}. 
According to Lattimer \cite{lattimer2012}, observational,
theoretical, and experimental results of neutron-star studies
have come to closer agreement during the last years.
In future, researchers hope that gravitational waves can 
give more information about the neutron-star equation of 
state \cite{maselli_2013}.


\vspace{1cm}

%\item According to our current understanding, most of the 
%  visible universe, about X mass-\%, is nucleonic matter. 

\begin{itemize}
\item Phase diagram of nuclear matter?   
\end{itemize}

Zero temperature is a good approximation if the temperature is 
much smaller than the Fermi temperature. The Fermi temperature 
for nuclear matter is very high.

\vspace{1cm}
Find latest literature!
\begin{itemize}
\item Ref. \cite{tews,erler_pre}
\item Temperature, relativity, electrons, muons, hyperons, 
  quarks, crust, etc.
\end{itemize}

\subsection{The infinite nuclear-matter problem} \label{sec:inf}

Infinite nuclear matter is, as we define it, a relativistic 
quantum system with an infinite number of interacting nucleons. 
Given the homogeneous structure, naively one may assume infinite
nuclear matter to be a very simple system to study. In fact, 
many theoretical many-body methods can be formulated easier for 
a homogeneous system than for, for example, finite nuclei. 
In particular, the single-particle basis, which for infinite 
nuclear matter is the plane-wave basis, is much simpler 
than for finite nuclei. Nevertheless, infinite
nuclear matter has been studied using microscopic approaches 
in more than half a century \cite{brueckner1954,brueckner_gammel}, 
and still there are large deviations \cite{fuchs2004,stone_2007} 
between predictions of different theoretical methods for this 
system.

Commonly, infinite nuclear matter is modelled using either an
\emph{ab initio} many-body method \cite{muther2000,fuchs2004} 
or a kind of mean-field approach \cite{bender2003,dalen2010}. 
The \emph{ab initio} methods use a Hamiltonian operator 
containing two- and many-body interactions. Normally,
the only adjustable parameters in an \emph{ab initio} calculation
are in the Hamiltonian. Optimal parameter values for the two-body 
interaction are obtained by fitting phase shifts to 
experimental nucleon-nucleon scattering data \cite{machleidt2001}.
The three-body interaction may have additional parameters which
are adjusted to, for example, observables of triton or helium
\cite{machleidt2011}, or to the saturation density and/or energy
of symmetric nuclear matter \cite{akmal1998,lejeune2000,soma2008,
gandolfi_2010}. In principle, \emph{ab initio} methods provide 
a systematic way of improving the result towards the exact 
solution, given a Hamiltonian operator. However, the results
are truly based on first principles only if that is also the case 
for the Hamiltonian operator. 

Self-consistent mean-field approaches  \cite{bender2003}, such 
as the Skyrme-Hartree-Fock method, the Gogny model, and 
relativistic mean-field theory, are methods closely related to 
density functional theory \cite{gross1995}, which is very
popular in quantum chemistry and condensed-matter physics, but 
here tailored to nuclear-physics applications.
In these methods \cite{bender2003}, an effective 
density-dependent interaction is 
used in self-consistent calculations. The effective interaction 
contains several free parameters, which are typically fitted to 
optimally reproduce observables of many different nuclei, and 
possibly also saturation properties of symmetric nuclear 
matter \cite{stone_2007}. In self-consistent mean-field
methods, the many-body problem is reduced to an effective
one-body problem, which makes these methods computationally
considerably simpler than \emph{ab initio} approaches. As a 
result, self-consistent mean-field approximations can be used
to study nuclei from the lightest to superheavy elements. 

Because self-consistent mean-field methods cannot be 
systematically improved, for example as function of a
converegence parameter, as
is common in \emph{ab initio} methods, the mean-field 
approaches are believed to have less predictive power than 
\emph{ab initio} approximations. \emph{Ab initio} calculations
provide therefore important tests for mean-field 
parametrizations. In this context, \emph{ab initio} 
predictions for the nuclear-matter equation of state play 
an important role \cite{stone_2007}. \emph{Ab initio} methods,
such as the nonrelativistic Brueckener-Hartree-Fock and 
the relativistic Dirac-Brueckner-Hartree-Fock approach, 
have been used \cite{dalen2010,baldo2004,cao2006,stone_2007} 
to extract density functionals based on the local density 
approximation. Even if the local density approximation has a 
microscopic foundation, in particular for systems where 
the density varies slowly, the quality of these functionals 
are currently not as good as more common mean-field 
approaches \cite{stone_2007}. The quality of local density
approximations for nuclear systems would probably improve
if we had more realistic nuclear-matter equations of state 
\cite{drut2010,hebeler2011}.

As pointed out by van Dalen and M{\"u}ther \cite{dalen2009}, 
because the saturation energy per particle in nuclear matter 
is much smaller than the nucleon mass in energy units, one might 
assume that relativistic effects are small in nucleonic 
systems. On the contrary, calculations show \cite{dalen2009} 
that relativistic effects give important contributions for 
infinite nuclear matter, in particular as compared to 
nonrelativistic approaches with only a two-body interaction.   
Nonnegligible relativistic effects are also found in
finite nuclei \cite{dalen2009}. According to
Ref.~\cite{dalen2009}, relativistic \emph{ab initio}
calculations for finite nuclei are complicated, and have 
therefore not yet been done. Instead, nonrelativistic
approaches for finite nuclei have been widely used 
\cite{pieper2001,dickhoff2004,navratil2009,hagen2013c},
giving results in good agreement with experiments, 
in particular when three-body forces are included 
\cite{pieper2001,hagen2012,roth2013,hammer2013,cipallone2013}.   
Relativistic and nonrelativistic mean-field methods
are used for finite nuclei and nuclear matter side by side,
and in the best calculations both approaches give 
approximately the same accuracy \cite{stone_2007}. 
As we see, nonrelativistic and relativistic microscopic
methods have a complementary role in describing
low-energy nuclear systems. Because this thesis considers
nonrelativistic \emph{ab initio} methods, we concentrate
on this group of methods in the rest of the present
section.

\begin{figure} 
  \centering
  \includegraphics[scale=0.5]{figures/coester_line/coester_line-crop.pdf}
  \caption{Even if different two-body interactions are 
    optimized to reproduce the same nucleon-nucleon
    phase shifts, they give very different predictions for
    the symmetric-nuclear-matter saturation point. 
    The figure shows results \cite{li2006} 
    obtained with
    the nonrelativistic Brueckner-Hartree-Fock method (BHF)
    and only two-body interactions, with the 
    Brueckner-Hartree-Fock method including
    three-body forces (BHF$+$TBF), and relativistic 
    results obtained with the Dirac-Brueckner-Hartree-Fock
    (DBHF) method.  The saturation
    points are given as the energy per particle $B/A$ at a 
    density $\rho $. Different points with the same color
    represent calculations with different two-body 
    interactions. The unfilled square represents the
    uncertainty region of the experimental saturation point.
    The figure is courtesy of Ref.~\cite{li2006}. Reprinted 
    by permission from $\dots $.
  }
  \label{fig:coester}
\end{figure}


When studying infinite nuclear matter, an important
test for the microscopic method is to compare the theoretical
saturation properties with experimental data. Inside a heavy 
nucleus, the density is approximately constant 
and similar for different nuclei \cite{walecka2004}. 
The density in the center of nuclei can be obtained from 
electron scattering experiments, and different estimates 
around 0.16--0.17 fm$^{-3}$ are used in the literature 
\cite{chabanat1997,mahaux1989b}. This is believed to be
the saturation density of nuclear matter, also in the
infinite-matter limit. In the special case of symmetric
nuclear matter, the semiempirical mass formula 
\cite{walecka2004} for nuclei contains only the volume 
term. The parameters in the semiempirical mass formula
can be determined by simultaneously optimizing with 
respect to many different finite nuclei \cite{chabanat1997}.   
Such an optimization gives an estimate for the volume
term, and thus also for the experimental binding
energy of symmetric nuclear matter. According to
Chabanat \emph{et al.}~\cite{chabanat1997}, the 
experimental binding energy of nuclear matter is 
$-(16\pm 0.2)$ MeV. 

Figure \ref{fig:coester} shows a typical set of saturation
points for symmetric nuclear matter, as obtained with 
different \emph{ab initio} many-body methods and different
interaction models. The results are taken from 
Ref.~\cite{li2006}. The figure presents saturation points 
obtained with the nonrelativistic Brueckner-Hartree-Fock 
(BHF) method \cite{brueckner1954,brueckner_gammel,bethe1971,
day1978,haftel_tabakin,mahaux1985,mahaux1989} and the 
Dirac-Brueckner-Hartree-Fock (DBHF) method 
\cite{anastasio1983,brockmann1984,horowitz1984,haar1987,
horowitz1987,brockmann1990}, which is 
a relativistic counterpart of the BHF approximation.
The uncertainty region of the experimental saturation 
point is marked by a blue unfilled square. As can be seen from the 
figure, saturation points obtained with the 
nonrelativistic BHF method and different two-body
interaction models (black circles) occur approximately on a 
line that does not cross the experimental uncertainty region. 
The phenomenon of saturation points obtained with
the same many-body method and different interaction
models aligning so that a higher saturation density 
means more binding is well-known for both 
nonrelativistic \cite{coester1970} 
and relativistic calculations \cite{fuchs2004}.
When including only a two-body interaction, 
nonrelativistic calculations typically give a so-called 
Coester band much farther from the experimental saturation 
region than relativistic calculations do \cite{fuchs2004}.
Figure \ref{fig:coester} also shows another commonly
observed feature:
When including three-body forces in nonrelativistic 
calculations, the saturation point is normally closer
to the experimental value than in calculations neglecting
three-body interactions. It is a prevailing understanding 
\cite{fuchs2004,hammer2013} that either three-body interactions 
are needed in nonrelativistic calculations or a relativistic 
many-body method is necessary to obtain saturation properties
in agreement with experimental data. 

In many-body perturbation theory, the energy can be 
expressed in terms of Goldsone diagrams \cite{bartlett_book}. 
The hole line approximation \cite{day1967,day1978} builds on 
the assumption that the total contribution from all Goldstone 
diagrams with 
$n$ independent hole lines is larger than the  
contribution from all diagrams containing $n+1$ hole lines. 
Given a truncation level $n$, all diagrams containing more
than $n$ hole lines are therefore neglected from the 
perturbational expansion. As Day explains \cite{day1967}, 
the truncation in the number of hole-lines is justified only 
at sufficiently low densities. According to Song 
\emph{et al.}~\cite{song}, the approximation breaks down 
above $3\rho_{0}$, where $\rho_{0}$ is the experimental 
nuclear-matter saturation density. The above-mentioned 
BHF method is equivalent to the 
lowest-order approximation of the hole-line expansion,
including diagrams with maximally two independent hole
lines. The BHF approach is one of the standard 
methods for infinite nuclear matter, and has been
used and developed in a large number of studies, starting 
in the 1950s with the publications of Brueckner and 
co-workers \cite{brueckner1954,brueckner_levinson,brueckner,
brueckner_gammel} and continuing until the present
\cite{day1967,haftel_tabakin,bethe1971,day1978,
jackson1983,mahaux1985,lejeune1986,mahaux1989,baldo1990,
engvik1996,engvik1997,song,schiller,vidana2000,suzuki,
zuo2002,zuo2003,li2006,burgio2008,vidana2011,inoue2013}.   
Mahaux \emph{et al.}~introduced continuous single-particle
energies \cite{mahaux1985,mahaux1989} in the $G$ matrix, 
giving a faster convergence \cite{baldo1990,song} in terms 
of hole lines than the traditional gap choice 
\cite{haftel_tabakin}. 
The BHF approximation has been extended to finite temperatures 
\cite{lejeune1986,zuo2003}, asymmetric nuclear matter
\cite{bombaci1991,engvik1996,zuo2002a}, and hyperonic 
matter \cite{schulze1998,vidana2000}, among others.
In Chapter \ref{ch:mbpt}, we review the theory and 
give details about the implementation of the BHF method.

In addition to the many BHF calculations, there are only 
a few studies of nuclear matter using higher-order 
hole-line approximations \cite{day1981,song1997,song}.     
By definition, Bethe-Brueckner-Goldstone (BBG) theory 
\cite{day1967,raja,day1978}
does not determine the exact form of the energy denominator. 
A necessary condition for convergence in the number of
hole lines is that the calculation does not depend on 
the choice of single-particle potentials in the energy
denominator. Song \emph{et al.}~have compared \cite{song} 
a three-hole-line approximation using the traditional
gap choice with a calculation using continuous 
single-particle energies. In this study, the two approaches 
gave approximately the same binding energy, differing by
less than 1 MeV around the saturation density. The weak
dependence on the choice of single-particle potentials
may indicate that a three-hole-line approximation is 
sufficient to obtain converged results \cite{song}.
In agreement with studies using other methods, 
the results of Song \emph{et al.}~show that it is 
necessary to take into account three-body forces in 
order to reproduce the experimental saturation point.

A method that is related to the hole-line approximation 
is coupled-cluster theory \cite{coester1958,coester1960,
  cizek1966,cizek1971,bartlett_review}. The first 
coupled-cluster calculations for nuclear matter were
done using the so-called Bochum truncation scheme 
\cite{kummel1978,day_cc}. Day and Zabolitzky did 
calculations \cite{day_cc} containing three-body equations, 
and included also an estimate of the four-body amplitudes. 
Whereas the inclusion of three-body 
terms gave a significant contribution to the binding
energy, their estimate including a subset of the 
four-body terms gave results very similar to the
three-body approximation. Provided that the  
approximation of the four-body equations was 
reasonable, the results showed convergence for the 
Bochum coupled-cluster method. Day and Zabolitzky 
also compared their coupled-cluster calculations with
the hole-line approximation. For Fermi momenta between
1.4 and 1.8 fm$^{-1}$, the four-hole line approximation 
gave binding energies differing at most about 2--3 MeV 
from their most accurate coupled-cluster results 
\cite{day_cc}. In Paper II \cite{baardsen}, we 
explain the differences between our coupled-cluster
implementation and the Bochum scheme. In Chapter 
\ref{ch:cc}, we give an introduction to coupled-cluster 
theory. 

Symmetric nuclear matter, pure neutron matter, and 
asymmetric neutron-star matter have been 
studied using many other nonrelativistic \emph{ab initio}
methods, and we will mention some of these only briefly.
There exist a large number of different methods derived from 
perturbation theory, some of which are obtained by partial
summation of certain diagrammatic classes to infinte order 
\cite{harris}. When using a bare interaction, nuclear 
matter is commonly known to be a nonperturbative system
\cite{bogner2005}. The $G$ matrix, which is used in BBG 
theory, was introduced to deal with this problem 
\cite{day1967}. Renormalization group (RG) theory provides 
an alternative, modern approach 
to obtain softer interactions that give the same phase 
shifts as bare nuclear interaction models. Nuclear-matter
calculations have been done \cite{bogner2005,bogner2010,
hebeler2010,hebeler2011,furnstahl2013} using two- and three-body 
interactions evolved to low momenta by using either the 
so-called $V_{\text{low} k}$ or the similarity renormalization 
group (SRG) method. When using a low-momentum interaction 
in symmetric nuclear matter,
the third-order particle-particle and hole-hole diagrams      
give only a small additional contribution compared to 
second-order perturbation theory \cite{hebeler2011}.
This perturbative behaviour in calculations with 
low-momentum potentials is potentially a great advantage.
On the other hand, in these calculations the RG interactions 
have a cutoff dependency that is nonnegligible and 
increasing for larger densities. Lately, Tews 
\emph{et al.}~have done \cite{tews} perturbation-theory 
calculations 
for neutron matter with nuclear interactions derived 
from chiral perturbation theory including the full 
next-to-next-to-next-to-leading order (N$^{3}$LO) 
contribution, with three- and four-body forces. 
Compared to calculations with three-body interactions
defined only to next-to-next-to-leading order (NNLO), the
inclusion of all N$^{3}$LO diagrams was found to be 
important.

The nonrelativistic particle-hole ring approximation has 
been implemented for nuclear matter to fourth order 
\cite{dickhoff1982}. Other methods derived from many-body 
perturbation theory are for example the model-space BHF
\cite{ma1983,kuo1986,engvik1997a} and the model-space 
particle-particle ring \cite{song1987,jiang1988,engvik1997a,
siu2009} approximations. In the particle-particle ring diagram
approximation, particle-particle and hole-hole diagrams
are summed to infinite order, and this method is therefore
similar to our coupled-cluster ladder approximation 
presented in Paper II \cite{baardsen}. 

In the self-consistent Green's function (SCGF) method 
\cite{dickhoff1992,muther2000,dickhoff2004}, single-particle
and two-particle propagators are used to evaluate expectation
values of different operators. The standard SCGF 
approximation for extended nuclear matter includes 
particle-particle and hole-hole ladder diagrams to infinite 
order \cite{dickhoff2004}. In contrast to for
example the BHF and coupled-cluster methods, in the SCGF
approach the Fermi sea is correlated and the propagators
are said to be 'dressed'. According to Dickhoff and 
M{\"u}ther \cite{dickhoff1992}, perturbations of the Fermi
surface can become important in systems with strong 
correlations. Due to the symmetry between particle and 
hole states, one can show that the number of particles
is a conserved quantity in SCGF calculations 
\cite{dickhoff1992}. Calculations in SCGF theory also have 
the advantage that they can be conveniently compared to 
experiments through spectral functions, which are evaluated 
using propagators. As reviewed by Dickhoff and Barbieri 
\cite{dickhoff2004}, there have been introduced several
different approaches to deal with pairing instabilities, which 
occur in the SCGF method. In Paper II \cite{baardsen}, we 
explain why the pairing instability problem is not present
in coupled-cluster equations.  The developments of the 
SCGF method for nuclear matter until 2004 are reviewed in, 
for example, Ref.~\cite{dickhoff2004}. During the last 
decade, several new studies of infinite 
nuclear and neutron matter have been presented 
\cite{bozek2004,hassaneen2004,
frick2005,bozek2006,soma2006,gad2007,soma2008,soma2009,
soma2009b,rios2009,carbone2012,carbone2013}. 

Beside BBG, CC, and SCGF theory, different variational 
and Monte Carlo methods have been important 
in the study of infinite nuclear matter. Let us first 
consider two variational methods: the Fermi hypernetted 
chain (FHNC) \cite{fantoni1998} and the variational Monte 
Carlo (VMC) \cite{pieper1998} approximations. In both the 
VMC and FHNC methods, the variational energy is written 
using a Jastrow-type wave-function ansatz. The Jastrow-type
ansatz of a nuclear system is often expanded as a sum 
of different operators \cite{pieper1998,fantoni1998}. 
In the VMC method, the variational energy is calculated 
using a Monte Carlo method \cite{pieper1998}. 
In contrast, the FHNC approach
results in a set of integral equations \cite{fantoni1998}.
The so-called variational chain summation (VCS) technique,
in which the Fermi-hypernetted-chain-single-operator
(FHNC-SOC) equations are solved, has been used in several
studies of nuclear and neutron star matter 
\cite{pandharipande1979,wiringa1988,akmal1997,
akmal1998,carlson_2003}. The VMC method has likewise been
applied to model infinite nuclear systems \cite{carlson_2003}. 
Whereas FHNC approximations can be formulated with integral 
equations at the thermodynamic limit, Monte Carlo methods are 
restricted to finite systems \cite{fantoni1998}. Infinite 
nuclear or neutron matter is therefore typically 
approximated by a box with a finite number of particles 
\cite{carlson_2003,gandolfi2007,gandolfi_2009}. Gandolfi 
\emph{et al.}~\cite{gandolfi_2009} have used twist-averaged 
boundary conditions to approximate the thermodynamic limit.

The variational energy estimate of the VMC method is 
restricted by the chosen Jastrow-type ansatz. The Green's 
function Monte Carlo (GFMC) method \cite{pieper1998} provides 
a recipe to improve the VMC energy to almost the exact value. 
After rewriting the Schr{\"o}dinger equation as a diffusion 
equation in imaginary time, the wave function is propagated 
towards a GFMC solution. Beacause of the fermion sign problem 
\cite{pudliner1997}, GFMC calculations cannot be systematically
improved to the exact solution. Carlson \emph{et al.}~have 
used the GFMC method to study pure neutron matter in both the 
normal \cite{carlson_2003} and superfluid phases 
\cite{gezerlis2010}. The GFMC method is computationally very 
expensive, and is therefore restricted to very small systems.
As suggested by Schmidt and Fantoni \cite{schmidt_1999},
the computational scaling of the GFMC approach can be
significantly improved by using a Hubbard-Stratonovich 
transformation, which makes it possible to sample both 
position and spin randomly. This approach, which is called
the auxiliary field diffusion Monte Carlo (AFDMC) method,
has been used by Gandolfi \emph{et al.}~in calculations
of symmetric nuclear matter \cite{gandolfi2007}, 
neutron matter \cite{gandolfi_2009}, and neutron-star 
matter \cite{gandolfi_2010}. The AFDMC method has
also been applied with a Jastrow-BCS wave-function ansatz
to model neutron matter in the superfluid phase 
\cite{gandolfi_2012b}. In Refs.~\cite{borasoy2008,
epelbaum2009}, another Monte Carlo projection method 
using lattice discretization has been applied to neutron
matter at low densities.   
 
Examples of other recent developments for infinite nuclear
matter are, for example, the chiral-perturbation-theory
approaches of Holt, Kaiser, and Weise \cite{holt2013}, and
the study of Inoue \emph{et al.}~\cite{inoue2013}, in which
a nuclear interaction model derived from lattice QCD is used.
In the latter calculations, the lattice-QCD two-body force 
still has unphysically large quark masses. 

%% More about this?
%% Let us finally
%% mention that neutron matter behaves as a universal Fermi 
%% gas at low densities. This property of neutron matter has been
%% investigated by Schwenk and Pethick \cite{schwenk_2005}.


\vspace{2cm}

A brief historical review $+$ current challenges
\begin{itemize}
\item K. Oyamatsu: Prog. Theor. Phys., Vol. 109, No. 4, 631, April 2003 
  \cite{oyamatsu2003}
\item Y. Dewulf: Saturation of nuclear matter and short-range correlations
\item State of the art. Results of other methods for infinite nuclear matter: including relativistic effects in the interaction $+$ $V_{lowk}$ (Dalen and M{\"u}ther) \cite{dalen2009}
\item Selfconsistent Green's function method: PPHH ladders and the pairing instability \cite{vonderfecht1993, alm1996, bozek1999}, thermodynamic consistency \cite{kadanoff1962, bozek2001}
\item N. Bassan, K. E. Schmidt: PRC 84, 035807, 2011
\item Virial expansion \cite{shen_2010}
\item Strongly interacting atomic fermion systems
  and neutron matter, which both are superfluid systems
  with large superfluidity gaps, behave in similar
  ways when the former system is suitably tuned 
  \cite{gezerlis_2008}
\end{itemize}


\section{The homogeneous electron gas}

The focus of this thesis is on coupled-cluster theory
for nuclear matter. Another important system with infinite 
extension is the electron gas \cite{giuliani2005}. 
The homogeneous electron gas is defined as a system of 
interacting electrons with a constant, neutralizing 
background charge \cite{fetter}. The electron gas, in one, 
two or three dimensions, is interesting as a test-bed for 
electron-electron correlations. The three-dimensional 
electron gas is particularly important as a cornerstone 
of the local-density approximation in density-functional 
theory \cite{gross1995}. In the physical world, systems 
similar to the three-dimensional electron gas can be 
found in, for example, alkali metals and doped 
semiconductors. Two-dimensional electron fluids are 
observed on metal and liquid-helium surfaces, as well as 
at metal-oxide-semiconductor interfaces. These and other 
physical realizations of the electron gas are presented and 
discussed in the textbook of Giuliani \cite{giuliani2005}.
We use the electron gas as a benchmark system
to compare coupled-cluster theory with other many-body 
methods. In particular, we concentrate on the two-dimensional
electron gas, for which there are very few coupled-cluster
studies \cite{freeman1978,freeman1983}.    

As we explain
in Sec.~\ref{sec:nuclear_int}, the nucleon-nucleon 
interaction has a range of only a few fermi, but is strongly 
repulsive when the nucleons are close to each other. 
Consequently, short-ranged correlations are important in 
nuclear matter, while longe-ranged correlations are less 
significant \cite{dewulf2003}. In contrast, the Coulomb 
interaction has an infinite range, and therefore 
long-ranged correlations play an essential role in the
electron gas. At low density, the electrons become 
localized and form a lattice. This so-called Wigner 
crystallization \cite{wigner1934} is a direct consequence 
of the long-ranged repulsive interaction. At higher
densities, the electron gas is better described as a
liquid \cite{ceperley1978,ceperley1980,
tanatar_ceperley_1989,bishop1982}. 
When using, for example, Monte Carlo methods 
\cite{foulkes2001}, the electron gas must be approximated 
by a finite system. The long-ranged Coulomb interaction 
in the electron gas causes additional finite-size effects 
\cite{fraser1996,chiesa2006,drummond2008} that are not
present in infinite nuclear matter (for the latter, see, 
for example, Paper III). Because of these differences, 
coupled-cluster approximations face other challenges when 
applied to the electron gas than when used to study 
infinite nuclear matter.


The electron gas has been studied using a large number 
of different approaches, and we will here only mention
some of the most important works that are relevant for
this thesis. We start with the three-dimensional electron 
gas, which has got most attention in the literature.
It is a well-kown fact that the correlation energy
of the three-dimensional electron gas diverges 
at second order in perturbation theory \cite{bruus2004}.
As is shown in the textbook of Bruus and Flensberg
\cite{bruus2004}, the particle-hole ring diagrams 
dominate at the limit of high density. Even though
all these diagrams diverge when calculated separately,
the energy converges when summing all direct 
particle-hole ring diagrams to infinite order.
Gell-Mann and Brueckner obtained for this so-called
random-phase approximation (RPA) the first terms of
the exact energy in the high-density limit 
\cite{gellmann1957}. 

%% Calculations in the RPA were
%% also done using dielectric-function theory 
%% \cite{pines1953}. 

At metallic and lower densities, short-ranged correlations
become significant. Singwi \emph{et al.} \cite{singwi1968}
and Lowy and Brown \cite{lowy1975} came up with early attemts 
to combine RPA with contributions for short-ranged 
correlations. The calculations of Singwi \emph{et al.}
were based on dielectric function theory, whereas Lowy and
Brown interpolated between short- and long-range models
using a diagrammatic technique. 

In 1978, Ceperley used \cite{ceperley1978} 
the variational Monte Carlo (VMC) method to 
study the two- and three-dimensional electron gas. The
obtained VMC ground-state energies were shown to be close
to other results at that time. The electron gas has also
been studied \cite{lantto1980} using the Fermi hypernetted 
chain (FHNC) method, which is another variational approch to the
quantum many-body problem. In both the VMC and the FHNC 
methods, the energy estimate is restricted by the chosen
Slater-Jastrow wave-function ansatz. As we discussed in 
Sec.~\ref{sec:inf}, higher accuracy can be obtained with
the diffusion Monte Carlo (DMC) method (in nuclear physics,
this method is commonly named Green's function Monte Carlo). 
The DMC results of Ceperley and Alder from 1980 
\cite{ceperley1980} are still, more than thirty years later, 
among the most accurate energy estimates for the 
three-dimensional electron gas. In Ref.~\cite{ceperley1980},
the fermion sign problem \cite{pudliner1997} is handled using 
a released-node approximation. Similar accuracy has been 
obtained in more recent calculations using the 
backflow-correlation technique \cite{kwon1998}, which is used 
to relax the simpler fixed-node approximation. Backflow 
correlations have also been used in other DMC studies, 
such as Refs.~\cite{holzmann2003,lopezrios2006,gurtubay2010}.
Ortiz and Ballone did DMC calculations \cite{ortiz1994} with 
a normal fixed-node approximation, but studied systems with
several different spin polarizations. 

When modelling the electron gas using a finite box, as is
commonly done in Monte Carlo methods, the energy has an error
compared to the electron gas at the thermodynamic limit. 
At the thermodynamic limit, plane-wave single-particle states
fill the Fermi sphere with a continuous spectrum. When the
system is approximated using a finite box, the 
single-particle spectrum becomes discrete. This 
discretization gives an 
error that is common to systems with short-ranged and 
long-ranged interactions. The perhaps most obvious way to
correct for finite-size effects related to a discrete 
single-particle basis, is by using an extrapolation formula.
As is described in Ref.~\cite{drummond2008}, results from
Hartree-Fock and density-functional theory calculations 
can be used to construct extrapolation methods. Another
approach that efficiently reduces the finite-size error
is a technique that utilizes so-called twisted boundary 
conditions \cite{lin2001}. Twisted boundary conditions 
means that the single-particle wave function is 
multiplied by a complex phase factor when moving from one 
simulation cell to a neighbouring cell. When averaging
over results obtained with different twist angles, the
energy estimates become much more accurate than when
using periodic boundary conditions \cite{lin2001}. In 
Paper III, twist-averaged boundary conditions are used to 
deal with finite-size effects in nuclear matter. 

Finite-size approximations of extended Coulombic systems, 
such as the electron gas, have additional errors that are 
caused by the long-ranged interaction \cite{fraser1996,
chiesa2006,drummond2008}. Interactions with electrons in
neighbouring cells can be summed using, for example,
Ewald's method \cite{ewald1921,fraser1996,wood2004,
drummond2008} (see also Sec.~\ref{sec:ccheg}). In 
Ewald's approach, each electron in the simulation cell
interacts with an infinite number of image charges located
at the same local position in all other cells. As is shown 
in Refs.~\cite{fraser1996,chiesa2006}, 
Ewald's interaction cannot describe the exchange-correlation
energy correctly. Chiesa \emph{et al.} \cite{chiesa2006}
used a static structure factor and a Jastrow factor
derived from RPA to estimate the correction to Ewald's
method. The correction technique of Chiesa \emph{et al.}
is directly applicable only to Monte Carlo methods.
Fraser \emph{et al.} \cite{fraser1996} suggested two 
different effective interactions that avoid the screening 
effects caused by Ewald's interaction . The most successful
alternative was to use a normal Coulomb interaction combined 
with the minimum-image convention \cite{fraser1996}.
The reader is referred to Ref.~\cite{drummond2008} for
more details about finite-size effects and different
approaches to correct for the related errors.

%% Chiesa \emph{et al.} \cite{chiesa2006} 
%% have derived simple correction terms to the Ewald energy 
%% of the electron gas using the random-phase approximation. The method of Chiesa 
%% \emph{et al.} is directly applicable only to Monte Carlo
%% calculations. 

%% Fraser \emph{et al.} 
%% \cite{fraser1996} suggest other approaches, in which the
%% Ewald interaction is replaced by a different effective
%% Coulomb interaction. Utilizing the minimum-image convention 
%% \cite{thijssen2007} together with a short-ranged 
%% approximation of the Coulomb interaction, they get  
%% more accurate results than with the Ewald's method.  

The full configuration-interaction quantum Monte Carlo 
(FCIQMC) met\-hod \cite{booth2009} is a new approach to the 
quantum many-body problem, in which the full 
configuration-interaction (FCI) equations \cite{harris} are 
solved approximately using a Monte Carlo technique. Similarly 
as in the DMC or GFMC methods, the Schr{\"o}dinger equation is 
written as a diffusion equation with an imaginary time 
variable \cite{shepherd_2012a}. In the FCI method, the total 
wave-function ansatz 
is expressed as a linear combination of Slater determinants 
constructed from a given single-particle basis. In FCIQMC, 
the coefficients in the expansion of Slater determinants 
are obtained as the large-time limit of a random walk 
\cite{shepherd_2012a}. As we discussed above, Monte Carlo
calculations of the electron gas have an error related to 
the finite number of particles in the simulation cell. 
Methods such as FCIQMC, CC, and partial summations derived 
from many-body perturbation theory have and additional error 
when studying systems with a discrete single-particle basis: 
The result depends on the given set of both occupied and 
unoccupied single-particle states \cite{shepherd_2012b}. 
This error can be corrected for by using, for example, the 
single-point extrapolation technique \cite{shepherd_2012b,
shepherd_2012c} introduced by Shepherd \emph{et al.} The 
FCIQMC method has recently been applied to finite 
electron-gas systems \cite{shepherd_2012a,shepherd_2012b,
shepherd_2012c}, giving results in close agreement
with DMC calculations utilizing backflow correlations 
\cite{shepherd_2012a}. In Sec.~\ref{sec:ccheg},
we compare our CC calculations for the electron gas with 
FCIQMC results of Leikanger \cite{leikanger_pc}. 

Singal and Das \cite{singal1973} were the first to study 
the electron gas using a CC approach. The approximation
they used is similar to the BHF method \cite{haftel_tabakin}, 
and does not properly include particle-hole ring diagrams.
Later, Freeman did CC calculations \cite{freeman1977} in 
which only ring diagrams and their exchange parts were 
retained in a CC doubles approximation. The results of both
Singal and Das \cite{singal1973} and Freeman 
\cite{freeman1977} compared well with dielectric-function
approaches. Presently, the most accurate CC caclulations of 
the three-dimensional electron gas are those of Bishop and
L{\"u}hrmann \cite{bishop1978,bishop1982}. Bishop and
L{\"u}hrmann derived a CC SUB2 approximation (also called
CC doubles (CCD) \cite{bartlett_book}) for the electron gas, 
extended with some ladder contributions from higher-order
amplitudes. As the authors describe in 
Ref.~\cite{bishop1978}, the CCD approximation 
contains many more diagrammatic classes than
partial-summation techniques derived from perturbation 
theory. The CCD approximation takes account of 
particle-particle and hole-hole ladders, particle-hole
ring diagrams including exchange terms, and many other
diagrams to infinite order in perturbation theory. 
In their CC approximations, Bishop and L{\"u}hrmann 
replaced summations over hole states by averages. In
fact, the nine-dimensional CC amplitudes were simplified
to one-dimensional objects, with the absolute value
of the transfer momentum being the only variable. 
The authors showed that the state-average approximation
is accurate in the RPA approximation.
Bishop and L{\"u}hrmann neglected some diagrammatic 
classes, such as the hole-hole ladders and mixed ladders, 
which they assumed to be small. Despite all these 
approximations, the final CC energies are very accurate 
in a typical metallic density range. Relative 
differences of less than one percent
compared to DMC calculations of Ceperley and Alder
may indicate that the extended CCD approximation describes
most of the relevant correlations in the three-dimensional
electron gas \cite{bishop1982}. 

The CC study of Bishop and L{\"u}hrmann was very successful,
but still the calculations are based on a large number of
approximations. It would be desireable to apply CC theory
to the electron gas without the same approximations. 
Recently, Shepherd \emph{et al.} \cite{shepherd_2012b,
shepherd2013a,shepherd2013b,shepherd2013c} have 
taken up again the CC effort for the three-dimensional 
electron gas. The CC calculations for the electron gas
from the 1970s and 1980s were all done at the 
thermodynamic limit \cite{singal1973,freeman1977,
bishop1978,bishop1982}. Unfortunately, singularities
related to the Coulomb two-body interaction may complicate
numerical calculations at the thermodynamic limit. 
Shepherd \emph{et al.} approximate the electron gas using
finite cubic boxes \cite{shepherd_2012b,
shepherd2013a,shepherd2013b,shepherd2013c}. When using 
finite-size systems, the CC equations are simpler and most 
of the problems with ill-bahaved functions are avoided. 
Instead, one has to deal with errors related to finite 
particle numbers and single-particle bases, similarly as 
in the FCIQMC method \cite{shepherd_2012a,shepherd_2012b,
shepherd_2012c}.

%% Shepherd and Gr{\"u}neis did CC calculations 
%% \cite{shepherd2013a} in the CCSD(T) approximation,
%% which is called the perturbative triples CC \cite{},
%% and found  

%% The same study also showed that the 
%% CCD approximation can describe important features 
%% of the Wigner crystal. 
%% In the first 
%% part of their extensive study, they derived the RPA method
%% as an approximation of CC theory. In addition to numerical
%% calculations similar to those of Freeman \cite{freeman1977}, 
%% they got the same analytical result as Gell-Mann and 
%% Brueckner \cite{gellmann1957} for the high-density limit.

Let us finally turn our attention to the two-dimensional
electron gas. The two-dimensional electron gas is defined
in the same way as the three-dimensional counterpart, and 
similar approaches can therefore often be used to 
study both systems. As a first example, the classical 
derivation of the high-density RPA approximation by Gell-Mann 
and Brueckner \cite{gellmann1957} has been extended to the 
two-dimensional electron gas by Rajagopal and Kimball 
\cite{rajagopal1977}. 

As far as we know, the only CC calculations that have been
done for the two-dimensional electron gas are the ring
\cite{freeman1978} and particle-particle ladder \cite{freeman1983}
approximations of Freeman. Beacuse of stronger correlations
in the purely two-dimensional system, the ring approximation
was not as reliable in two dimensions as it was in the 
three-dimensional case \cite{freeman1978}. According to the 
studies of Freeman, the CC ring-diagram approximation is 
less accurate than the CC ladder approximation at 
intermediate and low densities, whereas both methods give 
similar results at the high-density limit \cite{freeman1983}.
In Sec.~\ref{sec:ccheg}, we present results for the 
two-dimensional finite-size electron gas with all 
correlations of the CCD approximation included. 

As we mentioned above, Ceperley studied both two- and 
three-dimensional systems in the VMC calculations of
Ref.~\cite{ceperley1978}. More accurate calculations
were done by Tanatar and Ceperley, using DMC approximated
by the fixed-node technique \cite{tanatar_ceperley_1989}.
After extrapolation to the thermodynamic limit, the DMC 
correlation energies were close to, but variationally
lower than, CC ladder results of Freeman \cite{freeman1983}, 
dielectric calculations of Jonson \cite{jonson1976},
and FHNC energies of Sim \emph{et al.} \cite{sim1986}
for scaled average electron-electron distances 
$r_{s}$ between one and ten. The CC ring-diagram results 
of Freeman \cite{freeman1978} were found to differ 
significantly from the DMC energy estimates of 
Tanatar and Ceperley \cite{tanatar_ceperley_1989}. 

\begin{itemize}
\item More recent Monte Carlo calculations for the 2DEG
\item Other approaches
\end{itemize}


%% In contrast to the three-dimensional 
%% electron gas, in the two-dimensional electron gas
%% the correlation energy does not diverge at second order
%% in perturbation theory \cite{freeman1978,freeman1983}.
 


\chapter{Microscopic models of fermionic matter}

\section{A quantum many-particle system}

\begin{itemize}
\item Nucleons appropriate degree of freedom (citations etc.) \\
  $\Rightarrow $ many-fermion system of interacting nucleons
\item Many-body forces: Why truncation
\item Why nonrelativistic theory
\end{itemize}

\subsection{The Hamiltonian equation}
In this thesis, we define infinite nuclear matter as a system in which the $d$-dimensional real space $\mathbb{R}^{d}$ is filled by homogeneous nuclear matter. We assume that nucleons are an appropriate degree of freedom in the density domains we are going to study. As is well known, nucleons have spin $1/2$ and behave statistically as fermions. Infinite nuclear matter is therefore, as we define it, a system containing an infinite number of interacting fermions. On the many-particle level, we neglect relativistic effects. We can therefore use nonrelativistic many-fermion theory \cite{fetter,harris,bartlett_book} for describing the system. 

The physics of a general, nonrelativistic, and time-independent quantum mechanical system is described by the Hamiltonian eigenvalue equation 
\begin{equation} \label{eq:ham_eq}
  \hat{H}| \Psi \rangle = E| \Psi \rangle ,
\end{equation}
where $\hat{H}$ is the Hamiltonian operator, $| \Psi \rangle$ is the quantum state vector, and the eigenvalue $E$ is the energy. In infinite nuclear matter, the Hamiltonian operator can be written as
\begin{equation}
  \hat{H} = \hat{T} + \hat{V},
\end{equation}
where $\hat{T}$ is the kinetic energy operator and $\hat{V}$ is the interaction operator. Nucleons have an underlying quark-gluon structure, and are not point particles (citation). Interactions between three or more nucleons are therefore generally nonzero, and should be included in microscopic calculations on nuclear matter (check literature for references). The total nuclear interaction operator is of the form
\begin{align}
  \hat{V} = \hat{V}_{NN} + \hat{V}_{NNN} + \hat{V}_{NNNN} + \dots 
\end{align}
where $\hat{V}_{NN}$, $\hat{V}_{NNN}$, and $\hat{V}_{NNNN}$ are the two-, three-, and four-body interaction operators, respectively. (Something about evidence that higher-order terms become increasingly smaller.) In our calculations, we include only the two-body interaction (why?). 

In coordinate space, the Hamiltonian operator for infinite nuclear matter has the form
\begin{align} \label{eq:hamilt}
  \hat{H} =& -\frac{\hbar^{2}}{2m}\sum_{i=1}^{A}\nabla_{i}^{2} + \sum_{i<j}^{A}\hat{v}_{NN}(\mathbf{r}_{i}, \mathbf{r}_{j}) \nonumber \\
  & + \sum_{i<j<k}^{A}\hat{v}_{NNN}(\mathbf{r}_{i}, \mathbf{r}_{j}, \mathbf{r}_{k}) + \dots ,
\end{align}
where $A$ is the total number of nucleons, $m$ is the nucleon mass, $\hbar $ is the Planck constant, and $\mathbf{r}_{i}$ is the coordinate of nucleon $i$. The projection of the state vector $| \Psi \rangle $ to the position space depends on $A$ position vectors, i.e.
\begin{equation}
  \langle \mathbf{r}_{1}\dots \mathbf{r}_{A}| \Psi \rangle = \Psi(\mathbf{r}_{1}, \dots , \mathbf{r}_{A}),
\end{equation}
where $\mathbf{r}_{i}$ is particle $i$'s position vector.

In the many-body methods we are going to consider -- the hole-line expansion and the coupled-cluster method -- the $A$-particle wave function $\Psi(\mathbf{r}_{1}, \dots , \mathbf{r}_{A})$ is expanded in a basis, i.e.,
\begin{equation} \label{eq:phi_expansion}
  \Psi(\mathbf{r}_{1}, \mathbf{r}_{2}, \dots , \mathbf{r}_{A}) = \sum_{m}c_{m}\Phi_{m}(\mathbf{r}_{1}, \mathbf{r}_{2}, \dots , \mathbf{r}_{A}), 
\end{equation}
where the basis functions $\Phi_{m}$ are Slater determinants
\begin{align}
  \Phi_{m}(\mathbf{r}_{1}, \mathbf{r}_{2}, \dots , \mathbf{r}_{A}) = \frac{1}{\sqrt{A!}}\left| \begin{array}{cccc}
    \phi_{\alpha_{1}^{(m)} }(\mathbf{r}_{1}) & \phi_{\alpha_{2}^{(m)} }(\mathbf{r}_{1}) & \ldots & \phi_{\alpha_{A}^{(m)} }(\mathbf{r}_{1}) \\
    \phi_{\alpha_{1}^{(m)} }(\mathbf{r}_{2}) & \phi_{\alpha_{2}^{(m)} }(\mathbf{r}_{2}) & \ldots & \phi_{\alpha_{A}^{(m)}}(\mathbf{r}_{2}) \\
    \vdots & \vdots & \ddots & \vdots \\
    \phi_{\alpha_{1}^{(m)} }(\mathbf{r}_{A}) & \phi_{\alpha_{2}^{(m)} }(\mathbf{r}_{A}) & \ldots & \phi_{\alpha_{A}^{(m)} }(\mathbf{r}_{A}) \\
  \end{array} \right| 
\end{align}
constructed from a chosen single-particle basis $\{\phi_{\alpha }(\mathbf{r})\}_{\alpha }$. The structure of the Slater determinants ensures that the total wave function is antisymmtric, which is a requirement for a fermion system. 
  

\subsection{Second quantization}

When working with many-particle quantum systems, it is convenient to utilize the power of the second quantization formalism \cite{fetter,bartlett_book,schwabl_adv}. In second quantization, the quantum states belong to the Fock space \cite{schwabl_adv}, which is the direct product of the Hilbert spaces \cite{kreyszig} with zero, one, two, and up to arbitrarily many particles. The states are given in occupation representation, in which $ |n_{1}n_{2}\dots \rangle $ is a state with $n_{1}$ particles in the single-particle state 1, $n_{2}$ particles in single-particle state 2 and so on. The single-particle states are elements of a chosen basis, and because the system is fermionic, the occupation numbers can only take the values 0 or 1.  

Let us define the fermion creation and annihilation operators $a_{\alpha }^{\dagger }$ and $a_{\alpha }$, respectively, such that $a_{\alpha }^{\dagger }$ creates a fermion in the state $| \alpha \rangle $ and $a_{\alpha }$ annihilates a fermion in the same state. When operating on single-particle states, the operators have the properites
\begin{equation}
  a_{\alpha }^{\dagger }|0\rangle = |\alpha \rangle , \qquad a_{\alpha }^{\dagger }|\alpha \rangle = 0  
\end{equation}
and
\begin{equation}
  a_{\alpha }|0\rangle = 0, \qquad a_{\alpha }|\alpha \rangle = |0\rangle .
\end{equation}
Above, $|0\rangle $ is the physical vacuum state. The fermion creation and annihilation operators obey the anticommutator relations
\begin{align}
  \{ a_{\alpha }^{\dagger }, a_{\beta }^{\dagger }\} &= 0, \qquad \{ a_{\alpha }, a_{\beta }\} = 0, \nonumber \\
  \{ a_{\alpha }, a_{\beta }^{\dagger } \} &= \delta_{\alpha , \beta }, 
\end{align}
where the curly brackets denote anticommutation operators and $\delta_{\alpha , \beta }$ is the Kronecker delta function. The anticommutator relations are necessary for getting antisymmetric many-particle states, as required for Fermion systems. 

In second quantization, a vector with the single-particle states $\alpha_{1}, \alpha_{2}, \dots $, $\alpha_{A}$ occupied is written in terms of creation operators as
\begin{align}
  | \alpha_{1}\alpha_{2}\dots \alpha_{A}\rangle = a_{\alpha_{1}}^{\dagger }\dots a_{\alpha_{A}}^{\dagger }| 0\rangle ,
\end{align}
where $| 0\rangle $ is the vacuum state. Because the occupation number for fermions is always zero or one, one can label a vector by only those single-particle
states that are occupied. A general one-body operator is written as
\begin{align}
  \hat{U} = \sum_{p,q}\langle p|\hat{u}|q\rangle a_{p}^{\dagger }a_{q}
\end{align}
and a two-body operator as
\begin{align}
  \hat{W} = \frac{1}{4}\sum_{p,q,r,s}\langle pq|\hat{w}|rs\rangle_{AS}a_{p}^{\dagger }a_{q}^{\dagger}a_{s}a_{r},
  \label{eq:W_sec_quant}
\end{align}
where the brackets are inner products and the summations are taken over all single-particle states. In Eq. (\ref{eq:W_sec_quant}) we have used the definition
\begin{align}
  \langle pq|\hat{w}|rs\rangle_{AS} \equiv \langle pq|\hat{w}|rs\rangle - \langle pq|\hat{w}|sr\rangle
\end{align}
 to denote an antisymmetrized interaction matrix element.

Let the reference state $|\Phi_{0}\rangle $ be constructed of $A$ single-particle states chosen from a given basis. Here and in the following, we denote states occupied in the reference state by $i, j, k, \dots $, states not occupied in $|\Phi_{0}\rangle $ by $a, b, c, \dots $, whereas indices $p, q, r, \dots $ are used for arbitrary single-particle states in the given basis. Using creation operators, the reference state is
\begin{align}
  |\Phi_{0}\rangle = a_{i_{1}}^{\dagger }a_{i_{2}}^{\dagger }\dots a_{i_{A}}^{\dagger }|0\rangle ,
\end{align}
where $|0\rangle $ is the physical vacuum state. Let us further define states of the form
\begin{align}
  |\Phi_{ij\dots }^{ab\dots }\rangle &= a_{a}^{\dagger }a_{b}^{\dagger }\dots a_{j}a_{i}|\Phi_{0}\rangle 
\end{align}
as particle-hole excitations of the reference state. In a space spanned by the given single-particle basis, the total state vector $|\Psi \rangle $ may be written as
\begin{align}
  |\tilde{\Psi }\rangle = |\Phi_{0}\rangle + \sum_{ia}c_{i}^{a}|\Phi_{i}^{a}\rangle + \sum_{ijab} c_{ij}^{ab}|\Phi_{ij}^{ab}\rangle + \dots  . 
\end{align}
This expansion can be truncated after a finite number of terms, and the coefficients are typically found using a numerical method.



  %% |\Phi_{i}^{a}\rangle &= a_{a}^{\dagger }a_{i}|\Phi_{0}\rangle , \nonumber \\
  %% |\Phi_{ij}^{ab}\rangle &= a_{a}^{\dagger }a_{b}^{\dagger }a_{j}a_{i}|\Phi_{0}\rangle , \nonumber \\
  %% & \, \, \, \vdots \nonumber \\

%% The brackets are inner products, expressed in coordinate space as
%% \begin{align}
%%   \langle p|\hat{u}|q\rangle = \int d\mathbf{r}\phi_{p}^{*}(\mathbf{r})\hat{u}(\mathbf{r})\phi_{q}(\mathbf{r})
%% \end{align}
%% and 
%% \begin{equation}
%%   \langle pq|\hat{w}|rs\rangle = \int d\mathbf{r}\int d\mathbf{r}'\phi_{p}^{*}(\mathbf{r})\phi_{q}^{*}(\mathbf{r}')\hat{w}(\mathbf{r}, \mathbf{r}')\phi_{r}(\mathbf{r})\phi_{s}(\mathbf{r}'),
%% \end{equation}
%% where the star denotes the complex conjugate.

%% Assuming that each single-particle state $|\alpha \rangle $ is associated with a single-particle energy $\varepsilon_{\alpha } $, the Fermi vacuum state $| \Phi_{0}\rangle $ is by definition constructed from the $A$ single-particle states with lowest single-particle energies. Furthermore, we define the excitations
%% \begin{align}
%%   |\Phi_{ij\dots }^{ab\dots } \rangle &\equiv a_{a}^{\dagger }a_{b}^{\dagger }\dots a_{j}a_{i}|\Phi_{0}\rangle, 
%% \end{align}
%% where $i, j, \dots $ and $a, b, \dots $ denote states which are occupied and unoccupied, respectively, in the Fermi vacuum state $| \Phi_{0}\rangle $. The hole-line expansion \cite{day1978} and the coupled-cluster theory \cite{crawford} provide tools to obtain approximations for the expansion coefficients $c_{m}$ in Eq. \ref{eq:phi_expansion}. In those many-body methods, the $A$-particle basis consists of the Fermi vacuum state $| \Phi_{0}\rangle $ and corresponding excited states $|\Phi_{ij\dots }^{ab\dots } \rangle$. 

%\begin{itemize}
%\item Particle-hole formalism
%\end{itemize}

\subsection{Momentum single-particle basis}\label{sec:box}

Later, we study infinite-matter systems in two and three dimensions. Similarly as explained by, for example, Fetter and Walecka \cite{fetter} in three dimensions, we use $d$-dimensional hypercubic potential wells for modelling infinite-nuclear-matter systems. Fig.~\ref{fig:boxes} illustrates hypercubes in one, two, and three dimensions. The external potential is assumed to be zero inside the hypercube, and infinitely large outside the box. In coordinate representation, the Hamiltonian equation of a single-nucleon system is  
\begin{equation}
  -\frac{\hbar^{2}}{2m}\nabla^{2}\phi(\mathbf{x}) = \varepsilon\phi(\mathbf{x})
\end{equation}
inside the hypercube. Here $\phi $ and $\varepsilon $ are the single-particle wave function and energy, respectively, $\mathbf{x}$ is a $d$-dimensional position vector $\mathbf{x} \equiv (x_{1}, \dots, x_{d})$, $m$ is the nucleon mass, and $\hbar $ is the Planck constant. The single-particle Hamiltonian equation has solutions of the form
\begin{equation} \label{eq:planeWave}
\phi_{\mathbf{k}}(\mathbf{x}) = \frac{1}{L^{d/2}}e^{i\mathbf{k}\cdot \mathbf{x}},
\end{equation}
where $|\mathbf{k}| = \sqrt{2m\varepsilon /\hbar^{2}}$. Continuing the approach outlined by Fetter and Walecka, we use periodic boundary conditions 
\begin{align}
  \phi_{\mathbf{k}}(x_{1}+L, \dots , x_{d}) &= \phi_{\mathbf{k}}(x_{1}, \dots , x_{d}), \nonumber \\
  &\ \, \vdots \nonumber \\
  \phi_{\mathbf{k}}(x_{1}, \dots , x_{d}+L) &= \phi_{\mathbf{k}}(x_{1}, \dots , x_{d}+L),
\end{align} 
which give the conditions
\begin{align}
  k_{i} = \frac{2\pi }{L}n_{i}, \qquad n_{i} = 0, \pm 1, \pm 2, \dots
  \label{eq:mom_sp_discr}
\end{align}
for the component $i$ of the wave vector $\mathbf{k}$. A general single-particle state represents a set of quantum numbers $(n_{1}, \dots , n_{d})$.\cite{fetter, harris, liboff}  

\begin{figure} 
  \centering
  \includegraphics[scale=1.0]{figures/boxes/boxes-crop.pdf}
  \caption{Infinite-nuclear-matter systems are modelled using hypercubes in one, two and three dimensions.}
  \label{fig:boxes}
\end{figure}

Single-particle states in fermionic matter have a spin projection $m_{s}$, and for nuclear matter also an isospin projection $m_{t}$. In nuclear matter, we write a single-particle state vector as 
\begin{align} \label{eq:sp_mom_cart}
  |\mathbf{k}_{n} m_{s}m_{t}\rangle = |n_{1}\dots n_{d}m_{s}m_{t}\rangle , 
\end{align}
where the quantum numbers $n_{i}$ are as defined in Eq.~(\ref{eq:mom_sp_discr}). We choose to label neutrons by $m_{t} = +\frac{1}{2}$ and protons by $m_{t} = -\frac{1}{2}$. In the electron gas there is no isospin dependency, but otherwise 
the single-particle states are equal to those in nuclear matter.   

\begin{figure} 
  \centering
  \includegraphics[scale=1.0]{figures/fourier_2d/fourier_2d-crop}
  \caption{Fourier transform in two dimensions. The finite, countinuous, and rectangular region in coordinate space is mapped to a set of infinitely many discrete points in momentum space (Fourier space). The finite size of the coordinate space domain gives a finite distance between points in momentum space. The number of Fourier grid points inside the Fermi sea, denoted by a circle, is by definition the same as the number of particles in the physical box.}
  \label{fig:fourier_2d}
\end{figure}

We construct infinite-matter systems by filling hypercubes with 
interacting nucleons, as shown for two dimensions at left in 
Fig.~\ref{fig:fourier_2d}. Inifinite matter is obtained in the limit 
when the box length $L = L_{x} = L_{y}$ and the number of particles $A$ 
approach infinity, whereas the particle density $\rho \equiv A/L^{2}$ is 
kept constant. Expressions in momentum basis may be considered as 
Fourier transforms of corresponding coordinate-space equations. 
Fig.~\ref{fig:fourier_2d} shows how a finite number of particles
in the continuous coordinate space is transformed to an infinite number
of discrete points in the Fourier space. The spacing between points
in Fourier space is inversely related to the hypercube side.
In Fourier space, the physical particles are represented by all momentum
points inside the Fermi sphere.

\begin{figure} 
  \centering
  \includegraphics[scale=1.2]{figures/fermi_levels/fermi_levels-crop.pdf}
  \caption{The Fermi momentum for neutrons, $k_{F_{n}}$, is defined as the momentum of the highest-lying occupied neutron state. Similarly, the Fermi momentum for protons, $k_{F_{p}}$ is defined as the momentum of the highest-lying occupied proton state. In other words, the proton and neutron Fermi levels, $k_{F_{n}}$ and $k_{F_{p}}$, determine together the ratio of the two nucleon types in the considered system.}
  \label{fig:fermi_levels}
\end{figure}

Let us neglect spin and isospin degrees of freedom and write the reference state as
\begin{align}
  |\Phi_{0}\rangle = a_{\mathbf{k}_{1}}^{\dagger }a_{\mathbf{k}_{2}}^{\dagger }\dots a_{\mathbf{k}_{A}}^{\dagger }|0\rangle , 
\end{align}
where each particle $i \in \{ 1, 2, \dots , A\}$ has a unique momentum vector $\mathbf{k}_{i}$. Assume that the reference state is occupied by the $A$ single-particle states with the lowest single-particle energies 
\begin{align}
  \varepsilon_{\mathbf{k}} = \frac{\hbar^{2}k^{2}}{2m},
\end{align}
where $k \equiv |\mathbf{k}|$ is the length of the momentum vector. Then the Fermi momentum $k_{F}$ is defined as the momentum $|\mathbf{k}|$ of the highest-lying occupied state. As illustrated in Fig.~\ref{fig:fermi_levels}, in nuclear matter the Fermi momentum is defined separately for protons and neutrons. In a calculation, the proton-neutron ratio can be controlled by adjusting the neutron and proton Fermi momenta $k_{F_{n}}$ and $k_{F_{p}}$. Symmetric nuclear matter is obtained by choosing the same Fermi level for both protons and neutrons, whereas in pure neutron matter the proton Fermi momentum $k_{F_{p}}$ is set to zero. 

Assuming that the $d$-dimensional hypercube with side length $L$ is filled by $A$ fermions, the particle density becomes
\begin{align}
  \rho \equiv \frac{A}{L^{d}} = \frac{1}{L^{d}}\sum_{m_{s}m_{t}}\sum_{|\mathbf{k}|\leq k_{F}(m_{t})},
  \label{eq:density_discr}
\end{align}
where the Fermi momentum may be dependent on the isospin. The summation is taken over all occupied single-particle states, restricted by Eq.~(\ref{eq:mom_sp_discr}) and a Fermi momentum.  





\subsection{Other single-particle bases} \label{eq:other_sp_b}

In the expressions arising in many-body perturbation theory and 
coupled-cluster theory, there are summations over single-particle 
momenta $\mathbf{k}_{p}$. At the limit when the box side of the 
hypercube $L$ approaches infinity, a sum over discrete Fourier 
states $\mathbf{k}_{p}$ can be approximated by an integral according to
\begin{equation}
  \sum_{\mathbf{k}_{p}} \longrightarrow 
  \left( \frac{L}{2\pi }\right)^{d}\int d\mathbf{k}_{p},
   \label{eq:sum2int}
\end{equation}
where the integration is over the $d$-dimensional space spanned by the 
Fourier states. In the energy expressions, the summations are often taken 
over only particle or hole states. At the thermodynamic limit, a sum over 
two different hole states, $i$ and $j$, is replaced with two integrals, 
as in Eq.~(\ref{eq:sum2int}). The restriction to only occupied states is 
ensured by a hole-hole Pauli exclusion operator
\begin{align}
  Q_{hh}(|\mathbf{k}_{i}|, |\mathbf{k}_{j}|, k_{F}) \equiv \theta(k_{F}-|\mathbf{k}_{i}|)\theta(k_{F}-|\mathbf{k}_{j}|),
\end{align}
where $\theta(x)$ is the Heaviside step function. Similarly, a sum over
two particle states, $a$ and $b$, is transformed to two $d$-dimensional 
integrals with a particle-particle Pauli exlusion operator
\begin{align}
  Q_{pp}(|\mathbf{k}_{a}|, |\mathbf{k}_{b}|, k_{F})\equiv \theta(|\mathbf{k}_{a}|-k_{F})\theta(|\mathbf{k}_{b}|-k_{F}),
\end{align}
which ensures that the integration is taken over only unoccupied
states. As before, $k_{F}$ denotes the Fermi momentum.


%% In this limit, the particle denisty is given by
%% \begin{align}
%%   \rho = \frac{1}{(2\pi )^{2}}\sum_{m_{s}m_{t}}\int_{|\mathbf{k}| \leq k_{F}(m_{t})} d\mathbf{k},
%% \end{align}
%% where 
The coordinates of two particles, $\mathbf{x}_{p}$ and 
$\mathbf{x}_{q}$, may be expressed in terms of a relative 
vector, $\mathbf{x}$, and a centre-of-mass (CM) vector, 
$\mathbf{X}$, such that
\begin{align}
  \mathbf{x} = \mathbf{x}_{p}-\mathbf{x}_{q}, 
  \qquad \mathbf{X} = (\mathbf{x}_{p}+\mathbf{x}_{q})/2.
  \label{eq:xrcm}
\end{align}
Assuming that the Planck constant $\hbar $ is one, the momentum
$\mathbf{p} \equiv \hbar \mathbf{k}$ is equal to the wave
vector $\mathbf{k}$. Provided that the two particles have 
equal mass $m$, the reduced mass is $m_{r} = \frac{1}{2}m$
and the total mass $M = 2m$. Using the definitions in 
Eq.~(\ref{eq:xrcm}), the relative and CM momentum vectors,
$\mathbf{k}$ and $\mathbf{K}$, become
\begin{align}
  \mathbf{k} &= (\mathbf{k}_{p}-\mathbf{k}_{q})/2, 
\qquad \mathbf{K} = \mathbf{k}_{p} + \mathbf{k}_{q}, 
  \label{eq:lab2rcm}
\end{align}
where $\mathbf{k}_{p}$ and $\mathbf{k}_{q}$ are the momenta
of particles $p$ and $q$ in laboratory coordinates.  

%% \begin{align}
%%   \mathbf{k} &= (\mathbf{k}_{i}-\mathbf{k}_{j})/2, \qquad \mathbf{K} = \mathbf{k}_{i} + \mathbf{k}_{j}, \nonumber \\
%%   %\mathbf{k}' &= (\mathbf{k}_{a}-\mathbf{k}_{a})/2, \qquad \mathbf{K}' = \mathbf{k}_{a} + \mathbf{k}_{b},
%%   \label{eq:lab2rcm}
%% \end{align}

In relative and centre-of-mass (RCM) coordinates, 
the hole-hole Pauli exclusion operator becomes
\begin{align}
  Q_{hh}(\mathbf{k}, \mathbf{K}, k_{F})=\theta(k_{F}-|\mathbf{k}+\mathbf{K}/2|)
  \theta(k_{F}-|-\mathbf{k}+\mathbf{K}/2|)
  \label{eq:paulihh_rcm}
\end{align} 
and the particle-particle Pauli exclusion operator
\begin{align}
  Q_{pp}(\mathbf{k}, \mathbf{K}, k_{F})=\theta(|\mathbf{k}+\mathbf{K}/2|-k_{F})
  \theta(|-\mathbf{k}+\mathbf{K}/2|-k_{F}).
  \label{eq:paulipp_rcm}
\end{align}
Observe that in RCM coordinates, the Pauli operators are dependent on
the radial coordinates of both the relative and the CM momentum vectors,
as well as on the angle between these two vectors.

Methods derived from MBPT are often implemented in a partial-wave basis with relative momenta \cite{mackenzie, haftel_tabakin}. The basis commonly used is of the form
\begin{align}
  |k\mathcal{J}m_{\mathcal{J}}(lS)m_{t_{1}}m_{t_{2}}\rangle,
  \label{eq:coupled_basis1}
\end{align}
where $k \equiv |\mathbf{k}|$ is the length of a relative momentum vector $\mathbf{k}$, $l$ is the relative orbital angular momentum related to $\mathbf{k}$, $S$ is the total two-particle spin, $\mathcal{J}$ is equal to the total relative angular momentum $l+S$, $m_{\mathcal{J}}$ is the $z$ projection of $\mathcal{J}$, and $m_{t_{1}}$ and $m_{t_{2}}$ label the nucleon types of particles 1 and 2, respectively. If charge independence breaking and charge symmetry breaking are negelcted, $m_{t_{1}}$ and $m_{t_{2}}$ can be replaced with a coupled isospin $T$ with the projection $M_{T}$. Charge symmetry and charge independence are defined in for example Refs.~\cite{heyde, machleidt2011}.

% Since the two-particle interactions we use take into account charge independence breaking \cite{machleidt2011}, we will here write the equations with separate particle labels $m_{t_{1}}$ and $m_{t_{2}}$. The interactions depend only on the sum $M_{T} = m_{t_{1}} + m_{t_{2}}$, and we will therefore sometimes only write $M_{T}$ explicitly, even if we do not assume that pairs of single-particle isospins are coupled.   

 When doing transformations from an RCM basis $|\mathbf{k}\mathbf{K}m_{s_{1}}m_{t_{1}}m_{s_{2}}m_{t_{2}}\rangle $ to the basis of Eq.~(\ref{eq:coupled_basis1}), we need the completeness relations
\begin{align}
  \sum_{lm_{l}}|lm_{l}\rangle \langle lm_{l}| &= \hat{\mathbb{1}}, \label{eq:lm_complete}\\
  \sum_{SM_{S}}|SM_{S}\rangle \langle SM_{S}| &= \hat{\mathbb{1}}, \label{eq:sm_complete}
\end{align}
where $m_{l}$ and $M_{S}$ are $z$ projections of $l$ and $S$, respectively, and $\hat{\mathbb{1}}$ is the unity operator. Furthermore, we need the equalities
\begin{align}
  \langle \mathbf{\hat{k}}|lm_{l}\rangle &\equiv Y_{lm_{l}}(\mathbf{\hat{k}}), \label{eq:Ylm_bracket}\\
  \int d\mathbf{\hat{k}}Y_{lm_{l}}^{*}(\mathbf{\hat{k}})Y_{l'm_{l'}}(\mathbf{\hat{k}}) &= \delta_{ll'}\delta_{m_{l}m_{l'}}, \label{eq:y_ortho} \\
  Y_{lm_{l}}(\pi - \theta_{\mathbf{k}}, \phi_{\mathbf{k}} + \pi ) &= (-1)^{l}Y_{lm_{l}}(\theta_{\mathbf{k}}, \phi_{\mathbf{k}}), \label{eq:y_minus}
\end{align}
where $\mathbf{\hat{k}} = (\theta_{\mathbf{k}}, \phi_{\mathbf{k}})$ is the angular vector of the relative momentum $\mathbf{k}$, $Y_{lm_{l}}(\mathbf{\hat{k}})$ is the spherical harmonics function, and $\delta_{pq}$ is the Kronecker delta function. The angular momenta are coupled according to
\begin{align}
  |lm_{l}SM_{S}\rangle = \sum_{\mathcal{J}m_{\mathcal{J}}}\langle lm_{l}SM_{S}|\mathcal{J}m_{\mathcal{J}}lS\rangle |\mathcal{J}m_{\mathcal{J}}lS\rangle , \label{eq:ls_jm}
\end{align}
where the bracket denotes the Clebsch-Gordan coefficient. We also need the property 
\begin{align}
  \langle j_{p}m_{p}j_{q}m_{q}|JM_{J}\rangle = (-1)^{j_{p}+j_{q}-J}\langle j_{q}m_{q}j_{p}m_{p}|JM_{J}\rangle 
  \label{eq:clebsch_qp}
\end{align}
of Clebsch-Gordan coefficients.
The relations (\ref{eq:y_ortho}) -- (\ref{eq:clebsch_qp}) and other useful angular momentum relations are given in for example the text of Varshalovich \emph{et al.}~\cite{varshalovich}.

Sometimes we denote coupling of angular momenta using
square brackets, for example
\begin{align} \label{eq:coupling_squarebr}
  \left[ Y_{l_{p}m_{l_{p}}}(\mathbf{\hat{k}}_{p}) 
    Y_{l_{q}m_{l_{q}}}(\mathbf{\hat{k}}_{q}) \right]_{\lambda m_{\lambda } } \equiv 
  \sum_{m_{l_{p}}m_{l_{q}}} \langle l_{p}m_{l_{p}}l_{q}m_{l_{q}}|\lambda m_{\lambda} \rangle
  Y_{l_{p}m_{l_{p}}}(\mathbf{\hat{k}}_{p})Y_{l_{q}m_{l_{q}}}(\mathbf{\hat{k}}_{q}),
\end{align}
where a Clebsch-Gordan coefficient is used to couple two 
different orbital angular momenta.


\subsection{Interaction matrix elements}

%% A two-particle operator $\hat{V}$ can be written in second quantization as 
%% \begin{equation}
%%   \hat{V} = \frac{1}{4}\sum_{pqrs} \langle pq|\hat{v}|rs\rangle_{AS} a_{p}^{\dagger }a_{q}^{\dagger }a_{s}a_{r}, 
%% \end{equation}
%% where the antisymmetrized interaction is by definition
%% \begin{equation}
%%   \langle pq|\hat{v}|rs\rangle_{AS} = \langle pq|\hat{v}|rs\rangle - \langle pq|\hat{v}|sr\rangle .
%% \end{equation}

Next we derive a transformation between relative and centre-of-mass (RCM) and laboratory coordinates for the two-body interaction, generalizing the approach of Fetter and Walecka \cite{fetter} to $d$ dimensions. Interaction matrix elements are defined as
\begin{equation}
  \langle pq|\hat{v}|rs\rangle = \int d\mathbf{x}_{1} \int d\mathbf{x}_{2}\phi_{p}^{*}(\mathbf{x}_{1})\phi_{q}^{*}(\mathbf{x}_{2})v(\mathbf{x}_{1}, \mathbf{x}_{2})\phi_{r}(\mathbf{x}_{1})\phi_{s}(\mathbf{x}_{2}),
\end{equation}
where $\phi_{m}(\mathbf{x})$ is the coordinate space projection of the single-particle state $| m\rangle $ and a star denotes the complex conjugate. If we take the single-particle states to be eigenfunctions of a finite hypercube with side length $L$, i.e. plane waves as in Eq.~\ref{eq:planeWave}, the matrix elements become 
\begin{align}
  \langle \mathbf{k}_{p}\mathbf{k}_{q}|\hat{v}|\mathbf{k}_{r}\mathbf{k}_{s}\rangle &= \int d\mathbf{x}_{1}\int d\mathbf{x}_{2}\phi_{\mathbf{k}_{p}}^{*}(\mathbf{x}_{1})\phi_{\mathbf{k}_{q}}^{*}(\mathbf{x}_{2})v(\mathbf{x}_{1}, \mathbf{x}_{2})\phi_{\mathbf{k}_{r}}(\mathbf{x}_{1})\phi_{\mathbf{k}_{s}}(\mathbf{x}_{2}) \nonumber \\
  &= \frac{1}{L^{2d}}\int d\mathbf{x}_{1}\int d\mathbf{x}_{2} e^{-i\mathbf{k}_{p}\cdot \mathbf{x}_{1}}e^{-i\mathbf{k}_{q}\cdot \mathbf{x}_{2}}v(\mathbf{x}_{1}, \mathbf{x}_{2})e^{i\mathbf{k}_{r}\cdot \mathbf{x}_{1}}e^{i\mathbf{k}_{s}\cdot \mathbf{x}_{2}}. 
\end{align}
If we use the definitions (\ref{eq:lab2rcm}), and do the change of integration variables 
\begin{equation}
  \mathbf{r} = \mathbf{x}_{1}-\mathbf{x}_{2}, \qquad \mathbf{R} = (\mathbf{x}_{1}+\mathbf{x}_{2})/2,
\end{equation}
where $\mathbf{r}$ and $\mathbf{R}$ are relative and centre-of-mass coordinates, respectively, we get the matrix element into the form
 \begin{align}
   \langle \mathbf{k}_{p}\mathbf{k}_{q}|\hat{v}|\mathbf{k}_{r}\mathbf{k}_{s}\rangle &=\underbrace{\left[ \frac{1}{L^{d}}\int d\mathbf{r} e^{-i\left( \mathbf{k}-\mathbf{k}'\right) \cdot \mathbf{r}} v(\mathbf{r})\right]}_{\equiv \langle \mathbf{k}|\hat{v}|\mathbf{k}'\rangle} \delta_{\mathbf{K},\mathbf{K}'}  
 \end{align}
Here we have assumed that the interaction is invariant under translations, which means that the interaction depends only on the distance $\mathbf{r} = \mathbf{x}_{1}-\mathbf{x}_{2}$, and not on the specific positions of the two particles. We have also used the definition
\begin{equation}
\delta_{\mathbf{K}, \mathbf{K}'} = \frac{1}{L^{d}}\int_{\mathbb{R}^{d}} d\mathbf{R}e^{i\left(\mathbf{K}-\mathbf{K}'\right)\cdot \mathbf{R}}
\end{equation}
of the $d$-dimensional Kronecker delta function, where the integration is over the entire $d$-dimensional real space.~\cite{fetter}


%% \subsection{Many-body methods}
%% Owing to the many degrees of freedom, in most realistic cases it is impossible to solve the many-fermion Hamiltonian equation (\ref{eq:ham_eq}) numerically exactly, not to mention analytically. This is true also for electron systems, in which the Hamiltonian operator has a known analytical form. It is therefore necessary to use numerical approximations. If possible, it is desireable to have a numerical scheme that converges to the exact solution systematically in terms of one or a few convergence parameters. We here call such numerical approaches \emph{ab initio} methods \cite{barrett,wloch}. Note that these schemes do not necessarily converge towards the exact solution of a nuclear many-body problem, as the exact form of the nuclear interaction is unknown. Examples of \emph{ab initio} many-body methods that have been applied to nuclear systems are the no core shell model \cite{barrett}, the coupled-cluster method \cite{coester1958,coester1960,dean2004}, the self-consistent Green's function method \cite{dickhoff2004}, and different quantum Monte Carlo techniques \cite{carlson}. Numerical schemes derived from many-body perturbation theory \cite{brueckner,day1967,raja} may also be considered as \emph{ab initio} methods, but sometimes the convergence parameter must be infinite to get a reasonable result \cite{brueckner,raja}. 


\section{The nuclear interaction} \label{sec:nuclear_int}

More than hundred years after the atomic nucleus was 
discovered by Rutherford \cite{rutherford1911}, an exact 
theoretical description of the nuclear force is still lacking.
Ideally, a model for the interaction between baryons, such as
nucleons and hyperons, should be derived directly from QCD. 
However, at the energies relevant for nuclear physics, QCD is 
strongly nonperturbative \cite{machleidt2011}. Nonperturbative 
QCD problems can be solved using lattice QCD, and, for example, 
Ishii \emph{et al.}~have derived simple nuclear potentials 
\cite{ishii2007,nemura2009} 
using such stochastic calculations. Lattice-QCD calculations 
involving baryons are computationally  
expensive, and due to limitations in computing time, the 
calculations are done with unphysically large quark masses 
\cite{ishii2007,inoue2013}. Since realistic nuclear 
interaction models cannot be derived directly from QCD, we  
need other effective approaches to obtain more accurate 
potentials. 

As can be seen from Eq.~(\ref{eq:hamilt}), 
nuclear Hamiltonians for many-particle systems contain 
two-body, three-body, and many-body interactions. Normally, 
the two-nucleon interaction dominates, and the magnitude of 
many-body interactions decrease rapidly with increasing order. 
For example, the contribution from the three-body interaction 
is considerably smaller than that from the two-body 
part (approx. 10 percent) (\textbf{Reference}). Let us therefore start with the nucleon-nucleon 
force, which is the most important part of the full nuclear 
interaction.


\subsection{Background}

%A basic introduction to the nuclear force is given in,
%for example, Refs.~\cite{wong_book,ring_schuck}. 
Interactions between nucleons can be studied in, for 
example, elastic scattering experiments. The nuclear 
interaction is known to have a range of only a few
fermi. In the region between $1$ fm and $2$ fm it is 
attractive, and at distances comparable to the nucleon size, 
the force becomes strongly repulsive (see Figure 
\ref{fig:potential}). On a very general level, the 
nucleon-nucleon interaction obeys certain symmetries, 
such as conservation of centre of mass and total momentum,
rotational and time-reversal symmetry, and conservation
of parity. The nucleon-nucleon force 
has a rich operator structure, the most important 
terms being a central, a spin-orbit, and a tensor-force
part. The reader is referred to \cite{wong_book,
ring_schuck} for a more thorough introduction to basic 
properties of the nuclear force. Let us instead turn our  
attention to the development of nuclear interaction
models.    

Machleidt gives a historical review \cite{machleidt1989} 
from the first decades of nuclear-interaction models, 
starting with the discovery of the atomic nucleus in 1911 
and continuing until the year 1989. Here we very briefly 
sketch this development following Ref.~\cite{machleidt1989}. 
At an early stage, in 1935, Yukawa proposed \cite{yukawa1935} 
a theoretical model in which mesons are mediator particles 
for the nuclear 
force. It soon became an established fact that the long-range  
part of the nuclear interaction is accurately described 
by one-pion exchange. As Machleidt outlines, it was much 
less straightforward to obtain successful models for the 
intermediate- and short-range parts. Various more or less
phenomenological approaches were investigated, including
approaches with two- and multi-pion exchange as well as 
one-boson exchange models. In the latter models, 
two- and multi-pion exchange terms are replaced by 
heavy single-boson mediator particles, such as the 
$\omega $, $\rho $, and $\sigma $ mesons. While the
$\omega $ and $\rho $ particles were detected early, 
still in 2013 a sufficiently light $\sigma $ meson
had not been found \cite{abbas2013} in experiments. 
Beside the efforts in quantum field theory, 
dispersion theory was also used to derive nuclear 
interactions \cite{machleidt1989}. 

\begin{figure} 
  \centering
  \includegraphics[scale=1.0]{figures/potential/potential-crop.pdf}
  \caption{The nucleon-nucleon interaction
    consists of a strongly repulsive short-range part
    ($r \leq 1 $ fm), an attractive intermediate-range
    part ($1$ fm $< r < 2$ fm), and a rapidly vanishing 
    long-range part ($r \geq 2$ fm). Theoretically, each
    of these regions corresponds to different types of 
    contributions. For example, the long-range part is
    known to be dominated by one-pion exchange.}
  \label{fig:potential}
\end{figure}

In the early 1990s, a group at University of Nijmegen
collected a large amount of data from nucleon-nucleon (NN)
scattering experiments \cite{stoks1993}. The Nijmegen data 
basis, and in some cases also more recent data, were used
to parametrize several very accurate NN
potentials. These potentials, named Nijm I, Nijm II, Reid93 
\cite{stoks1994}, Argonne $v_{18}$ 
\cite{wiringa1995}, and CD-Bonn \cite{machleidt2001b},
are fitted to the experimental data with a $\chi^{2}$
per datum close to one. Machleidt and Slaus compare
\cite{machleidt2001} these interaction models with each
other. In all of these modern potentials, the long-range
part is described by one-pion exchange, whereas the
intermediate- and short-range parts are treated 
differently. The CD-Bonn potential has a nonlocal 
one-pion-exchange contribution, whereas the other 
interactions use local approximations for this part 
\cite{machleidt2001}. In Ref.~\cite{machleidt2001b} 
it is shown that the nonlocality gives triton and 
$\alpha $-particle binding energies closer to experiments. 
The short- and intermediate-range parts of the
CD-Bonn, Nijm I, and Nijm II potentials are based
on the one-boson-exchange model, whereas in the Reid93 
and Argonne $v_{18}$ potentials this part is treated
phenomenologically. These modern NN potentials are
all charge dependent, but the approximations 
are different \cite{machleidt2001}.

Recently, different variations of the Urbana 
three-nucleon force \cite{carlson1983,pudliner1995,
pudliner1997} have been used in nuclear-matter 
studies \cite{soma2008,gandolfi_2009}. In the Urbana 
three-body interaction, an attractive two-pion exchange 
part is combined with a repulsive phenomenological term. 
According to Epelbaum \emph{et al.}~\cite{epelbaum_2009b}, 
the Urbana three-nucleon force has the disadvantage that 
it is not consistent with any two-body forces.
Approximately 25 years ago, Grang{\'e} 
\emph{et al.}~constructed a three-body interaction 
\cite{grange1989} 
that was consistent with the Paris two-body potential.
During the last decade, nuclear interactions derived 
from chiral perturbation theory \cite{epelbaum2006,
epelbaum_2009b,machleidt2011} have become increasingly 
popular. Chiral perturbation theory gives, in a natural 
way, two- and many-nucleon forces as terms 
of the same perturbational expansion. In our studies, 
we have used mostly nuclear interaction models derived 
from chiral perturbation theory, and we therefore devote 
the next subsection to chiral interactions.


%contains a two-pion exchange 
%part and a phenomenological term.    




%% Using renormalization theory, one can show that the form
%% of a nuclear potential is not unique \cite{}. Instead, 
%% as ... \emph{et al.} stress, infinitely many potentials 
%% can be physically equally correct. 
%% \item There is no single correct low-momentum potential 
%%   (Bogner \emph{et al.})

\subsection{Chiral perturbation theory} \label{sec:chiral}

Machleidt and Entem \cite{machleidt2011} and Epelbaum 
\cite{epelbaum2010} have written pedagogical introductions to
chiral perturbation theory for nuclear forces. In the current
subsection, we follow these texts when not referring 
explicitly to other works. 
Chiral effective field theory has also been reviewed in, 
for example, Refs.~\cite{epelbaum2006,epelbaum_2009b}.

When dealing with energy scales typical in atomic nuclei,
nucleons and pions are more relevant degrees of freedom 
than quarks and gluons. In chiral perturbation theory,
as described in Refs.~\cite{epelbaum2010,machleidt2011}, 
nucleons and pions are considered as particles in an 
effective field theory. If the quark masses were zero,
right- and left-handed quark fields would be independent
in QCD. In nuclear systems, the quark masses are relatively 
small but nonzero. This results in coupling of right-
and left-handed fields, and consequently the chiral 
symmetry is broken. When setting up the Lagrangian in
chiral perturbation theory, the approximative chiral
symmetry is utilized in a crucial way. 

%% In this theory,
%% the pions may be considered as Nambu-Goldstone bosons 
%% associated with the symmetry breaking. 

To limit the number of terms, the diagrams are ordered
using a perturbative smallness parameter $Q/\Lambda_{\chi }$,
where $Q$ is a momentum and $\Lambda_{\chi }$ is a 
fixed number, which is chosen to be approximately $1$ GeV. 
Chiral perturbation theory results in so-called contact 
terms for the short-range part of the interaction, as well
as one-pion, two-pion, and mulit-pion exchange terms. 
According to Machleidt and Entem \cite{machleidt2011},
the contact terms are used for renormalization purposes
as well as to model heavy mesons, which are not included
explicitly in the theory.
When using an order parameter as defined in 
Refs.~\cite{epelbaum2010,machleidt2011}, the lowest-order
(LO) contribution contains contact terms and one-pion
exchange. Two-pion exchange occurs at next-to-lowest 
order (NLO), three-body interactions are introduced at
next-to-next-to-lowest order (NNLO), and four-body
interactions are present for the first time at 
next-to-next-to-next-to-lowest order (N$^{3}$LO).

To avoid singularities in the Lippmann-Schwinger equation,
chiral interactions are regularized. The regulator may be
defined in different ways, and is typically dependent on
a cutoff parameter that has to be chosen appropriately. 
Ideally, the nuclear interaction should be independent 
on the specific form and chosen cutoff of the regulator. 
However, as we report in Paper III, for example, the results 
of a many-body calculation may be strongly dependent on 
regulator properties.  

In some of our nuclear-matter calculations, 
we have used the chiral nucleon-nucleon interaction of 
Entem and Machleidt \cite{machleidt2003}, which contains 
all two-body contributions to N$^{3}$LO. Epelbaum,
Gl{\"o}ckle, and Mei{\ss}ner have later developed another
chiral two-nucleon interaction \cite{epelbaum2005}, which, 
among other, uses different regularization schemes than
the interaction of Entem and Machleidt. 
In Paper I, we present an NNLO chiral interaction that is
parametrized with respect to phase shifts using a new, 
efficient optimization tool. We have used this  
two-body interaction, called NNLO$_{\text{opt}}$, in 
Papers I-III, and we therefore say a few words about the
optimization procedure in the next subsection.   


\subsection{Optimized interaction model}

Nuclear interactions derived from chiral perturbation theory
are dependent on a number of so-called low-energy constants.
These parameters can be fitted to reproduce phase shifts
that have been determined using experimental data. When using 
the optimization algorithm called Practical Optimization Using 
No Derivatives (for Squares) (POUNDERS) 
\cite{kortelainen2010,munson2012}, Ekstr{\"o}m 
\emph{et al.}~obtained \cite{ekstrom2013} better agreement 
with experimentally extracted 
phase shifts compared to previous parametrizations (see Paper I).
As is explained in Paper I, the parameters of the 
NNLO$_{\text{opt}}$ two-body interaction were obtained 
by comparing phase shifts calculated with the new interaction
with the experimentally extracted phase shifts of the Nijmegen
group \cite{stoks1993}. The POUNDERS optimization algorithm
is described in Refs.~\cite{kortelainen2010,munson2012}.
This algorithm is designed to do least-squares optimizations
without calculating derivatives. According to Kortelainen
\emph{et al.}~\cite{kortelainen2010}, the POUNDERS algorithm
is a quasi-Newton method in which the derivatives in the 
second-order Taylor expansion are replaced by parametrized
functions. The approximative second-order Taylor polynomial 
is parametrized so that it gives exact values 
at certain known data points. The POUNDERS algorithm is a
so-called trust-region method \cite{cohn2009}. Roughly, this 
means that at each step, the optimization parameter space is 
restricted to trust regions defined around the previous 
values of the optimization parameters 
\cite{kortelainen2010,munson2012}.  
According to Ref.~\cite{cohn2009}, trust-regions are used 
among other to get a faster optimization method. 


\subsection{The Minnesota potential}

In Paper III, we compare different coupled-cluster 
approximations with the auxiliary-field diffusion Monte 
Carlo method using the simple phenomenological Minnesota 
potential \cite{thompson1977}. This potential has the form
\begin{align} \label{eq:minnesota}
  v(r) &= \left( v_{R} + \left( 1 
  + P_{12}^{\sigma }\right)v_{T}/2 + \left( 
  1 - P_{12}^{\sigma }\right) v_{S}/2\right) \nonumber \\
  & \times \left( \alpha 
  + \left( 2-\alpha \right) P_{12}^{r}\right) /2 
  + \left( 1 + m_{t,1}\right) \left( 1 
  + m_{t,2}\right) \frac{e^{2}}{4r},
\end{align}
where $r = |\mathbf{r}_{1}-\mathbf{r}_{2}|$ is the distance
between particles 1 and 2, $m_{t,1}$ and $m_{t,2}$ are 
isospin projections ($\pm 1$) of the two particles, and 
$P_{12}^{\sigma }$ and $P_{12}^{r}$ are exchange operators 
for spin and position, respectively. In 
Eq.~(\ref{eq:minnesota}), we have used the definitions
\begin{align}
  v_{R} = v_{0R}e^{-k_{R}r^{2}}, \qquad v_{T} = 
  - v_{0T}e^{-k_{T}r^{2}}, \qquad v_{S} = -v_{0S}e^{-k_{S}r^{2}},
\end{align} 
where the constants $v_{0R}$, $v_{0T}$, $v_{0S}$, $k_{R}$,
$k_{T}$, and $k_{S}$ are as given in Ref.~\cite{thompson1977}.
In our calculations, we have chosen the parameter $\alpha $ 
to be one. Then the potential for pure neutron matter 
can be written as
\begin{align}
  v(r) = \left\{ \begin{array}{ll}
    v_{0R}e^{-k_{R}r^{2}} - v_{0S}e^{-k_{S}r^{2}}, & \text{ if } S = 0, \\
    v_{0R}e^{-k_{R}r^{2}} - v_{0T}e^{-k_{T}r^{2}}, & \text{ if } S = 1,
  \end{array} \right.
\end{align}
where $S$ is the total two-particle spin. In this special 
case, the Minnesota potential is a function of only the total 
spin and the interparticle distance.


%% \subsection{Three-nucleon forces in applications}

%% \begin{itemize}
%% \item Recent applications of three-body forces
%% \end{itemize}


\chapter{The Brueckner-Hartree-Fock approximation for nuclear matter} \label{ch:mbpt}

Rayleigh--Schr{\"o}dinger perturbation theory (RSPT) \cite{bartlett_book} 
is one of the standard many-body methods for nonrelativistic quantum
systems. RSPT has the advantage that it is size-extensive 
\cite{bartlett_book}, which roughly means that the theory scales 
correctly with the number of particles in the system. Unfortunately, 
the strong short-range interaction between nucleons makes nuclear 
systems nonperturbative, and generally it is necessary to renormalize 
the interaction before doing perturbative many-body calculations. 
Brueckner-Goldstone theory and the hole-line approximation 
\cite{day1967,day1978} are approaches related to RSPT in which the 
interaction has been replaced with a so-called $G$ matrix. The $G$ matrix 
is obtained by summing 
to infinite order in perturbation theory certain correlation 
contributions that are important for short-range correlations.

Being the lowest-order approximation of both Brueckner-Goldstone theory
and the hole-line approximation \cite{day1967,day1978}, the 
Brueckner-Hartree-Fock (BHF) 
approximation \cite{brueckner1954,brueckner_levinson,brueckner,
brueckner_gammel,day1967,bethe1971,day1978,jackson1983,baldo1990,
song,li2006,baldo2012} is one of the standard many-body methods 
for infinite-nuclear-matter studies.  
In Ref.~\cite{baardsen} we have derived a coupled-cluster ladder
approximation in a partial-wave basis using exact Pauli exclusion
operators. As we show in the paper, the coupled-cluster ladder 
approximation is closely related to the BHF method. In the work with
Ref.~\cite{baardsen}, we have used the BHF method to verify important 
parts of our coupled-cluster ladder implementations. 
We have also compared the coupled-cluster results with BHF calculations. 
In this chapter, we discuss the basic principles of RSPT and 
the BHF approximation. Furthermore, we show how we have implemented
the BHF method and the lowest orders of perturbation theory for 
nuclear-matter systems. Apart from minor details related to the 
practical implementations, most of the material presented in this 
chapter is based on well-known theory.
 

\section{Many-body perturbation theory}

We follow Shavitt and Bartlett \cite{bartlett_book} when introducing many-body perturbation theory (MBPT). For the interested reader, the textbook of Harris, Monkhorst, and Freeman \cite{harris} also gives a thorough introduction to MBPT. If there is no external potential, the physics of a time-independent nuclear system is determined by a Hamiltonian operator 
\begin{equation}
  \hat{H} = \hat{T} + \hat{V},
\end{equation}
where $\hat{T}$ is the kinetic energy operator and $\hat{V}$ is the interaction operator. The Hamiltonian operator $\hat{H}$ can be regrouped into a noninteracting part $\hat{H}_{0}$ and a perturbation operator $\hat{H}_{I}$, i.e.
\begin{equation}
  \hat{H} = \hat{H}_{0} + \hat{H}_{I},
\end{equation}
where
\begin{align}
  \hat{H}_{0} &= \hat{T} + \hat{U}, \\
  \hat{H}_{I} &= \hat{V} - \hat{U}.
\end{align}
Here $\hat{U}$ is an arbitrary single-particle potential operator that preferably is chosen such that the contribution of the perturbation $\hat{H}_{I}$ is as small as possible. 

Let the state $| \Psi \rangle $ be the eigenvector of the total Hamiltonian $\hat{H}$ corresponding to the lowest eigenvalue, i.e.
\begin{equation}
  \hat{H}| \Psi \rangle = E| \Psi \rangle, 
\end{equation}
and let $| \Phi_{0}\rangle $ be the eigenvector of the noninteracting part $\hat{H}_{0}$ corresponding to the lowest eigenvalue, i.e.
\begin{equation}
  \hat{H}_{0}|\Phi_{0}\rangle = E_{0}|\Phi_{0}\rangle . 
\end{equation}
Here $E$ and $E_{0}$ are the ground state total and noninteracting energy, respectively. In the following, we assume that the ground state is nondegenerate. We define further the operators $\hat{P}$ and $\hat{Q}$ such that
\begin{align}
  \hat{P}| \Psi \rangle = |\Phi_{0}\rangle 
\end{align}
and $\hat{P} + \hat{Q} = \hat{I}$, where $\hat{I}$ is the unity operator.

 As shown by Shavitt and Bartlett \cite{bartlett_book}, the total energy can be written as 
\begin{equation}
  E = E_{0} + \Delta E,
\end{equation}
where the energy perturbation
\begin{equation}
  \Delta E = \sum_{n=1}^{\infty }\Delta E^{(n)}
\end{equation}
is expanded in powers of the perturbation $\hat{H}_{I}$. The energy contribution to the $n$:th power of $\hat{H}_{I}$ is
\begin{equation} \label{eq:pert_contr_n}
  \Delta E^{(n)} = \langle \Phi_{0}|\hat{H}_{I}\left( \hat{R}\left( E-\mu - \hat{H}_{I}\right)\right)^{n-1}|\Phi_{0} \rangle ,
\end{equation}
where $\hat{R}$ is the resolvent operator, defined as
\begin{equation} \label{eq:resolvent}
  \hat{R} \equiv \left( \hat{Q}\left( \hat{H}_{0}-\mu \right) \hat{Q}\right)^{-1},
\end{equation}
and $\mu $ is an arbitrary constant. The Rayeligh-Schr{\"o}dinger perturbation theory (RSPT) is defined such that $\mu = E_{0}$. It can be shown that RSPT is a size-extensive method, which roughly means that the energy scales correctly with the number of particles. Shavitt and Bartlett \cite{bartlett_book} explain the concept of size-extensivity, and show that RSPT is size-extensive for a special system with equal, noninteracting atoms. In this thesis, we consider only perturbative methods derived from RSPT.

% = \sum_{i\neq 0}\frac{|\Phi_{i}\rangle \langle \Phi_{i}|}{E-E_{i}}
%% The last expression of Eq. \ref{eq:resolvent} is valid if the states $|\Phi_{i}\rangle $ are distinct eigenfunctions of the operator $H_{0}$. The energy $E_{i}$ is the eigenvalue of $\hat{H}_{0}$ corresponding to the vector $|\Phi_{i}\rangle $. Assuming that the eigenfunctions of $\hat{H}_{0}$ span a complete basis, it holds that
%% \begin{equation}
%%   \hat{Q}|\Psi \rangle = \sum_{i=1}^{\infty }|\Phi_{i}\rangle .
%% \end{equation}

Expressed in second quantization, the single-particle operators can be written as 
\begin{align}
  \hat{T} &= \sum_{p,q}\langle p|\hat{t}|q\rangle a_{p}^{\dagger }a_{q}, \\
  \hat{U} &= \sum_{p,q}\langle p|\hat{u}|q\rangle a_{p}^{\dagger }a_{q},
\end{align}
and the two-particle operator becomes
\begin{align}
  \hat{V} &= \sum_{p,q,r,s}\langle pq|v|rs\rangle_{AS}a_{p}^{\dagger }a_{q}^{\dagger }a_{s}a_{r}.
\end{align}
Let us define the Fock operator as 
\begin{equation}
  \hat{F} = \sum_{p,q}\langle p|\hat{f}|q\rangle a_{p}^{\dagger }a_{q},
\end{equation}
where
\begin{equation} \label{eq:fock_def}
  \langle p|\hat{f}|q\rangle = \langle p|\hat{t}|q\rangle + \langle p|\hat{u}|q\rangle .
\end{equation}
In the laboratory coordinate momentum basis, the matrix elements of the kinetic energy operator are diagonal.

Observe that the explicit expressions of $\Delta E^{(n)}$ in Eq.~(\ref{eq:pert_contr_n}) depend on which operator $\hat{U}$ is used. Let us first choose the single-particle potential
\begin{equation} 
  \langle p|\hat{u}|q\rangle = \sum_{i}\langle pi|\hat{v}|qi\rangle_{AS},
  \label{eq:u_hf}
\end{equation}
where $i$ denotes a state that is occupied in the Fermi vacuum. Provided the Fock operator contains the single-particle potential (\ref{eq:u_hf}), the Hartree-Fock equations \cite{harris} are
\begin{align} 
  \langle p|\hat{f}|q\rangle &= \varepsilon_{q}\langle p|q\rangle , \nonumber \\
  \langle p|q\rangle &= \delta_{pq},
  \label{eq:hf_eq}
\end{align}
where the single-particle states are restricted to be orthonormal. A Hartree-Fock single-particle basis is a basis that fulfills Eq.~(\ref{eq:hf_eq}). If the interaction $\hat{v}$ conserves the total momentum, the Fock matrix becomes diagonal in the plane wave basis. Consequently, the Hartree-Fock equations (\ref{eq:hf_eq}) are fulfilled in this basis, and the momentum basis is a Hartree-Fock single-particle basis for infinite nuclear matter.

 From now on, we use a notation with explicit momentum states. For brevity, we do not write out the spin and isospin degrees of freedom explicitly. The state vectors and sums  
\begin{equation}
  |\mathbf{k}\rangle \quad \text{ and } \quad \sum_{\mathbf{k}}
\end{equation}
should therefore be read as  
\begin{equation}
  |\mathbf{k}\rangle |m_{s}m_{t}\rangle \quad \text{ and } \quad \sum_{m_{s}m_{t}}\sum_{\mathbf{k}}, 
\end{equation}
 where $m_{s}$ and $m_{t}$ are the $z$ projections of the single-particle spin and isospin, respectively. Let us define the reference energy as
\begin{align}
  E_{REF} \equiv \langle \Phi_{0}|\hat{H}|\Phi_{0}\rangle = E_{0} + \Delta E^{(1)}.
\end{align}
As shown by Shavitt and Bartlett \cite{bartlett_book}, the explicit expression for the reference energy is 
\begin{equation} 
  E_{REF} = \sum_{\mathbf{k}_{i}}\langle \mathbf{k}_{i}|\hat{t}|\mathbf{k}_{i}\rangle + \frac{1}{2}\sum_{\mathbf{k}_{i},\mathbf{k}_{j}}\langle \mathbf{k}_{i}\mathbf{k}_{j}|\hat{v}|\mathbf{k}_{i}\mathbf{k}_{j}\rangle_{AS}. 
  \label{eq:eneRef}
\end{equation}
By definition, the Hartree-Fock energy is the reference 
energy of a system calculated in a Hartree-Fock basis 
\cite{harris}. For infinite nuclear matter, the 
plane-wave basis is a Hartree-Fock basis and the reference 
energy (\ref{eq:eneRef}) is therefore also the Hartree-Fock 
energy. As can be seen from Eq.~(\ref{eq:hf_eq}), the Fock 
matrix defined by Eqs.~(\ref{eq:fock_def}) and 
(\ref{eq:u_hf}) is diagonal in a Hartree-Fock basis. 
Consequently, one-particle-one-hole diagrams vanish when
using the plane-wave basis in infinite matter. 
With the chosen operator $\hat{U}$, the second-order 
contribution is
\begin{align} 
  \Delta E^{(2)} = \frac{1}{4}\sum_{\mathbf{k}_{i},\mathbf{k}_{j}}\sum_{\mathbf{k}_{a},\mathbf{k}_{b}}\frac{\langle \mathbf{k}_{i}\mathbf{k}_{j}|\hat{v}|\mathbf{k}_{a}\mathbf{k}_{b}\rangle_{AS} \langle \mathbf{k}_{a}\mathbf{k}_{b}|\hat{v}|\mathbf{k}_{i}\mathbf{k}_{j}\rangle_{AS}}{\varepsilon_{\mathbf{k}_{i}}+\varepsilon_{\mathbf{k}_{j}}-\varepsilon_{\mathbf{k}_{a}}-\varepsilon_{\mathbf{k}_{b}}},
  \label{eq:enePt2}
\end{align}
where we have used the definition 
\begin{align}
  \varepsilon_{\mathbf{k}} \equiv \langle \mathbf{k}|\hat{f}|\mathbf{k}\rangle . 
  \label{eq:sp_pot_def}
\end{align}
In this special case, the contribution to third order is 
\begin{align}
  \Delta E^{(3)} = \Delta E_{pp}^{(3)} + \Delta E_{hh}^{(3)} + \Delta E_{ph}^{(3)}.
\end{align}
Here the third-order energy consists of the so-called particle-particle term 
\begin{align}
  \Delta E_{pp}^{(3)} &= \frac{1}{8} \sum_{\mathbf{k}_{a},\mathbf{k}_{b}}\sum_{\mathbf{k}_{c},\mathbf{k}_{d}}\sum_{\mathbf{k}_{i},\mathbf{k}_{j}} \frac{\langle \mathbf{k}_{i}\mathbf{k}_{j}|\hat{v}|\mathbf{k}_{a}\mathbf{k}_{b}\rangle_{AS}\langle \mathbf{k}_{a}\mathbf{k}_{b}|\hat{v}|\mathbf{k}_{c}\mathbf{k}_{d}\rangle_{AS}}{\left( \varepsilon_{\mathbf{k}_{i}}-\varepsilon_{\mathbf{k}_{j}}-\varepsilon_{\mathbf{k}_{a}}-\varepsilon_{\mathbf{k}_{b}} \right)} \nonumber \\
  & \times \frac{\langle \mathbf{k}_{c}\mathbf{k}_{d}|\hat{v}|\mathbf{k}_{i}\mathbf{k}_{j}\rangle_{AS} }{\left( \varepsilon_{\mathbf{k}_{i}}-\varepsilon_{\mathbf{k}_{j}} +\varepsilon_{\mathbf{k}_{c}}-\varepsilon_{\mathbf{k}_{d}}\right)},
  \label{eq:mbpt3_pp}
\end{align}
 the hole-hole term
\begin{align}
  \Delta E_{hh}^{(3)} &= \frac{1}{8}\sum_{\mathbf{k}_{a},\mathbf{k}_{b}}\sum_{\mathbf{k}_{i},\mathbf{k}_{j}}\sum_{\mathbf{k}_{k},\mathbf{k}_{l}}\frac{\langle \mathbf{k}_{i}\mathbf{k}_{j}|\hat{v}|\mathbf{k}_{a}\mathbf{k}_{b}\rangle_{AS}\langle \mathbf{k}_{a}\mathbf{k}_{b}|\hat{v}|\mathbf{k}_{k}\mathbf{k}_{l}\rangle_{AS}}{\left( \varepsilon_{\mathbf{k}_{i}}+\varepsilon_{\mathbf{k}_{j}}-\varepsilon_{\mathbf{k}_{a}}-\varepsilon_{\mathbf{k}_{b}}\right)} \nonumber \\
  & \times \frac{\langle \mathbf{k}_{k}\mathbf{k}_{l}|\hat{v}|\mathbf{k}_{i}\mathbf{k}_{j}\rangle_{AS} }{\left( \varepsilon_{\mathbf{k}_{k}}+\varepsilon_{\mathbf{k}_{l}}-\varepsilon_{\mathbf{k}_{a}}-\varepsilon_{\mathbf{k}_{b}}\right) },
\end{align}
and the particle-hole term
\begin{align}
  \Delta E_{ph}^{(3)} &= - \sum_{\mathbf{k}_{a},\mathbf{k}_{b}}\sum_{\mathbf{k}_{c},\mathbf{k}_{i}}\sum_{\mathbf{k}_{j},\mathbf{k}_{k}} \frac{\langle \mathbf{k}_{i}\mathbf{k}_{j}|\hat{v}|\mathbf{k}_{a}\mathbf{k}_{b}\rangle_{AS} \langle \mathbf{k}_{k}\mathbf{k}_{b}|\hat{v}|\mathbf{k}_{i}\mathbf{k}_{c}\rangle_{AS}}{\left( \varepsilon_{\mathbf{k}_{i}}+\varepsilon_{\mathbf{k}_{j}}-\varepsilon_{\mathbf{k}_{a}}-\varepsilon_{\mathbf{k}_{b}}\right)} \nonumber \\
  & \times \frac{\langle \mathbf{k}_{a}\mathbf{k}_{c}|\hat{v}|\mathbf{k}_{i}\mathbf{k}_{j}\rangle_{AS}}{\left( \varepsilon_{\mathbf{k}_{k}}+\varepsilon_{\mathbf{k}_{j}}-\varepsilon_{\mathbf{k}_{a}}-\varepsilon_{\mathbf{k}_{c}}\right)}.
\end{align}
The reference and second-order energy terms of Eqs. (\ref{eq:eneRef}) and (\ref{eq:enePt2}) are given diagrammatically in Fig.~\ref{fig:diag_ene_pt2}. We use the diagrammatic rules defined in Ref. \cite{bartlett_book}. A vertical line with an arrow pointing up (down) represents the summation over a particle (hole) state, whereas a ring without an arrow represents a summation over a hole state. The kinetic energy operator is denoted by a horisontal dashed line with a cross in one of the ends, and the antisymmetrized two-body operator is represented by a horisontal dashed line that is connected to four vertical lines. Between every pair of interaction lines, there is an energy denominator. For an introduction to the diagrammatic rules, the reader is referred to Refs.~\cite{harris,bartlett_book,crawford}. The third-order diagrams for this special case are given in Fig.~\ref{fig:diag_ene_pt3}. Shavitt and Bartlett \cite{bartlett_book} show all MBPT diagrams up to fourth order for the general case.
 
\begin{figure}
  \centering
  \includegraphics[scale=1.1]{diagrams/ene_pt2/ene_term-crop.pdf}
  \caption{Diagrammatic representation of the energy to second order in perturbation theory, as obtained when assuming a Hartree-Fock single-particle potential $\hat{U}$.}  
  \label{fig:diag_ene_pt2}
\end{figure}

%  & + \sum_{\mathbf{k}_{i}}\sum_{\mathbf{k}_{a}}\frac{\langle \mathbf{k}_{i}|\hat{f}|\mathbf{k}_{a}\rangle \langle \mathbf{k}_{a}|\hat{f}|\mathbf{k}_{i}\rangle }{\varepsilon_{\mathbf{k}_{i}}-\varepsilon_{\mathbf{k}_{a}}}

\begin{figure}
  \centering
  \includegraphics[scale=1.1]{diagrams/ene_pt3/ene_term-crop.pdf}
  \caption{MBPT diagrams for the third-order energy correction, when the operator $\hat{U}$ is as in Eq. \ref{eq:u_hf}. For a more general operator $\hat{U}$, there are additional third-order diagrams containing the Fock operator.}  
  \label{fig:diag_ene_pt3}
\end{figure}

A diagram is said to be \emph{closed} if both ends of all particle/hole 
lines are connected to an operator, and the diagram is \emph{connected} 
if it cannot be divided into two topologically unconnected parts without 
breaking lines. The linked-diagram theorem theorem \cite{goldstone1957} 
states that the RSPT energy correction $\Delta E$ is the sum of all 
closed and connected diagrams which give different correlation 
contributions. As explained in for example Refs.~\cite{harris,bartlett_book}, 
inclusion of only connected diagrams ensures that the energy is 
\emph{size-consistent}. Size-consistency is defined as the property 
that the total energy of two subsystems, calculated independently, 
must be equal to the energy of a system where both parts are present, 
but so far from each other that they do not interact \cite{bartlett_book}. 
It follows therefore from the linked-diagram theorem that RSPT 
is a size-consistent method. Size-consistency becomes increasingly important
for larger systems, and is a critical property when studying 
infinite matter. As Harris \emph{et al.}~show in their textbook 
\cite{harris}, the configuration interaction doubles approximation, 
which is not size-consistent, gives zero correlation energy per particle 
for the electron gas. 
 



\section{Brueckner-Hartree-Fock approximation}

The perturbation series contains diagrams similar to the particle-particle diagram $\Delta E_{pp}^{(3)}$, given in Fig.~\ref{fig:diag_ene_pt3}, with one, two, three, and up to infinitely many interaction matrix elements. That class of terms are commonly called particle-particle ladder diagrams, and a few of them are shown in Fig.~\ref{fig:ppladders}. In a similar way, the RSPT contains hole-hole ladder diagrams to infinite order. The hole-hole ladder diagrams are obtained from the particle-particle ladders by exchanging all particle and hole lines with each other. The complete set of particle-particle ladder diagrams can be constructed using the so-called $G$ matrix \cite{brueckner_levinson}, which is defined by the implicit equation
 \begin{align}
   \langle \mathbf{k}_{p}\mathbf{k}_{q}|\hat{g}|\mathbf{k}_{r}\mathbf{k}_{s}\rangle &= \langle \mathbf{k}_{p}\mathbf{k}_{q}|\hat{v}|\mathbf{k}_{r}\mathbf{k}_{s}\rangle \nonumber \\
     &+ \sum_{\mathbf{k}_{c}\mathbf{k}_{d}}\frac{\langle \mathbf{k}_{p}\mathbf{k}_{q}|\hat{v}|\mathbf{k}_{c}\mathbf{k}_{d}\rangle \langle \mathbf{k}_{c}\mathbf{k}_{d}|\hat{g}|\mathbf{k}_{r}\mathbf{k}_{s}\rangle }{\varepsilon_{\mathbf{k}_{p}}+\varepsilon_{\mathbf{k}_{q}}-\varepsilon_{\mathbf{k}_{c}}-\varepsilon_{\mathbf{k}_{d}}},
   \label{eq:gmat1}
 \end{align}
where the single-particle energies are defined as in Eq.~(\ref{eq:sp_pot_def}) and the summation is over only particle states. Observe that the $G$ matrix is on both sides of the equation and the single-particle eneriges are dependent on the auxiliary potential $\hat{U}$. In terms of antisymmetric matrix elements, the same equation can be written as
\begin{align} 
  &\langle \mathbf{k}_{p}\mathbf{k}_{q}|\hat{g}|\mathbf{k}_{r}\mathbf{k}_{s}\rangle_{AS} = \langle \mathbf{k}_{p}\mathbf{k}_{q}|\hat{v}|\mathbf{k}_{r}\mathbf{k}_{s}\rangle_{AS} \nonumber \\
  &+ \frac{1}{2}\sum_{\mathbf{k}_{c}\mathbf{k}_{d}}\frac{\langle \mathbf{k}_{p}\mathbf{k}_{q}|\hat{v}|\mathbf{k}_{c}\mathbf{k}_{d}\rangle_{AS}\langle \mathbf{k}_{c}\mathbf{k}_{d}|\hat{g}|\mathbf{k}_{r}\mathbf{k}_{s}\rangle_{AS}}{\varepsilon_{\mathbf{k}_{p}}+\varepsilon_{\mathbf{k}_{q}}-\varepsilon_{\mathbf{k}_{c}}-\varepsilon_{\mathbf{k}_{d}}},
  \label{eq:gmatrix_as}
\end{align}
where the sum is now multiplied by a factor of one half. The $G$-matrix equation is shown diagrammatically in Fig.~\ref{fig:gmatrix}.


\begin{figure}
  \centering
  \includegraphics[scale=1.1]{diagrams/ppladders/ene_term-crop.pdf}
  \caption{Particle-particle ladder diagrams up to fourth order, as they occur in the MBPT energy expansion.}  
  \label{fig:ppladders}
\end{figure}

\begin{figure}
  \centering
  \includegraphics[scale=1.1]{diagrams/gmatrix/gmatrix-crop.pdf}
  \caption{Diagrammatic representation of the $G$-matrix equation, which is used in Brueckner theory \cite{brueckner_gammel,goldstone1957} (Figure 4 in Paper II). The wave-shaped line represents the $G$ matrix, which is an effective interaction obtained by resummation of short-range correlation contributions. Reprinted by permission from \dots  The figure is also used in Ref.~\cite{baardsen}.}  
  \label{fig:gmatrix}
\end{figure}

The strong short-range forces of the nuclear interaction induce correlations 
that cannot be accounted for in a mean-field theory. By replacing the bare 
interaction with a $G$ matrix, important parts of the correlations are included 
already at the lowest order of perturbation theory. The Brueckner-Goldstone 
(BG) expansion \cite{day1967} is obtained by replacing all interaction 
matrix elements in the RSPT with $G$ matrices. In BG theory, the diagonal 
matrix elements of the auxiliary potential $\hat{U}$ are commonly chosen to be 
\begin{align}
  \langle \mathbf{k}_{p}|\hat{u}|\mathbf{k}_{p}\rangle = \sum_{\mathbf{k}_{i}}\langle \mathbf{k}_{p}\mathbf{k}_{i}|\hat{g}|\mathbf{k}_{p}\mathbf{k}_{i}\rangle_{AS} 
  \label{eq:u_gmat}
\end{align}
for single-particle states that are occupied in the uncorrelated Fermi vacuum. 
In the first-order approximation, which corresponds to the reference energy 
in RSPT, the energy becomes
\begin{equation}
  E_{BHF} = \sum_{\mathbf{k}_{i}}\langle \mathbf{k}_{i}|\hat{t}|\mathbf{k}_{i}\rangle + \frac{1}{2}\sum_{\mathbf{k}_{i},\mathbf{k}_{j}}\langle \mathbf{k}_{i}\mathbf{k}_{j}|\hat{g}|\mathbf{k}_{i}\mathbf{k}_{j}\rangle_{AS}.
  \label{eq:ene_bhf}
\end{equation}
This approximation is commonly called the Brueckner-Hartree-Fock (BHF) 
method \cite{hjensen1995}, because the interaction in the 
Hartree-Fock energy has been replaced with a self-consistently solved 
Brueckner $G$ matrix. Observe that the 
matrix elements of the auxiliary potential is not yet defined for 
single-particle states above the Fermi level. There have been different
approaches to handling the single-particle potential for particle
states, such as the conventional gap option \cite{hjensen1995}, in which
the single-particle potential is set to zero above the Fermi level,
and the continuous option \cite{mahaux1985,mahaux1989}, in which the 
potential in Eq.~(\ref{eq:u_gmat}) is used for both particle and 
hole single-particle states. In our calculations, we use the latter 
approach. Day has written a pedagogical introduction \cite{day1967} 
to Brueckner-Goldstone theory and its application to infinite nuclear matter.

According to Rajaraman and Bethe \cite{raja}, the Brueckner-Goldstone 
expansion does not converge when using the $G$ matrix as a convergence 
parameter. Instead, they argue that the number of independent hole lines 
is a more appropriate cutoff parameter in the perturbation series. The 
first-order energy in the Brueckner-Goldstone theory, the BHF approximation,
 is equal to the Brueckner-Bethe approximation including up to two hole 
lines \cite{day1978}.

\begin{itemize}
\item Give a short heuristic argumentation for the hole-line expansion
\item The hole-line approximation is a good assumption only at \\
  sufficiently low densities
\end{itemize}

%% \begin{displaymath}
%%   \langle \mathbf{k}_{p}|\hat{u}|\mathbf{k}_{q}\rangle = \left\{ \begin{array}{ll} 
%%     \sum_{\mathbf{k}_{i}}\langle \mathbf{k}_{p}\mathbf{k}_{i}|\hat{g}|\mathbf{k}_{q}\mathbf{k}_{i}\rangle_{AS}, & \text{if } p=q \text{ and } |\mathbf{k}_{p}| \leq k_{F}, \\
%%     0, & \text{else.}
%%     \end{array} \right.
%% \end{displaymath}



%In Brueckner theory \cite{brueckner,brueckner_gammel,goldstone1957},  
%The energy in Eq.~(\ref{eq:ene_bhf}) can be considered as the Hartree-Fock energy, where the interaction matrix has been replaced by the G-matrix. The lowest-order energy in the Brueckner-Goldstone expansion is therefore commonly named the Brueckner-Hartree-Fock approximation \cite{li2012}. 

\section{Transformation to partial-wave expansion}
The nuclear interaction is often given explicitly in the partial-wave basis
(\ref{eq:coupled_basis1}). Next, we rewrite some of the previously defined 
expressions derived from RSPT and the BHF approximation in the partial-wave 
basis. In this basis, the equations can be considerably simplified by 
approximating the
Pauli exclusion operators (\ref{eq:paulihh_rcm}) and 
(\ref{eq:paulipp_rcm}) by averages over the angle between the relative
and CM momentum vectors. In this section, we consider only 
three-dimensional nuclear matter.

\subsection{The first orders of perturbation theory}
  
Infinite nuclear matter contains in theory an infinite number of particles, 
and one is therefore normally interested in the energy per nucleon. 
The kinetic energy per particle can be calculated analytically, and is 
for both PNM and SNM simply
\begin{align}
  \frac{E_{kin}}{A} &= \frac{\Omega }{A}\frac{1}{(2\pi )^{3}}\sum_{m_{s},m_{t}}\int_{|\mathbf{k}|\leq k_{F}}d\mathbf{k}\frac{\hbar^{2}k^{2}}{2m} \nonumber \\
  &= \frac{3\hbar^{2}k_{F}^{2}}{10m},
  \label{eq:ene_kin_expr}
\end{align}
where $\Omega \equiv L^{3}$ is the volume, $A$ is the number of nucleons, $k_{F}$ is the Fermi momentum, $m$ is the nucleon mass, and $\hbar $ is the Planck constant. To obtain the final expression in Eq.~(\ref{eq:ene_kin_expr}), the number of particles $A$ is calculated as
\begin{align}
  A = \frac{\Omega }{(2\pi )^{3}}\sum_{m_{s}, m_{t}}\int_{|\mathbf{k}|\leq k_{F}}d\mathbf{k},
\end{align}
that is, by counting all nucleons under the Fermi surface. In PNM calculations,
we use the neutron mass $m_{n}$, and in SNM calculations we use the average 
of the proton and neutron masses $\overline{m}=(m_{n}+m_{p})/2$. 

\begin{itemize}
\item Give an estimate of the error coming from using the average nucleon mass
\end{itemize}
 
The expressions for the potential energy are obtained by first writing the 
equations in terms of relative momenta, and then transforming to a 
partial-wave basis. When transforming the potential energy expressions to 
the desired form, we first rewrite the sums as integrals using the 
transformation (\ref{eq:sum2int}). Next, the equations are written in an 
RCM basis, as defined in Eq.~(\ref{eq:lab2rcm}), and the identity operators 
(\ref{eq:lm_complete}) and (\ref{eq:sm_complete}) are used to to introduce 
angular momenta. A coupled angular momentum basis is obtained by using the 
relation (\ref{eq:ls_jm}). In this chapter, we approximate all Pauli 
exclusion operators (\ref{eq:paulihh_rcm}) and (\ref{eq:paulipp_rcm}) 
by averages over the angle between the relative and CM momentum vectors, 
$\theta_{\mathbf{k}\mathbf{K}}$. That is, the exact Pauli operators are 
replaced with the angular-averaged hole-hole operator
\begin{align}
  \overline{Q}_{hh}(k, K, k_{F}) &= \frac{1}{2}\int_{-1}^{1}d\cos \theta_{\mathbf{k}\mathbf{K}} \nonumber \\
  & \times \theta(k_{F}-|\mathbf{k}+\mathbf{K}/2|)\theta(k_{F}-|-\mathbf{k}+\mathbf{K}/2|),
  \label{eq:pauli_ave_hh}
\end{align}
and the angular-averaged particle-particle operator
\begin{align}
  \overline{Q}_{pp}(k', K, k_{F}) &= \frac{1}{2}\int_{-1}^{1}d\cos \theta_{\mathbf{k}'\mathbf{K}} \nonumber \\
  & \times \theta(|\mathbf{k}'+\mathbf{K}/2|-k_{F})\theta(|-\mathbf{k}'+\mathbf{K}/2|-k_{F}).
  \label{eq:pauli_ave_pp}
\end{align}
Explicit expressions for the angular-averaged Pauli exclusion operators 
are given in for example Ref.~\cite{ramos_polls}.
Finally, the dependency on the angular coordinates of the relative momentum 
vectors are integrated out using Eq.~(\ref{eq:y_ortho}). 

With a transformation as described above, the first-order correction to 
the energy per nucleon becomes
\begin{align}
  \frac{\Delta E^{(1)}}{A} &= C \sum_{\mathcal{J}Sl}\sum_{M_{T}} (2\mathcal{J}+1) \nonumber \\
  & \times \int_{0}^{k_{F}}dk k^{2}\left( 1 - \frac{3}{2}\frac{k}{k_{F}} + \frac{1}{2}\left( \frac{k}{k_{F}}\right)^{3} \right) \nonumber \\
  & \times \langle k\mathcal{J}lSM_{T}|\hat{v}|k\mathcal{J}lSM_{T}\rangle ,
  \label{eq:ene_hf_pw}
\end{align}
where the constant $C$ is 1 for symmetric nuclear matter and 2 for pure 
neutron matter. The summation over $M_{T}$ takes the values -1, 0, 1 for 
symmetric nuclear matter and only 0 for pure neutron matter. Here and in 
the following, all interaction matrix elements are assumed to be multiplied 
by the operator (\ref{eq:antisymm_app}), which ensures proper antisymmetry 
and conservation of parity. A similar expression as Eq.~(\ref{eq:ene_hf_pw}) 
is given for symmetric nuclear matter by MacKenzie \cite{mackenzie} using 
coupled isospin $T$. The differing factor $4/\pi $ comes from a different 
definition of the Fourier-Bessel transform in the interaction matrix elements. 

Similarly, the second-order perturbation correction (\ref{eq:enePt2}) can be written as
\begin{align}
  \frac{\Delta E^{(2)}}{A} &= C\frac{3}{16k_{F}^{3}}\sum_{\mathcal{J}S}\sum_{ll'}\sum_{M_{T}}(2\mathcal{J}+1)\int_{0}^{2k_{F}}dK K^{2} \nonumber \\
  & \times \int_{0}^{\sqrt{k_{F}^{2}-K^{2}/4)}}dk k^{2} \int_{\sqrt{k_{F}^{2}-K^{2}/4}}^{\infty }dk' k'^{2} \nonumber \\
  & \times |\langle k\mathcal{J}lSM_{T}|\hat{v}|k'\mathcal{J}l'SM_{T}\rangle |^{2} \nonumber \\
  & \times \frac{\overline{Q}_{hh}(k, K, k_{F})\overline{Q}_{pp}(k', K, k_{F})}{\Delta \varepsilon_{ave}(k, k', K, M_{T})},
  \label{eq:pt2_pw}
\end{align}
where we have used angular-averaged Pauli exclusion operators and 
the constant $C$ is as defined for the Hartree-Fock term. In 
Eq.~(\ref{eq:pt2_pw}), the 
energy denominator has been approximated by the expression
\begin{align}
   \Delta \varepsilon_{ave}(k, k', K) &= \varepsilon(\overline{|\mathbf{k}+\mathbf{K}/2|}) + \varepsilon(\overline{|-\mathbf{k}+\mathbf{K}/2|}) \nonumber \\
   &- \varepsilon(\overline{|\mathbf{k}'+\mathbf{K}/2|}) - \varepsilon(\overline{|-\mathbf{k}'+\mathbf{K}/2|}),
   \label{eq:ene_denom_av}
\end{align}
where angular-averaged arguments are marked with an overline. Above, the isospin dependence in the energy denominator has been neglected for brevity. A similar expression as Eq.~(\ref{eq:pt2_pw}) is given in Ref.~\cite{mackenzie}, in which they assume that the interaction is a function of the total isospin only.

The exact energy denominator contains the single-particle energy terms
\begin{align}
  \varepsilon(k_{i}) + \varepsilon(k_{j}) = \varepsilon(|\mathbf{k}+\mathbf{K}/2|) + \varepsilon(|-\mathbf{k}+\mathbf{K}/2|),
\end{align}
where the laboratory frame momenta $k_{i}$ and $k_{j}$ have been written using RCM momenta on the right-hand side of the equation. When approximating the arguments of the single-particle energies, we use a technique first suggested by 
Brueckner and Gammel \cite{brueckner_gammel} and explained in the PhD thesis of Ramos \cite{ramos_phd,ramos_polls}. In this method, the single-particle energy is first approximated by a polynomial expansion. Because nuclear matter is an isotropic medium, the single-particle energy must be a symmetric function, and therefore only even powers in the polynomial have a nonzero contribution. The single-particle energy is approximated by a polynomial
\begin{align}
  \varepsilon(k) = \alpha + \beta k^{2} + \gamma k^{4} + \dots ,
\end{align}
where $\alpha $, $\beta $, and $\gamma $ are constants, and consequently 
\begin{align}
  \varepsilon(k_{i}) + \varepsilon(k_{j}) &= 2\alpha + \beta (k_{i}^{2} + k_{j}^{2}) + \gamma (k_{i}^{4} + k_{j}^{4}) + \dots \nonumber \\
  & = \dots \nonumber \\
  & = 2\alpha + 2\beta (k^{2} + K^{2}/4) \nonumber \\
  & + 2\gamma \left( (k^{2}+K^{2}/4)^{2} + (\mathbf{k}\cdot \mathbf{K})^{2}\right) + \dots .
\end{align}
As Ramos points out, the first dependence on the angle between $\mathbf{k}$ and $\mathbf{K}$ comes at second order.~\cite{ramos_phd} 

Following Ramos, we do the angular-average approximation
\begin{align}
  (\mathbf{k}\cdot \mathbf{K})^{2} \approx K^{2}k^{2}\overline{\cos^{2}\theta_{\mathbf{k}\mathbf{K}}} ,
\end{align}
where
\begin{align}
  \overline{ \cos^{2}\theta_{\mathbf{k}\mathbf{K}}} &= \frac{1}{2}\int_{-1}^{1}d\cos \theta_{\mathbf{k}\mathbf{K}} \cos^{2}\theta_{\mathbf{k}\mathbf{K}} \nonumber \\
  & \times \theta(k_{F}-|\mathbf{k}+\mathbf{K}/2|)\theta(k_{F}-|-\mathbf{k}+\mathbf{K}/2|)
\end{align}
for hole states and
\begin{align}
  \overline{ \cos^{2}\theta_{\mathbf{k}\mathbf{K}}} &= \frac{1}{2}\int_{-1}^{1}d\cos \theta_{\mathbf{k}\mathbf{K}} \cos^{2}\theta_{\mathbf{k}\mathbf{K}} \nonumber \\
  & \times \theta(|\mathbf{k}+\mathbf{K}/2|-k_{F})\theta(|-\mathbf{k}+\mathbf{K}/2|-k_{F})
\end{align}
for particle states. The angular-averaged value can be shown to be 
\begin{align}
  \overline{ \cos^{2}\theta_{\mathbf{k}\mathbf{K}}} = \frac{1}{3}\overline{Q}^{3}(k, K, k_{F}),
\end{align}
where $\overline{Q}$ is replaced with the angular-averaged Pauli exclusion operator $\overline{Q}_{hh}$ for two-hole states, and by $\overline{Q}_{pp}$ for two-particle states. Now we get the angular-averaged input momenta
\begin{align}
  k_{i}^{2} &= \overline{|\mathbf{k}+\mathbf{K}/2|^{2}} \nonumber \\
  & = k^{2} + K^{2}/4 + \frac{1}{\sqrt{3}}kKQ_{hh}^{3/2}(k, K, k_{F})
  \label{eq:ki_ave}
\end{align} 
and
\begin{align}
  k_{j}^{2} &= \overline{|-\mathbf{k}+\mathbf{K}/2|^{2}} \nonumber \\
  & = k^{2} + K^{2}/4 - \frac{1}{\sqrt{3}}kKQ_{hh}^{3/2}(k, K, k_{F}).
  \label{eq:kj_ave}
\end{align}
Angular-averaged input momenta for particle states $k_{a}$ and $k_{b}$
are defined in a similar way \cite{ramos_phd}. Explicit expressions
for the single-particle energies are obtained from Eqs.~(\ref{eq:wiso})
and (\ref{eq:uiso}) by replacing the $G$ matrix with a bare interaction.

The particle-particle term in third-order perturbation theory, Eq.~(\ref{eq:mbpt3_pp}), can be written as 
\begin{align}
  \frac{\Delta E_{pp}^{(3)}}{A} &= C \frac{3}{32k_{F}^{3}}\sum_{\mathcal{J}S}\sum_{ll'l''}\sum_{M_{T}}(2\mathcal{J} + 1) \int_{0}^{2k_{F}}dK K^{2} \nonumber \\
  & \times \int_{0}^{\sqrt{k_{F}^{2}-K^{2}/4}}dk k^{2}\int_{\sqrt{k_{F}^{2}-K^{2}/4}}^{\infty }dp p^{2} \int_{\sqrt{k_{F}^{2}-K^{2}/4}}^{\infty }dp' p'^{2} \nonumber \\
  & \times \frac{\langle k\mathcal{J}(lS)M_{T}|\hat{v}|p\mathcal{J}(l'S)M_{T}\rangle \langle p\mathcal{J}(l'S)M_{T}|\hat{v}|p'\mathcal{J}(l''S)M_{T}\rangle }{\Delta \varepsilon_{ave}(k, p, K) \Delta \varepsilon_{ave}(k, p', K)} \nonumber \\
  & \times \langle p'\mathcal{J}(l''S)M_{T}|\hat{v}|k\mathcal{J}(lS)M_{T}\rangle \nonumber \\
  & \times \overline{Q}_{hh}(k, K, k_{F})\overline{Q}_{pp}(p, K, k_{F})\overline{Q}_{pp}(p', K, k_{F}), 
  \label{eq:pt3pp_pw}
\end{align}
where $C$ is 1 for symmetric nuclear matter and 2 for pure neutron matter. In Eq.~(\ref{eq:pt3pp_pw}), the Pauli exclusion operators in the numerator have been approximated by angular-averaged operators. A similar expression for the particle-particle term is given by MacKenzie \cite{mackenzie} for symmetric nuclear matter with total two-particle isospin as a good quantum number. In Ref.~\cite{mackenzie}, there is an additional factor $(4/\pi )^{3}$ owing to a different definition of the interaction matrix elements. 


\subsection{The BHF approximation}
The Brueckner-Hartree-Fock approximation is a standard approach for calculating the binding energy of infinite nuclear matter. Implementations of the $G$-matrix equation have therefore been discussed many places in the literature \cite{haftel_tabakin, ramos_phd, mahaux1985, hjensen1995}. In addition to being an essential part of the Brueckner-Goldstone theory \cite{day1967}, the $G$ matrix is encountered also in for example the self-consistent Greens function method \cite{ramos_polls}. We have solved the $G$-matrix equation using the matrix inversion technique of Haftel and Tabakin \cite{haftel_tabakin}, but using continuous single-particle energy spectra \cite{mahaux1985, mahaux1989}. Using continuous single-particle spectra means that the operator $\hat{U}$ is chosen such that the definition (\ref{eq:u_gmat}) is used for both hole and particle single-particle states.

Our implementation of the BHF approximation follows to a large degree Haftel and Tabakin \cite{haftel_tabakin}, using angular-averaged Pauli exclusion operators. As given in Ref.~\cite{haftel_tabakin}, the $G$-matrix equation (\ref{eq:gmatrix_as}) can be written in a partial-wave expansion basis as
\begin{align}
  & \langle k\mathcal{J}(lS)M_{T}|\hat{g}(K)|k'\mathcal{J}(l'S)M_{T}\rangle \nonumber \\
  &= \langle k\mathcal{J}(lS)M_{T}|\hat{v}|k'\mathcal{J}(l'S)M_{T}\rangle \nonumber \\
  & + \frac{1}{2}\sum_{l''}\int_{0}^{\infty }dp p^{2} Q_{pp}(p, K, k_{F}) \nonumber \\
  & \times \frac{\langle k\mathcal{J}(lS)M_{T}|\hat{v}|p\mathcal{J}(l''S)M_{T}\rangle \langle p\mathcal{J}(l''S)M_{T}|\hat{g}(K)|k'\mathcal{J}(l'S)M_{T}\rangle }{\Delta \varepsilon_{ave}(k', p, K, M_{T})},
  \label{eq:gmat_pw}
\end{align}
where we notice that the $G$ matrix depends on the radial part of the CM momentum. The energy denominator is defined as in Eq.~(\ref{eq:ene_denom_av}).

Next, let us consider the energy denominator. The single-particle energy of a state $|\mathbf{k}_{p}, m_{s}, m_{t}\rangle $ is 
\begin{align}
  \varepsilon(\mathbf{k}_{p}, m_{s}, m_{t}) &= \frac{\hbar^{2}k_{p}^{2}}{2m} + \sum_{m_{s'}}\sum_{m_{t'}}\int d\mathbf{k}_{j}\theta(k_{F}-|\mathbf{k}_{j}|) \nonumber \\
  & \times \langle \mathbf{k}_{p}m_{s}m_{t}\mathbf{k}_{j}m_{s'}m_{t'}|\hat{g}|\mathbf{k}_{p}m_{s}m_{t}\mathbf{k}_{j}m_{s'}m_{t'}\rangle_{AS},
\end{align}
where $m_{s}$ and $m_{t}$ are the $z$ projections of spin and isospin, respectively, and a factor $\Omega /(2\pi)^{3}$ is included in the two-body interaction. If the integration variable is changed to relative coordinates, i.e.,
\begin{align}
  \mathbf{k} = (\mathbf{k}_{p}-\mathbf{k}_{j})/2, \qquad \mathbf{K} = \mathbf{k}_{p} + \mathbf{k}_{j},
\end{align}
where $\mathbf{k}$ and $\mathbf{K}$ are relative and CM coordinates, respectively, the single-particle potential becomes 
\begin{align}
  U(\mathbf{k}_{p}, m_{s}, m_{t}) &= 8\sum_{m_{s'}}\sum_{m_{t'}}\int_{\min\{ 0, (k_{p}-k_{F})/2\}}^{(k_{p}+k_{F})/2}dk k^{2}\int_{0}^{2\pi }d\phi_{\mathbf{k}}\int_{-1}^{1}d(\cos\theta_{\mathbf{k}}) \nonumber \\
  & \times \langle \mathbf{k}m_{s}m_{t}m_{s'}m_{t'}|\hat{g}(\mathbf{K})|\mathbf{k}m_{s}m_{t}m_{s'}m_{t'}\rangle_{AS}  \nonumber \\
  & \times \theta(k_{F} - |-2\mathbf{k} + \mathbf{k}_{p}|). 
  \label{eq:u_bhf_2}
\end{align}
Because infinite nuclear matter is an isotropic medium, the single-particle energy should not depend on the direction of $\mathbf{k}_{p}$. The input momentum vector is chosen to be directed along the positive $z$ axis. 
 
In our calculations we have replaced the CM momentum with $\overline{K^{2}}$, defined as the average of $|\mathbf{K}|^{2}$ when integrating with respect to the vector $\mathbf{k}$. If $\mathbf{k}_{p}$ is directed along the $z$ axis, the average of the squared CM momentum becomes 
\begin{align}
  \overline{K^{2}} \equiv \frac{1}{2}\int_{-1}^{1}d\cos \theta_{\mathbf{k}}\left|2(\mathbf{k}_{p}-\mathbf{k})\right|^{2}\theta(k_{F}-|-2\mathbf{k}+\mathbf{k}_{p}|).
  \label{eq:Kav2_def}
\end{align}
By evaluating this expression, we got the explicit formula
\begin{align}
  \overline{K^{2}} = \left\{ \begin{array}{ll}
    4(k_{p}^{2} + k^{2}), & \text{ if } 0 \leq 2k \leq k_{F}-k_{p} \\ 
    & \text{ and } k_{p} \leq k_{F}, \\
    4(k_{p}^{2} + k^{2}) + \frac{1}{8kk_{p}}\left[ -16k_{p}k^{3} \right. & \\
     -4(7k_{p}^{2} + k_{F}^{2})k^{2} - 16k_{p}^{3}k & \\
     + \left. k_{F}^{4} + 2k_{F}^{2}k_{p}^{2} - 3k_{p}^{4} \right], & \text{ if } |k_{F} - k_{p}| \leq 2k \leq k_{F} + k_{p}.
    \end{array} \right.
  \label{eq:Kav2}
\end{align}
Haftel and Tabakin \cite{haftel_tabakin} have given a similar averaged CM momentum that is defined only when $k_{p} \leq k_{F}$. Our expression differs from Haftel and Tabakin's in the second interval, but the limits of the intervals are the same. In Fig.~\ref{fig:cm_mom_ave}, our formula for $\overline{K^{2}}$ is compared with that of Ref.~\cite{haftel_tabakin}. Observe that our definition of the CM momentum vector $\mathbf{K}$ is different from the definition
\begin{equation}
  \mathbf{P} = (\mathbf{k}_{i}-\mathbf{k}_{j})/2
  \label{eq:Pcm}
\end{equation}
used by Haftel and Tabakin. In Fig.~\ref{fig:cm_mom_ave}, both expressions 
were calculated using the definition (\ref{eq:Pcm}). In contrast to the 
expression given by Haftel and Tabakin, our average CM momentum vanishes 
at $k = (k_{p} + k_{F})/2$, as is required by the definition 
(\ref{eq:Kav2_def}). It seems therefore that the formula for $\overline{K^{2}}$
given in Ref.~\cite{haftel_tabakin} is wrong.


\begin{figure}
  \centering
  \includegraphics[scale=0.75]{figures/cm_mom_ave/plot_krel_Pav2.pdf}
  \caption{Average of the squared CM momentum, $\overline{P^{2}} = \overline{K^{2}}/4$, plotted as a function of the relative momentum for given lab momenta $k_{p}$. The Fermi momentum $k_{F}$ was fixed to 1.6 fm$^{-1}$. The dashed lines represent results calculated with the formula for $\overline{P^{2}}$ given in Eq.~(3.19) of Ref.~\cite{haftel_tabakin}. In contrast to the formula given in Ref.~\cite{haftel_tabakin}, our formula for $\overline{K^{2}}$ gives the correct values at the end points of the relative momentum interval.}  
  \label{fig:cm_mom_ave}
\end{figure}

In addition to using the averaged CM momentum (\ref{eq:Kav2_def}), we replace the Pauli exclusion operator in Eq.~(\ref{eq:u_bhf_2}) with the angular-averaged operator 
\begin{align}
  Q(k, k_{p}, k_{F}) = \frac{1}{2}\int_{-1}^{1}d\cos \theta_{\mathbf{k}} \theta(k_{F} - |-2\mathbf{k} + \mathbf{k}_{p}|),
\end{align}
similarly as done in Ref.~\cite{haftel_tabakin}. Applying angular momentum algebra, we got the expression
\begin{align}
  U(k_{p}, m_{t}) = & 16\sum_{m_{t'}}\sum_{\mathcal{J}lS}(2\mathcal{J} + 1) \left[ \theta(k_{F}-|k_{p}|)\int_{0}^{|k_{F}-k_{p}|/2}dk k^{2} \right. \nonumber \\
    & \left. + \int_{|k_{F}-k_{p}|/2}^{(k_{F}+k_{p})/2}dk k^{2} \left( \frac{-4k^{2}+4kk_{p}-k_{p}^{2}+k_{F}^{2}}{8kk_{p}}\right) \right] \nonumber \\
  & \times \left[ \langle k\mathcal{J}(lS)m_{t}m_{t'}|g(K_{av})|k\mathcal{J}(lS)m_{t}m_{t'}\rangle \right. \nonumber \\
    & \left. - (-1)^{1+S+l} \langle k\mathcal{J}(lS)m_{t}m_{t'}|g(K_{av})|k\mathcal{J}(lS)m_{t'}m_{t}\rangle \right]. 
\end{align}
Above, the single-particle potential is given as a function of a single-particle isospin projection $m_{t}$. In the energy denominator, the single-particle potential always occurs in pairs
\begin{align}
  W(k_{p}, m_{t_{p}}, k_{q}, m_{t_{q}}) \equiv U(k_{p}, m_{t_{p}}) + U(k_{q}, m_{t_{q}}).
\end{align}
The function $W$ is a function of the two-particle isospin projection $M_{T} = m_{t_{p}} + m_{t_{q}}$, i.e.
\begin{align}
  W^{M_{T}}(k_{p}, k_{q}) \equiv U^{M_{T}, +}(k_{p}) + U^{M_{T}, -}(k_{q}),
  \label{eq:wiso}
\end{align}
where the explicit expressions of the single-particle potentials are 
\begin{align}
  U^{M_{T}, \pm }(k_{p}) = &16 \sum_{\mathcal{J}lS}(2\mathcal{J} + 1) \left[ \theta(k_{F} - |k_{p}|) \int_{0}^{|k_{F}-k_{p}|/2}dk k^{2} \right. \nonumber \\
    & \left. + \int_{|k_{F}-k_{p}|/2}^{(k_{F}+k_{p})/2}dk k^{2} \left( \frac{-4k^{2} + 4kk_{p} - k_{p}^{2} + k_{F}^{2}}{8kk_{p}}\right) \right] \nonumber \\
  & \times \mathcal{B}^{M_{T}, \pm }\langle k\mathcal{J}(lS)|g(K_{av})|k\mathcal{J}(lS)\rangle ,
  \label{eq:uiso}
\end{align}
and the antisymmetrization operator $\mathcal{B}^{M_{T}, \pm }$ is defined in
Eqs.~(\ref{eq:bopsnm}) and (\ref{eq:boppnm}). 

%%  For simplicity, in our BHF calculations we have approximated the antisymmetrization operator for symmetric nuclear matter by 
%% \begin{align}
%%   & \tilde{\mathcal{B}}\langle k\mathcal{J}(lS)|\hat{O}|k\mathcal{J}(lS)\rangle = \langle k\mathcal{J}(lS)|\hat{O}(M_{T'}=0)|k\mathcal{J}(lS)\rangle \nonumber \\ 
%%   & + \frac{1}{2}\left(1 - (-1)^{1+l+S}\right)\left[ \langle k\mathcal{J}(lS)|\hat{O}(M_{T'}=-1)|k\mathcal{J}(lS)\rangle \right. \nonumber \\
%%     & \left. + \langle k\mathcal{J}(lS)|\hat{O}(M_{T'}=1)|k\mathcal{J}(lS)\rangle \right] ,
%%   \label{eq:b_approx}
%% \end{align}
%% which makes the energy denominator independent on isospin.

Finally, let us write the BHF energy (\ref{eq:ene_bhf}) in the partial-wave basis. If the angular-average approxmation (\ref{eq:pauli_ave_hh}) is used for the hole-hole Pauli exclusion operator, the BHF energy per nucleon is
\begin{align}
  \frac{\Delta E_{BHF}}{A} &= \frac{3\pi }{2k_{F}^{3}}\sum_{\mathcal{J}lS}\sum_{M_{T}}(2\mathcal{J}+1)\int_{0}^{k_{F}}dk k^{2} \nonumber \\
  & \times \left\{ \int_{0}^{2(k_{F}-k)}dK K^{2} - \int_{2(k_{F}-k)}^{2\sqrt{k_{F}^{2}-k^{2}}}dK K^{2}\frac{k^{2}-k_{F}^{2}+K^{2}/4}{kK} \right\} \nonumber \\
  & \times \langle k\mathcal{J}lSM_{T}|g(K)|k\mathcal{J}lSM_{T}\rangle ,
\end{align} 
where the $G$-matrix elements are assumed to be multiplied by the 
antisymmetrization and parity conservation operator 
$\mathcal{A}^{llSM_{T}}$, defined in Eq.~(\ref{eq:antisymm_app}). 
Haftel and Tabakin \cite{haftel_tabakin} have given
a similar expression for a system in which the interaction depends only
on the total two-particle isospin.

\begin{itemize}
\item A figure with results: HF, MBPT(2), MBPT(3) with only pp ladders, \\
  and BHF with NNLO$_{\text{opt}}$ and AV18 
\end{itemize}

\section{Implementations}

\subsection{The first orders of perturbation theory}

The energy expressions for the lowest orders of perturbation theory, given in Eqs. (\ref{eq:ene_hf_pw}), (\ref{eq:pt2_pw}), and (\ref{eq:pt3pp_pw}), were straightforwardly implemented using numerical integration with Gauss-Legendre quadratures \cite{num_recipes}. The expressions for the angular-averaged Pauli exclusion operators are discontinuous with respect to the first derivative. In our implementation, we split the integration intervals at the discontinuity points of the Pauli operators to avoid convergency problems in the numerical integrations.  

\subsection{The BHF approximation}
 We solve the $G$-matrix equation using 
a matrix inversion technique introduced by Haftel and Tabakin 
\cite{haftel_tabakin}. As outlined in 
Ref.~\cite{haftel_tabakin}, the $G$-matrix equation 
(\ref{eq:gmat_pw}) may be written as 
\begin{align}
  \mathbf{V} = \mathbf{U}\mathbf{G},
\end{align} 
where, for coupled channels, $\mathbf{G}$ is a $2(N+1) \times 2(N+1)$ matrix
\begin{align}
  \mathbf{G} = \left[ \begin{array}{cc}
      \mathbf{G}_{l_{\text{min}}, l_{\text{min}}} & \mathbf{G}_{l_{\text{min}}, l_{\text{max}}} \\
      \mathbf{G}_{l_{\text{max}}, l_{\text{min}}} & \mathbf{G}_{l_{\text{max}}, l_{\text{max}}} \\
    \end{array} \right] , 
  \label{eq:gmat_coupled}
\end{align}
where $l_{\text{min}} = \mathcal{J}-1$ and 
$l_{\text{max}} = \mathcal{J}+1$. The matrices $\mathbf{V}$ 
and $\mathbf{U}$ are set up similarly for coupled channels. 
For uncoupled channels, the matrices $\mathbf{V}$, 
$\mathbf{U}$, and $\mathbf{G}$ contain only one submatrix. 
In Eq.~(\ref{eq:gmat_coupled}), the submatrices have matrix 
elements
\begin{align}
  \left[ \mathbf{G}_{l, l'}\right]_{i, j} \equiv \langle k_{i}\mathcal{J}(lS)M_{T}|\hat{g}(K)|k_{j}\mathcal{J}(l'S)M_{T}\rangle ,
  \label{eq:submat_g}
\end{align}
for $i, j \in \left\{ 1, \dots , N+1\right\}$. Similarly, the matrix $\mathbf{V}$ is defined by the interaction matrix elements
\begin{align}
  \left[ \mathbf{V}_{l,l'}\right]_{i, j} \equiv \langle k_{i}\mathcal{J}(lS)M_{T}|\hat{v}|k_{j}\mathcal{J}(l'S)M_{T}\rangle
  \label{eq:submat_v}
\end{align}
and $\mathbf{U}$ by the elements
\begin{align}
  \left[ \mathbf{U}_{l,l'}\right]_{i, j} \equiv \delta_{ij} + u_{j}\left[ \mathbf{V}_{l,l'}\right]_{i, j}, 
  \label{eq:submat_u}
\end{align}
where 
\begin{align}
  u_{j} = \left\{ \begin{array}{ll}
    -\frac{\omega_{k_{j}}k_{j}^{2}Q_{pp}(k_{j}, K, k_{F})}{\Delta \varepsilon_{av}(k_{j}, k_{0}, K)}, & \text{ if } j \leq N, \\
      \sum_{p=1}^{N} \frac{\omega_{k_{p}}k_{0}^{2}Q_{pp}(k_{0}, K, k_{F})}{\Delta \varepsilon_{av}(k_{p}, k_{0}, K)}, & \text{ if } j = N+1,
    \end{array} \right. 
\end{align}
and $u_{j+N+1} = u_{j}$ for $j = 1, 2, \dots , N+1$. The coefficients $\omega_{k_{j}}$ are the quadrature weights corresponding to the grid points $k_{j}$, and $k_{0}$ is the relative momentum point we want to evaluate the $G$ matrix at. Following Haftel and Tabakin, we calculate only the principal value of the integral in the $G$-matrix equation (\ref{eq:gmat_pw}). As shown in Ref.~\cite{haftel_tabakin}, the singularity in the integrand is removed  by adding a term that integrates to zero. A more proper way of dealing with the singularity is to solve a complex $G$-matrix equation, as is explained, for example, in the PhD thesis of Engvik \cite{engvik_phd}. In the submatrices (\ref{eq:submat_g}), (\ref{eq:submat_v}), and (\ref{eq:submat_u}), the momentum mesh points are $[k_{1}, k_{2}, \dots , k_{N}, k_{0}]$, where the elements $k_{1}, k_{2}, \dots , k_{N}$ are integration points used to integrate from zero to infinity and $k_{0}$ is the chosen evaluation point.

\begin{figure}
  \centering
  \includegraphics[scale=1.0]{figures/setup_grid/setup_grid-crop.pdf}
  \caption{We use this algorithm to set up the momentum grid 
    points for the $G$ matrix. The principles of this setup
    are similar as in Refs.~\cite{ramos_phd, engvik_phd},
    but here we automatically find a suitable grid.}  
  \label{fig:setup_grid}
\end{figure}


When solving the $G$-matrix equation, the grid points must 
be set up carefully to get good convergence in the numerical 
integrations. The angular-averaged Pauli exclusion operator 
$Q_{pp}(k, K, k_{F})$ has a discontinuous first derivative at 
certain momenta $k$ that depend on the CM momentum $K$ and 
the Fermi momentum $k_{F}$. Therefore, we split the 
integration interval in relative momentum $k$ at the 
discontinuity points of the Pauli operator. Even if the 
singularity point of the $G$-matrix equation is removed by 
calculating only the principal value, the region where 
the energy denominator is small still may cause convergence 
problems in a numerical integration. According to Engvik 
\cite{engvik_phd}, better numerical accuracy can be obtained by choosing 
integration points symmetrically around the singularity 
point where the denominator vanishes. Using this technique, 
the setup of the integration points $[k_{1}, \dots , k_{N}]$ 
becomes therefore dependent on both $K$, which is the CM 
momentum, and $k_{0}$, which is the singularity point of 
the integrand. We set up the relative momentum grid points 
using the algorithm given in Fig.~\ref{fig:setup_grid}. In 
her PhD thesis \cite{ramos_phd}, Ramos has set up the 
momentum grid points in a similar way. However, we 
have written an algorithm to find a grid with certain
conditions.

\vspace{1cm}
\begin{algorithmic}
  
  %\begin{algorithm}
   % \caption{Algorithm for the BHF approximation.}
    \State     Selfconsistency loop for the BHF approximation.
    \While{not converged} 
    \State Set up a vector of the single-particle potential.
    \State Loop over input laboratory frame momenta
    \State Get average CM momentum given $k_{lab}$ and $k_{rel}$ 
    
    
    
    \EndWhile
  
  %\end{algorithm}
  
\end{algorithmic}

  %% while not converged
  
  %%      #     Set up a vector of the single-particle 
  %%      #     potential as a function of input momentum.

  %%      #     Loop over input laboratory frame momenta 
  %%      #     k_lab.


\chapter{Coupled-cluster approximations for infinite matter} \label{ch:cc}

The coupled-cluster (CC) method is one of the most popular \emph{ab initio}
many-body approaches in quantum chemistry \cite{bartlett_review}. 
A major advantage of the CC approach is that it combines high accuracy 
with moderate computational cost. The method scales polynomially 
with the number of occupied and unoccupied orbitals, and is currently the 
most accurate many-body method for medium-size molecules 
\cite{bartlett_review}. Since the late 1990s, CC theory has been used 
to calculate the binding energy of light and 
medium-mass nuclei \cite{dean2004,heisenberg1999,hagen2007_1,
hagen2007_2,roth2009}, nuclei around the neutron drip line 
\cite{hagen2012,hagen2013}, calculations of spectroscopic 
factors \cite{jensen2010}, and a study of time-dependent behaviour 
of simple nuclear models \cite{pigg2012}, among other. In contrast, 
there are very few applications of CC theory for infinite 
nuclear and neutron-star matter. To our knowledge, there has been
no coupled-cluster studies of nuclear matter since Day and Zabolitzky 
did their calculations in the early 1980s \cite{day_cc}. In 
Papers I-III we present new applications 
of CC theory for nuclear matter, using interaction models derived
from chiral perturbation theory.


\section{Coupled-cluster theory} \label{sec:cct}
%The Coupled-Cluster theory is applied to infinite nuclear matter systems in one, two, and three dimensions.

\begin{itemize}
\item Why CC theory?
\end{itemize}

As mentioned in Chapter \ref{ch:mbpt}, the hole-line expansion 
\cite{day1967,raja,rajaraman1963,bethe1965} provides a physically 
motivated convergence parameter in the number of independent hole lines. 
Unfortunately, the hole-line 
expansion requires a rapidly increasing number of diagrams for higher-order
approximations. In addition, the assumption that the number of hole lines 
is a good convergence parameter is only valid at sufficiently low densities, 
when the average distance between the nucleons is large compared to the 
range of the interaction \cite{day1967,muther2000}. A method that is 
related to the perturbative techniques is coupled-cluster theory 
\cite{coester1958,coester1960,bartlett_review}. Coupled-cluster theory has 
some of the same properties as many-body methods derived from 
Rayleigh-Schr{\"o}dinger perturbation theory, such as size consistency and 
a nonvariational energy, but it suggests a convenient scheme for obtaining 
a physically justified truncation. The theory gives a system of nonlinear 
equations, which describe all possible correlations in a model where 
arbitrary many one-, two-, and up to $n$-particle clusters can 
be formed \cite{harris,crawford}. Already in an approximation including
only one- and two-particle clusters, large sets of diagrams are 
included to infinite order. 
\textbf{Compare with classes of diagrams in MBPT}

\begin{itemize}
\item The SUB2 approximation ($=$ CCSD ?) contains among others RPA ring diagrams, the Bethe-Goldstone approximation, and the Brueckner-Bethe-Goldstone theory \cite{bishop1978, bishop_lahoz_1987}
\end{itemize}

We follow here Refs.~\cite{harris,crawford} when introducing the general concepts of CC theory. Coupled-cluster theory starts from the assumption that the total wave function can be written in the exponential form 
\begin{equation}
  |\Psi_{CC}\rangle = e^{\hat{T}}|\Phi_{0}\rangle ,
\end{equation}  
where $|\Phi_{0}\rangle $ is the Fermi vacuum state, the total cluster operator $\hat{T}$ is defined as the sum of cluster operators
\begin{equation}
  \hat{T} = \sum_{m=1}^{A}\hat{T}_{m},
  \label{eq:sum_cluster}
\end{equation} 
and a single $m$-particle-$m$-hole cluster operator is defined as
\begin{align}
  \hat{T}_{m} = \left( \frac{1}{m!}\right)^{2}\sum_{i_{1}, \dots ,i_{m}\atop a_{1}, \dots , a_{m}}t_{i_{1}, \dots , i_{m}}^{a_{1}, \dots , a_{m}}a_{a_{1}}^{\dagger }\dots a_{a_{m}}^{\dagger }a_{i_{m}}\dots a_{i_{1}}. 
\end{align} 
Here the labels $i_{1}, \dots , i_{m}$ and $a_{1}, \dots , a_{m}$ denote states occupied and unoccupied in $|\Phi_{0}\rangle $, respectively, $a_{p}^{\dagger }$ and $a_{p}$ are creation and annihilation operators, respectively, and the CC amplitudes $t_{i_{1}, \dots , i_{m}}^{a_{1}, \dots , a_{m}}$ are unknowns that can be obtained by solving a set of nonlinear equations. As shown in Refs.~\cite{harris,crawford}, the exponential form arises naturally if the wave function ansatz is required to include all possible correlations of arbitrary many one-particle, two-particle and up to $A$-particle clusters. 

%Provided that there are no clusters with more than $n$ particles that are inseparably correlated, the exponential ansatz provides a physically well-motivated truncation.

The CC equations are obtained by projecting the Schr{\"o}dinger equation
\begin{equation}
  \hat{H}e^{\hat{T}}|\Phi_{0}\rangle = Ee^{\hat{T}}|\Phi_{0}\rangle
\end{equation}
onto the bra vectors 
\begin{align}
  \langle \Phi_{0}|e^{-\hat{T}}, \qquad \langle \Phi_{i_{1},\dots ,i_{m}}^{a_{1},\dots ,a_{m}}|e^{-\hat{T}},
\end{align}
where $m = 1, 2, \dots , A$ in the latter vector. In other words, the Hamiltonian operator $\hat{H}$ is replaced with a similarity-transformed operator $\overline{H}$, defined as
\begin{equation}
  \overline{H} = e^{-\hat{T}}\hat{H}e^{\hat{T}},
\end{equation}
and the Hamiltonian equation of $\overline{H}$ is projected against the bra vectors $\langle \Phi_{0}|$ and $\langle \Phi_{i_{1},\dots ,i_{m}}^{a_{1},\dots ,a_{m}}|$. The CC equations consist of the energy equation
\begin{align}
  \langle \Phi_{0}|\overline{H}|\Phi_{0}\rangle = E
  \label{eq:cc_ene}
\end{align} 
and the amplitude equations
\begin{align}
  \langle \Phi_{i_{1},\dots ,i_{m}}^{a_{1},\dots ,a_{m}}|\overline{H}|\Phi_{0}\rangle = 0, \qquad m = 1, 2, \dots A.
  \label{eq:cc_tampl}
\end{align}
In practical calculations, the CC amplitudes $t_{i_{1}, \dots , i_{m}}^{a_{1}, \dots , a_{m}}$ are first obtained by solving Eq.~(\ref{eq:cc_tampl}). Thereafter the CC amplitudes are used for calculating the energy from Eq.~(\ref{eq:cc_ene}).

%As MBPT, the coupled-cluster method is also size extensive and nonvariational. 



The coupled-cluster equations as presented above is an exact reformulation of
the Schr{\"o}dinger energy eigenvalue problem. Except for very simple model 
systems, few- or many-body problems cannot be solved exactly. In quantum 
chemistry, the most common approach for approximating the coupled-cluster 
equations is to truncate the cluster operator (\ref{eq:sum_cluster}) after 
$m = n$, where $n$ is smaller than the number of particles 
\cite{crawford,bartlett_review}. Approximations including 
$\hat{T}_{1}$, $\hat{T}_{2}$, $\hat{T}_{3}$, etc., are in the CC literature 
commonly denoted by singles (S), doubles (D), triples (T), etc., 
respectively \cite{crawford}. For example, the CCSD approximation includes 
only $\hat{T}_{1}$ and $\hat{T}_{2}$, whereas in the CCD approximation only 
$\hat{T}_{2}$ is nonzero. The same approximations are sometimes called
SUB$n$ truncations \cite{bishop1978}, where for example the SUB2 approximation
is equal to CCSD. An alternative family of approximations is the so-called 
Bochum truncation scheme \cite{kummel1978}, which is tailored for systems
with hard-core interaction models. In this work, we consider only 
the SUB$n$ truncation scheme of CC theory. 

In the following, we give the explicit CC equations in the CCD 
approximation, as has been derived in for example Ref.~\cite{crawford}. 
The same equations have been given in Ref.~\cite{baardsen}, but we 
repeat the basic equations to make it easier to follow the discussions. 
In addition, these equations are also used in our studies of the 
homogeneous electron gas, which are presented in Sec.~\ref{sec:ccheg}.  
As was explained in Ref.~\cite{baardsen}, conservation of momentum 
and symmetry makes the $\hat{T}_{1}$ amplitude to vanish for infinite 
homogeneous matter. For homogeneous matter, the CCD approximation is 
therefore equal to CCSD. The general equations are given in a 
momentum basis. For brevity, we neglect the spin and 
isospin degrees of freedom in the equations. 

The total CCD energy is split into two terms, i.e. 
\begin{equation}
  E_{CCD} = E_{REF} + \Delta E_{CCD},
\end{equation}
where $E_{REF}$ is the reference energy, defined as the Fermi
vacuum expectation value 
\begin{equation}
  E_{REF} = \langle \Phi_{0}|\hat{H}|\Phi_{0}\rangle ,
\end{equation}
and $\Delta E_{CCD}$ is the remaining correction in the given
approximation. Assuming that the Hamiltonian is of the form
\begin{align}
  \hat{H} &= \sum_{\mathbf{k}_{p}\mathbf{k_{q}}}\langle 
  \mathbf{k}_{p}|h_{0}|\mathbf{k}_{q}\rangle 
  a_{\mathbf{k}_{p}}^{\dagger }a_{\mathbf{k}_{q}} \nonumber \\
  &+ \frac{1}{4}\sum_{\mathbf{k}_{p}\mathbf{k}_{q}}
  \sum_{\mathbf{k}_{r}\mathbf{k}_{s}}\langle \mathbf{k}_{p}
  \mathbf{k}_{q}|v|\mathbf{k}_{r}\mathbf{k}_{q}\rangle_{AS}
  a_{\mathbf{k}_{p}}^{\dagger }a_{\mathbf{k}_{q}}^{\dagger }
  a_{\mathbf{k}_{s}}a_{\mathbf{k}_{r}} 
  + \text{constant},
\end{align}
the reference energy is
\begin{align}
  E_{REF} = \sum_{\mathbf{k}_{i}}\langle 
\mathbf{k}_{i}|\hat{h}_{0}|\mathbf{k}_{i}\rangle 
+ \frac{1}{2}\sum_{\mathbf{k}_{i},\mathbf{k}_{j}}
\langle \mathbf{k}_{i}\mathbf{k}_{j}|\hat{v}|
\mathbf{k}_{i}\mathbf{k}_{j}\rangle_{AS} 
+ \text{constant}
  \label{eq:ene_ref}
\end{align}
and the CCD correction  
\begin{align}
  \Delta E_{CCD} = \frac{1}{4}\sum_{\mathbf{k}_{i},\mathbf{k}_{j}}\sum_{\mathbf{k}_{a},\mathbf{k}_{b}}\langle \mathbf{k}_{i}\mathbf{k}_{j}|\hat{v}|\mathbf{k}_{a}\mathbf{k}_{b}\rangle_{AS}\langle \mathbf{k}_{a}\mathbf{k}_{b}|\hat{t}|\mathbf{k}_{i}\mathbf{k}_{j}\rangle ,
  \label{eq:ene_ccd}
\end{align}
where in the latter equation the CC amplitude is written using a bracket notation. In the following, we always use the bracket notation for the CC amplitudes. Fig.~\ref{fig:ene_ccd} shows the CCD energy diagrammatically. In a Hartree-Fock single-particle basis, as is the plane wave basis in infinite nuclear matter, the reference energy $E_{0}$ equals the Hartree-Fock energy and the rest is by definition the correlation energy.

\begin{figure} 
  \centering
  \includegraphics[scale=1.0]{diagrams/diagram_cc_ene/ene_term-crop}
  \caption{A diagrammatic expression for the CCD energy (Figure 2 in Paper II). Reprinted by permission from \dots as given in Ref.~\cite{baardsen}.}
  \label{fig:ene_ccd}
\end{figure}

Let us define the Fock operator as
\begin{equation}
  \hat{F} = \sum_{p,q}\langle p|\hat{f}|q\rangle 
  a_{p}^{\dagger }a_{q},
\end{equation}
where 
\begin{equation}
  \langle p|\hat{f}|q\rangle = \langle p|\hat{h}_{0}|q\rangle 
  + \sum_{i}\langle pi|\hat{v}|qi\rangle_{AS}.
  \label{eq:fock_matr_ele}
\end{equation}
The explicit expression of the CC $\hat{T}_{2}$ amplitude equation is given in for example Ref.~\cite{crawford}. Restricting ourself to the CCD approximation and writing the equations in a momentum basis, we get the $\hat{T}_{2}$ amplitude equation

\begin{align} \label{eq:t2ampl}
  & \left\{ \langle \mathbf{k}_{i}|\hat{f}|\mathbf{k}_{i}\rangle +
  \langle \mathbf{k}_{j}|\hat{f}|\mathbf{k}_{j}\rangle -
  \langle \mathbf{k}_{a}|\hat{f}|\mathbf{k}_{a}\rangle -
  \langle \mathbf{k}_{b}|\hat{f}|\mathbf{k}_{b}\rangle
  \right\} \langle \mathbf{k}_{a}\mathbf{k}_{c}|\hat{t}|\mathbf{k}_{i}\mathbf{k}_{j}\rangle \nonumber \\
  &= \langle \mathbf{k}_{a}\mathbf{k}_{b}|\hat{v}|\mathbf{k}_{i}\mathbf{k}_{j}\rangle_{AS} \nonumber \\
  & + \frac{1}{2}\sum_{\mathbf{k}_{c},\mathbf{k}_{d}}\langle \mathbf{k}_{a}\mathbf{k}_{b}|\hat{v}|\mathbf{k}_{c}\mathbf{k}_{d}\rangle_{AS}\langle \mathbf{k}_{c}\mathbf{k}_{d}|\hat{t}|\mathbf{k}_{i}\mathbf{k}_{j}\rangle \nonumber \\
  & + \frac{1}{2}\sum_{\mathbf{k}_{k},\mathbf{k}_{l}} \langle \mathbf{k}_{a}\mathbf{k}_{b}|\hat{t}|\mathbf{k}_{k}\mathbf{k}_{l}\rangle \left\{ \langle \mathbf{k}_{k}\mathbf{k}_{l}|\hat{v}|\mathbf{k}_{i}\mathbf{k}_{j}\rangle_{AS} 
+ \frac{1}{2}\sum_{\mathbf{k}_{c},\mathbf{k}_{d}} \langle \mathbf{k}_{k}\mathbf{k}_{l}|\hat{v}|\mathbf{k}_{c}\mathbf{k}_{d}\rangle_{AS} \langle \mathbf{k}_{c}\mathbf{k}_{d}|\hat{t}|\mathbf{k}_{i}\mathbf{k}_{j}\rangle \right\} \nonumber \\ %\displaybreak \\
  & + \hat{P}(\mathbf{k}_{i}\mathbf{k}_{j})\hat{P}(\mathbf{k}_{a}\mathbf{k}_{b})\sum_{\mathbf{k}_{k},\mathbf{k}_{c}}\langle \mathbf{k}_{a}\mathbf{k}_{c}|\hat{t}|\mathbf{k}_{i}\mathbf{k}_{k}\rangle \nonumber \\
    & \times \left\{ \langle \mathbf{k}_{k}\mathbf{k}_{b}|\hat{v}|\mathbf{k}_{c}\mathbf{k}_{j}\rangle_{AS} + \frac{1}{2} \sum_{\mathbf{k}_{l},\mathbf{k}_{d}}
\langle \mathbf{k}_{k}\mathbf{k}_{l}|\hat{v}|\mathbf{k}_{c}\mathbf{k}_{d}\rangle_{AS}
\langle \mathbf{k}_{d}\mathbf{k}_{b}|\hat{t}|\mathbf{k}_{l}\mathbf{k}_{j}\rangle
\right\} \nonumber \\
& - \frac{1}{2}\hat{P}(\mathbf{k}_{i}\mathbf{k}_{j})\sum_{\mathbf{k}_{k}} 
\langle \mathbf{k}_{a}\mathbf{k}_{b}|\hat{t}|\mathbf{k}_{i}\mathbf{k}_{k}\rangle
\left\{ \sum_{\mathbf{k}_{l}}\sum_{\mathbf{k}_{c},\mathbf{k}_{d}} 
\langle \mathbf{k}_{k}\mathbf{k}_{l}|\hat{v}|\mathbf{k}_{c}\mathbf{k}_{d}\rangle_{AS}
\langle \mathbf{k}_{c}\mathbf{k}_{d}|\hat{t}|\mathbf{k}_{j}\mathbf{k}_{l}\rangle
\right\} \nonumber \\
& - \frac{1}{2}\hat{P}(\mathbf{k}_{a}\mathbf{k}_{b})\sum_{\mathbf{k}_{c}}
\langle \mathbf{k}_{a}\mathbf{k}_{c}|\hat{t}|\mathbf{k}_{i}\mathbf{k}_{j}\rangle
\left\{ \sum_{\mathbf{k}_{k},\mathbf{k}_{l}}\sum_{\mathbf{k}_{d}}
\langle \mathbf{k}_{k}\mathbf{k}_{l}|\hat{v}|\mathbf{k}_{c}\mathbf{k}_{d}\rangle_{AS}
\langle \mathbf{k}_{b}\mathbf{k}_{d}|\hat{t}|\mathbf{k}_{k}\mathbf{k}_{l}\rangle
\right\} ,
  \end{align}
where the permutation operator $\hat{P}$ is defined such that
\begin{equation}
  \hat{P}(x, y)\eta(x, y) = \eta(x, y) - \eta(y, x),
  \label{eq:perm}
\end{equation}
given an arbitrary function $\eta(x, y)$. In Eq.~(\ref{eq:t2ampl}) we have 
assumed that the Fock operator is diagonal, as it is in homogeneous matter
when the interaction conserves total momentum. The CC amplitude equations
are commonly factorized to reduce the computational cost \cite{crawford}.
Let us denote the expressions in brackets on right-hand side of 
Eq.~(\ref{eq:t2ampl}) by
\[
\langle \mathbf{k}_{k}\mathbf{k}_{l}|I_{1}|\mathbf{k}_{i}\mathbf{k}_{j}\rangle ,
\langle \mathbf{k}_{k}\mathbf{k}_{b}|I_{2}|\mathbf{k}_{c}\mathbf{k}_{j}\rangle ,
\langle \mathbf{k}_{k}|I_{3}|\mathbf{k}_{j}\rangle , 
\langle \mathbf{k}_{b}|I_{4}|\mathbf{k}_{c} \rangle ,
\]
where the intermediate terms are given in the same order as in the amplitude 
equation above. As was suggested first by Scuseria 
\emph{et al.}~\cite{scuseria_1988}, the computational 
cost is significantly reduced by computing the intermediates separately
before substituting them into the amplitude equation. For example,
the computational cost of evaluating the last term may be reduced from
the order $\sim n_{h}^{4}n_{p}^{4}$ to $\sim n_{h}^{3}n_{p}^{2}$, where $n_{h}$ 
and $n_{p}$ are the number of occupied and unoccupied orbitals, respectively. 
Typically, $n_{p}$ is much larger than $n_{h}$, and the scaling in terms 
of $n_{p}$ is therefore the dominant contribution to the computational cost.  
The CCD $\hat{T}_{2}$ amplitude equation is given diagrammatically in 
Fig.~\ref{fig:ampl_cc}. 

Let us for a moment consider only the terms that are linear in the 
$t$ amplitude on right-hand side of Eq.~(\ref{eq:t2ampl}). The terms 
on the second, third and fifth to sixth row on right-hand side  
have summations over two particle states, two hole states and
one hole and one particle state, respectively. We call the term with
summation over two particle states the particle-particle ladder
contribution. 
Similarly, the linear term with summation over two hole
states is called the hole-hole ladder contribution. The term with 
summation over one particle and one hole state gives particle-hole
ring contributions. The particle-particle and hole-hole ladder 
approximation (PPHH-LAD) is obtained from the CCD equations by setting 
the second term of $I_{1}$, as well as $I_{2}$, $I_{3}$, and $I_{4}$ to zero. 
The particle-particle ladder approximation (PP-LAD) is otherwise like 
PPHH-LAD, but in PP-LAD the entire intermediate $I_{1}$ is neglected. 
Explicit CC ladder equations are given in for example Ref.~\cite{baardsen}.

\begin{figure} 
  \centering
  \includegraphics[scale=1.0]{diagrams/diagram_cc_ampl/t_amplitude_all2-crop}
  \caption{The CCD $\hat{T}_{2}$ amplitude equation (Figure 1 in Paper II). Reprinted by permission from \dots given in a diagrammatic representation \cite{baardsen}.}
  \label{fig:ampl_cc}
\end{figure}

As explained in Refs.~\cite{bishop1978,baardsen}, the PP-LAD approximation 
generates short-range correlation contributions in a similar way as in 
the Brueckner $G$-matrix equation. In the PPHH-LAD approximation, 
particle-particle and hole-hole ladder diagrams are summed to infinite order, 
similarly as for example in the ladder approximation of the self-consistent 
Green's function (SCGF) method \cite{dickhoff2004}. The CC ladder 
approximations differ from the BHF and SCGF methods in that the 
single-particle potentials are not solved self-consistently,
but instead include only the Hartree-Fock contribution. As shown in
Refs.~\cite{freeman1977,bishop1978}, the linear particle-hole term
together with the interaction term generate the direct and exchange 
parts of ring diagrams with forward time-order. Bishop and L{\"u}hrmann 
\cite{bishop1978} explain thoroughly how the full CCD approximation contains 
ladder diagrams, the random phase approximation (RPA), and a large number
of other diagrams to infinite order. Observe that in CC theory, the different
contributions also couple to each other, thereby generating additional 
diagrams. 


%The particle-hole 
%ring diagrams are known to account for long-range correlation contributions \cite{}.
 
%In the following sections, we will derive explicit CC equations in the doubles approximation for one- and two-dimensional nuclear matter. In addition, we present an approximation of CCD for infinite nuclear matter in three dimensions.   

In the following sections, we discuss applications of the CC 
method for nuclear matter and the electron gas.



%% \subsection{CCSD approximation in one dimension}
%% As explained in Sec. \ref{sec:box}, we use a one-dimensional potential well to model the one-dimensional nuclear matter system. In one dimension, the single-particle eigenvectors are 
%% \begin{equation}
%%   \phi_{n}(x) = \frac{1}{(L)^{1/2}}e^{ik_{n}x},  
%% \end{equation}
%% where the discrete state vector $k_{n}$ is 
%% \begin{equation}
%%   k_{n} = \frac{2\pi }{L}n, \qquad n = 0, \pm 1, \pm 2, \dots .
%% \end{equation}
%% The number of particles is obtained as the sum over occupied states, which by definition are all states with momentum $k_{n}$ smaller than or equal to the Fermi momentum $k_{F}$. Using the transformation of sums given in Eq. \ref{eq:sum2int}, the density of a spin- and isospinless one-dimensional nuclear matter system can be calculated as 
%% \begin{equation}
%%   \rho \equiv \frac{A}{L} = \frac{1}{L}\sum_{n\atop |k_{n}|\leq k_{F}} \longrightarrow \frac{1}{2\pi }\int_{-k_{F}}^{k_{F}}dk = \frac{k_{F}}{\pi },
%% \end{equation}
%% where $A$ is the number of nucleons.


%% %% As a model for the one-dimensional nuclear matter system, we use a simple potential well which is filled by interacting nucleons. Figure \ref{fig:box_1d} illustrates how nucleons are restricted to a finite region of the $x$ axis by assuming that the external potential is infinite outside the box and zero in the interval $[-L/2, L/2]$. The corresponding single-particle system is a standard textbook example, which is explained by for example Liboff \cite{liboff}. Inside the one-dimensional box, the Hamiltonian equation of the single nucleon  is 
%% %% \begin{equation}
%% %%   -\frac{\hbar^{2}}{2m}\frac{\partial^{2}}{\partial x^{2}}\phi_{p}(x) = \varepsilon_{p}\phi_{p}(x),
%% %% \end{equation}
%% %% where $\varepsilon_{p}$ is the energy. This equation has solutions of the form 
%% %% \begin{equation}
%% %% \phi_{p}(x) = \frac{1}{\sqrt{L}}e^{ik_{p}x},
%% %% \end{equation}
%% %% where $k_{p} = \sqrt{2m\varepsilon_{p}/\hbar^{2}}$. Using the periodic boundary condition
%% %% \begin{equation}
%% %%   \phi_{p}(x+L) = \phi_{p}(x),
%% %% \end{equation}
%% %% it follows that 
%% %% \begin{equation}
%% %%   k_{p} = \frac{2\pi }{L}p, \qquad p = 0, \pm 1, \pm 2, \dots .
%% %% \end{equation}

%% %% \begin{figure} \label{fig:box_1d}
%% %%   \centering
%% %%   \includegraphics[scale=1.0]{figures/box_1d/box_1d-crop.pdf}
%% %%   \caption{The nucleons are restricted to a one-dimensional potential well. Here $U(x)$ is the external potential.}
%% %% \end{figure}

%% %% In one dimension, sums are transformed to integrals as
%% %% \begin{equation}
%% %%   \sum_{p} \longrightarrow \frac{L}{2\pi }\int_{-k_{F}}^{k_{F}}dk
%% %% \end{equation}
%% %% when $L$ approaches infinity. Using this transformation, we get for the density the expression
%% %% \begin{equation}
%% %%   \rho \equiv \frac{A}{L} = \frac{1}{L}\sum_{k_{p}} \longrightarrow \frac{1}{2\pi }\int_{-k_{F}}^{k_{F}}dk = \frac{k_{F}}{\pi }.
%% %% \end{equation}

%% Let us in the following use indices $i, j, k, \dots $ and $a, b, c, \dots $ to indicate occupied and unoccupied states, respectively. The kinetic energy per particle is for the one-dimensional nuclear matter system
%% \begin{equation}
%%   \frac{\Delta E_{kin}}{A} = \frac{1}{A}\sum_{i}\frac{k_{i}^{2}}{2m} \longrightarrow \frac{k_{F}^{2}}{6m},
%% \end{equation}
%% where the summation over occupied states $i$ means a summation over states with $|k_{i}| \leq k_{F}$. The Hartree-Fock potential contribution is 
%% \begin{align}
%%   \frac{\Delta E_{HF}}{A} &= \frac{1}{2A}\sum_{ij}\langle k_{i}k_{j}|v|k_{i}k_{j}\rangle_{AS} \nonumber \\
%%   &\longrightarrow \frac{1}{2A}\left( \frac{L}{2\pi }\right)^{2}\int dk_{i}\int dk_{j} \langle k_{i}k_{j}|v|k_{i}k_{j}\rangle_{AS} \nonumber \\
%%   &\times \theta\left( k_{F}-|k_{i}|\right)\theta\left( k_{F}-|k_{j}|\right),
%% \end{align}
%% where $\theta(x)$ is the Heavyside step function.

%% Let us next define the relative and Centre-of-Mass (CM) coordinates $r$ and $R$ such that
%% \begin{equation}
%%   r = x_{1} - x_{2} ; \qquad R = \frac{1}{2}(x_{1} + x_{2}),
%% \end{equation}
%% where $x_{1}$ and $x_{2}$ are the coordinates of particles 1 and 2, respectively. Correspondingly, the relative and CM momenta $k$ and $K$ are
%% \begin{equation}
%%   k = \frac{1}{2}(k_{1} - k_{2}) ; \qquad K = k_{1} + k_{2},
%% \end{equation}
%% where $k_{1}$ and $k_{2}$ are the momenta of particles 1 and 2, given in the laboratory coordinate system.

%% In laboratory coordinates, the two-particle interaction matrix elements are transformed from coordinate to momentum representation using the transformation 
%% \begin{align}
%%   \langle k_{1}k_{2}|v|k_{3}k_{4} \rangle &= \frac{1}{L^{2}}\int_{-L/2}^{L/2}dx_{1}\int_{-L/2}^{L/2}dx_{2} \nonumber \\
%%   &\times e^{-ik_{1}x_{1}}e^{-ik_{2}x_{2}}v(x_{1}, x_{2})e^{ik_{3}x_{1}}e^{ik_{4}x_{2}}.
%% \end{align}
%% Here we have used the momentum single-particle basis, which gives the projection
%% \begin{equation}
%%   \langle x|k\rangle = \phi_{k}(x) = \frac{1}{\sqrt{L}}e^{ikx},  
%% \end{equation}
%% and we have assumed that the interaction operator $\hat{v}$ conserves the positions of particles, that is,
%% \begin{equation}
%%   \langle x_{1}x_{2}|\hat{v}|x_{3}x_{4}\rangle = v(x_{1}, x_{2})\delta(x_{1}-x_{3})\delta(x_{2}-x_{4}),
%% \end{equation}
%% where $\delta(x)$ is the Dirac delta distribution. 

%% Using RCM coordinates, the Fourier transform of the two-particle interaction can be written as
%% \begin{align}
%%   \langle kK|v|k'K'\rangle &= \frac{1}{L^{2}}\int_{-L}^{L}dr e^{-i(k-k')r}v(r) \nonumber \\
%%   &\times \int_{-\frac{1}{2}(L-|r|)}^{\frac{1}{2}(L-|r|)}dR e^{i(K-K')R}.
%% \end{align}   
%% Assuming that the interaction range is short compared to the length of the box, $L$, we can use the definition
%% \begin{equation}
%%   \delta(K-K') = \frac{1}{2\pi }\int_{-\infty }^{\infty }dR e^{i(K-K')R}
%% \end{equation}
%% and obtain the expression 
%% \begin{equation} \label{eq:vkk_rcm}
%%   \langle kK|v|k'K'\rangle = \lim_{L\rightarrow \infty } \frac{2\pi }{L^{2}}\int_{-L}^{L}e^{i(k-k')r}v(r)\delta(K-K')
%% \end{equation}
%% for interaction matrix elements in RCM momentum coordinates. The corresponding exchange term is 
%%  \begin{equation} \label{eq:vkk_rcm}
%%   \langle kK|v|-k'K'\rangle = \lim_{L\rightarrow \infty } \frac{2\pi }{L^{2}}\int_{-L}^{L}e^{i(k+k')r}v(r)\delta(K-K').
%% \end{equation}
%% Above we have assumed that the two-particle interaction is invariant under translations, which means that the interaction depends only on the distance $r = x_{1}-x_{2}$, and not on the specific positions of the two particles.

%% Expressed in RCM coordinates, the HF potential contribution to the binding energy becomes
%% \begin{align}
%%   \frac{\Delta E_{HF}}{A} &= \frac{1}{8\pi k_{F}}\int_{-k_{F}}^{k_{F}}dk\int_{-2(k_{F}-|k|)}^{2(k_{F}-|k|)}dK \nonumber \\
%%   &\times \left\{ \langle k|v|k\rangle - \langle k|v|-k\rangle \right\},
%% \end{align}  
%% where the direct and exchange terms have been written explicitly.

%% A general expression of the CCD correlation energy is given in Eq.~(\ref{eq:ene_ccd}). In one dimension, the CCD correlation energy per particle becomes 
%% \begin{align}
%%   \frac{\Delta E_{CCD}}{A} &= \frac{1}{4A}\sum_{ijab}\langle k_{i}k_{j}|v|k_{a}k_{b}\rangle_{AS} \langle k_{a}k_{b}|t|k_{i}k_{j}\rangle_{AS} \nonumber \\
%%   &\longrightarrow \frac{1}{4A}\left( \frac{L}{2\pi }\right)^{4}\int dk_{i}\int dk_{j}\int dk_{a} \int dk_{b} \nonumber \\
%%   &\times \langle k_{i}k_{j}|v|k_{a}k_{b}\rangle_{AS} \langle k_{a}k_{b}|t|k_{i}k_{j}\rangle_{AS} \nonumber \\
%%   &\times \theta(k_{F}-|k_{i}|)\theta(k_{F}-|k_{j}|)\theta(|k_{a}|-k_{F})\theta(|k_{b}|-k_{F}).
%% \end{align}
%% As can be seen from the Heaviside step functions, $i$ and $j$ denote sinlge-particle states that are occupied in the Fermi vacuum Slater determinant, whereas $a$ and $b$ stand for excited states. Using RCM coordinates and explicit integration limits, the CCD correlation contribution becomes
%% \begin{align}
%%   \frac{\Delta E_{CCD}}{A} &= \frac{1}{32\pi^{2}k_{F}}\int_{-2k_{F}}^{2k_{F}}dK\int_{-\left(-\frac{1}{2}|K|+k_{F}\right)}^{-\frac{1}{2}|K|+k_{F}}dk \nonumber \\
%%   &\times \left\{ \int_{-\infty }^{-\left( \frac{1}{2}|K|+k_{F}\right)}dk' + \int_{\frac{1}{2}|K|+k_{F}}^{\infty }dk' \right\} \nonumber \\
%%   &\times \langle k|v|k'\rangle_{AS}\langle k'|t(K)|k\rangle_{AS}.
%% \end{align} 
%% Observe that the coupled-cluster $t_{ij}^{ab}$-amplitude has an explicit dependency on the Centre-of-Mass momentum K. 

%% The $T_{2}$ amplitude equation (\ref{eq:t2ampl}) can be written in one-dimensional RCM momentum coordinates as
%% \begin{align}
%%   0 &= \langle k'|\hat{v}|k\rangle_{AS} + \left( -\varepsilon\left( k+K/2\right) - \varepsilon\left( -k + K/2\right)\right. \nonumber \\
%%   & \left. + \varepsilon\left( k' + K/2\right) + \varepsilon\left( -k' + K/2\right) \right) \langle k'|\hat{t}(K)|k\rangle \nonumber \\
%%   & + \frac{1}{2}\sum_{h}\langle k'|\hat{t}(K)|h\rangle \left( \langle h|\hat{v}|k\rangle_{AS} + \sum_{p}\langle h|\hat{v}|p\rangle_{AS}\langle p|\hat{t}(K)|k\rangle \right) \nonumber \\
%%   & + \frac{1}{2}\sum_{p}\langle k'|\hat{v}|p\rangle_{AS}\langle p|\hat{t}(K)|k\rangle \nonumber \\
%%   & + \hat{P}(ij)\hat{P}(ab)\sum_{k_{k}}\langle p_{1}|\hat{t}(K)|p_{2}\rangle \nonumber \\
%%   &\times \left( \langle p_{3}|\hat{v}|p_{4}\rangle_{AS} + \frac{1}{2}\sum_{k_{l}}\langle p_{5}|\hat{t}(K)|p_{6}\rangle \langle p_{7}|\hat{v}|p_{8}\rangle_{AS} \right) \nonumber \\
%%   & - \hat{P}(ab)\frac{1}{2}\sum_{k_{c}}\langle p_{9}|\hat{t}(K)|k\rangle \left( \sum_{h}\langle p_{10}|\hat{t}(K)|h\rangle \langle h|\hat{v}|p_{11}\rangle_{AS} \right) \nonumber \\
%%   & - \hat{P}(ij)\frac{1}{2}\sum_{k_{k}}\langle k'|\hat{t}(K)|p_{12}\rangle \left( \sum_{p}\langle p_{13}|\hat{v}|p\rangle_{AS} \langle p|\hat{t}(K)|p_{14}\rangle \right), 
%% \end{align}
%% where $h$ and $p$ are laboratory momenta with the restrictions $|h| \leq k_{F}$ and $|p| > k_{F}$. In the expression above, we have also used the definitions
%% \begin{align}
%%   p_{1} &= (k_{a}-k_{c})/2, \nonumber \\
%%   p_{2} &= (k_{i}-k_{k})/2, \nonumber \\
%%   p_{3} &= (k_{k}-k_{b})/2, \nonumber \\
%%   p_{4} &= (k_{c}-k_{j})/2, \qquad k_{c} = k_{i} + k_{k} - k_{a}, \nonumber \\
%%   p_{5} &= (k_{d}-k_{b})/2, \qquad k_{d} = k_{l} + k_{j} - k_{b}, \nonumber \\
%%   p_{6} &= (k_{l}-k_{j})/2, \nonumber \\
%%   p_{7} &= (k_{k}-k_{l})/2, \nonumber \\
%%   p_{8} &= (k_{c}-k_{d})/2, \nonumber \\
%%   p_{9} &= (k_{a}-k_{c})/2 = p_{1}, \nonumber \\
%%   p_{10} &= (k_{b}-k_{d})/2, \nonumber \\
%%   p_{11} &= (k_{c}-k_{d})/2 = p_{8}, \nonumber \\
%%   p_{12} &= (k_{i}-k_{k})/2, \nonumber \\
%%   p_{13} &= (k_{k}-k_{l})/2, \qquad k_{l} = k_{c} + k_{d} - k_{k} = K - k_{k}, \nonumber \\
%%   p_{14} &= (k_{j}-k_{l})/2.
%% \end{align}
%% The antisymmetrisation operator $\hat{P}(x, y)$ is as defined in Eq. \ref{eq:perm}. Using integrals and explicit integration limits, the $T_{2}$ amplitude equation for one-dimensional nuclear matter becomes
%% \begin{align}
%%   &\left( \varepsilon\left( k + K/2\right) + \varepsilon\left( -k + K/2\right) - \varepsilon\left( k' + K/2\right) - \varepsilon\left( -k' + K/2\right) \right) \nonumber \\
%%   &\times \langle k'|t(K)|k\rangle \nonumber \\
%%   & = \langle k'|v|k\rangle_{AS} \nonumber \\
%%   & + \frac{L}{4\pi }\int_{-k_{max, h}}^{k_{max, h}}dh \langle k'|t(K)|h\rangle \bigg( \langle h|v|k\rangle_{AS}  \nonumber \\
%%   & + \left. \frac{L}{2\pi }\left[ \int_{-\infty }^{-k_{min,p}}dp + \int_{k_{min,p}}^{\infty }dp \right] \langle h|v|p\rangle_{AS}\langle p|t(K)|k\rangle \right) \nonumber \\
%%   & + \frac{L}{4\pi }\left[ \int_{-\infty }^{-k_{min,p}}dp + \int_{k_{min,p}}^{\infty }dp \right] \langle k'|v|p\rangle_{AS} \langle p|t(K)|k\rangle \nonumber \\
%%   & + P(ij)P(ab)\frac{L}{2\pi }\int_{-k_{F}}^{k_{F}}dk_{k}\langle p_{1}|t(K)|p_{2}\rangle \nonumber \\
%%   & \times \left( \langle p_{3}|v|p_{4}\rangle_{AS} + \frac{L}{4\pi }\int_{-k_{F}}^{k_{F}}dk_{l}\langle p_{5}|t(K)|p_{6}\rangle \langle p_{7}|v|p_{8}\rangle_{AS} \right) \nonumber \\
%%   & - P(ab)\frac{L}{4\pi }\left[ \int_{-\infty }^{k_{F}}dk_{c} + \int_{k_{F}}^{\infty }dk_{c} \right] \langle p_{9}|t(K)|k\rangle \nonumber \\
%%   & \times \left( \frac{L}{2\pi }\int_{-k_{max,h}}^{k_{max,h}}dh \langle p_{10}|t(K)|h\rangle \langle h|v|p_{11}\rangle_{AS} \right) \nonumber \\
%%   & - P(ij)\frac{L}{4\pi }\int_{-k_{F}}^{k_{F}}dk_{k}\langle k'|t(K)|p_{12}\rangle \nonumber \\
%%   & \times \left( \frac{L}{2\pi }\left[ \int_{-\infty }^{-k_{min,p}}dp + \int_{k_{min,p}}^{\infty }dp \right]\langle p_{13}|v|p\rangle_{AS} \langle p|t(K)|p_{14}\rangle \right), 
%% \end{align}
%% where the explicit expressions of the integration limits $k_{max,h}$ and $k_{min,p}$ are 
%% \begin{equation}
%%   k_{max,h} = -|K|/2 + k_{F}, \qquad k_{min,p} = |K|/2 + k_{F}.
%% \end{equation}
%% The single-particle energy $\varepsilon(p)$ is defined as
%% \begin{equation}
%%   \varepsilon(p) \equiv \langle p|\hat{f}|p\rangle ,
%% \end{equation}
%% where $\hat{f}$ is the Fock operator of Eq.~(\ref{eq:fock_matr_ele}).

%% The single-particle energy can be written in a momentum basis as
%% \begin{align}
%%   \varepsilon(k_{i}) &= \frac{k_{i}^{2}}{2m} + \sum_{j}\langle k_{i}k_{j}|v|k_{i}k_{j}\rangle_{AS} \nonumber \\
%%   & \longrightarrow \frac{k_{i}^{2}}{2m} + \frac{L}{2\pi }\int_{-k_{F}}^{k_{F}}dk' \langle k_{i}k'|v|k_{i}k'\rangle_{AS},  
%% \end{align}
%% where the two-particle interaction has laboratory frame momenta as inputs.

%% \subsection{CCSD approximation in two dimensions}
%% Consider a two-dimensional rectangle in coordinate space with sides $L_{x}$ and $L_{y}$, as illustrated at left in Fig.~\ref{fig:fourier_2d}. Doing a Fourier transform, this finite domain in coordinate space results in a discrete momentum grid (Fourier grid)
%% \begin{equation}
%%   \mathbf{k}_{n_{x},n_{y}} = \frac{2\pi }{L_{x}}n_{x}\mathbf{u}_{x} + \frac{2\pi }{L_{y}}n_{y}\mathbf{u}_{y},
%% \end{equation}
%% where $n_{i} = 0, \pm 1, \pm 2, \dots $ for $i \in \{x, y\}$ and $\mathbf{u}_{x}$ and $\mathbf{u}_{y}$ are the unit vectors of the $k_{x}$ and $k_{y}$ axes. As seen from Fig.~\ref{fig:fourier_2d}, the finite and continuous domain in coordinate space is transformed to a discrete but infinite domain in momentum space. In the limit when $L_{x}$ and $L_{y}$ go towards infinity, the Fourier space domain approaches the entire two-dimensional real momentum space.  



%% In a two-dimensional space, the orbital angular momentum operator $\hat{L}_{z}$ is 
%% \begin{equation}
%%   \hat{L}_{z} = -i\hbar \frac{\partial }{\partial \phi },
%% \end{equation}
%% where $\phi $ is the angular coordinate in a chosen polar coordinate system, as illustrated in Fig.~\ref{fig:polar}. This operator has eigenfunctions of the form $\psi_{m}(\phi ) = Ce^{im\phi }$, where $C$ and $m$ are constants. The corresponding eigenvalues are $m\hbar $. From the uniqueness condition
%% \begin{equation}
%%   \psi_{m}(\phi + 2\pi ) = \psi_{m}(\phi ),
%% \end{equation}    
%% it follows that $m = 0, \pm 1, \pm 2, \dots $. Thus, we get the angular wave functions
%% \begin{equation}
%%   \psi_{m}(\phi ) = \frac{1}{\sqrt{2\pi }}e^{im\phi }, \qquad m = 0, \pm 1, \pm 2, \dots ,
%% \end{equation}
%% where the constant $C$ has been chosen such that the basis is orthonormal. One can easily show that this function is also an eigenfunction of the operator $\hat{L}^{2}$ with the eigenvalue $-m^{2}\hbar^{2}$, and therefore there are no other angular symmetries. 

%% \begin{figure} 
%%   \centering
%%   \includegraphics[scale=1.0]{figures/polar/polar-crop}
%%   \caption{Polar coordinates in position and momentum space. (Illustration by the author.)}
%%   \label{fig:polar}
%% \end{figure}


%% Here we choose to derive the coupled-cluster equations using the single-particle basis $|km\rangle $, where $k$ is the radial momentum coordinate and $m$ is an eigenvalue of $\hat{L}_{z}$. To be able to calculate with larger systems, spin and isospin degrees of freedom are neglected. As most other quantum many-body methods, the coupled-cluster formalism is normally defined in laboratory coordinates \cite{ harris,crawford,kummel1978}. The equations are simplest in a laboratory coordinate representation, and we will therefore derive CC equations for a two-dimensional system in laboratory coordinates. 

%% In Eq.~(\ref{eq:ene_ref}), an expression was given for the reference energy in a general momentum space. For an $n$-dimensional space, a sum over momenta can generally be approximated by an integral according to 
%% \begin{equation}
%%   \sum_{\mathbf{k}} \longrightarrow \left(\frac{L}{2\pi }\right)^{n}\int d\mathbf{k}
%% \end{equation}
%% in the limit when the side length $L$ approaches infinity. In two dimensions, the total number of particles without spin and isospin is 
%% \begin{equation}
%%   A = \left( \frac{L}{2\pi }\right)^{2}\int_{|\mathbf{k}|\leq k_{F}}d\mathbf{k},
%% \end{equation} 
%% which gives the density of particles
%% \begin{equation}
%%   \rho = \frac{L^{2}}{A} = \frac{2\pi }{k_{F}^{2}}.
%% \end{equation}
%% At the limit $L \rightarrow \infty $, the kinetic energy per particle becomes
%% \begin{align}
%%   \frac{E_{kin}}{A} &= \frac{1}{A}\sum_{|\mathbf{k}_{i}|\leq k_{F}}\frac{k_{i}^{2}}{2m} \nonumber \\
%%   & \longrightarrow \frac{1}{A}\left( \frac{L}{2\pi }\right)^{2}\int_{0}^{k_{F}}dk k\int_{0}^{2\pi }d\phi \frac{k^{2}}{2m} \nonumber \\
%%   & = \frac{k_{F}^{2}}{4m}.
%% \end{align} 
%% The Hartree-Fock potential energy per particle is
%% \begin{align}
%%   \frac{\Delta E_{HF}}{A} &= \frac{1}{2A}\sum_{|\mathbf{k}_{i}|\leq k_{F}}\sum_{|\mathbf{k}_{j}|\leq k_{F}}\langle \mathbf{k}_{i}\mathbf{k}_{j}|v|\mathbf{k}_{i}\mathbf{k}_{j}\rangle_{AS} \nonumber \\
%%   & \longrightarrow \frac{1}{2A}\left( \frac{L}{2\pi }\right)^{4}\int d\mathbf{k}_{i}\int d\mathbf{k}_{j}\langle \mathbf{k}_{i}\mathbf{k}_{j}|v|\mathbf{k}_{i}\mathbf{k}_{j}\rangle_{AS} \nonumber \\
%%   & \times \theta(k_{F}-|\mathbf{k}_{i}|)\theta(k_{F}-|\mathbf{k}_{j}|).
%% \end{align}
%% Using the relations 
%% \begin{align}
%%   \sum_{m}|m\rangle \langle m| &= \hat{\mathbb{1}}, \\
%%   \langle \phi|m \rangle &= \psi_{m}(\phi ), \\
%%   \int_{0}^{2\pi }d\phi \psi_{m}^{*}(\phi )\psi_{m'}(\phi ) &= \delta_{m,m'},
%% \end{align}
%% where $\hat{\mathbb{1}}$ is the unity operator, $\langle \phi|m \rangle $ is the projection of the state $|m\rangle $ to the state $| \phi \rangle $, and the star denotes the complex conjugate, we get as an expression for the HF potential per particle
%% \begin{align}
%%   \frac{\Delta E_{HF}}{A} &= \frac{L^{2}}{8\pi^{3}k_{F}^{2}}\sum_{m_{i}}\sum_{m_{j}}\int_{0}^{k_{F}}dk_{i} k_{i}\int_{0}^{k_{F}}dk_{j} k_{j} \nonumber \\
%%   & \times \langle k_{i}m_{i}k_{j}m_{j}|v|k_{i}m_{i}k_{j}m_{j}\rangle_{AS}. 
%% \end{align}
%% In the coupled-cluster doubles approximation, the correlation energy per particle is
%% \begin{align}
%%   \frac{\Delta E_{CCD}}{A} &= \frac{1}{4A}\sum_{|\mathbf{k}_{i}|\leq k_{F}}\sum_{|\mathbf{k}_{j}|\leq k_{F}}\sum_{|\mathbf{k}_{a}|>k_{F}}\sum_{|\mathbf{k}_{b}|>k_{F}} \nonumber \\
%%   & \times \langle \mathbf{k}_{i}\mathbf{k}_{j}|v|\mathbf{k}_{a}\mathbf{k}_{b}\rangle_{AS}\langle \mathbf{k}_{a}\mathbf{k}_{b}|t|\mathbf{k}_{i}\mathbf{k}_{j}\rangle .
%% \end{align}
%% This expression can be written in terms of the basis $|km\rangle $ similarly as the Hartree-Fock potential, and we get
%% \begin{align}
%%   \frac{\Delta E_{CCD}}{A} &= \frac{1}{4A}\left( \frac{L}{2\pi }\right)^{8}\sum_{m_{i}}\sum_{m_{j}}\sum_{m_{a}}\sum_{m_{b}}  \nonumber \\
%%   & \times \int_{0}^{k_{F}}dk_{i}k_{i} \int_{0}^{k_{F}}dk_{j} k_{j}\int_{k_{F}}^{\infty }dk_{a}k_{a}\int_{k_{F}}^{\infty }dk_{b}k_{b} \nonumber \\
%%   & \times \langle k_{i}m_{i}k_{j}m_{j}|v|k_{a}m_{a}k_{b}m_{b}\rangle_{AS} \nonumber \\
%%   & \times \langle k_{a}m_{a}k_{b}m_{b}|t|k_{i}m_{i}k_{j}m_{j}\rangle ,
%% \end{align}
%% where the $t$-amplitude is implicitly antisymmetric. In this angular momentum basis, the $T_{2}$ amplitude equation is
%% \begin{align}
%%   & 0 = \langle k_{a}m_{a}k_{b}|v|k_{i}m_{i}k_{j}m_{j}\rangle_{AS} \nonumber \\
%%   & - \left( \langle k_{i}m_{i}|f|k_{i}m_{i}\rangle + \langle k_{j}m_{j}|f|k_{j}m_{j}\rangle \right. \nonumber \\
%%   & \left. - \langle k_{a}m_{a}|f|k_{a}m_{a}\rangle - \langle k_{j}m_{j}|f|k_{j}m_{j}\rangle \right)\langle k_{a}m_{a}k_{b}m_{b}|t|k_{i}m_{i}k_{j}m_{j}\rangle \nonumber \\
%%   & + \frac{1}{2}\left( \frac{L}{2\pi }\right)^{4}\sum_{m_{k}}\sum_{m_{l}}\int_{0}^{k_{F}}dk_{k} k_{k}\int_{0}^{k_{F}}dk_{l}k_{l}\langle k_{a}m_{a}k_{b}m_{b}|t|k_{k}m_{k}k_{l}m_{l}\rangle \nonumber \\
%%   & \times \bigg( \langle k_{k}m_{k}k_{l}m_{l}|v|k_{i}m_{i}k_{j}m_{j}\rangle_{AS} \nonumber \\
%%   & + \left( \frac{L}{2\pi }\right)^{4}\sum_{m_{c}}\sum_{m_{d}}\int_{k_{F}}^{\infty }dk_{c}k_{c}\int_{k_{F}}^{\infty }dk_{d}k_{d}\langle k_{k}m_{k}k_{l}m_{l}|v|k_{c}m_{c}k_{d}m_{d}\rangle_{AS} \nonumber \\
%%   & \times \langle k_{c}m_{c}k_{d}m_{d}|t|k_{i}m_{i}k_{j}m_{j}\rangle \bigg) \nonumber \\
%%   & + \frac{1}{2}\left( \frac{L}{2\pi }\right)^{4}\sum_{m_{c}}\sum_{m_{d}}\int_{k_{F}}^{\infty }dk_{c}k_{c}\int_{k_{F}}^{\infty }dk_{d}k_{d}\langle k_{a}m_{a}k_{b}m_{b}|v|k_{c}m_{c}k_{d}m_{d}\rangle_{AS} \nonumber \\
%%   & \times \langle k_{c}m_{c}k_{d}m_{d}|t|k_{i}m_{i}k_{j}m_{j}\rangle \nonumber \\
%%   & + P(ij)P(ab)\left( \frac{L}{2\pi }\right)^{4}\sum_{m_{k}}\sum_{m_{c}}\int_{0}^{k_{F}}dk_{k}k_{k}\int_{k_{F}}^{\infty }dk_{c}k_{c}\langle k_{a}m_{a}k_{c}m_{c}|t|k_{i}m_{i}k_{k}m_{k}\rangle \nonumber \\
%% %  & \times \langle k_{a}m_{a}k_{c}m_{c}|t|k_{i}m_{i}k_{k}m_{k}\rangle \nonumber \\
%%   & \times \bigg( \langle k_{k}m_{k}k_{b}m_{b}|v|k_{c}m_{c}k_{j}m_{j}\rangle_{AS} + \frac{1}{2}\left( \frac{L}{2\pi }\right)^{4}\sum_{m_{l}}\sum_{m_{d}}\int_{0}^{k_{F}}dk_{l}k_{l}\int_{k_{F}}^{\infty }dk_{d}k_{d} \nonumber \\
%%   & \times \langle k_{d}m_{d}k_{b}m_{b}|t|k_{l}m_{l}k_{j}m_{j}\rangle \langle k_{k}m_{k}k_{l}m_{l}|v|k_{c}m_{c}k_{d}m_{d}\rangle_{AS} \bigg) \nonumber \\
%%   & - P(ab)\frac{1}{2}\left( \frac{L}{2\pi }\right)^{2}\sum_{c}\int_{k_{F}}^{\infty }dk_{c}k_{c}\langle k_{a}m_{a}k_{c}m_{c}|t|k_{i}m_{i}k_{j}m_{j}\rangle \nonumber \\
%%   & \times \Bigg( \left( \frac{L}{2\pi}\right)^{6}\sum_{m_{k}}\sum_{m_{l}}\sum_{m_{d}}\int_{0}^{k_{F}}dk_{k}k_{k}\int_{0}^{k_{F}}dk_{l}k_{l}\int_{k_{F}}^{\infty }dk_{d}k_{d} \nonumber \\
%%   & \times \langle k_{b}m_{b}k_{d}m_{d}|t|k_{k}m_{k}k_{l}m_{l}\rangle \langle k_{k}m_{k}k_{l}m_{l}|v|k_{c}m_{c}k_{d}m_{d}\rangle_{AS} \Bigg) \nonumber \\
%%   & - P(ij)\frac{1}{2}\left( \frac{L}{2\pi }\right)^{2}\sum_{m_{k}}\int_{0}^{k_{F}}dk_{k}k_{k}\langle k_{a}m_{a}k_{b}m_{b}|t|k_{i}m_{i}k_{k}m_{k}\rangle \nonumber \displaybreak  \\
%%   & \times \Bigg( \left( \frac{L}{2\pi }\right)^{6}\sum_{m_{l}}\sum_{m_{c}}\sum_{m_{d}}\int_{0}^{k_{F}}dk_{l}k_{l}\int_{k_{F}}^{\infty }dk_{c}k_{c}\int_{k_{F}}^{\infty }dk_{d}k_{d} \nonumber \\
%%   & \times \langle k_{k}m_{k}k_{l}m_{l}|v|k_{c}m_{c}k_{d}m_{d}\rangle_{AS}\langle k_{c}m_{c}k_{d}m_{d}|t|k_{j}m_{j}k_{l}m_{l}\rangle \Bigg). 
%% \end{align}

%\displaybreak


\section{Applications for nuclear matter}

%After having been invented within the nuclear physics community \cite{coester1958,coester1960}, coupled-cluster theory soon became popular in the field of quantum chemistry \cite{bartlett_review,cizek1966}.  


A long-standing problem in nuclear physics \cite{gandolfi2007,
hebeler2011,dickhoff2004,day1967,jackson1983,akmal1998,dalen2010,
baldo2012} 
has been to properly model the nuclear-matter equation of state, 
starting from a realistic nuclear interaction model. In this context,
an important quantity is the energy per nucleon as a function of the 
particle density. In this thesis, we assume that the temperature is
much smaller than the Fermi temperature, and model the interactions
between nucleons using nonrelativistic quantum mechanics. 
In most many-body quantum approaches, including the CC method, the energy 
at zero temperature is obtained from the basic formulation of the theory. 
In principle, it should therefore be possible to calculate the energy per 
particles using the CC approach, and thereby get essential input to the
nuclear-matter equation of state. 

More than thirty years ago, Day and Zabolitzky studied the nuclear-matter
equation of state \cite{day_cc} using the Bochum truncation 
\cite{kummel1978} of CC theory. Day and Zabolitzky did calculations 
including reduced two- and three-particle subsystem amplitudes, 
as well as an estimate of the four-particle amplitude. One aim of this 
thesis has been to take up again the work on coupled-cluster theory for nuclear
matter, now using the $n$-particle-$n$-hole truncation scheme 
explained in Sec.~\ref{sec:cct} together with modern nuclear 
interaction models. The latter truncation scheme has been successfully
applied to finite nuclei, and it is therefore of interest to
extend the method to the limit of infinitely many nucleons. In the
next subsection, we give an introduction to Ref.~\cite{baardsen},
where we have done CC calculations for nuclear matter at the
thermodynamic limit. The results of Paper II have also been
used to validate the CC method used in Paper III, where the
infinite system was modelled using finite-size boxes with 
twisted-angle boundary conditions.  


%% In infinite matter, the density is related to the Fermi momentum through 
%% Eq.~(\ref{eq:density_discr}). The same equation of state 
%% can therefore be rewritten as a relation between the Fermi momentum 
%% and the energy per particle. We are therefore interested in 


\subsection{Coupled-cluster ladder approximation in three dimensions}

As a first step towards new and improved CC calculations for nuclear 
matter, we have approximated the CCD amplitude equations (\ref{eq:t2ampl}) 
by including only particle-particle ladder diagrams, which are 
necessary to properly treat the short-range correlations between 
nucleons \cite{dickhoff2004,brueckner1954}, and hole-hole 
ladder diagrams, which constitute the symmetric counterpart 
for hole states. Our work has been described in detail in 
Ref.~\cite{baardsen}, and we here only give a brief introduction. 

In Chapter \ref{ch:mbpt}, we explained how the BHF equations are commonly
simplified using an angular-average approximation of the Pauli exclusion 
operators (\ref{eq:paulihh_rcm}) and (\ref{eq:paulipp_rcm}). The 
angular-average approximation, defined by Eqs.~(\ref{eq:pauli_ave_hh})
and (\ref{eq:pauli_ave_pp}), is obtained by replacing the exact Pauli 
operators with averages integrated over the angle between the relative 
and CM momentum vectors. This approximation makes the BHF equations 
considerably simpler, as the $G$ matrix becomes diagonal in the 
total angular momentum $\mathcal{J}$ and its projection $m_{\mathcal{J}}$ 
\cite{suzuki}. As shown in Refs.~\cite{suzuki,schiller},
the error in the BHF binding energy introduced by using angular-averaged 
Pauli exclusion operators is of the order $0.2-0.5$ MeV.

Suzuki \emph{et al.}~\cite{suzuki} have suggested an approach to
expand the exact Pauli operators of the BHF equations in partial waves. 
In Ref.~\cite{baardsen}, we derive explicit expressions for the CC 
ladder equations using the same technique. When using exact Pauli operators,
the CC ladder equations become complex, and storage of the amplitude matrix 
requires a huge amount of memory. To make the computations manageable,
we simplify the amplitude matrix by inferring an angular-average
approximation of the momentum arguments in the single-particle
energies, as shown in Eqs.~(\ref{eq:ene_denom_av}), (\ref{eq:ki_ave}) 
and (\ref{eq:kj_ave}). When using these approximated single-particle 
energies, the amplitude matrix gets a symmetry that makes it possible
to evaluate the matrix elements for only one direction of the 
CM momentum vector, and then use a rotation matrix to obtain
matrix elements for other CM momentum directions \cite{suzuki,baardsen}. 
Using the rotation matrix, we are therfore able to decrease the size
of the amplitude matrix further. The angular-average approximation
in the single-particle energies also opens the possibility
to use simplified formulae for the exact Pauli operators,
as introduced by Suzuki \emph{et al.}~\cite{suzuki}.

Taken that convergence is obtained in the number of momentum 
grid points, the only approximations of the CC ladder equations
in Ref.~\cite{baardsen} are the angular-average approximation in 
the single-particle potential and a finite cutoff in angular 
momentum. As seen from Fig.~6 of Ref.~\cite{baardsen}, the error 
owing to the cutoff in angular momentum is smaller than approximately 
$0.2-0.3$ MeV. We have not been able to directly quantify the error 
related to the 
angular-average approximation in the single-particle potentials. 
However, in Paper III we show that a CC ladder approximation with a finite 
number of particles in a box and twisted-averaged boundary conitions 
gives energies within $0.1-0.2$ MeV from the ladder approximation 
described above. The good agreement between these two quite different
approches indicates that the error owing to the angular-averaged
single-particle potentials is small.  
 
In Ref.~\cite{baardsen} we also compare CC ladder approximations with 
exact and angular-averaged Pauli exclusion operators. Using exact 
Pauli operators, we get energies that are at most about $0.2$ MeV
from the energies obtained using angular-averaged Pauli operators.
Similarly as observed in the BHF method \cite{suzuki,schiller},
exact treatment of the Pauli exclusion operators gives more
binding for symmetric nuclear matter. In pure neutron matter,
we find that the angular-average approximation of Pauli operators 
has a much smaller effect on the binding energy.

\begin{itemize}
\item Figure: MBPT(2) with exact and angular-averaged Pauli operators 
\end{itemize}


\subsection{Problems with particle-hole diagrams}
In three-dimensional nuclear matter, the CC $\hat{T}_{2}$ amplitude is a 
function of four three-dimensional momentum vectors. One vector can be 
removed owing to momentum conservation, but still the size of the 
$\hat{T}_{2}$ amplitude matrix may become very large. The realistic 
two-particle nuclear interactions we use are given in a coupled 
angluar-momentum basis with relative momentum coordinates. We have 
therefore chosen to transform the CC equations to the coupled 
angular-momentum and relative momentum basis, given in 
Eq.~(\ref{eq:coupled_basis1}). As discussed in the previous 
subsection, the partial-wave expanded CC ladder equations  
may be considerably simplified using angular-averaged single-particle 
potentials. In this approximation, the size of the amplitude matrix 
decreases by two dimensions. Unfortunately, as we show in the
following, the complete CCD equations become considerably more
complicated in the partial-wave basis.

In the following, we repeat parts of the theory explained in 
Ref.~\cite{baardsen}, but we extend the discussion to include terms 
that are negelcted in the article. For simplicity, let us for a 
moment consider only those terms in the the CCD $\hat{T}_{2}$ 
amplitude equation (\ref{eq:t2ampl}) that are linear in the CC 
amplitude. Using integrals at the limit when the box size approaches 
infinity, as in Eq.~(\ref{eq:sum2int}), the linear part of the 
amplitude equation becomes
\begin{align}
  0 &= \langle \mathbf{k}_{a}\mathbf{k}_{b}|\hat{v}|\mathbf{k}_{i}\mathbf{k}_{j}\rangle_{AS} \nonumber \\
  & + \left( \varepsilon(\mathbf{k}_{a})+\varepsilon(\mathbf{k}_{b})-\varepsilon(\mathbf{k}_{i})-\varepsilon(\mathbf{k}_{j}) \right) \langle \mathbf{k}_{a}\mathbf{k}_{b}|\hat{t}|\mathbf{k}_{i}\mathbf{k}_{j}\rangle \nonumber \\
  & + \frac{1}{2}\left( \frac{L}{2\pi }\right)^{6} \int d\mathbf{k}_{k}\int d\mathbf{k}_{l} \langle \mathbf{k}_{a}\mathbf{k}_{b}|\hat{t}|\mathbf{k}_{k}\mathbf{k}_{l}\rangle \langle \mathbf{k}_{k}\mathbf{k}_{l}|\hat{v}|\mathbf{k}_{i}\mathbf{k}_{j}\rangle_{AS} \nonumber \\
  & \times \theta(k_{F}-|\mathbf{k}_{k}|)\theta(k_{F}-|\mathbf{k}_{l}|) \nonumber \\
  & + \frac{1}{2}\left( \frac{L}{2\pi }\right)^{6}\int d\mathbf{k}_{c}\int d\mathbf{k}_{d}\langle \mathbf{k}_{a}\mathbf{k}_{b}|\hat{v}|\mathbf{k}_{c}\mathbf{k}_{d}\rangle_{AS}\langle \mathbf{k}_{c}\mathbf{k}_{d}|\hat{t}|\mathbf{k}_{i}\mathbf{k}_{j}\rangle \nonumber \\
  & \times \theta(|\mathbf{k}_{c}|-k_{F})\theta(|\mathbf{k}_{d}|-k_{F}) \nonumber \\
  & + \hat{P}(\mathbf{k}_{i}, \mathbf{k}_{j})\hat{P}(\mathbf{k}_{a}, \mathbf{k}_{b})\left( \frac{L}{2\pi }\right)^{6}\int d\mathbf{k}_{k}\int d\mathbf{k}_{c}\langle \mathbf{k}_{a}\mathbf{k}_{c}|\hat{t}|\mathbf{k}_{i}\mathbf{k}_{k}\rangle \nonumber \\
  & \times \langle \mathbf{k}_{k}\mathbf{k}_{b}|\hat{v}|\mathbf{k}_{c}\mathbf{k}_{j}\rangle_{AS}\theta(k_{F}-|\mathbf{k}_{k}|)\theta(|\mathbf{k}_{c}|-k_{F}), 
  \label{eq:t2ampl_linear}
\end{align}
where $\theta(x)$ is the Heaviside step function. In 
Eq.~(\ref{eq:t2ampl_linear}) we have used the definition 
$\varepsilon(\mathbf{k}) \equiv \langle \mathbf{k}|f|\mathbf{k}\rangle$. 
The term on the third and fourth lines of Eq.~(\ref{eq:t2ampl_linear}) 
has summation over two hole states. The correponding diagram 
is the one in the middle on the second row of 
Fig.~\ref{fig:t2ampl_linear}. Let us call this term the hole-hole 
ladder (HHLAD) diagram, because the term generates diagrams similar 
to the hole-hole ladders in many-body perturbation theory 
\cite{harris}. In the same way, there is a summation over two 
particle states in the term on the fifth and sixth lines of 
Eq.~(\ref{eq:t2ampl_linear}). The corresponding diagram is given 
on the second row at left of Fig.~\ref{fig:t2ampl_linear}. This 
term generates diagrams similar to the particle-particle ladders 
encountered in many-body perturbation theory \cite{harris}, and 
therefore we call the term the particle-particle ladder (PPLAD) 
diagram. In the last term of Eq.~(\ref{eq:t2ampl_linear}), there 
is a summation over one hole and one particle state. The 
corresponding diagram is the last one in 
Fig.~\ref{fig:t2ampl_linear}. Let us call this term the 
particle-hole (PH) diagram.

\begin{figure} 
  \centering
  \includegraphics[scale=1.0]{diagrams/diagram_cc_ampl_linear/t_amplitude_all2-crop.pdf}
  \caption{Diagrams of the CCD $\hat{T}_{2}$ amplitude equation 
that are linear in the CC amplitude. 
The complete figure with all diagrams is given in 
Ref.~\cite{baardsen}.}
  \label{fig:t2ampl_linear}
\end{figure}

Similarly as in Ref.~\cite{baardsen}, we write the linear part 
of the CCD amplitude equation in RCM coordinates, but 
including here also the PH term. The amplitude equation becomes
\begin{align}
  0 &= \langle \mathbf{k}'|\hat{v}|\mathbf{k}\rangle_{AS} 
+ \left( \varepsilon(|\mathbf{k}'+\mathbf{K}/2|) 
+ \varepsilon(|-\mathbf{k}'+\mathbf{K}/2|) \right. \nonumber \\
  & \left. -\varepsilon(|\mathbf{k}+\mathbf{K}/2|) 
- \varepsilon(|-\mathbf{k}+\mathbf{K}/2|)\right)
\langle \mathbf{k}'|\hat{t}|\mathbf{k}\rangle \nonumber \\
  & + \frac{1}{2}\int d\mathbf{h}
\langle \mathbf{k}'|t(\mathbf{K})|\mathbf{h}\rangle 
\langle \mathbf{h}|\hat{v}|\mathbf{k}\rangle_{AS} \nonumber \\
  & \times \theta(k_{F}-|\mathbf{h}+\mathbf{K}/2|)
\theta(k_{F}-|-\mathbf{h}+\mathbf{K}/2|) \nonumber \\
  & + \frac{1}{2}\int d\mathbf{p}
\langle \mathbf{k}'|\hat{v}|\mathbf{p}\rangle_{AS}\langle 
\mathbf{p}|\hat{t}|\mathbf{k}\rangle \nonumber \\
  & \times \theta(|\mathbf{p}+\mathbf{K}/2|-k_{F})
\theta(|-\mathbf{p}+\mathbf{K}/2|-k_{F}) \nonumber \\
  & + \int d\mathbf{B} \langle \mathbf{B}/4
+3\mathbf{k}'/4-\mathbf{k}/4
-\mathbf{K}/4|\hat{v}|\mathbf{B}/4-\mathbf{k}'/4
+3\mathbf{k}/4-\mathbf{K}/4 \rangle_{AS} \nonumber \\
  & \times \langle -\mathbf{B}/4+3\mathbf{k}'/4
-\mathbf{k}/4+\mathbf{K}/4|\hat{t}|\mathbf{B}/4
+3\mathbf{k}/4-\mathbf{k}'/4+\mathbf{K}/4\rangle \nonumber \\
  & \times \theta(k_{F}-|\mathbf{B}/2+(\mathbf{k}'
-\mathbf{k})/2|)\theta(|\mathbf{B}/2-(\mathbf{k}'
-\mathbf{k})/2|-k_{F}) \nonumber \\
  & + \dots ,
  \label{eq:t2ampl_rcm}
\end{align} 
where the last term is the unpermuted particle-hole part. 
Here we have used the definitions 
\begin{align}
  \mathbf{h} = (\mathbf{k}_{k}-\mathbf{k}_{l})/2, 
\qquad \mathbf{p} = (\mathbf{k}_{c}-\mathbf{k}_{d})/2,
\end{align}
and the relative momentum vectors $\mathbf{k}$, $\mathbf{k}'$ 
and CM momentum vector $\mathbf{K}$ are defined in 
Eq.~(\ref{eq:lab2rcm}).

In Ref.~\cite{baardsen} we wrote the amplitude equation into such a form that both the interaction and the CC amplitude were given in a coupled angular momentum and relative momentum basis
\begin{align}
  |k\mathcal{J}m_{\mathcal{J}}(lS)M_{T}\rangle,
  \label{eq:coupled_basis}
\end{align}
where $k \equiv |\mathbf{k}|$ is the length of a relative momentum vector $\mathbf{k}$, $l$ is the orbital angular momentum related to $\mathbf{k}$, $S$ is the total two-particle spin, $\mathcal{J}$ is equal to the total angular momentum $l+S$, $m_{\mathcal{J}}$ is the $z$ projection of $\mathcal{J}$, and $M_{T}$ is the projection of the total two-particle isospin.

Let us now consider the simpler linear equation in which the PH diagrams are negelcted, as was done in Ref.~\cite{baardsen}. When using an angular-average approximation for the inputs in the single-particle potentials $\varepsilon(|\mathbf{p}|) $ of the amplitude equation, all the vectors $| \mathbf{k}\rangle$, $|\mathbf{k}'\rangle $, and their conjugates can simply be replaced with vectors as in Eq.~(\ref{eq:coupled_basis}). The resulting CC amplitude equation is of the form  
  \begin{align} \label{eq:pphhlad_exact}
    &\Delta \tilde{\varepsilon }(k,k',K)\langle k'\mathcal{J}'m_{\mathcal{J}'}(l'S)M_{T}|\hat{t}(K)|k\mathcal{J}m_{\mathcal{J}}(lS)M_{T}\rangle \nonumber \\
    &=\langle k'\mathcal{J}'m_{\mathcal{J}'}(l'S)M_{T}|\hat{v}|k\mathcal{J}m_{\mathcal{J}}(lS)M_{T}\rangle \delta_{\mathcal{J}\mathcal{J}'}\delta_{m_{\mathcal{J}}m_{\mathcal{J}'}} \nonumber \\
    &+ \frac{1}{2}\sum_{\mathcal{J}''m_{\mathcal{J}''}}\sum_{l''l'''}\int_{0}^{k_{F}}h^{2}dh \nonumber \\
    &\times \langle k'\mathcal{J}'m_{\mathcal{J}'}(l'S)M_{T}|\hat{t}(K)|h\mathcal{J}''m_{\mathcal{J}''}(l''S)M_{T}\rangle \nonumber \\
    &\times \langle h\mathcal{J}m_{\mathcal{J}}(l'''S)M_{T}|\hat{v}|k\mathcal{J}m_{\mathcal{J}}(lS)M_{T}\rangle \nonumber \\
    &\times Q_{hh}(l''\mathcal{J}''m_{\mathcal{J}''},l'''\mathcal{J}m_{\mathcal{J}};SM_{T}hK\theta_{K}\phi_{K}) \nonumber \\
    &+ \frac{1}{2}\sum_{\mathcal{J}''m_{\mathcal{J}''}}\sum_{l''l'''}\int_{0}^{\infty }p^{2}dp \nonumber \\
    &\times \langle k'\mathcal{J}'m_{\mathcal{J}'}(l'S)M_{T}|\hat{v}|p\mathcal{J}'m_{\mathcal{J}'}(l''S)M_{T}\rangle \nonumber \\
    &\times \langle p\mathcal{J}''m_{\mathcal{J}''}(l'''S)M_{T}|\hat{t}(K)|k\mathcal{J}m_{\mathcal{J}}(lS)M_{T}\rangle \nonumber \\
    &\times Q_{pp}(l''\mathcal{J}'m_{\mathcal{J}'},l'''\mathcal{J}''m_{\mathcal{J}''};SM_{T}pK\theta_{K}\phi_{K}),
  \end{align}
where the Pauli operators $Q_{hh}$ and $Q_{pp}$ and the function $\tilde{\varepsilon }(k,k',K)$ are defined in Ref.~\cite{baardsen}. In the expression above, we have used a technique introduced by Suzuki \emph{et al.}~\cite{suzuki}. Unfortunately, the PH terms in Eq.~\ref{eq:t2ampl_rcm} have complicated dependencies on the angular parts of $\mathbf{k}$ and $\mathbf{k}'$. When including the PH term, the angular parts of $\mathbf{k}$ and $\mathbf{k}'$ can therefore no longer be separated out in a simple way. Because of this problem, we have omitted the PH diagrams from our CC amplitude equation in Ref.~\cite{baardsen}. 

An advantage with using partial waves is that the CC amplitude matrix can be calculated for only one direction of the CM momentum vector $\mathbf{K}$, for example only vectors $\mathbf{K} = (0, 0, K)$, and then the matrix elements for other directions can be obtained by using rotation matrices. In Ref.~\cite{baardsen}, the CC $\hat{T}_{2}$ amplitude matrix was rotated as 
\begin{align} 
  &\langle k'\mathcal{J}'m_{\mathcal{J}'}(l'S)|\hat{t}(\mathbf{K})|k\mathcal{J}m_{\mathcal{J}}(lS)\rangle \nonumber \\
  &= \sum_{m_{\mathcal{J}'''}m_{\mathcal{J}''}}D_{m_{\mathcal{J}'}m_{\mathcal{J}'''}}^{\mathcal{J}'}(\phi_{K},\theta_{K},0)D_{m_{\mathcal{J}}m_{\mathcal{J}''}}^{\mathcal{J} *}(\phi_{K},\theta_{K},0) \nonumber \\
  &\times \langle k'\mathcal{J}'m_{\mathcal{J}'''}(l'S)|\hat{t}(K)|k\mathcal{J}m_{\mathcal{J}''}(lS)\rangle , 
  \label{eq:rotation}
\end{align}
where $D_{m_{\mathcal{J}}m_{\mathcal{J}'}}^{\mathcal{J}}(\phi ,\theta ,\sigma )$ is a rotation matrix defined by Varshalovich \emph{et al.}~\cite{varshalovich}. It is possible to solve the amplitude equation using three-dimensional momentum vectors, but then we loose the rotational symmetry of the CC amplitude matrix. Consequently, the CC amplitude matrix needs two degrees more of freedom, which increases the demand for computer memory. 




%% \nonumber \\
%%   & + \frac{1}{2}\hat{P}(\mathbf{k}_{i}, \mathbf{k}_{j})\hat{P}(\mathbf{k}_{a}, \mathbf{k}_{b})\left( \frac{L}{2\pi }\right)^{12}\int d\mathbf{k}_{k}\int d\mathbf{k}_{l}\int d\mathbf{k}_{c}\int d\mathbf{k}_{d} \nonumber \\
%%   & \times \langle \mathbf{k}_{k}\mathbf{k}_{l}|\hat{v}|\mathbf{k}_{c}\mathbf{k}_{d}\rangle_{AS}\langle \mathbf{k}_{a}\mathbf{k}_{c}|\hat{t}|\mathbf{k}_{i}\mathbf{k}_{k}\rangle \langle \mathbf{k}_{d}\mathbf{k}_{b}|\hat{t}|\mathbf{k}_{l}\mathbf{k}_{j}\rangle \nonumber \\
%%   & \times \theta(k_{F}-|\mathbf{k}_{k}|)\theta(k_{F}-|\mathbf{k}_{l}|)\theta(|\mathbf{k}_{c}|-k_{F})\theta(|\mathbf{k}_{d}|-k_{F}) \nonumber \\
%%   & - \frac{1}{2}\hat{P}(\mathbf{k}_{i}, \mathbf{k}_{j})\left( \frac{L}{2\pi }\right)^{12}\int d\mathbf{k}_{k}\int d\mathbf{k}_{l}\int d\mathbf{k}_{c}\int d\mathbf{k}_{d}



\begin{itemize}
\item How recoupling can be done. What problems arise? 
\end{itemize}


\subsection{Implementations, verification, and results}

Let us discuss the implementation of the CC 
ladder equations with exact Pauli operators, given
as Eqs.~(18) and (23) in Paper II. The ladder 
equations using angular-averaged Pauli operators
may be implemented in a similar way. As stated above,
the nuclear interaction is diagonal in the total
two-particle spin $S$, in the isospin projection  
$M_{T}$, and in the total relative angular momentum
$\mathcal{J}$. Because of this symmetry, we store
the interaction matrices in blocks of the conserved
quantum numbers. Utilization of block diagonality
decreases the memory consumption significantly, and
it can also reduce the number of floating-point
operations greatly. However, as shown in Paper II,
the Pauli exclusion operators and the $t$-amplitude
matrix are not diagonal in $\mathcal{J}$ and its 
projection $m_{\mathcal{J}}$. Furthermore, these 
matrices depend on the radial CM momentum $K$,
and the Pauli operators are also functions of the
angle $\theta_{K}$ related to the CM momentum. 

The CC energy equation (18) of Paper II may be 
written as
\begin{align} \label{eq:cc_ene_impl1}
  &\Delta E_{CCD}/A = \text{Constant}\times 
  \sum_{SM_{T}}\sum_{\mathcal{J}m_{\mathcal{J}}}
  \sum_{\mathcal{J}''m_{\mathcal{J}''}}
  \sum_{\mathcal{J}'''m_{\mathcal{J}'''}}\sum_{m_{\mathcal{J}'}} 
  \nonumber \\
  & \times \sum_{K_{i}}K_{i}^{2}\omega_{K_{i}}\sum_{\theta_{j}}
  \sin \theta_{j}\omega_{\theta_{j}}
  d_{m_{\mathcal{J}''}m_{\mathcal{J}'}}^{\mathcal{J}''}(\theta_{j})
  d_{m_{\mathcal{J}'''}m_{\mathcal{J}'}}^{\mathcal{J}'''}(\theta_{j})
  \nonumber \\
  & \times \text{Tr} \left[ \mathbf{V}^{SM_{T}\mathcal{J}}
    \mathbf{Q}_{pp}^{SM_{T}K_{i}\mathcal{J}m_{\mathcal{J}}
      \mathcal{J}''m_{\mathcal{J}''}\theta_{j}}
    \mathbf{T}^{SM_{T}K_{i}\mathcal{J}''m_{\mathcal{J}'}\mathcal{J}'''}
    \mathbf{Q}_{hh}^{SM_{T}K_{i}\mathcal{J}'''m_{\mathcal{J}'''}
    \mathcal{J}m_{\mathcal{J}}\theta_{j}}
    \right],
\end{align} 
where the two-body interaction, the Pauli operators and 
the $t$ amplitude are expressed as matrices. Here
$K_{i}$ and $\theta_{j}$ are radial and angular coordinates
of the CM momentum, $\omega_{K_{i}}$ and $\omega_{\theta_{j}}$
are the corresponding quadrature weights, and the function 
$d$ is part of the Wigner $D$ function, as defined in
Refs.~\cite{varshalovich,baardsen}.
Using the
relation 
\begin{align}
  \text{Tr} \left[ \mathbf{A}\mathbf{B}\right] = 
  \text{Tr} \left[ \mathbf{B}\mathbf{A}\right]
\end{align} 
for the trace of a matrix-matrix product, we can replace
the trace in Eq.~(\ref{eq:cc_ene_impl1}) by
\begin{align}
  \text{Tr}\left[ \mathbf{M}^{SM_{T}K_{i}\mathcal{J}m_{\mathcal{J}}
    \mathcal{J}''m_{\mathcal{J}''}\mathcal{J}'''m_{\mathcal{J}'''}
    \theta_{j}} 
    \mathbf{T}^{SM_{T}K_{i}\mathcal{J}''m_{\mathcal{J}'}\mathcal{J}'''}\right],
\end{align}
where 
\begin{align}
  \mathbf{M}^{SM_{T}K_{i}\mathcal{J}m_{\mathcal{J}}
    \mathcal{J}''m_{\mathcal{J}''}\mathcal{J}'''
    m_{\mathcal{J}'''}\theta_{j}} = 
  \mathbf{Q}_{hh}^{SM_{T}K_{i}\mathcal{J}'''m_{\mathcal{J}'''}
    \mathcal{J}m_{\mathcal{J}}}
  \mathbf{V}^{SM_{T}\mathcal{J}}
  \mathbf{Q}_{pp}^{SM_{T}K_{i}\mathcal{J}m_{\mathcal{J}}
    \mathcal{J}''m_{\mathcal{J}''}\theta_{j}}.
\end{align}
The matrix $\mathbf{M}$ does not change during the CC
self-consistency loop, and is therefore set up only
once at the beginning of the calculations. The trace
is evaluated as
\begin{align} \label{eq:tr_sum}
  \text{Tr} \left[ \mathbf{M}\mathbf{T}\right] 
  = \sum_{\alpha }\left( \sum_{\beta }M_{\alpha \beta }T_{\beta \alpha }\right),
\end{align} 
and a matrix-matrix multiplication is therefore not
necessary. If the dimension of $\mathbf{T}$ is 
$m\times n$, the simplification in Eq.~(\ref{eq:tr_sum}) 
reduces the number of floating-point operations 
from $m^{2}n^{2}$ to $mn$ \cite{golub}. 

We further simplify the CC energy equation and write it
as
\begin{align} \label{eq:cc_ene_impl2}
  \Delta E_{CCD}/A &= \text{Constant} \times \sum_{K_{i}}
  \sum_{SM_{T}}\sum_{\mathcal{J}''m_{\mathcal{J}'}}
  \sum_{\mathcal{J}'''} \nonumber \\
  & \times \sum_{\alpha } \left( 
  \sum_{\beta }\tilde{T}_{\alpha \beta }^{SM_{T}K_{i}\mathcal{J}''m_{\mathcal{J}'}
    \mathcal{J}'''}O_{\beta \alpha }^{SM_{T}K_{i}\mathcal{J}''
    m_{\mathcal{J}'}\mathcal{J}'''}
  \right),
\end{align}
where the matrix elements $\tilde{T}_{\alpha \beta }$ are defined as
\begin{align}
  \tilde{T}_{\alpha \beta }^{SM_{T}K_{i}\mathcal{J}''m_{\mathcal{J}'}
    \mathcal{J}'''} = T_{\alpha \beta }^{SM_{T}K_{i}
    \mathcal{J}''m_{\mathcal{J}'}\mathcal{J}'''}
  /\Delta \tilde{\varepsilon}(k_{\alpha }, k_{\beta }, K_{i}),
\end{align} 
that is, the $t$ amplitude divided by the energy denominator.
The matrix $\mathbf{O}$ is defined such that
\begin{align}
  \mathbf{O}^{SM_{T}K_{i}\mathcal{J}''m_{\mathcal{J}'}\mathcal{J}'''} 
  = \sum_{\mathcal{J}m_{\mathcal{J}}}\sum_{m_{\mathcal{J}''}m_{\mathcal{J}'''}}
  \sum_{\theta_{j}}\mathbf{N}^{(SM_{T}K_{i}\mathcal{J}''m_{\mathcal{J}'}
    \mathcal{J}''')(\mathcal{J}m_{\mathcal{J}}m_{\mathcal{J}''}
    m_{\mathcal{J}'''}\theta_{j})},
\end{align}
where
\begin{align}
  & \mathbf{N}^{(SM_{T}K_{i}\mathcal{J}''m_{\mathcal{J}'}
    \mathcal{J}''')(\mathcal{J}m_{\mathcal{J}}m_{\mathcal{J}''}
    m_{\mathcal{J}'''}\theta_{j})} \nonumber \\
  & = K_{i}^{2}\omega_{K_{i}}\sin \theta_{j}\omega_{\theta_{j}} 
  d_{m_{\mathcal{J}''}m_{\mathcal{J}'}}^{\mathcal{J}''}(\theta_{j})
  d_{m_{\mathcal{J}'''}m_{\mathcal{J}'}}^{\mathcal{J}'''}(\theta_{j}) 
  \nonumber \\
  & \times \mathbf{M}^{SM_{T}K_{i}\mathcal{J}m_{\mathcal{J}}
    \mathcal{J}''m_{\mathcal{J}''}\mathcal{J}'''
    m_{\mathcal{J}'''}\theta_{j}}. 
\end{align}
Similarly as for the energy equation, we write the 
amplitude equation as
\begin{align} \label{eq:tampl_impl1}
  \mathbf{T}^{SM_{T}K_{i}\mathcal{J}'m_{\mathcal{J}}\mathcal{J}} 
  & = \mathbf{V}^{SM_{T}\mathcal{J}}\delta_{\mathcal{J}\mathcal{J}'}
  \nonumber \\
  & + \frac{1}{2}\sum_{\mathcal{J}''}
  \mathbf{\tilde{T}}^{SM_{T}K_{i}\mathcal{J}'m_{\mathcal{J}}\mathcal{J}''}
  \mathbf{W}^{SM_{T}K_{i}\mathcal{J}''m_{\mathcal{J}}\mathcal{J}}
  \nonumber \\
  & + \frac{1}{2}\sum_{\mathcal{J}''}
  \mathbf{U}^{SM_{T}K_{i}\mathcal{J}'m_{\mathcal{J}}\mathcal{J}''}
  \mathbf{\tilde{T}}^{SM_{T}K_{i}\mathcal{J}''m_{\mathcal{J}}\mathcal{J}},
\end{align}
where we have used the definitions
\begin{align}
  \mathbf{W}^{SM_{T}K_{i}\mathcal{J}''m_{\mathcal{J}}\mathcal{J}}
  = \mathbf{Q}_{hh}^{SM_{T}K_{i}\mathcal{J}''m_{\mathcal{J}}
    \mathcal{J}m_{\mathcal{J}}}
  \mathbf{V}^{SM_{T}\mathcal{J}}
\end{align}
and 
\begin{align}
  \mathbf{U}^{SM_{T}K_{i}\mathcal{J}'m_{\mathcal{J}}\mathcal{J}''}
  = \mathbf{V}^{SM_{T}\mathcal{J}'}
  \mathbf{Q}_{pp}^{SM_{T}K_{i}\mathcal{J}'m_{\mathcal{J}}
  \mathcal{J}''m_{\mathcal{J}}}.
\end{align}
The different matrices are stored in lists of matrices,
where the list index is determined by the superscript 
quantum numbers. For example, the matrices 
$\mathbf{O}^{SM_{T}K_{i}\mathcal{J}''m_{\mathcal{J}'}\mathcal{J}'''} $ 
are stored as elements in a list in which the quantum 
numbers $S$, $M_{T}$, $K_{i}$, $\mathcal{J}''$, 
$m_{\mathcal{J}'}$, and $\mathcal{J}'''$ determine the
location. To obtain sufficient accuarcy, the grid in 
relative momentum must be set up such that the total
interval is split at the points where the derivatives
of the Pauli operators are discontinuous. In our numerical
integrations, we use Gauss-Legendre quadratures 
\cite{num_recipes}. 

In our calculations, we set up the matrices 
$\mathbf{O}$, $\mathbf{W}$, and $\mathbf{U}$ only once, 
and then we use Eqs.~(\ref{eq:cc_ene_impl2}) 
and (\ref{eq:tampl_impl1}) to evaluate the energy in 
each iteration of the CC self-consistency loop. 
In large calculations, most of the computing time is 
used to set up the matrix $\mathbf{O}$. 
By setting up this matrix only once, the computing 
time of a typical symmetric-nuclear-matter calculation 
therefore decreases by approximately a factor of ten. 
To utilize the computing power of large computing clusters,
we have parallelized the code using both the 
Message Passing Interface (MPI) Standard 
\cite{gropp1999,traff2012} and OpenMP \cite{chapman}.
MPI is a standard for parallelization that is designed
to work on computers with distributed memory. Therefore, 
we use MPI to share work load between different computing
nodes on a supercomputer. We split the CC ladder 
equations into parts corresponding to different CM 
momentum values $K_{i}$, and distribute these parts
to different MPI processes. As mentioned above, most
of the computing time is used to set up the matrix
$\mathbf{O}$. The code can therefore be further
optimized by parallelizing the setup of $\mathbf{O}$
on each computing node. In our program, we use OpenMP 
to parallelize the setup of the matrices
$\mathbf{O}^{SM_{T}K_{i}\mathcal{J}''m_{\mathcal{J}'}\mathcal{J}'''} $ 
on the different computing nodes. 



%% \begin{align}
%%   \text{Tr} \left[ \mathbf{A}\mathbf{B}\right] &= 
%%   \sum_{i}\sum_{j}A_{ij}B_{ji} \nonumber \\
%%   & = \sum_{j}\sum_{i}B_{ji}A_{ij} \nonumber \\
%%   & = \text{Tr} \left[ \mathbf{B}\mathbf{A}\right]
%% \end{align}

\begin{itemize}
\item Other optimizations
\item Verification: Calculations with Minnesota potential, 
  comparison with CC in momentum grid, comparison with
  BHF 
\end{itemize}


\subsection{Other approaches to CCSD in three dimensions}

As discussed in the previous subsection, the CCSD amplitude 
equations become very complicated in the partial-wave 
expansion using RCM coordinates. In Paper III, a cartesian
momentum basis was successfully applied to nuclear-mattter
systems. We have studied two other approches to 
implementing coupled-cluster theory for infinite matter.
In these approaches, the interaction matrix elements are
transformed from the relative partial-wave basis 
(\ref{eq:coupled_basis1}) to a single-particle basis in 
laboratory coordinates. Once the interaction matrix elements
have been transformed to a laboratory frame basis, it is 
straightforward to derive and implement explicit 
coupled-cluster equations.

A seemingly appealing approach is to transform the 
interaction matrix to the laboratory frame single-particle basis
\begin{align}
  |k_{p}j_{p}m_{j_{p}}l_{p}s_{p}m_{t_{p}}\rangle , 
\end{align}
where $k_{p}$ is the laboratory momentum, $l_{p}$ is the orbital
angular momentum, $s_{p}$ is the spin, $j_{p}$ is the angular 
momentum, $m_{j_{p}}$ is the $z$ projection of $j_{p}$, and 
$m_{t_{p}}$ is the isospin projection. Hagen \emph{et al.}~have
shown \cite{hagen2010} how more efficient coupled-cluster
implementations can be obtained by coupling single-particle
angular momenta $j_{p}$ and $j_{q}$ to a total angular momentum
$J$. This kind of angular momentum coupling is possible because 
the nuclear interaction is rotationally invariant 
\cite{ring_schuck} and therefore conserves the total momentum. 
We would therefore like to write the two-nucleon states in the 
basis
\begin{align} \label{eq:coupled_lab}
  |k_{p}j_{p}l_{p},k_{q}j_{q}l_{q};Jm_{t_{p}}m_{t_{q}} \rangle ,
\end{align} 
where the single-particle angular momenta $j_{p}$ and $j_{q}$ 
have been coupled to a total angular momentum. 

We show how the interaction matrix elements can
be transformed to the basis (\ref{eq:coupled_lab}).  Assume 
for a while that we have the continuous plane wave basis 
\begin{align} \label{eq:planew_c}
  \psi_{\mathbf{k}}(\mathbf{x})=\frac{1}{(2\pi )^{3/2}}
  e^{i\mathbf{k}\cdot \mathbf{x}},
\end{align}
instead of the discrete box potential plane wave basis of 
Eq.~(\ref{eq:planeWave}). A bra vector 
$\langle \mathbf{k}_{p}\mathbf{k}_{q}|$ can be expanded as \cite{wong1972}
\begin{align} \label{eq:angExp}
  \langle \mathbf{k}_{p}\mathbf{k}_{q}| &= \langle \mathbf{k}_{p}\mathbf{k}_{q}|
  \left( \sum_{l_{p}m_{l_{p}}}\sum_{l_{q}m_{l_{q}}}|l_{p}m_{l_{p}}l_{q}m_{l_{q}}\rangle
  \langle l_{p}m_{l_{p}}l_{q}m_{l_{q}}| \right) \nonumber \\
  &= \langle \mathbf{k}_{p}\mathbf{k}_{q}|\sum_{l_{p}m_{l_{p}}}\sum_{l_{q}m_{l_{q}}}
  |l_{p}m_{l_{p}}l_{q}m_{l_{q}}\rangle \nonumber \\
  & \times \left( \sum_{\lambda \mu } \langle l_{p}m_{l_{p}}l_{q}m_{l_{q}}|
      \lambda \mu \rangle \langle \lambda \mu l_{p}l_{q}| \right) \nonumber \\
  &= \sum_{l_{p}l_{q}\lambda \mu }\langle k_{p}l_{p}k_{q}l_{q},\lambda \mu |
      \left[ Y_{l_{p}m_{l_{p}}}(\mathbf{\hat{k}}_{p})
        Y_{l_{q}m_{l_{q}}}(\mathbf{\hat{k}}_{q})\right]_{\lambda m_{\lambda } },
\end{align} 
where the functions 
$Y_{lm_{l}}(\mathbf{\hat{k}}) \equiv \langle \mathbf{\hat{k}}|lm_{l}\rangle $ 
are spherical harmonics, the bracket denotes a Clebsch-Gordan 
coefficient \cite{varshalovich}, $l_{p}$ and $l_{q}$ are orbital angular 
momenta of the two particles, $\lambda $ is the total orbital angular 
momentum, and $m_{\lambda } $ its projection in the $z$ direction.
In Eq.~(\ref{eq:angExp}) the square brackets represent coupling
of orbital angular momenta, as in 
Eq.~(\ref{eq:coupling_squarebr}). We define the radial
states with the normalization condition
\begin{align} \label{eq:radial_norm2}
  \langle klm_{l}|k'l'm_{l'}\rangle = 
  \frac{1}{kk'}\delta(k-k')\delta_{ll'}\delta_{m_{l}m_{l'}},
\end{align} 
and Eq.~(\ref{eq:angExp}) differs therefore from the expression 
in Ref.~\cite{wong1972} by a factor $(k_{p}k_{q})^{-1}$. When
projecting to the real space, we get \cite{liboff}
\begin{align}
  \varphi_{klm_{l}}(\mathbf{r}) &\equiv 
  \langle \mathbf{r}|klm_{l}\rangle = 
  j_{l}(kr)Y_{lm_{l}}(\theta_{r} ,\phi_{r} ),
\end{align}
where $j_{l}(x)$ is spherical Bessel function.  
Observe that continuous momentum variables give a Dirac delta 
distribution in the normalization expression. 

Radial ket vectors can be transformed between the laboratory 
and RCM coordinate systems using the relation \cite{wong1972} 
\begin{equation} \label{eq:labRCM}
  |k_{p}l_{p}k_{q}l_{q},\lambda m_{\lambda }\rangle =
  \int k^{2} dk \int K^{2} dK 
  \sum_{lL}|klKL,\lambda m_{\lambda } \rangle
  \langle klKL,\lambda |k_{p}l_{p}k_{q}l_{q},\lambda \rangle ,
\end{equation}
where $k$ and $K$ are radial coordinates of relative and 
CM momenta, respectively, $l$ and $L$ are corresponding
orbtial angular momenta, and the coefficient denoted by
a bracket is called a vector bracket \cite{wong1972,kung1979}. 
The vector bracket can be written as
\begin{align} \label{eq:vb}
  \langle klKL,\lambda |k_{p}l_{p}k_{q}l_{q},\lambda \rangle 
  = \left( 4\pi \right)^{2} \delta(u)\theta(1-v^{2})A(v),
\end{align}
where 
\begin{align}
  A(v)&=\frac{1}{(2\lambda +1)}\frac{1}{k_{p}k_{q}kK}
  \sum_{m_{\lambda } }\left[ Y_{lm_{l}}(\mathbf{\hat{k}})
    \times Y_{LM_{L}}(\mathbf{\hat{K}})\right]_{\lambda m_{\lambda }}^{*} 
  \nonumber \\
  & \times \left[ Y_{l_{p}m_{l_{p}}}(\mathbf{\hat{k}}_{p})
    \times Y_{l_{q}m_{l_{q}}}(\mathbf{\hat{k}}_{q})\right]_{\lambda m_{\lambda }},
\end{align}
as formulated by Balian and Brezin \cite{wong1972, balian1969}, 
and 
\begin{align}\label{eq:v_theta}
  u&=k^{2}+\frac{1}{4}K^{2}-\frac{1}{2}(k_{p}^{2}+k_{q}^{2}), \\
  v&=\frac{1}{kK}\left( k_{p}^{2}-k^{2}-\frac{1}{4}K^{2} \right). 
\end{align}
The delta distribution ensures that kinetic energy is conserved 
in the transformation between coordinate systems. The variable 
$v$ may also be written as
\[
v = \cos(\theta_{\mathbf{k}\mathbf{K}}),
\] 
where $\theta_{\mathbf{k}\mathbf{K}}$ is the angle between the 
relative and CM momentum vectors $\mathbf{k}$ 
and $\mathbf{K}$. The step funtion $\theta(1-v^{2})$ therefore 
gives the geometric restriction 
$|\cos(\theta_{\mathbf{k}\mathbf{K}})| \leq 1$. 
%% Eq.~(\ref{eq:v_theta}) is easily obtained by considering the 
%% vector relations 
%% \[
%% \mathbf{k}_{1}=\mathbf{k}+\frac{1}{2}\mathbf{K}, \qquad 
%% \mathbf{k}_{2}=-\mathbf{k}+\frac{1}{2}\mathbf{K}
%% \] 
%% and using the cosine rule as well as the conservation of
%% kinetic energy.    

We want to write a vector 
$|(k_{p}l_{p}j_{p})(k_{p}l_{p}j_{p})(JM_{J})\rangle $ as a linear 
combination of vectors $|klKL(\mathcal{J})SJM_{J}\rangle $.
This transformation has been given in, for example, 
Refs.~\cite{kung1979,hjensen1993}. In the following, we
sketch a derivation of the transformation between the two
bases. The derivation is done in several steps. First we do 
a recoupling from the $j\!-\!j$ scheme to the $L\! -\! S$ 
scheme, as in Eq. (4.13) of Ref.~\cite{lawson}. Further, we 
do the recoupling 
$JM_{J}\lambda S \longrightarrow \lambda \mu SM_{S}$, and get
\begin{align}
  |(k_{p}l_{p}j_{p})(k_{q}l_{q}j_{q})(JM_{J})\rangle &=
  \sum_{\lambda m_{\lambda }}\sum_{S M_{S}} 
  \hat{j_{p}}\hat{j_{q}}\hat{\lambda }\hat{S} 
  \left\{ \begin{array}{ccc}
    l_{p} & \frac{1}{2} & j_{p} \\
    l_{q} & \frac{1}{2} & j_{q} \\
    \lambda & S & J 
    \end{array} \right\} \nonumber \\
   & \times \langle \lambda m_{\lambda }SM_{S}|JM_{J}\rangle 
   |k_{p}l_{p}k_{q}l_{q},\lambda m_{\lambda } \rangle |SM_{S}\rangle , 
\end{align}
where we have used the definition $\hat{x}=\sqrt{2x+1}$. 
If we now use the laboratory to RCM transformation 
(\ref{eq:labRCM}), do the recoupling 
$\lambda m_{\lambda }SM_{S} \longrightarrow JM_{J}\lambda S$, 
and use the $9j$ coefficient relation (A4.27) in 
Ref.~\cite{lawson}, we get 
\begin{align}
  |(k_{p}l_{p}j_{p})(k_{q}l_{q}j_{q})(JM_{J})\rangle &=
  \sum_{\lambda S}\sum_{lL}\int k^{2}dk \int K^{2}dK 
  \hat{j_{p}}\hat{j_{q}}\hat{\lambda }\hat{S}
  \left\{ \begin{array}{ccc}
    l_{p} & l_{q} & \lambda \\
    \frac{1}{2} & \frac{1}{2} & S \\
    j_{p} & j_{q} & J 
    \end{array} \right\} \nonumber \\
    & \times 
  \langle klKL,\lambda |k_{p}l_{p}k_{q}l_{q},\lambda \rangle 
  |klKL(\lambda )SJM_{J}\rangle .
\end{align}     
To get the desired basis on the right-hand side, we must further do the recoupling of three angular momenta $lL(\lambda )SJM_{J} \longrightarrow lL(\mathcal{J})SJM_{J}$, diagrammatically sketched as \cite{lawson}

\begin{tikzpicture}[
        thick,
        % Set the overall layout of the tree
        level/.style={level distance=0.5cm},
        %level 2/.style={sibling distance=1cm},
        level 2/.style={sibling distance=1.5cm},
        level 3/.style={sibling distance=1.5cm}
    ]

    \coordinate (A) at (-0.75,0);
    \coordinate (B) at (0.75,0);
    \coordinate (C) at (0,1.05);
    \coordinate (D) at (0.8,0.7);

    
    \draw [line1] (A) -- (B) node[below,midway,text centered] {($JM_{J}$)}; 
    \draw [line1] (A) -- (C);
    \node at (barycentric cs:A=0.8,C=1,B=-0.3) {$S$};
    \node at (barycentric cs:C=1,A=0.3,B=0.8) {$\lambda $};
    \draw [line1] (B) -- (C);
    \node at (barycentric cs:C=1,D=0.7,A=-0.25) {$l$};
    \draw [line1] (B) -- (D);
    \node at (barycentric cs:B=0.6,D=1.1,C=-0.3) {$L$};
    \draw [line1] (C) -- (D);

    \draw (3.8,.5) node {$= \sum_{\mathcal{J}}\hat{\mathcal{J}}\hat{\lambda }\: W(SlJL;\mathcal{J}\lambda ) \times $};
    \node at (-2,0) {};
    \node at (0,1.85) {};
    \node at (0,-.8) {};

    \coordinate (E) at (6.75,0);
    \coordinate (F) at (8.25,0);
    \coordinate (G) at (7.5,1.05);
    \coordinate (H) at (6.7,0.7);

    \draw [line1] (E) -- (F) node[below,midway,text centered] {($JM_{J}$)}; 
    \draw [line1] (E) -- (G);
    \node at (barycentric cs:E=0.8,G=1,F=0.4) {$\mathcal{J}$};
    \node at (barycentric cs:G=1,E=-0.3,F=0.8) {$L$};
    \draw [line1] (F) -- (G);
    \node at (barycentric cs:G=1,H=0.7,F=-0.25) {$l$};
    \draw [line1] (E) -- (H);
    \node at (barycentric cs:E=0.6,H=1.1,G=-0.4) {$S$};
    \draw [line1] (G) -- (H);
\end{tikzpicture} \\
where $W(SlJL;\mathcal{J}\lambda )$ is a Racah coefficient 
\cite{racah1942,varshalovich}. When we apply this recoupling, 
we get the expression
\begin{align}
  |(k_{p}l_{p}j_{p})(k_{q}l_{q}j_{q})(JM_{J})\rangle &=
  \sum_{\lambda S}\sum_{lL}\sum_{\mathcal{J}}
  \int k^{2}dk \int K^{2}dK \hat{j_{p}}\hat{j_{q}}
  \hat{\lambda }^{2}\hat{S}\hat{\mathcal{J}} \nonumber \\
  & \times 
  \left\{ \begin{array}{ccc}
    l_{p} & l_{q} & \lambda \\
    \frac{1}{2} & \frac{1}{2} & S \\
    j_{p} & j_{q} & J 
  \end{array} \right\}   
  \langle klKL,\lambda |k_{p}l_{p}k_{q}l_{q},\lambda \rangle 
  \nonumber \\
  & \times W(LlJS;\lambda \mathcal{J}) 
  |klKL(\mathcal{J})SJM_{J}\rangle .
\end{align}
Here we have used some symmetry relations for Racah coefficients. 
 If we now use the relation between Racah coefficients and 
$6j$ symbols, and include the isospin degree of freedom, we 
obtain
\begin{align} \label{eq:rcm2lab_pw3}
  |(k_{p}l_{p}j_{p})(k_{q}l_{q}j_{q})(JM_{J}m_{t_{p}}m_{t_{q}})\rangle
  &= \sum_{\lambda S}\sum_{lL}\sum_{\mathcal{J}}
  \int k^{2}dk \int K^{2}dK \: \hat{j_{p}}\hat{j_{q}}
  \hat{\lambda }^{2}\hat{S}\hat{\mathcal{J}} \nonumber \\
  & \times (-1)^{L+l+J+S} 
  \left\{ \begin{array}{ccc}
    l_{p} & l_{q} & \lambda \\
    \frac{1}{2} & \frac{1}{2} & S \\
    j_{p} & j_{q} & J 
  \end{array} \right\} \nonumber \\
  & \times 
  \langle klKL,\lambda |k_{p}l_{p}k_{q}l_{q},\lambda \rangle
  \left\{ \begin{array}{ccc}
    L & l & \lambda \\
    S & J & \mathcal{J} \\
  \end{array} \right\}  \nonumber \\
  & \times |klKL(\mathcal{J})SJM_{J}m_{t_{p}}m_{t_{q}}\rangle .
\end{align}
The corresponding antisymmetrized vector is obtained by 
multiplying the right-hand side of Eq.~(\ref{eq:rcm2lab_pw3}) 
by $\mathcal{A}^{lSM_{T}}$, where $M_{T} = m_{t_{p}}+m_{t_{q}}$ 
and the antisymmetrization factor is defined in 
Eq.~(\ref{eq:antisymm_app}). The transformation in 
Eq.~(\ref{eq:rcm2lab_pw3}) differs from the expressions in 
Refs.~\cite{kung1979,hjensen1993} by a phase factor
$(-1)^{(J-\lambda )-(\mathcal{J}-l)}$, which is one owing to
triangular relations.

Using the transformation (\ref{eq:rcm2lab_pw3}), a 
two-body interaction matrix element may now be written
as 
\begin{eqnarray} \label{eq:int_v_relcm2lab}
  \lefteqn{  \langle k_{p}l_{p}j_{p}k_{q}l_{q}j_{q}JTM_{T}|
  V|k_{r}l_{r}j_{r}k_{s}l_{s}j_{s}JTM_{T}\rangle } \nonumber \\
  & & {} =\sum_{l L \lambda S \mathcal{J}}\int_{0}^{\infty } k^{2}dk 
  \int_{0}^{\infty } K^{2}dK 
  \left\{ \begin{array}{ccc}
    l_{p} & l_{q} & \lambda \\
    \frac{1}{2} & \frac{1}{2} & S \\
    j_{p} & j_{q} & J 
  \end{array} \right\} \nonumber \\
  & & {} \times (-1)^{\lambda +\mathcal{J}-L-S}
  \hat{\mathcal{J}}\hat{\lambda }^{2}\hat{j}_{p}
  \hat{j}_{q}\hat{S} \left\{ 
  \begin{array}{ccc}
    L & l & \lambda \\
    S & J & \mathcal{J} 
  \end{array} \right\} 
  4\pi^{2}\delta\left( k^{2}+\frac{1}{4}K^{2}
  -\frac{1}{2}(k_{p}^{2}+k_{q}^{2})\right) \nonumber \\
  & & {} \times \theta \left( 1-\frac{(k_{p}^{2}-k^{2}
    -\frac{1}{4}K^{2})^{2}}{k^{2}K^{2}}\right)A
  \left( \frac{k_{p}^{2}-k^{2}-\frac{1}{4}K^{2}}{kK}\right) 
  \nonumber \\
  & & {} \times \sum_{l' \lambda' }\int_{0}^{\infty } 
  k'^{2}dk' \int_{0}^{\infty } K'^{2}dK' 
  \left\{ \begin{array}{ccc}
    l_{r} & l_{s} & \lambda ' \\
    \frac{1}{2} & \frac{1}{2} & S \\
    j_{p} & j_{q} & J 
  \end{array} \right\} \nonumber \\
    & & {} \times (-1)^{\lambda' +\mathcal{J}-L-S}
  \hat{\mathcal{J}}\hat{\lambda }'^{2}\hat{j}_{r}
  \hat{j}_{s}\hat{S} 
  \left\{ \begin{array}{ccc}
    L & l' & \lambda' \\
    S & J & \mathcal{J} 
  \end{array} \right\} 4\pi^{2}\delta 
  \left( k'^{2}+\frac{1}{4}K'^{2}
  -\frac{1}{2}(k_{r}^{2}+k_{s}^{2})\right) \nonumber \\
  & & {} \times \theta \left( 1-\frac{(k_{r}^{2}-k'^{2}
    -\frac{1}{4}K'^{2})^{2}}{k'^{2}K'^{2}}\right) A
  \left( \frac{k_{r}^{2}-k'^{2}-\frac{1}{4}K'^{2}}{k'K'}\right) 
  \nonumber \\
  & & {} \times \langle klKL(\mathcal{J})SJTM_{T}|V
  |k'l'K'L(\mathcal{J})SJTM_{T}\rangle ,
\end{eqnarray}
where the vector brackets are given explicitly using
Eq.~(\ref{eq:vb}). We assume that the antisymmetrization 
factors are incorporated implicitly in the relative 
coordinate two-body interaction. 


The nuclear interaction is generally diagonal in $\mathcal{J}$, 
$S$, $M_{T}$, $K$, $L$, $J$, and $T$, and independent on
the latter four. Let us separate the quantum numbers that the
interaction is independent on, and use the normalization 
\[
\langle K|K'\rangle = \delta(K-K')/(KK'). 
\]
After reorganising the different terms, the transformation 
becomes
\begin{eqnarray} \label{eq:int_v_relcm2lab3}
  \lefteqn{  \langle k_{p}l_{p}j_{p}k_{q}l_{q}j_{q}JTM_{T}|
  V|k_{r}l_{r}j_{r}k_{s}l_{s}j_{s}JTM_{T}\rangle } \nonumber \\
  & & {} = \left(4\pi^{2}\right)^{2} \sum_{l L \lambda S \mathcal{J}}
  \sum_{l' \lambda' } (-1)^{\lambda + \lambda'}
  \hat{\mathcal{J}}^{2}\hat{\lambda }^{2}\hat{\lambda }'^{2} 
  \hat{S}^{2} \hat{j}_{p}\hat{j}_{q}\hat{j}_{r}\hat{j}_{s} 
  \nonumber \\
  & & {} \times \left\{ 
  \begin{array}{ccc}
    L & l & \lambda \\
    S & J & \mathcal{J} 
  \end{array} \right\}
   \left\{ \begin{array}{ccc}
    L & l' & \lambda' \\
    S & J & \mathcal{J} 
  \end{array} \right\}
   \left\{ \begin{array}{ccc}
    l_{p} & l_{q} & \lambda \\
    \frac{1}{2} & \frac{1}{2} & S \\
    j_{p} & j_{q} & J 
  \end{array} \right\}   
  \left\{ \begin{array}{ccc}
    l_{r} & l_{s} & \lambda ' \\
    \frac{1}{2} & \frac{1}{2} & S \\
    j_{p} & j_{q} & J 
  \end{array} \right\} \nonumber \\
  & & {} \times  \int_{0}^{\infty }dk' k'^{2}
  \tilde{K}\tilde{k} 
  \langle \tilde{k}l(\mathcal{J})SM_{T}|V|
  k'l'(\mathcal{J})SM_{T}\rangle \nonumber \\
  & & {} \times \theta \left( 1-\frac{(k_{p}^{2}
    -\tilde{k}^{2}-\frac{1}{4}\tilde{K}^{2})^{2}}
  {\tilde{k}^{2}\tilde{K}^{2}}\right)A
  \left( \frac{k_{p}^{2}-\tilde{k}^{2}-\frac{1}{4}
    \tilde{K}^{2}}{\tilde{k}\tilde{K}}\right) \nonumber \\
  & & {} \times \theta \left( 1-\frac{(k_{r}^{2}-k'^{2}
    -\frac{1}{4}\tilde{K}^{2})^{2}}{k'^{2}\tilde{K}^{2}}
  \right) A\left( \frac{k_{r}^{2}-k'^{2}
    -\frac{1}{4}\tilde{K}^{2}}{k'\tilde{K}}\right),
\end{eqnarray} 
where we have used the definitions
\begin{align}
  \tilde{K} &= 2\left( \frac{1}{2}\left( k_{c}^{2}+k_{d}^{2}\right) 
  - k'^{2}\right)^{1/2}, \nonumber \\
  \tilde{k}&=\left( \frac{1}{2}\left( k_{a}^{2}+k_{b}^{2}
  -k_{c}^{2}-k_{d}^{2} \right) + k'^{2} \right)^{1/2}.
\end{align}
To remove the Dirac delta distributions, 
we have defined the new integration variables
\begin{align}
  s &= \frac{1}{4}K^{2}+k^{2}-\frac{1}{2}
  \left( k_{p}^{2}+k_{q}^{2}\right), \nonumber \\
  t &= k'^{2}+\frac{1}{4}K^{2}-\frac{1}{2}
  \left( k_{r}^{2}+k_{s}^{2}\right),
\end{align}
and simplified the integral as explained in 
Appendix~\ref{sec:coupled_delta}.

The reference energy is given algebraically in
Eq.~(\ref{eq:ene_ref}). Next we want to write the reference
energy in the coupled partial-wave basis (\ref{eq:coupled_lab}).
A bra vector can be transformed as
\begin{eqnarray} \label{eq:bra_jj}
  \lefteqn{\sum_{m_{s_{p}} m_{s_{q}}}\sum_{m_{t_{p}} m_{t_{q}}} 
    \langle \mathbf{k}_{p}\mathbf{k}_{q}|
    \langle m_{s_{p}}m_{s_{q}}|
    \langle m_{t_{p}}m_{t_{q}}| } \nonumber \\
  & & {} = \sum_{l_{p}l_{q}\atop m_{l_{p}}m_{l_{q}}} 
  \sum_{m_{s_{p}}m_{s_{q}}\atop m_{t_{p}}m_{t_{q}}} 
  \sum_{j_{p}j_{q}\atop m_{j_{p}}m_{j_{q}}} \sum_{JM_{J}}  \nonumber \\
  & & {} \times \langle l_{p}s_{p}m_{l_{p}}m_{s_{p}}|
  j_{p}m_{j_{p}}\rangle
  \langle l_{q}s_{q}m_{l_{q}}m_{s_{q}}|j_{q}m_{j_{q}}\rangle  
  \langle j_{p}j_{q}m_{j_{p}}m_{j_{q}}|JM_{J}\rangle  \nonumber \\
  & & {} \times \langle k_{p}j_{p}l_{p},k_{q}j_{q}l_{q};JM_{J}|
  \langle m_{t_{p}}m_{t_{q}}|Y_{l_{p}m_{l_{p}}}
  ( \mathbf{\hat{k}}_{p})Y_{l_{q}m_{l_{q}}}
  ( \mathbf{\hat{k}}_{q}),
\end{eqnarray}
where we have used completeness relations similar to 
Eqs.~(\ref{eq:lm_complete}) and (\ref{eq:sm_complete}), 
as well as coupling of angular momenta as in 
Eq.~(\ref{eq:ls_jm}). If we use the angular momentum 
expansion (\ref{eq:bra_jj}), the orthogonality of spherical 
harmonics, and the relation \cite{lawson}
\begin{equation}
  \sum_{m_{p}m_{q}}(j_{p}j_{q}m_{p}m_{q}|JM_{J})
  (j_{p}j_{q}m_{p}m_{q}|J'M_{J}')=\delta_{JJ'}\delta_{M_{J}M_{J}'}
\end{equation}
for Clebsch-Gordan coefficients, the reference energy
per particle becomes
\begin{align} \label{eq:ene_A_pt1_ang}
  \frac{E_{REF}}{A} &= \frac{1}{5\pi^{2}}
  \frac{\hbar^{2}k_{F}^{5}}{m\rho } 
  +\frac{1}{2A}\int_{0}^{k_{F}}k_{1}^{2}dk_{1}
  \int_{0}^{k_{F}}k_{2}^{2}dk_{2} \nonumber \\
  & \times \sum_{l_{1}l_{2}\atop j_{1}j_{2}}
  \sum_{J}\sum_{m_{t_{1}}m_{t_{2}}}(2J+1) \nonumber \\
  & \times \langle k_{1}j_{1}l_{1},k_{2}j_{2}l_{2};Jm_{t_{1}}m_{t_{2}}|
  \tilde{v}
  |k_{1}j_{1}l_{1},k_{2}j_{2}l_{2};Jm_{t_{1}}m_{t_{2}}\rangle ,
\end{align}
where $A$ is the number of nucleons and $\rho =A/V$ is the 
nucleon density. In Eq.~(\ref{eq:ene_A_pt1_ang}), we
need an explicit expression for the number of particles,
which in principle is an infinitely large number.
When evaluating Eq.~(\ref{eq:ene_A_pt1_ang}), we hope that
the expansion in partial waves may be truncated after
a reasonably small number of angular momenta.
On the other hand, above we have assumed that the 
momentum points are infinitely dense. The total number
of particles therefore becomes infinitely large.
Assuming that convergence has been obtained
in the partial-wave expansion, the number of nucleons is
\begin{align} \label{eq:number_inf}
  A &= \sum_{m_{s}}\sum_{m_{t}}\frac{V}{(2\pi )^{3}}
  \int_{|\mathbf{k}| \leq k_{F}}d\mathbf{k} \nonumber \\
  &= V \frac{2k_{F}^{3}}{3\pi^{2}}
\end{align}
in the special case of symmetric nuclear matter. 
A practical problem arises here: The expression for
the number of particles (\ref{eq:number_inf}) depends on 
the volume of the nuclear-matter system. When deriving
Eqs.~(\ref{eq:int_v_relcm2lab3}) and (\ref{eq:ene_A_pt1_ang}),
we have assumed that the volume is infinitely large. 
Unfortunately, there is no volume term in 
Eq.~(\ref{eq:int_v_relcm2lab3}) that cancels the volume
in the expression for the number of particles. Dimensional
analysis of the given expressions also gives correct
units for the energy per particle. We have therefore
not been able to calculate the binding energy of
nuclear matter using the approach described above.

The problem with infinite numbers can be avoided 
by discretizing the radial momentum explicitly 
\cite{papenbrock_notes}. This
corresponds to using eigenstates of a spherical well.
Following the textbook of Liboff \cite{liboff}, a
spherical well has the single-particle wave functions
\begin{align} \label{eq:psi_radial_discr}
  \psi_{nlm_{l}}(r, \theta_{r}, \phi_{r}) \equiv
  \langle \mathbf{r}| nlm_{l}\rangle = 
  j_{l}(k_{nl}r)Y_{lm_{l}}(\theta_{r}, \phi_{r})
\end{align} 
and single-particle energies
\begin{align}
  \varepsilon_{nl} = \frac{\hbar^{2}k_{nl}^{2}}{2m},
\end{align}
where $j_{l}(x)$ is the spherical Bessel function.  
The discrete momenta $k_{nl}$ are obtained from the 
Dirichlet boundary condition
\begin{align}
  \psi_{nlm_{l}}(R, \theta_{r}, \phi_{r}) = 0,
\end{align}
where $R$ is the radius of the spherical box. 
Including spin and isospin, the single-particle states
may be written as
\begin{align} \label{eq:discr_njl}
  \psi_{njm_{j}lsm_{t}}(r, \theta_{r}, \phi_{r}) = 
  \sum_{m_{l}m_{s}}\langle lm_{l}sm_{s}|jm_{j}ls\rangle
  j_{l}(k_{nl}r)Y_{lm_{l}}(\theta_{r}, \phi_{r})|m_{s}m_{t}\rangle ,
\end{align}
where spin and orbital angular momentum have been
coupled to a total single-particle angular momentum.

Similarly as in a more general case, the uncorrelated 
Fermi vacuum state is constructed by choosing the $A$
single-particle states $|njm_{j}lm_{t}\rangle $ with the 
lowest single-particle energies. To be able to use
single-reference coupled-cluster theory, the number of 
particles must be chosen such that all energy shells 
below the Fermi level are fully occupied. Similarly, one may
choose the unoccupied single-particle states such that
a given number of energy shells above the Fermi level
are completely filled. 
Given a density $\rho $ and a number of nucleons $A$,
the radius is
\begin{align}
  R = \left( \frac{3A}{4\pi \rho }\right)^{1/3}.
\end{align}
The discrete spherical Bessel single-particle basis 
is useful for nuclear-matter studies if the energy
per particle converges fast with the number of occupied
and unoccupied single-particle states. 

Above we concluded that the continuous radial basis
$|kjm_{j}lm_{t}\rangle $ gives problems with infinite numbers, 
and we therefore want to replace this basis with the
discrete equivalent $|njm_{j}lm_{t}\rangle $, which 
corresponds to eigenstates of a spherical well. 
Our task is now to rewrite the transformation given in 
Eq.~(\ref{eq:rcm2lab_pw3}) into a discrete form. The 
only part that is problematic is the vector bracket,
which is defined for continuous radial states. One
alternative is to derive a new discrete counterpart to the
vector bracket. However, there is another well-known 
transformation coefficient between discrete laboratory
and RCM states, which is the Moshinsky bracket \cite{lawson}, 
defined for harmonic oscillator states. If we use
Moshinsky coefficients to transform the interaction
matrix elements to laboratory frame harmonic oscillator
states, matrix elements in the discrete radial basis
$|njm_{j}lm_{t}\rangle $ can be obtained as 
\begin{align}
\langle pq|v|rs\rangle = 
\sum_{\alpha \leq \beta \atop \gamma \leq \delta }
\langle pq|\alpha \beta \rangle 
\langle \alpha \beta |v|\gamma \delta \rangle 
\langle \gamma \delta |rs\rangle ,
\end{align}
where $p,q,r,s$ represent radial single-particle states
and $\alpha , \beta , \gamma , \delta $ harmonic
oscillator states. Hagen \emph{et al.}~\cite{hagen2006} 
have used a similar transformation from the harmonic oscillator 
basis to a Gamow basis. As given in 
Ref.~\cite{hagen2006}, the two-particle overlaps are 
\begin{align}
  \langle pq|\alpha \beta \rangle = \left\{ \begin{array}{ll}
    \frac{\langle p|\alpha \rangle \langle q|\beta \rangle 
      - (-1)^{J-j_{\alpha }-j_{\beta }}\langle p|\beta \rangle 
    \langle q|\alpha \rangle }
         {\sqrt{ (1+\delta_{pq})(1+\delta_{\alpha \beta })
           }}, & \text{ if } m_{t_{p}} = m_{t_{q}}, \\
         \langle p|\alpha \rangle \langle q|\beta \rangle ,
         & \text{ if } m_{t_{p}} \neq m_{t_{q}}.
         \end{array} \right.
\end{align}
When doing the transformation from the harmonic oscillator 
basis to the discrete radial single-particle basis, 
the single-particle overlaps are
\begin{align}
  \langle p|\alpha \rangle = 
  \delta_{j_{p}j_{\alpha}}\delta_{l_{p}l_{\alpha }}
  \delta_{m_{t_{p}}m_{t_{\alpha }}} \int_{0}^{R}dr r^{2}
  j_{l_{p}}(k_{n_{p}l_{p}}r)R_{n_{\alpha }l_{\alpha }}(Cr),
\end{align}
where
\[
C = \sqrt{m\omega /\hbar },
\]
the variable $\omega $ is the harmonic oscillator strength,
and $R_{n_{\alpha }l_{\alpha }}(Cr)$ is the radial part of the
harmonic oscillator wave function. Numerical calculations
should include sufficiently many harmonic oscillator 
single-particle states to get a result that is only
weakly dependent on the oscillator strength. Similarly
as done in Ref.~\cite{hagen2006} with the Gamow basis,
we calculate the CM correction term directly in the 
Bessel basis.




%% \begin{align}
%%   u_{j} = \left\{ \begin{array}{ll}
%%     -\frac{\omega_{k_{j}}k_{j}^{2}Q_{pp}(k_{j}, K, k_{F})}{\Delta \varepsilon_{av}(k_{j}, k_{0}, K)}, & \text{ if } j \leq N, \\
%%       \sum_{p=1}^{N} \frac{\omega_{k_{p}}k_{0}^{2}Q_{pp}(k_{0}, K, k_{F})}{\Delta \varepsilon_{av}(k_{p}, k_{0}, K)}, & \text{ if } j = N+1,
%%     \end{array} \right. 
%% \end{align}  
 




\vspace{4cm}
\begin{itemize}
\item Show results of bad convergence, explain problems with dimensionality
\item Discuss the spherical Bessel basis. Show results of bad convergence
\end{itemize}


\section{Applications for the homogeneous electron gas} \label{sec:ccheg}

Another important homogeneous system beside nuclear matter 
is the electron gas. In the earliest CC studies of the 
electron gas \cite{singal1973,freeman1977,bishop1978,
freeman1978,bishop1982,freeman1983}, the system was treated 
at the thermodynamic limit. In Paper III, we approximated 
infinite nuclear matter using finite boxes of nucleons. 
Some of the most accurate calculations of the electron gas
\cite{ceperley1980,tanatar_ceperley_1989,lopezrios2006,
shepherd_2012a} have been done using finite boxes and Monte
Carlo methods. Recently, CC theory has been applied in 
several studies \cite{shepherd_2012b,shepherd2013a,
shepherd2013b,shepherd2013c,roggero2013} to finite-particle 
approximations of the homogeneous electron gas. 
 Freeman has done CC studies of the two-dimensional 
electron gas including only ring \cite{freeman1978} or 
only particle-particle ladder \cite{freeman1983} diagrams 
from the CCD approximation, but to the best of our knowledge, 
nobody has done a complete CCD calculation of the 
two-dimensional homogeneous electron gas. 
As an extension of the CC calculations by Shepherd 
\emph{et al.}~\cite{shepherd_2012b,
shepherd2013a,shepherd2013b,shepherd2013c}
and Roggero \emph{et al.}~\cite{roggero2013}, we study
the two-dimensional electron gas in the CCD approximation
using a finite number of particles.
We validate our methods by comparing results 
for the three-dimensional electron gas with Shepherd 
\emph{et al.} \cite{shepherd_2012b,shepherd_2012c}.
The single-particle basis consists of 
state vectors as given in Eq.~(\ref{eq:sp_mom_cart}), with the 
momentum discretized in cartesian coordinates. 
Similarly as in Refs.~\cite{shepherd_2012b,shepherd2013a,
shepherd2013b,shepherd2013c,roggero2013}, we neglect 
finite-size effects. We discuss different ways to approximate
finite-size effects later in this section.

%%  Finite hypercubes and discretized cartesian momentum vectors 
%% have also been used to model infinite matter with other 
%% many-body methods: For the electron gas, there have been done calculations with the full 
%% configuration interaction quantum Monte Carlo method 
%% \cite{shepherd_2012a,shepherd_2012b,shepherd_2012c},
%% second-order perturbation theory and random-phase approximation
%% \cite{shepherd_2012b}, and the same single-particle basis
%% has been used with the auxiliary-field diffusion Monte Carlo
%% \cite{schmidt_1999,gandolfi2007,gandolfi_2009}, Green's 
%% function Monte Carlo, and variational chain summation 
%% \cite{carlson_2003} methods to study symmetric nuclear 
%% and/or pure neutron matter.  

%% We have implemented CC theory for the
%% two- and three-dimensional electron gas using finite hypercubic 
%% cells. 
%% Our studies of the electron gas is an extension of 
%% the CC calculations by Shepherd \emph{et al.}~\cite{shepherd_2012b,
%% shepherd2013a,shepherd2013b,shepherd2013c}
%% and Roggero \emph{et al.}~\cite{roggero2013}
%% to two-dimensional systems. Similarly as in Ref.~\cite{shepherd_2012b}, 
%% we approximate the electron gas by a finite number of electrons 
%% in a finite hypercube.

The homogeneous electron gas is defined as an infinite system
consisting of electrons distributed with a constant density 
through the entire real space. The system is assumed to 
contain a constant positive background charge which cancels
the negative charges of the electrons. The Hamiltonian 
operator of the homogeneous electron gas can be written as
\cite{fetter}
\begin{align}
  \hat{H} = \hat{H}_{\text{kin}} + \hat{H}_{ee} + \hat{H}_{eb} 
  + \hat{H}_{bb},
\end{align}
where $\hat{H}_{\text{kin}}$ is the kinetic-energy operator,
$\hat{H}_{ee}$ models the electron-electron interaction,
$\hat{H}_{eb}$ represents the interaction between electrons 
and the positive background charge, and the operator 
$\hat{H}_{bb}$ gives the interaction energy of the background
charge with itself. 


\begin{figure} 
  \centering
  \includegraphics[scale=1.0]{figures/ewald/ewald-crop.pdf}
  \caption{The electron gas is approximated using a finite 
    box (marked gray in the figure), with periodic boundary 
    conditions. The Ewald interaction is an effective Coulomb 
    interaction, which is obtained by summing the contribution
    from infinitely many image charges. The translational vector
    is $\mathbf{R} = L_{x}n_{x}\mathbf{u}_{x} + L_{y}n_{y}
    \mathbf{u}_{y}$, where $n_{x}$ and $n_{y}$ are integers.}
  \label{fig:ewald}
\end{figure}


Similarly as in Refs.~\cite{shepherd_2012b,shepherd2013a,
shepherd2013b,shepherd2013c,roggero2013}, we approximate
the homogeneous electron gas by a finite hypercube filled
with electrons. Using periodic boundary conditions, we should 
also take into account interactions between electrons in the
finite box and image charges in other boxes in the infinite 
space. The setting is illustrated in Figure \ref{fig:ewald}.
The summation over an infinite number of Coulomb interaction 
terms can be efficiently calculated using, for example, 
Ewald's summation technique \cite{ewald1921,fraser1996}. 
In Ewald's approach, the Coulomb interaction is splitted into a 
short-ranged part, which is calculated in real space,
and a long-ranged part, which is summed in Fourier space. 
The kinetic energy operator is 
\begin{align}
  \hat{H}_{\text{kin}} = -\frac{\hbar^{2}}{2m}\sum_{i=1}^{N}
  \nabla_{i}^{2},
\end{align}
where the sum is taken over all particles in the finite
box. The Ewald electron-electron interaction operator 
can be written as \cite{drummond2008}
\begin{align}
  \hat{H}_{ee} = \sum_{i<j}^{N} 
  v_{E}\left( \mathbf{r}_{i}-\mathbf{r}_{j}\right)
  + \frac{1}{2}Nv_{0},
\end{align}
where $v_{E}(\mathbf{r})$ is the effective two-body 
interaction and $v_{0}$ is the self-interaction, defined 
as $v_{0} = \lim_{\mathbf{r} \rightarrow 0} 
\left\{ v_{E}(\mathbf{r}) - 1/r\right\} $. The negative 
electron charges are neutralized by a positive, homogeneous 
background charge. Fraser \emph{et al.} explain 
\cite{fraser1996} how the
electron-background and background-background terms, 
$\hat{H}_{eb}$ and $\hat{H}_{bb}$, vanish
when using Ewald's interaction for the three-dimensional
electron gas. Using the same arguments, one can show that
these terms are also zero in the corresponding 
two-dimensional system. 

In the three-dimensional electron gas, the Ewald 
interaction is \cite{drummond2008}
\begin{align}
  v_{E}(\mathbf{r}) = \sum_{\mathbf{k} \neq \mathbf{0}}
  \frac{4\pi }{L^{3}k^{2}}e^{i\mathbf{k}\cdot \mathbf{r}}
  e^{-\eta^{2}k^{2}/4}
  + \sum_{\mathbf{R}}\frac{1}{\left| \mathbf{r}
    -\mathbf{R}\right| } \erfc \left( \frac{\left| 
    \mathbf{r}-\mathbf{R}\right|}{\eta }\right)
  - \frac{\pi \eta^{2}}{L^{3}},
\end{align}
where $L$ is the box side length, $\erfc(x)$ is the 
complementary error function, and $\eta $ is a free
parameter that can take any value in the interval 
$(0, \infty )$. The translational vector 
\begin{align}
  \mathbf{R} = L\left(n_{x}\mathbf{u}_{x} + n_{y}
  \mathbf{u}_{y} + n_{z}\mathbf{u}_{z}\right) ,
\end{align}
where $\mathbf{u}_{i}$ is the unit vector for dimension $i$,
is defined for all integers $n_{x}$, $n_{y}$, and 
$n_{z}$. These vectors are used to obtain all image
cells in the entire real space. 
The parameter $\eta $ decides how 
the Coulomb interaction is divided into a short-ranged
and long-ranged part, and does not alter the total
function. However, the number of operations needed
to calculate the Ewald interaction with a desired 
accuracy depends on $\eta $, and $\eta $ is therefore 
often chosen to optimize the convergency as a function
of the simulation-cell size \cite{drummond2008}. In
our calculations, we choose $\eta $ to be an infinitesimally
small positive number, similarly as was done in 
Refs.~\cite{shepherd_2012b,shepherd2013a,shepherd2013b,
shepherd2013c,roggero2013}.
This gives an interaction that is evaluated only in
Fourier space. 

When studying the two-dimensional electron gas, we
use an Ewald interaction that is quasi two-dimensional.
The interaction is derived in three dimensions, with 
Fourier discretization in only two dimensions 
\cite{parry1975,parry1976,wood2004}. The Ewald effective
interaction has the form \cite{wood2004}
\begin{align}
  v_{E}(\mathbf{r}) &= \sum_{\mathbf{k} \neq \mathbf{0}} 
  \frac{\pi }{L^{2}k}\left\{ e^{-kz} \erfc \left(
  \frac{\eta k}{2} - \frac{z}{\eta }\right)
  + e^{kz}\erfc \left( \frac{\eta k}{2} + \frac{z}{\eta }
  \right) \right\} e^{i\mathbf{k}\cdot \mathbf{r}_{xy}} 
  \nonumber \\
  & + \sum_{\mathbf{R}}\frac{1}{\left| \mathbf{r}-\mathbf{R}
    \right| } \erfc \left( \frac{\left| \mathbf{r}-\mathbf{R}
    \right|}{\eta }\right) \nonumber \\
  & - \frac{2\pi}{L^{2}}\left\{ z\erf \left( \frac{z}{\eta }
  \right) + \frac{\eta }{\sqrt{\pi }}e^{-z^{2}/\eta^{2}}\right\},
\end{align}
where the Fourier vectors $\mathbf{k}$ and the position vector
$\mathbf{r}_{xy}$ are defined in the $(x,y)$ plane. When
applying the interaction $v_{E}(\mathbf{r})$ to two-dimensional
systems, we set $z$ to zero. Similarly as in the 
three-dimensional case, and as suggested in 
Ref.~\cite{drummond2008} for two dimensions, also here we 
choose $\eta $ to approach zero from above. The resulting 
Fourier-transformed interaction is
\begin{align}
  v_{E}^{\eta = 0, z = 0}(\mathbf{r}) = \sum_{\mathbf{k} \neq \mathbf{0}} 
  \frac{2\pi }{L^{2}k}e^{i\mathbf{k}\cdot \mathbf{r}_{xy}}. 
\end{align}
The self-interaction $v_{0}$ is a constant that can be 
included in the reference energy.


In the three-dimensional electron gas, the antisymmetrized
matrix elements are
\begin{align} \label{eq:vmat_3dheg}
  & \langle \mathbf{k}_{p}m_{s_{p}}\mathbf{k}_{q}m_{s_{q}}
  |\tilde{v}|\mathbf{k}_{r}m_{s_{r}}\mathbf{k}_{s}m_{s_{s}}\rangle_{AS} 
  \nonumber \\
  & = \frac{4\pi }{L^{3}}\delta_{\mathbf{k}_{p}+\mathbf{k}_{q},
    \mathbf{k}_{r}+\mathbf{k}_{s}}\left\{ 
  \delta_{m_{s_{p}}m_{s_{r}}}\delta_{m_{s_{q}}m_{s_{s}}}
  \left( 1 - \delta_{\mathbf{k}_{p}\mathbf{k}_{r}}\right) 
  \frac{1}{|\mathbf{k}_{r}-\mathbf{k}_{p}|^{2}}
  \right. \nonumber \\
  & \left. - \delta_{m_{s_{p}}m_{s_{s}}}\delta_{m_{s_{q}}m_{s_{r}}}
  \left( 1 - \delta_{\mathbf{k}_{p}\mathbf{k}_{s}} \right)
  \frac{1}{|\mathbf{k}_{s}-\mathbf{k}_{p}|^{2}} 
  \right\} ,
\end{align}
where the Kronecker delta functions 
$\delta_{\mathbf{k}_{p}\mathbf{k}_{r}}$ and
$\delta_{\mathbf{k}_{p}\mathbf{k}_{s}}$ ensure that the 
contribution with zero momentum transfer vanishes.
Similarly, the matrix elements for the two-dimensional
electron gas are
\begin{align} \label{eq:vmat_2dheg}
  & \langle \mathbf{k}_{p}m_{s_{p}}\mathbf{k}_{q}m_{s_{q}}
  |v|\mathbf{k}_{r}m_{s_{r}}\mathbf{k}_{s}m_{s_{s}}\rangle_{AS} 
  \nonumber \\
  & = \frac{2\pi }{L^{2}}
  \delta_{\mathbf{k}_{p}+\mathbf{k}_{q},\mathbf{k}_{r}+\mathbf{k}_{s}}
  \left\{ \delta_{m_{s_{p}}m_{s_{r}}}\delta_{m_{s_{q}}m_{s_{s}}} 
  \left( 1 - \delta_{\mathbf{k}_{p}\mathbf{k}_{r}}\right)
  \frac{1}{
    |\mathbf{k}_{r}-\mathbf{k}_{p}|} \right.
  \nonumber \\
  & - \left. \delta_{m_{s_{p}}m_{s_{s}}}\delta_{m_{s_{q}}m_{s_{r}}}
  \left( 1 - \delta_{\mathbf{k}_{p}\mathbf{k}_{s}}\right)
  \frac{1}{ 
    |\mathbf{k}_{s}-\mathbf{k}_{p}|}
  \right\} ,
\end{align}
where the single-particle momentum vectors $\mathbf{k}_{p,q,r,s}$
are now defined in two dimensions. In our calculations, the 
self-interaction constant is included in the reference 
energy, as in Eq.~(\ref{eq:ene_ref}). We therefore get the 
Fock-operator matrix elements 
\begin{align}
  \langle \mathbf{k}_{p}|f|\mathbf{k}_{q} \rangle
  = \frac{\hbar^{2}k_{p}^{2}}{2m}\delta_{\mathbf{k}_{p},
  \mathbf{k}_{q}} + \sum_{\mathbf{k}_{i}}\langle 
  \mathbf{k}_{p}\mathbf{k}_{i}|v|\mathbf{k}_{q}
  \mathbf{k}_{i}\rangle_{AS}.
  \label{eq:fock_heg}
\end{align}
In Ref.~\cite{shepherd2013b}, the matrix elements were 
defined with the self-interaction constant included in the
two-body interaction. This gives Fock-operator matrix 
elements with a gap constant. The definition used in 
Ref.~\cite{shepherd2013b} may give numerically more
stable calculations, as the gap constant 
prevents the energy denominator from becoming too small
in the vicinity of the Fermi surface. However, when using
Fock matrix elements as defined in Eq.~(\ref{eq:fock_heg}), 
the energy denominator in the CC equations never vanishes 
unless the numerator is zero. We discuss this point in 
Paper II.


%% \begin{align} 
%%   \hat{H}_{eb} = -\sum_{i=1}^{N}\int d\mathbf{r}
%%   \rho(\mathbf{r}) v\left( 
%%   |\mathbf{r}-\mathbf{r}_{i}|\right)
%% \end{align}
%% models the interaction between the electrons and the 
%% positive background charge, and 
%% \begin{align}
%%   \hat{H}_{bb} = \frac{1}{2}\int d\mathbf{r}
%%   \int d\mathbf{r}'\rho(\mathbf{r})\rho(\mathbf{r}')
%%   v\left( |\mathbf{r}-\mathbf{r}'|\right)
%% \end{align}
%% represents the interaction between the background 
%% charge and itself. Above, $\rho(\mathbf{r}) $ is the 
%% electron density and $v(|\mathbf{r}|)$ is the 
%% Coulomb interaction. In the following, we replace
%% the Coulomb interaction by 
%% \begin{align}
%%   v(r) = e^{2}\frac{\exp\left( -\mu r\right)}{r},
%% \end{align}
%% where the Coulomb potential is obtained at the limit 
%% in which $\mu $ approaches zero.

%% Let us write the Hamiltonian operator in the discrete
%% momentum basis (\ref{eq:sp_mom_cart}). Fetter and Walecka 
%% show \cite{fetter} that the total Hamiltonian
%% for the three-dimensional system can be written as
%% \begin{align} \label{eq:ham_heg2}
%%   \hat{H} = \hat{H}_{0} + \hat{H}_{I} - 
%%   \frac{2\pi e^{2}N^{2}}{L^{3}\mu^{2}},
%% \end{align} 
%% where the kinetic energy operator is
%% \begin{align} \label{eq:h0_mom}
%%   \hat{H}_{0} = \sum_{\mathbf{k}_{p}}\sum_{m_{s_{p}}}
%%   \frac{\hbar^{2}k_{p}^{2}}{2m}
%%   a_{\mathbf{k}_{p}m_{s_{p}}}^{\dagger }a_{\mathbf{k}_{p}m_{s_{p}}},
%% \end{align}
%% the two-body interaction operator is
%% \begin{align} \label{eq:hi_mom}
%%   \hat{H}_{I} = \sum_{\mathbf{k}_{p}\mathbf{k}_{q} \atop 
%%   \mathbf{k}_{r}\mathbf{k}_{s}}
%%   \sum_{m_{s_{p}}m_{s_{q}}\atop m_{s_{r}}m_{s_{s}}}
%%   \langle \mathbf{k}_{p}m_{s_{p}}\mathbf{k}_{q}m_{s_{q}}
%%   |v|\mathbf{k}_{r}m_{s_{r}}\mathbf{k}_{s}m_{s_{s}}\rangle_{AS}
%%   a_{\mathbf{k}_{p}m_{s_{p}}}^{\dagger }
%%   a_{\mathbf{k}_{q}m_{s_{q}}}^{\dagger }
%%   a_{\mathbf{k}_{s}m_{s_{s}}}a_{\mathbf{k}_{r}m_{s_{r}}},
%% \end{align}
%% the variable $N$ represents the number of electrons, 
%% and $L$ is the side length of the cube. The interaction 
%% matrix elements are obtained as Fourier transformations
%% of the Yukawa interaction.
%% Further, they show that the part of the two-body interaction 
%% operator with zero momentum transfer gives a constant 
%% \begin{align} \label{eq:constants_3deg}
%%   \frac{2\pi e^{2}}{L^{3}\mu^{2}}\left( N^{2}-N\right),  
%% \end{align}
%% where the the part with $N^{2}$ exacly cancels
%% the constant in Eq.~(\ref{eq:ham_heg2}). The 
%% Hamiltonian operator for the three-dimensional electron
%% gas may now be written as
%% \begin{align} \label{eq:ham_3d_q0}
%%   \hat{H} = \hat{H}_{0} + \hat{H}_{I}^{\mathbf{k}_{\text{rel}}\neq 0} 
%%   - \frac{2\pi e^{2}N}{L^{3}\mu^{2}},
%% \end{align}
%% where $\hat{H}_{I}^{\mathbf{k}_{\text{rel}}\neq 0}$ is an interaction
%% operator in which states that have zero momentum transfer 
%% are removed.
%% The latter operator can be written as  
%% \begin{align} \label{eq:hi_mom}
%%   \hat{H}_{I}^{\mathbf{k}_{\text{rel}}\neq 0} &= 
%%   \sum_{\mathbf{k}_{p}\mathbf{k}_{q} \atop \mathbf{k}_{r}\mathbf{k}_{s}}
%%   \sum_{m_{s_{p}}m_{s_{q}}\atop m_{s_{r}}m_{s_{s}}}
%%   \langle \mathbf{k}_{p}m_{s_{p}}\mathbf{k}_{q}m_{s_{q}}
%%   |\tilde{v}|\mathbf{k}_{r}m_{s_{r}}\mathbf{k}_{s}m_{s_{s}}\rangle_{AS}
%%   \nonumber \\
%%   & \times a_{\mathbf{k}_{p}m_{s_{p}}}^{\dagger }
%%   a_{\mathbf{k}_{q}m_{s_{q}}}^{\dagger }
%%   a_{\mathbf{k}_{s}m_{s_{s}}}a_{\mathbf{k}_{r}m_{s_{r}}},
%% \end{align}
%% with the antisymmetrized interaction matrix elements
%% \begin{align} \label{eq:vmat_3dheg}
%%   & \langle \mathbf{k}_{p}m_{s_{p}}\mathbf{k}_{q}m_{s_{q}}
%%   |\tilde{v}|\mathbf{k}_{r}m_{s_{r}}\mathbf{k}_{s}m_{s_{s}}\rangle_{AS} 
%%   \nonumber \\
%%   & = \frac{e^{2}}{L^{3}}\delta_{\mathbf{k}_{p}+\mathbf{k}_{q},
%%     \mathbf{k}_{r}+\mathbf{k}_{s}}\left\{ 
%%   \delta_{m_{s_{p}}m_{s_{r}}}\delta_{m_{s_{q}}m_{s_{s}}}
%%   \left( 1 - \delta_{\mathbf{k}_{p}\mathbf{k}_{r}}\right) 
%%   \frac{4\pi }{\mu^{2} + |\mathbf{k}_{r}-\mathbf{k}_{p}|^{2}}
%%   \right. \nonumber \\
%%   & \left. - \delta_{m_{s_{p}}m_{s_{s}}}\delta_{m_{s_{q}}m_{s_{r}}}
%%   \left( 1 - \delta_{\mathbf{k}_{p}\mathbf{k}_{s}} \right)
%%   \frac{4\pi }{\mu^{2} + |\mathbf{k}_{s}-\mathbf{k}_{p}|^{2}} 
%%   \right\} .
%% \end{align}
%% The Kronecker delta functions 
%% $\delta_{\mathbf{k}_{p}\mathbf{k}_{r}}$ and
%% $\delta_{\mathbf{k}_{p}\mathbf{k}_{s}}$ ensure that the 
%% contribution with zero momentum transfer vanishes.
%% If the side length $L$ of the cube is finite, the last term 
%% of Eq.~(\ref{eq:ham_3d_q0}) remains finite for nonzero 
%% $\mu $ values, and becomes infinite as $\mu $ approaches
%% zero. On the other hand, if we first let 
%% $L \rightarrow \infty $ and thereafter take the limit 
%% $\mu \rightarrow 0$, the constant in 
%% Eq.~(\ref{eq:ham_3d_q0}) vanishes in the expression 
%% for the energy per particle \cite{fetter}. Observe 
%% that the last term of Eq.~(\ref{eq:ham_3d_q0}) gives
%% just a constant in the total energy per particle, 
%% and does not affect the correlation energy of the system. 

%% Following a similar approach as outlined above for the 
%% three-dimensional case, the Hamiltonian of a two-dimensional 
%% electron gas can be written as 
%% \begin{align} \label{eq:ham_2deg}
%%   \hat{H} = \hat{H}_{0} + \hat{H}_{I}^{\mathbf{k}_{\text{rel}}\neq 0} 
%%   - \frac{\pi e^{2}N}{L^{2}\mu },
%% \end{align}  
%% where $\hat{H}_{0}$ and $\hat{H}_{I}^{\mathbf{k}_{\text{rel}}\neq 0}$ 
%% are defined as in Eqs.~(\ref{eq:h0_mom}) and 
%% (\ref{eq:hi_mom}), but with two-dimensional momentum vectors 
%% and different interaction matrix elements. Similarly as in 
%% the three-dimensional case \cite{fetter}, the momentum-space 
%% interaction matrix elements for the two-dimensional system 
%% are
%% \begin{align} \label{eq:vmat_2dheg}
%%   & \langle \mathbf{k}_{p}m_{s_{p}}\mathbf{k}_{q}m_{s_{q}}
%%   |v|\mathbf{k}_{r}m_{s_{r}}\mathbf{k}_{s}m_{s_{s}}\rangle_{AS} 
%%   \nonumber \\
%%   & = \frac{e^{2}}{L^{2}} 
%%   \delta_{m_{s_{p}}m_{s_{r}}}\delta_{m_{s_{q}}m_{s_{s}}} 
%%   \int d\mathbf{r}\int d\mathbf{r}'
%%   \exp\left( -i\mathbf{k}_{p}\cdot \mathbf{r}\right)
%%   \exp\left( -i\mathbf{k}_{q}\cdot \mathbf{r}'\right) 
%%   \nonumber \\
%%   & \times 
%%   \frac{\exp\left( -\mu |\mathbf{r}-\mathbf{r}'|\right) }
%%        {|\mathbf{r}-\mathbf{r}'|} 
%%        \exp\left( i\mathbf{k}_{r}\cdot \mathbf{r}\right)
%%        \exp\left( i\mathbf{k}_{s}\cdot \mathbf{r}'\right) 
%%        \nonumber \\
%%        & = \frac{e^{2}}{L^{2}} 
%%        \delta_{m_{s_{p}}m_{s_{r}}}\delta_{m_{s_{q}}m_{s_{s}}} 
%%        \delta_{\mathbf{k}_{p}+\mathbf{k}_{q},\mathbf{k}_{r}+\mathbf{k}_{s}}
%%        \int d\mathbf{r} 
%%        \exp\left( i(\mathbf{k}_{r}-\mathbf{k}_{p})\cdot 
%%        \mathbf{r}\right) \frac{\exp \left( -\mu r\right) }{r},
%% \end{align} 
%% where $r$ is the radial coordinate of the position vector.
%% Using Eqs.~(9.1.21) and (9.1.10) of Ref.~\cite{abramowitz}, 
%% that is,
%% \begin{align}
%%   \int_{0}^{2\pi }\exp\left( ipr \cos \phi \right) 
%%   = 2\pi J_{0}(pr)
%% \end{align}
%% and 
%% \begin{align}
%%   J_{0}(x) = J_{0}(-x),
%% \end{align}
%% where $J_{0}(x)$ is the Bessel function of the first kind, as
%% well as the relation 
%% \begin{align}
%%   \int_{0}^{\infty }dr \exp \left( -\mu r\right) J_{0}(pr)
%%   = \frac{1}{\sqrt{\mu^{2} + p^{2}}},
%% \end{align}
%% which is given in Eq.~(6.611) of Ref.~\cite{gradshteyn},
%% the interaction matrix elements of 
%% $\hat{H}_{I}^{\mathbf{k}_{\text{rel}}\neq 0}$ in 
%% Eq.~(\ref{eq:ham_2deg}) may be written as
%% \begin{align}
%%   & \langle \mathbf{k}_{p}m_{s_{p}}\mathbf{k}_{q}m_{s_{q}}
%%   |v|\mathbf{k}_{r}m_{s_{r}}\mathbf{k}_{s}m_{s_{s}}\rangle_{AS} 
%%   \nonumber \\
%%   & = \frac{e^{2}}{L^{2}}
%%   \delta_{\mathbf{k}_{p}+\mathbf{k}_{q},\mathbf{k}_{r}+\mathbf{k}_{s}}
%%   \left\{ \delta_{m_{s_{p}}m_{s_{r}}}\delta_{m_{s_{q}}m_{s_{s}}} 
%%   \left( 1 - \delta_{\mathbf{k}_{p}\mathbf{k}_{r}}\right)
%%   \frac{2\pi }{\sqrt{\mu^{2} 
%%       + |\mathbf{k}_{r}-\mathbf{k}_{p}|^{2}}} \right.
%%   \nonumber \\
%%   & - \left. \delta_{m_{s_{p}}m_{s_{s}}}\delta_{m_{s_{q}}m_{s_{r}}}
%%   \left( 1 - \delta_{\mathbf{k}_{p}\mathbf{k}_{s}}\right)
%%   \frac{2\pi }{\sqrt{\mu^{2} 
%%       + |\mathbf{k}_{s}-\mathbf{k}_{p}|^{2}}}
%%   \right\} .
%% \end{align}
%% Interaction matrix elements for the Coulomb interaction,
%% with $\mu $ set to zero, have been derived in a similar
%% way in for example the textbook by Hamaguchi \cite{hamaguchi}.
%% As we explain in the next paragraph, we study only electrons
%% in finite boxes. \textbf{Strictly, $\dots $}    

When using periodic boundary conditions, the 
discrete-momentum single-particle basis functions 
\[
\phi_{\mathbf{k}}(\mathbf{r}) =
e^{i\mathbf{k}\cdot \mathbf{r}}/L^{d/2}
\]
are associated with 
the single-particle energy   
\begin{align}
  \varepsilon_{n_{x}, n_{y}} = \frac{\hbar^{2}}{2m} 
  \left( \frac{2\pi }{L}\right)^{2}
  \left( n_{x}^{2} + n_{y}^{2}\right)
\end{align}
for two-dimensional sytems and 
\begin{align}
  \varepsilon_{n_{x}, n_{y}, n_{z}} = \frac{\hbar^{2}}{2m}
  \left( \frac{2\pi }{L}\right)^{2}
  \left( n_{x}^{2} + n_{y}^{2} + n_{z}^{2}\right)
\end{align} 
for three-dimensional systems. Similarly as in, for example, 
Refs.~\cite{shepherd_2012a,roggero2013,hagen2014}, we choose 
the single-particle basis such that both the occupied and 
unoccupied single-particle spaces have a closed-shell 
structure. This means that all single-particle states 
corresponding to energies below a chosen cutoff are
included in the basis. We study only the unpolarized spin
phase, in which all orbitals are occupied with one spin-up 
and one spin-down electron. Table \ref{tab:orbitals_2dheg}
shows the lowest-lying spin-orbitals and the cumulated
numbers of single-particle states for a two-dimensional
electron box with periodic boundary conditions.
The CCD energy is obtained by solving the energy
equations (\ref{eq:ene_ref}) and (\ref{eq:ene_ccd})
together with the $\hat{T}_{2}$ amplitude equation
(\ref{eq:t2ampl}) using the discrete momentum basis.

\begin{table}
  \begin{center}
  \begin{tabular}{c|c|c|c|c}
    $n_{x}^{2}+n_{y}^{2}$ & $n_{x}$ & $n_{y}$ & 
    $N_{\uparrow \downarrow }$ & $N_{\uparrow \uparrow }$ \\
    \hline
    0 & 0 & 0 & 2 & 1 \\
    \hline
    1 & -1 &  0 & & \\
    &  1 &  0 & & \\
    &  0 & -1 & & \\
    &  0 &  1 & 10 & 5 \\
    \hline
    2  & -1 & -1 & & \\
    & -1 &  1 & & \\
    &  1 & -1 & & \\
    &  1 &  1 & 18 & 9 \\
    \hline
    4  & -2 & 0 & & \\
    &  2 & 0 & & \\
    &  0 & -2 & & \\
    &  0 &  2 & 26 & 13 \\
    \hline
    5  & -2 & -1 & & \\
    &  2 & -1 & & \\
    & -2 &  1 & & \\
    &  2 &  1 & & \\
    & -1 & -2 & & \\
    & -1 &  2 & & \\
    &  1 & -2 & & \\
    &  1 &  2 & 42 & 21 \\
    \hline
  \end{tabular}
  \end{center}
  \caption{Illustration of how single-particle energies
    fill energy shells in a two-dimensional electron box.
  Here $n_{x}$ and $n_{y}$ are the momentum quantum numbers,
  $n_{x}^{2} + n_{y}^{2}$ determines the single-particle 
  energy level, $N_{\uparrow \downarrow }$ represents the 
  cumulated number of spin-orbitals in an unpolarized spin
  phase, and $N_{\uparrow \uparrow }$ stands for the
  cumulated number of spin-orbitals in a spin-polarized
  system.} 
  \label{tab:orbitals_2dheg}
\end{table}

\begin{itemize}
\item General introduction: Why and what has been done
\item Discrete momentum coordinates: definition, advantages and \\
  disadvantages
\item How finite-size effects could have been handled
\end{itemize}


\subsection{Implementation, verification, and results}

As can be seen from Eqs.~(\ref{eq:vmat_3dheg}) and 
(\ref{eq:vmat_2dheg}), the Coulomb interaction is diagonal 
in the CM momentum $\mathbf{K}$ and spin projection $M_{S}$. 
Because of these symmetries, we store the two-body 
interaction and $t$-amplitude matrices in blocks of 
$(\mathbf{K}, M_{S})$. From a computational point of view,
the storage in blocks saves a considerable amount of 
memory and processor time. In the CCD equations 
(\ref{eq:ene_ccd}) and (\ref{eq:t2ampl}), the terms with
summation over two particle states or two hole states
can straightforwardly be written as matrix-matrix 
multiplications. Unfortunately, the terms in the 
CCD amplitude equation with summation over one particle 
and one hole state cannot be calculated directly using 
matrix-matrix multiplications. To circumvent this 
problem, we use a similar cross-coupling technique as
introduced by Kuo \emph{et al.}~for 
coupling of angular momenta \cite{kuo1981}. In the discretized 
Cartesian momentum basis, all diagrams of the CCD 
equations can therefore be implemented using matrix-matrix 
multiplications \cite{hagen2014}. 

Let us explain the basic principles of the cross-coupling
of matrix elements in the discrete momentum basis. 
Consider a matrix element
\[
\langle pq|v|rs\rangle .
\] 
In a diagram with summation over the two ket 
states, for example, we set up the matrix elements as 
\begin{align}
  V_{\alpha(p, q) , \beta(r, s) } \equiv 
  \langle pq|v|rs\rangle \nonumber
\end{align}
and sum over the two-particle states labelled with 
$\beta $. In this case, the blocks are set up such that
the conservation requirements
\[
\mathbf{k}_{p}+\mathbf{k}_{q} = \mathbf{k}_{r}+\mathbf{k}_{s} 
\]
and 
\[
m_{s_{p}}+m_{s_{q}} = m_{s_{r}}+m_{s_{s}} 
\]
are fulfilled. This gives blocks in total momentum 
$\mathbf{K}$ and total spin projection $M_{S}$. On the other 
hand, if the original summation is over the states $s$ and 
$q$, for example, the matrix elements can be set up as
\begin{align} \label{eq:v_crosscoupl}
  V_{\gamma(p, r), \delta(s, q)} \equiv
  \langle pq|v|rs\rangle ,
\end{align} 
where the summation is taken over the two-particle states 
$\delta $. To ensure conservation of total momentum and 
spin projection, the matrices must be stored in blocks such 
that
\[
\mathbf{k}_{p}-\mathbf{k}_{r} = \mathbf{k}_{s}-\mathbf{k}_{q} 
\]
and
\[
m_{s_{p}}-m_{s_{r}} = m_{s_{s}}-m_{s_{q}}.
\]
The blocks are now in relative momentum $\mathbf{\tilde{k}}$ and 
relative spin projection $\tilde{m_{s}}$. We call matrix 
elements set up as in Eq.~(\ref{eq:v_crosscoupl}) 
cross-coupled matrix elements.

Let us show how the particle-hole diagrams are calculated with
matrix-matrix multiplications using cross-coupled matrices.  
Consider the particle-hole term
\begin{align}
  \langle ab|I_{ph}|ij \rangle \equiv \sum_{kc}
  \langle ac|t|ik\rangle \left\{ \langle kb|v|cj\rangle
  + \frac{1}{2}\sum_{ld}\langle kl|v|cd\rangle
  \langle db|t|lj\rangle \right\} ,
\end{align}
which is part of the CCD amplitude equation. We cross-couple
the matrix $I_{ph}$ using the matrix-element transformation 
\begin{align} \label{eq:pphh2phhp}
  \langle ab|I_{ph}|ij\rangle \longrightarrow 
  \langle bj|I_{ph}^{*}|ia\rangle ,
\end{align} 
where the star denotes that the matrix has been cross-coupled
from a particle-particle-hole-hole form to a 
particle-hole-hole-particle form. Technically, the 
matrix-element transformation is a relocation of matrix 
elements. The cross-coupled matrix $I_{ph}^{*}$ is now 
written as the matrix-matrix product
\begin{align} \label{eq:iphstar}
  \langle bj|I_{ph}^{*}|ia\rangle = \sum_{kc}
  \langle bj|I_{2}^{*}|ck\rangle \langle ck|t^{*}|ia\rangle ,
\end{align} 
where the matrix $t^{*}$ is obtained using a similar 
element-transformation as in Eq.~(\ref{eq:pphh2phhp})
and 
\begin{align} \label{eq:i2star}
  \langle bj|I_{2}^{*}|ck\rangle = \langle bj|v^{\# }|ck\rangle
  + \frac{1}{2}\sum_{ld}\langle bj|t^{*}|ld\rangle
  \langle ld|v^{*}|ck\rangle .
\end{align}
Again, the matrices $v^{*}$ and $v^{\# }$ are obtained with an 
element-transformation of the type shown in 
Eq.~(\ref{eq:pphh2phhp}). The matrix $v^{*}$ has been 
obtained by cross-coupling a matrix with
hole-hole-particle-particle configuration, whereas
the matrix $v^{\# }$ has been obtained by cross-coupling 
a hole-particle-particle-hole matrix. The equations
(\ref{eq:iphstar}) and (\ref{eq:i2star}) are 
straightforwardly implemented as matrix-matrix 
multiplications. Finally, the matrix $I_{ph}^{*}$ is 
transformed back to the normal coupling scheme, and
blocks of $I_{ph}$ are added to the CCD amplitude equation.
The implementation of matrix-matrix multiplications
for all diagrams gives a significant speedup of the
computer program. Figure \ref{fig:time_scaling} shows
that the computing time scales quadatically with the
number of unoccupied orbitals, $n_{\mathrm{unocc}}$,
when utilizing block diagonalization and matrix-matrix
multiplications as described above.  


\begin{figure} 
  \centering
  \includegraphics[scale=0.9]{figures/mpi_processes/mpi_processes-crop}
  \caption{The big matrices are divided into matrix blocks, 
    each being an element of an array. 
    Subarrays of the different matrix lists are associated 
    with different MPI processes. There are different matrix 
    lists for normally coupled and cross-coupled matrices.}
  \label{fig:mpi_processes}
\end{figure}

\begin{figure} 
  \centering
  \includegraphics[scale=0.7]{results/ccd_periodic/time_2dheg/plot_opt.pdf}
  \caption{The computing time scales quadratically as a function
    of unoccupied single-particle orbits $n_{\mathrm{unocc}}$.
    The CCD calculations were done for the two-dimensional 
    electron gas with ten electrons at $r_{s} = 0.5$,
    without MPI parallelization and with a fixed number of cpus
    on a single computing node.}
  \label{fig:time_scaling}
\end{figure}


To be able to utilize large-scale distributed computing 
clusters, we have parallelized the CC program in Cartesian
momentum coordinates using the Message Passing Interface
(MPI) Standard \cite{gropp1999,traff2012}. In our 
implementation, both the cross-coupled and normally
coupled matrices are stored in arrays containing matrix
blocks. Most of the matrix arrays are divided into 
subarrays that are distributed to different MPI processes 
(see Figure \ref{fig:mpi_processes}). 
A given subarray containing matrices is then stored 
only locally on the computing node where the corresponding
MPI process resides. In addition to parallelizing the 
computation operations, this arrangement reduces the memory
consumption on each node. Some of the matrices, such as the
cross-coupled amplitude matrix and the normally coupled 
matrix of the particle-hole diagrams must be communicated
to all nodes. We therefore chose to store complete matrix
arrays of these parts on every node. For example, different
parts of the normally coupled particle-hole diagrams, 
$I_{ph}$, are calculated locally on different computing 
nodes, and finally all parts are summed up and the resulting 
matrix array is distributed to all nodes. 


\begin{itemize}
\item Precomputation of different parts
\item Other optimizations
\end{itemize}

\begin{itemize}
\item Tests for the 3D electron gas: comparison \\
  with Alavi \emph{et al.}
\item Results for two-dimensional electron gas: \\
  CCD, ring and ladder approximations
\item Comparisons with SRG and FCIQMC?
\item Comparison with other results, discussion
\end{itemize}




\chapter{Conclusions}
This work showed that $\dots $
\begin{itemize}
\item Limitations in our methods
\item The main results
\end{itemize}
What are the next steps?
\begin{itemize}
\item Calculation of astrophysical observables
\item Finite size effects in the electron gas
\item Triples in the electron gas 
\item Pair-correlation functions for electron gas
  (and nuclear matter?)
\item Three-body forces in N3LO for neutron matter 
\item CC at finite temperatures in cartesian momentum 
  coordinates? (Simen Kvaal's method for time-dependent
  CC, change time to temperature)
\item CC for other observables in cartesian momentum
  coordinates?
\item Extract a density functional??
\item Relativistic coupled-cluster theory for infintie 
  nuclear matter (similar to RBHF)?
\end{itemize}


\bibliographystyle{unsrt}
\bibliography{references}

\appendix

\chapter{Technical details} \label{sec:app_ang_mom}

\section{Antisymmetrization}
When transforming the Brueckner-Hartree-Fock and coupled-cluster equations to a coupled angular momentum - relative momentum basis, one needs to consider antisymmetrized interaction matrix elements
\begin{align}
  &\langle \mathbf{k}_{p}m_{s_{p}}m_{t_{p}}\mathbf{k}_{q}m_{s_{q}}m_{t_{q}}|\hat{v}|\mathbf{k}_{r}m_{s_{r}}m_{t_{r}}\mathbf{k}_{s}m_{s_{s}}m_{t_{s}}\rangle_{AS} \nonumber \\
  = &\langle \mathbf{k}_{p}m_{s_{p}}m_{t_{p}}\mathbf{k}_{q}m_{s_{q}}m_{t_{q}}|\hat{v}|\mathbf{k}_{r}m_{s_{r}}m_{t_{r}}\mathbf{k}_{s}m_{s_{s}}m_{t_{s}}\rangle \nonumber \\
  - &\langle \mathbf{k}_{p}m_{s_{p}}m_{t_{p}}\mathbf{k}_{q}m_{s_{q}}m_{t_{q}}|\hat{v}|\mathbf{k}_{s}m_{s_{s}}m_{t_{s}}\mathbf{k}_{r}m_{s_{r}}m_{t_{r}}\rangle .
  \label{eq:vmat_as}
\end{align}
Let us transform only the ket vector of Eq.~(\ref{eq:vmat_as}) to the coupled angular momentum - relative momentum basis 
\begin{align}
  |k\mathcal{J}m_{\mathcal{J}}(lS)m_{t_{1}}m_{t_{2}}\rangle. \nonumber
  \label{eq:pw_basis_app}
\end{align}
Using the angular momentum algebra relations of Eqs. (\ref{eq:lm_complete}) - (\ref{eq:Ylm_bracket}) and (\ref{eq:y_minus}) - (\ref{eq:clebsch_qp}) and assuming that the two nucleons are both either protons or neutrons, we get
\begin{align}
  &\left[ |\mathbf{k}_{p}m_{s_{p}}\mathbf{k}_{q}m_{s_{q}}\rangle - |\mathbf{k}_{q}m_{s_{q}}\mathbf{k}_{p}m_{s_{p}}\rangle \right] \nonumber \\
  = &\left[ |\mathbf{k}\mathbf{K}m_{s_{p}}m_{s_{q}}\rangle - |-\mathbf{k}\mathbf{K}m_{s_{q}}m_{s_{p}}\rangle \right] \nonumber \\
  = &\sum_{SM_{S}} \left[ |\mathbf{k}\mathbf{K}\rangle - (-1)^{1-S}|-\mathbf{k}\mathbf{K}\rangle \right] \nonumber \\
  & \times |SM_{S}\rangle \langle SM_{S}|sm_{s_{p}}sm_{s_{q}}\rangle \nonumber \\
  = &\sum_{SM_{S}}\sum_{lm_{l}} \left[ |\mathbf{k}\mathbf{K}\rangle  - (-1)^{1-S+l}|-\mathbf{k}\mathbf{K}\rangle \right] \nonumber \\
  & \times \langle lm_{l}|\mathbf{\hat{k}}\rangle |SM_{S}lm_{l}\rangle \langle SM_{S}|sm_{s_{p}}sm_{s_{q}}\rangle \nonumber \\ 
  = & \sum_{SM_{S}}\sum_{lm_{l}}\sum_{\mathcal{J}m_{\mathcal{J}}} (1-(-1)^{1-S+l}) |k\mathbf{K}\mathcal{J}m_{\mathcal{J}}lS\rangle \nonumber \\
  & \times \langle lm_{l}|\mathbf{\hat{k}}\rangle \langle SM_{S}|sm_{s_{p}}sm_{s_{q}}\rangle \langle \mathcal{J}m_{\mathcal{J}}lS|lm_{l}SM_{S}\rangle ,
\end{align}
where $\mathbf{k}$ and $\mathbf{K}$ are relative and CM momenta, respectively, as defined in Eq.~(\ref{eq:lab2rcm}). Whereas two creation operators related to one type of particle anticommute, creation operators of different particle species commute with each other \cite{bishop_lahoz_1987}. Instead of antisymmetric matrix elements as defined in Eq.~(\ref{eq:vmat_as}), we get matrix elements
\begin{align}
  2\langle \mathbf{k}_{p}m_{s_{p}}m_{t_{p}}\mathbf{k}_{q}m_{s_{q}}m_{t_{q}}|\hat{v}|\mathbf{k}_{r}m_{s_{r}}m_{t_{r}}\mathbf{k}_{s}m_{s_{s}}m_{t_{s}}\rangle
\end{align} 
when the particles $r$ and $s$ are of different nucleon types. The ket vector then becomes
\begin{align}
  & 2|\mathbf{k}_{p}m_{s_{p}}\mathbf{k}_{q}m_{s_{q}}\rangle \nonumber \\
  = & 2\sum_{SM_{S}}\sum_{lm_{l}}\sum_{\mathcal{J}m_{\mathcal{J}}} |k\mathbf{K}\mathcal{J}m_{\mathcal{J}}lS\rangle \nonumber \\
  & \times \langle lm_{l}|\mathbf{\hat{k}}\rangle \langle SM_{S}|sm_{s_{p}}sm_{s_{q}}\rangle \langle \mathcal{J}m_{\mathcal{J}}lS|lm_{l}SM_{S}\rangle ,
\end{align}
where we now have a factor 2 insted of the antisymmetrization 
factor $(1-(-1)^{1-S+l})$. In the proton-neutron representation, 
we define the antisymmetrization operator
\begin{align}
  \mathcal{A}^{lSM_{T}} = \left\{ \begin{array}{ll}
    \sqrt{2}, & \text{ if } M_{T} = 0, \\
    \left( 1-(-1)^{l+S+1}\right)/\sqrt{2}, 
    & \text{ if } |M_{T}| = 1.
  \end{array} \right.
  \label{eq:antisymm_app}
\end{align}
When written in the basis (\ref{eq:pw_basis_app}), all 
interaction matrix elements are assumed to be multiplied by 
a product $\mathcal{A}^{lSM_{T}}\mathcal{A}^{l'SM_{T}}$ to get 
the correct antisymmetrization. Observe that the 
antisymmetrization operator $\mathcal{A}$ is defined 
differently here than in Paper II.

The antisymmetrization operator $\mathcal{B}^{M_{T}, \pm }$ is defined for symmetric nuclear matter as
\begin{align}
  & \mathcal{B}^{M_{T}, \pm }\langle k\mathcal{J}(lS)|\hat{O}|k\mathcal{J}(lS)\rangle = \langle k\mathcal{J}(lS)|\hat{O}(M_{T'}=0)|k\mathcal{J}(lS)\rangle \nonumber \\
  & + \left(1 - (-1)^{1+l+S}\right)\langle k\mathcal{J}(lS)|\hat{O}(M_{T'}=M_{T}\pm \delta_{M_{T}0})|k\mathcal{J}(lS)\rangle 
  \label{eq:bopsnm}
\end{align}
and for pure neutron matter as
\begin{align}
  & \mathcal{B}^{M_{T}, \pm }\langle k\mathcal{J}(lS)|\hat{O}|k\mathcal{J}(lS)\rangle = \left(1 - (-1)^{1+l+S}\right) \nonumber \\
  &\times \langle k\mathcal{J}(lS)|\hat{O}(M_{T'}=1)|k\mathcal{J}(lS)\rangle ,
  \label{eq:boppnm}
\end{align}
where $\hat{O}$ is a general two-particle operator. 

\chapter{Mathematical tools} 

\section{Coupled delta distributions} \label{sec:coupled_delta}
Assume that $f:(x, y) \rightarrow \mathbb{R}$ and 
$g:(x, y) \rightarrow \mathbb{R}$ are functions with 
continuous derivatives on the domain 
$(x, y) \in \Omega(x,y) \subset \mathbb{R}^{2}$,
and $h:(x, y) \rightarrow \mathbb{R}$ is an arbitrary function. 
We want to calculate an integral with two coupled delta 
distributions
\begin{eqnarray}
  \lefteqn{ \iint_{\Omega(x,y)}\delta(f(x,y))
    \delta(g(x,y))h(x,y) dx dy } \nonumber \\
  & & {} =\iint_{\Omega(x,y)}\delta^{(2)}(f(x,y),g(x,y))
  h(x,y)dx dy. 
\end{eqnarray}
On right-hand side we have defined the two-dimensional 
delta distribution $\delta^{(2)}(s,t)$. This integral can 
be evaluated by changing to the variables
\begin{equation}
  s=f(x,y), \qquad t=g(x,y).  
\end{equation}
The integration measure transforms accordingly as \cite{callahan2010}
\begin{equation}
  dx dy \quad \longrightarrow \quad \left\vert \begin{array}{cc}
    \frac{\partial x}{\partial s} & \frac{\partial x}{\partial t} \\
    \frac{\partial y}{\partial s} & \frac{\partial y}{\partial t} \\
  \end{array} \right\vert ds dt,
\end{equation}
where the new measure contains the absolute value of a 
Jacobian determinant. Let the direction of integration of 
the new integration domain $\tilde{\Omega}(s,t)$ be the same 
as in the old domain $\Omega(x,y)$. If the initial integration 
domain $\Omega(x,y)$ is chosen such that there exist unique 
inverse mappings
\begin{align}
  \eta &=x(s,t), \nonumber \\
  \xi &=y(s,t), 
\end{align}
we can write the integral as
\begin{eqnarray}
  \lefteqn{ \iint_{\Omega(x,y)} \delta(f(x,y))
    \delta(g(x,y))h(x,y) dx dy } \nonumber \\
  & & {} =\iint_{\tilde{\Omega }(s,t)} \delta^{(2)}(s,t)
  \tilde{h}(s,t)
  \left\vert \begin{array}{cc}
    \frac{\partial x}{\partial s} & 
    \frac{\partial x}{\partial t} \\
    \frac{\partial y}{\partial s} & 
    \frac{\partial y}{\partial t} \\
  \end{array} \right\vert ds dt \nonumber \\
  & & {} =\tilde{h}(s,t)\left \vert 
  \frac{\partial x}{\partial s}\cdot \frac{\partial y}{\partial t}
  -\frac{\partial x}{\partial t}\cdot \frac{\partial y}{\partial s}
  \right \vert \Bigg\vert_{s=t=0}\nonumber \\
  & & {} =\frac{\tilde{h}(s,t)}{\left \vert 
    \frac{\partial s}{\partial x}\cdot \frac{\partial t}{\partial y}
    -\frac{\partial t}{\partial x}\cdot \frac{\partial s}{\partial y}
    \right \vert}\Bigg\vert_{s=t=0},
\end{eqnarray}
where we have defined $\tilde{h}(s,t)\equiv h(x(s,t),y(s,t))$. 
In the last equality we have applied the inverse function 
theorem \cite{callahan2010}. 
  

\end{document}
