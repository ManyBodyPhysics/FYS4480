\chapter{Density functional theory}

\section{Introduction}
Hohenberg and Kohn proved that the total energy of a system including that of the many-body 
effects of electrons (exchange and correlation) in the presence of static external potential (for example, the atomic nuclei) 
is a unique functional of the charge density. The minimum value of the total energy functional 
is the ground state energy of the system. The electronic charge density which yields this 
minimum is then the exact single particle ground state energy.

In Hartree-Fock theory  (HF) one works with large basis sets. This
poses a problem for large systems. An alternative to the HF methods is
density functional theory (DFT). DFT  takes into 
account electron correlations but is less demanding computationally
than full scale diagonalization or Monte Carlo methods.

The electronic energy $E$ is said to be a \emph{functional} of the
electronic density, $E[\rho]$, in the sense that for a given function
$\rho(r)$, there is a single corresponding energy. The  
\emph{Hohenberg-Kohn theorem} confirms that such
a functional exists, but does not tell us the form of the
functional. As shown by Kohn and Sham, the exact ground-state energy
$E$ of an $N$-electron system can be written as
\begin{equation*}
  E[\rho] = -\frac{1}{2} \sum_{i=1}^N\int
  \Psi_i^*(\mathbf{r_1})\nabla_1^2 \Psi_i(\mathbf{r_1}) d\mathbf{r_1}
  - \int \frac{Z}{r_1} \rho(\mathbf{r_1}) d\mathbf{r_1} +
  \frac{1}{2} \int\frac{\rho(\mathbf{r_1})\rho(\mathbf{r_2})}{r_{12}}
  d\mathbf{r_1}d\mathbf{r_2} + E_{EXC}[\rho]
\end{equation*}
with $\Psi_i$ the \emph{Kohn-Sham} (KS) \emph{orbitals}.
 The ground-state charge density is given by
\begin{equation*}
  \rho(\mathbf{r}) = \sum_{i=1}^N|\Psi_i(\mathbf{r})|^2, 
  %\label{}
\end{equation*}
where the sum is over the occupied Kohn-Sham orbitals. The last term,
$E_{EXC}[\rho]$, is the \emph{exchange-correlation energy} which in
theory takes into account all non-classical electron-electron
interaction. However, we do not know how to obtain this term exactly,
and are forced to approximate it. The KS orbitals are found by solving
the \emph{Kohn-Sham (KS) equations}, which can be found by applying a
variational principle to the electronic energy $E[\rho]$. This approach
is similar to the one used for obtaining the HF equation, which we repeat here

The KS equations read
\begin{equation*}
  \left\{ -\frac{1}{2}\nabla_1^2 - \frac{Z}{r_1} + \int 
  \frac{\rho(\mathbf{r_2})}{r_{12}} d\mathbf{r_2} +
  V_{EXC}(\mathbf{r_1}) \right\} \Psi_i(\mathbf{r_1}) =
  \epsilon_i \Psi_i(\mathbf{r_1})
  %\label{}
\end{equation*}
where $\epsilon_i$ are the KS orbital energies, and where the 
\emph{exchange-correlation potential} is given by
\begin{equation*}
  V_{EXC}[\rho] = \frac{\delta E_{EXC}[\rho]}{\delta \rho}.
  %\label{}
\end{equation*}
The KS equations are solved in a self-consistent fashion. A variety of
basis set functions  can be used, and the experience gained in HF
calculations are often useful. The computational time needed for a DFT
calculation formally scales as the third power of the number of basis
functions. 

The main source of error in DFT usually arises from the approximate
nature of $E_{EXC}$. In the \emph{local density approximation} (LDA) it
is approximated as
\begin{equation*}
  E_{EXC} = \int \rho(\mathbf{r})\epsilon_{EXC}[\rho(\mathbf{r})]
  d\mathbf{r},
  %\label{}
\end{equation*}
where $\epsilon_{EXC}[\rho(\mathbf{r})]$ is the exchange-correlation
energy per electron in a homogeneous electron gas of constant density.
The LDA approach is clearly an approximation as the charge is not
continuously distributed. To account for the inhomogeneity of the
electron density, a nonlocal correction involving the gradient of
$\rho$ is often added to the exchange-correlation energy.

\section{Hohenberg-Kohn theorem}
Assume a Hamiltonian of a many-fermion system
\[
\hat{H} = \hat{T}+\hat{U}_{\mathrm{ext}}+\hat{V},
\]
where $\hat{T}$ is the familiar operator for the kinetic energy, $\hat{U}_{\mathrm{ext}}$ is an external potential
that can for example confine our system and $\hat{V}$ a two-body potential. 
As an example, for a system of electrons confined within an atom, the external potential takes the form
\[
\hat{U}_{\mathrm{ext}}=\sum_i^{N_e}\hat{u}_{\mathrm{ext}}({\bf r}_i)=-\sum_i^{N_e}\frac{Zke}{r_i}),
\]
where the constant $k=1.44$ eVnm and $r_i$ is the distance from electron $i$ to the mass center of the atomic nucleus with charge $Z$. Similarly, the two-body interaction is  given by the electron-electron repulsive Coulomb interaction
\[
\hat{V}=\sum_{i < j} v(r_{ij})=\sum_{i < j} \frac{ke^2}{r_{ij}},
\]
where the $r_{ij}$ is the relative distance between electron $i$ and electron $j$.
\subsection{Theorem I}
The first of the important theorems that lay the foundation for density functional theory states that the external
potential $\hat{U}_{\mathrm{ext}}$ and the hence the total energy are unique functionals of the particle density
$\rho({\bf r})$, that is
\[
E[\rho({\bf r})]= F[\rho({\bf r})] +\int \hat{u}_{\mathrm{ext}}({\bf r})\rho({\bf r})d{\bf r},
\] 
where $F$ is an unknown function but otherwise a universal function of $\rho({\bf r})$ only. 

A Hamiltonian for the system can be written such that the electron wave function $\Psi_0$ that minimizes  expectation
value of the $\hat{H}$ gives the ground state energy
\[
E[\rho({\bf r})] = \langle \Psi_0 | \hat{H} |\Psi_0 \rangle=\langle \Psi_0 | \hat{F}+\hat{U}_{\mathrm{ext}} |\Psi_0 \rangle,
\]
where $F$ is the same for all electron systems. This means that the Hamiltonian is completely 
defined by the number of electrons and their distribution. The proof is as follows. We assume we have two
external potentials  $\hat{U}^{(1)}_{\mathrm{ext}}$ and $\hat{U}^{(2)}_{\mathrm{ext}}$ which differ by more than just a constant.
These two potentials yield the final Hamiltonians 
$\hat{H}^{(1)}$ and $\hat{H}^{(2)}$, respectively. Similalry, we postulate now that the two lowest eigenstates 
$\Psi^{(1)}$ and $\Psi^{(2)}$ result in the same ground state density $\rho_0$, where we have skipped the explicit radial
dependence. The state  $\Psi^{(2)}$ is not the ground state of 
$\hat{H}^{(1)}$, meaning that
\[
E^{(1)}=\langle \Psi^{(1)} | \hat{H}^{(1)} |\Psi^{(1)} \rangle <  \langle \Psi^{(2)} | \hat{H}^{(1)} |\Psi^{(2)} \rangle.
\]
Now, 
\[
\langle \Psi^{(2)} | \hat{H}^{(1)} |\Psi^{(2)} \rangle= \langle \Psi^{(2)} | \hat{H}^{(2)} |\Psi^{(2)} \rangle+\langle \Psi^{(2)} |(\hat{H}^{(1)} \hat{H}^{(2)}) |\Psi^{(2)} \rangle,
\]
which can be rewritten as
\[
\langle \Psi^{(2)} | \hat{H}^{(1)} |\Psi^{(2)} \rangle= E^{(2)}+\int \left(\hat{u}^{(1)}_{\mathrm{ext}}({\bf r})-\hat{u}^{(2)}_{\mathrm{ext}}({\bf r}) \right)
\rho_0({\bf r})d{\bf r},
\]
resulting in
\[
E^{(1)} < E^{(2)}+\int \left(\hat{u}^{(1)}_{\mathrm{ext}}({\bf r})-\hat{u}^{(2)}_{\mathrm{ext}}({\bf r}) \right)
\rho_0({\bf r})d{\bf r}.
\]

Let us then repeat the above exercise, but now replacing $E^{(1)}$ with $E^{(2)}$. This results in
\[
E^{(2)} < E^{(1)}+\int \left(\hat{u}^{(2)}_{\mathrm{ext}}({\bf r})-\hat{u}^{(1)}_{\mathrm{ext}}({\bf r}) \right)
\rho_0({\bf r})d{\bf r}.
\]
Adding these two inequalities results in 

a clear  inconsistency and the theorem is proved (proof by reductio ad absurdum).
The latter is from Latin and means reduction to absurdity. This is a common form of argument 
which seeks to demonstrate that a statement is true by showing that a false, untenable, or absurd result follows from 
its denial, or in turn to demonstrate that a statement is false by 
showing that a false, untenable, or absurd result follows from its acceptance. We can thus not have two different potentials that result in the same ground state density $\rho_0$.

There is a corollary  to this theorem that since $H$ is now fully determined (except for a constant shift), then all properties of a many-particle system are determined via the ground state density $\rho_0$.  
\subsection{Theorem II}
A universal functional for the energy $E[\rho]$, that is a function of density $\rho$, can be defined to be valid for any external potential  $\hat{U}_{\mathrm{ext}}$. For a given external potential, the ground state of a system is a global
minimum of such a functional. The density which minimizes $E[\rho]$ is $\rho_0$.  

The proof is again rather simple. From Theorem I, we know that $\rho$ determines 
\section{Applications}

\subsection{The local density approximation and the electron gas}
The electron gas is perhaps the only realistic model of a 
system of many interacting particles that allows for a solution
of the Hartree-Fock equations on a closed form. Furthermore, to first order in the interaction, one can also
compute on a closed form the total energy and several other properties of a many-particle systems. 
The model gives a very good approximation to the properties of valence electrons in metals.
The assumptions are
\begin{itemize}
\item System of electrons that is not influenced by external forces except by an attraction provided by a uniform background of ions. These ions give rise to a uniform background charge. The ions are stationary.
\item The system as a whole is neutral.
\item We assume we have $N_e$ electrons in a cubic box of length $L$ and volume $\Omega=L^3$. This volume contains also a
uniform distribution of positive charge with density $N_ee/\Omega$. 
\end{itemize}

This is a homogeneous system and the one-particle wave functions are given by plane wave functions normalized to a volume $\Omega$ 
for a box with length $L$ (the limit $L\rightarrow \infty$ is to be taken after we have computed various expectation values)
\[
\psi_{{\bf k}\sigma}({\bf r})= \frac{1}{\sqrt{\Omega}}\exp{(i{\bf kr})}\xi_{\sigma}
\]
where ${\bf k}$ is the wave number and  $\xi_{\sigma}$ is a spin function for either spin up or down
\[ 
\xi_{\sigma=+1/2}=\left(\begin{array}{c} 1 \\ 0 \end{array}\right) \hspace{0.5cm}
\xi_{\sigma=-1/2}=\left(\begin{array}{c} 0 \\ 1 \end{array}\right).\]

We assume that we have periodic boundary conditions which limit the allowed wave numbers to
\[
k_i=\frac{2\pi n_i}{L}\hspace{0.5cm} i=x,y,z \hspace{0.5cm} n_i=0,\pm 1,\pm 2, \dots
\]
We assume first that the electrons interact via a central, symmetric and translationally invariant
interaction  $V(r_{12})$ with
$r_{12}=|{\bf r}_1-{\bf r}_2|$.  The interaction is spin independent.

The total Hamiltonian consists then of kinetic and potential energy
\[
\hat{H} = \hat{T}+\hat{V}.
\]
The operator for the kinetic energy can be written as
\[
\hat{T}=\sum_{{\bf k}\sigma}\frac{\hbar^2k^2}{2m}a_{{\bf k}\sigma}^{\dagger}a_{{\bf k}\sigma}.
\]

The Hamilton operator is given by
\[
\hat{H}=\hat{H}_{el}+\hat{H}_{b}+\hat{H}_{el-b},
\]
with the electronic part
\[
\hat{H}_{el}=\sum_{i=1}^N\frac{p_i^2}{2m}+\frac{e^2}{2}\sum_{i\ne j}\frac{e^{-\mu |{\bf r}_i-{\bf r}_j|}}{|{\bf r}_i-{\bf r}_j|},
\]
where we have introduced an explicit convergence factor
(the limit $\mu\rightarrow 0$ is performed after having calculated the various integrals).
Correspondingly, we have
\[
\hat{H}_{b}=\frac{e^2}{2}\int\int d{\bf r}d{\bf r}'\frac{n({\bf r})n({\bf r}')e^{-\mu |{\bf r}-{\bf r}'|}}{|{\bf r}-{\bf r}'|},
\]
which is the energy contribution from the positive background charge with density
$n({\bf r})=N/\Omega$. Finally,
\[
\hat{H}_{el-b}=-\frac{e^2}{2}\sum_{i=1}^N\int d{\bf r}\frac{n({\bf r})e^{-\mu |{\bf r}-{\bf x}_i|}}{|{\bf r}-{\bf x}_i|},
\]
is the interaction between the electrons and the positive background.

Last week we demonstrated that the Hartree-Fock energy can be written as 
\[
\varepsilon_{k}^{HF}=\frac{\hbar^{2}k^{2}}{2m_e}-\frac{e^{2}}
{\Omega^{2}}\sum_{k'\leq
k_{F}}\int d\vec{r}e^{i(\vec{k'}-\vec{k})\vec{r}}\int
d\vec{r'}\frac{e^{i(\vec{k}-\vec{k'})\vec{r'}}}
{\vert\vec{r}-\vec{r'}\vert}
\]
resulting in
\[
\varepsilon_{k}^{HF}=\frac{\hbar^{2}k^{2}}{2m_e}-\frac{e^{2}
k_{F}}{2\pi}
\left[
2+\frac{k_{F}^{2}-k^{2}}{kk_{F}}ln\left\vert\frac{k+k_{F}}
{k-k_{F}}\right\vert
\right]
\]

We introduced a convergence factor $e^{-\mu\vert\vec{r}-\vec{r'}\vert}$ and used
$\sum_{\vec{k}}\rightarrow
\frac{\Omega}{(2\pi)^{3}}\int d\vec{k}$. The results were also rewritten in terms of the density
\[
n= \frac{k_F^3}{3\pi^2}=\frac{3}{4\pi r_s^3},
\]
where $n=N_e/\Omega$, $N_e$ being the number of electrons, and $r_s$ is the radius of a sphere which represents the volum per conducting electron.  
It can be convenient to use the Bohr radius $a_0=\hbar^2/e^2m_e$.
For most metals we have a relation $r_s/a_0\sim 2-6$.

We wish to show first  that
\[
\hat{H}_{b}=\frac{e^2}{2}\frac{N_e^2}{\Omega}\frac{4\pi}{\mu^2},
\]
and
\[
\hat{H}_{el-b}=-e^2\frac{N_e^2}{\Omega}\frac{4\pi}{\mu^2}.
\]
And then that the final Hamiltonian can be written as 
\[
H=H_{0}+H_{I},
\]
with
\[
H_{0}={\displaystyle\sum_{{\bf k}\sigma}}
\frac{\hbar^{2}k^{2}}{2m_e}a_{{\bf k}\sigma}^{\dagger}
a_{{\bf k}\sigma},
\]
and
\[
H_{I}=\frac{e^{2}}{2\Omega}{\displaystyle\sum_{\sigma_{1}
\sigma_{2}}}{\displaystyle
\sum_{{\bf q}\neq 0,{\bf k},{\bf p}}}\frac{4\pi}{q^{2}}
a_{{\bf k}+{\bf q},\sigma_{1}}^{\dagger}
a_{{\bf p}-{\bf q},\sigma_{2}}^{\dagger}
a_{{\bf p}\sigma_{2}}a_{{\bf k}\sigma_{1}}.
\] 
Finally, we want to
calculate
$E_0/N_e=\bra{\Phi_{0}}H\ket{\Phi_{0}}/N_e$ for
for this system to first order in the interaction.  
Using
\[
\rho= \frac{k_F^3}{3\pi^2}=\frac{3}{4\pi r_0^3},
\]
with $\rho=N_e/\Omega$, $r_0$
being the radius of a sphere representing the volume an electron occupies 
and the Bohr radius $a_0=\hbar^2/e^2m$, 
that the energy per electron can be written as 
\[
E_0/N_e=\frac{e^2}{2a_0}\left[\frac{2.21}{r_s^2}-\frac{0.916}{r_s}\right].
\]
Here we have defined
$r_s=r_0/a_0$ to be a dimensionless quantity.

Let us now calculate the following part of the Hamiltonian
\[ \hat H_b = \frac{e^2}{2} \iint \frac{n(\vec r) n(\vec r')e^{-\mu|\vec r - \vec r'|}}{|\vec r - \vec r'|} \dr \dr' , \]
%
where $n(\vec r) = N_e/\Omega$, the density of the positive backgroun charge. We define $\vec r_{12} = \vec r - \vec r'$, reulting in $\md^3 \vec r_{12} = \md^3 r$, and allowing us to rewrite the integral as
\[ \hat H_b = \frac{e^2 N_e^2}{2\Omega^2} \iint \frac{e^{-\mu |\vec r_{12}|}}{|\vec r_{12}|} \dr_{12} \dr' = \frac{e^2 N_e^2}{2\Omega} \int \frac{e^{-\mu |\vec r_{12}|}}{|\vec r_{12}|} \dr_{12} . \]
%
Here we have used that $\int \! \dr = \Omega$. We change to spherical coordinates and the lack of angle 
dependencies yields a factor $4\pi$, resulting in
\[ \hat H_b = \frac{4\pi e^2 N_e^2}{2\Omega} \int_0^\infty re^{-\mu r} \, \md r . \]

Solving by partial integration
\[ \int_0^\infty re^{-\mu r} \, \md r = \left[ -\frac{r}{\mu} e^{-\mu r} \right]_0^\infty + \frac{1}{\mu} \int_0^\infty e^{-\mu r} \, \md r
%
= \frac{1}{\mu} \left[ - \frac{1}{\mu} e^{-\mu r} \right]_0^\infty = \frac{1}{\mu^2}, \]
%
gives
\[
\hat H_b = \frac{e^2}{2} \frac{N_e^2}{\Omega} \frac{4\pi}{\mu^2} .
\]
%
The next term is 
\[ \hat H_{el-b} = -e^2 \sum_{i = 1}^N \int \frac{n(\vec r) e^{-\mu |\vec r - \vec x_i|}}{|\vec r - \vec x_i|} \dr . \]
%
Inserting  $n(\vec r)$ and changing variables in the same way as in the previous integral $\vec y = \vec r - \vec x_i$, we get $\md^3 \vec y = \md^3 \vec r$. This gives
\[ \hat H_{el-b} = -\frac{e^2 N_e}{\Omega} \sum_{i = 1^N} \int \frac{e^{-\mu |\vec y|}}{|\vec y|} \, \md^3 \vec y
%
=  -\frac{4\pi e^2 N_e}{\Omega} \sum_{i = 1}^N \int_0^\infty y e^{-\mu y} \md y. \]
%
We have already seen this  type of integral. The answer is 
\[ \hat H_{el-b} = -\frac{4\pi e^2 N_e}{\Omega} \sum_{i = 1}^N \frac{1}{\mu^2}, \]
which gives
\[
\hat H_{el-b} = -e^2 \frac{N_e^2}{\Omega} \frac{4\pi}{\mu^2} .
\]

Finally, we need to evaluate $\hat H_{el}$. This term reads
\[ \hat H_{el} = \sum_{i=1}^{N_e} \frac{\hat{\vec p}_i^2}{2m_e} + \frac{e^2}{2} \sum_{i \neq j} \frac{e^{-\mu |\vec r_i - \vec r_j|}}{\vec r_i - \vec r_j} . \]
%
The last term represents the repulsion between two electrons. It is a central symmetric interaction
and is translationally invariant. The potential is given by the expression
\[ v(|\vec r|) = e^2 \frac{e^{\mu|\vec r|}}{|\vec r|}, \]
which we derived last week in connection with the Hartree-Fock derivation.

The results becomes
\[ \int v(|\vec r|) e^{-i \vec q \cdot \vec r} \dr =
e^2 \int \frac{e^{\mu |\vec r|}}{|\vec r|} e^{-i \vec q \cdot \vec r} \dr =
e^2 \frac{4\pi}{\mu^2 + q^2} , \]
%
which gives us
\begin{align*}
\hat H_{el} &= \sum_{\vec k \sigma} \frac{\hbar^2 k^2}{2m} \hat a_{\vec k \sigma}^\dagger \hat a_{\vec k \sigma} +
\frac{e^2}{2\Omega} \sum_{\sigma \sigma'} \sum_{\vec k \vec p \vec q} \frac{4\pi}{\mu^2 + q^2}
\hat a_{\vec k + \vec q, \sigma}^\dagger \hat a_{\vec p - \vec q, \sigma'}^\dagger \hat a_{\vec p \sigma'} \hat a_{\vec k \sigma} \\
%
&= \sum_{\vec k \sigma} \frac{\hbar^2 k^2}{2m_e} \hat a_{\vec k \sigma}^\dagger \hat a_{\vec k \sigma} +
\frac{e^2}{2\Omega} \sum_{\sigma \sigma'} \sum_{\substack{\vec k \vec p \vec q \\ q \neq 0}} \frac{4\pi}{q^2}
\hat a_{\vec k + \vec q, \sigma}^\dagger \hat a_{\vec p - \vec q, \sigma'}^\dagger \hat a_{\vec p \sigma'} \hat a_{\vec k \sigma} + \\
&\quad\,\,
\frac{e^2}{2\Omega} \sum_{\sigma \sigma'} \sum_{\vec k \vec p} \frac{4\pi}{\mu^2}
\hat a_{\vec k, \sigma}^\dagger \hat a_{\vec p, \sigma'}^\dagger \hat a_{\vec p \sigma'} \hat a_{\vec k \sigma} ,
\end{align*}
%
where in the last sum we have split the sum over $\vec q$ in two parts, one with $\vec q\ne 0$ and one with 
$\vec q=0$. In the first term we also let $\mu\rightarrow 0$.

The last term has the following set of creation and annihilation operatord
\[ \hat a_{\vec k, \sigma}^\dagger \hat a_{\vec p, \sigma'}^\dagger \hat a_{\vec p \sigma'} \hat a_{\vec k \sigma} =
- \hat a_{\vec k, \sigma}^\dagger \hat a_{\vec p, \sigma'}^\dagger \hat a_{\vec k \sigma} \hat a_{\vec p \sigma'} =
- \hat a_{\vec k, \sigma}^\dagger \hat a_{\vec p \sigma'} \delta_{\vec p \vec k} \delta_{\sigma \sigma'} + \hat a_{\vec k, \sigma}^\dagger \hat a_{\vec k \sigma} \hat a_{\vec p, \sigma'}^\dagger \hat a_{\vec p \sigma'} , \]
which gives
\[ \sum_{\sigma \sigma'} \sum_{\vec k \vec p} \hat a_{\vec k, \sigma}^\dagger \hat a_{\vec p, \sigma'}^\dagger \hat a_{\vec p \sigma'} \hat a_{\vec k \sigma} =
\hat N^2 - \hat N , \]
where we have used the expression for the number operator.  The term to the first power in $\hat N$ 
goes to zero in the thermodynamic limit since we are interested in the energy per electron $E_0/N_e$. This term will then be proportional with $1/(\Omega \mu^2)$. 
In the thermodynamical limit $\Omega\rightarrow \infty$ we can set this term equal to zero.
We then get
%
\[ \hat H_{el} = \sum_{\vec k \sigma} \frac{\hbar^2 k^2}{2m} \hat a_{\vec k \sigma}^\dagger \hat a_{\vec k \sigma} +
\frac{e^2}{2\Omega} \sum_{\sigma \sigma'} \sum_{\substack{\vec k \vec p \vec q \\ q \neq 0}} \frac{4\pi}{q^2}
\hat a_{\vec k + \vec q, \sigma}^\dagger \hat a_{\vec p - \vec q, \sigma'}^\dagger \hat a_{\vec p \sigma'} \hat a_{\vec k \sigma} +
\frac{e^2}{2} \frac{N_e^2}{\Omega} \frac{4\pi}{\mu^2}. \]
%
The total Hamiltonian is $\hat H = \hat H_{el} + \hat H_{b} + \hat H_{el-b}$. 
Collecting all our terms we end up with
\[
\hat H_0 = \sum_{\vec k \sigma} \frac{\hbar^2 k^2}{2m_e} \hat a_{\vec k \sigma}^\dagger \hat a_{\vec k \sigma},
\]
and
\[
\hat H_I = \frac{e^2}{2\Omega} \sum_{\sigma \sigma'} \sum_{\substack{\vec k \vec p \vec q \\ q \neq 0}} \frac{4\pi}{q^2}
\hat a_{\vec k + \vec q, \sigma}^\dagger \hat a_{\vec p - \vec q, \sigma'}^\dagger \hat a_{\vec p \sigma'} \hat a_{\vec k \sigma},
\]
Now we need $E_0 = \bra{\Phi_0} \hat H \ket{\Phi_0}$.
The kinetic energy gives  simply
\[
\bra{\Phi_0} \hat H_0 \ket{\Phi_0} =
\frac{\hbar^2 \Omega}{10\pi^2 m_e} k_F^5 .
\]
The expectation value for  $\hat H_I$ is
\begin{align*}
\bra{\Phi_0} \hat H_I \ket{0} &=
\bra{\Phi_0} \left( \frac{e^2}{2\Omega} \sum_{\sigma \sigma'} \sum_{\substack{\vec k \vec p \vec q \\ q \neq 0}} \frac{4\pi}{q^2}
\hat a_{\vec k + \vec q, \sigma}^\dagger \hat a_{\vec p - \vec q, \sigma'}^\dagger \hat a_{\vec p \sigma'} \hat a_{\vec k \sigma} \right) \ket{\Phi_0} \\
%
&= \frac{e^2}{2\Omega} \sum_{\sigma \sigma'} \sum_{\substack{\vec k \vec p \vec q \\ q \neq 0}} \frac{4\pi}{q^2} \bra{\Phi_0}
\hat a_{\vec k + \vec q, \sigma}^\dagger \hat a_{\vec p - \vec q, \sigma'}^\dagger \hat a_{\vec p \sigma'} \hat a_{\vec k \sigma} \ket{\Phi_0}.
\end{align*}

 For the matrix element to be different from zero $0$, we must have $\vec k + \vec q = \vec p$ and  $\sigma = \sigma'$. We must also have $p \leq k_F$ and $k \leq k_F$. We get
\[ \bra{\Phi_0} \hat H_I \ket{0} =
-\frac{4\pi e^2}{2\Omega} \sum_{\sigma} \sum_{\substack{\vec k, \vec p \neq \vec k \\ k, p \leq k_F}} \frac{1}{|\vec p - \vec k|^2} =
-\frac{4\pi e^2}{\Omega} \sum_{\substack{\vec k, \vec p \neq \vec k \\ k, p \leq k_F}} \frac{1}{|\vec p - \vec k|^2}. \]
%
Changing to an integral we get
\[ \bra{\Phi_0} \hat H_I \ket{\Phi_0} =
-\frac{4\pi e^2}{\Omega} \left( \frac{\Omega}{(2\pi)^3} \right)^2 \int_0^{k_F} \!\!\!\! \int_0^{k_F} \frac{1}{|\vec p - \vec k|^2} \, \md^3 \vec k \, \md^3 \vec p . \]
%
Using spherical coordinates 
\[ \int_0^{k_F} \!\!\! \int_0^{k_F} \frac{1}{|\vec p - \vec k|^2} \, \md^3 \vec k \, \md^3 \vec p =
2\pi \int_0^{k_F} \!\!\! \int_0^\pi \!\!\! \int_0^{k_F} \frac{k^2 \sin \theta}{p^2 + k^2 - 2kp \cos \theta} \, \md k \md \theta \, \md^3 \vec p , \]
%
since $p$ is a constant in the integral over $k$. First we integrate over  $\theta$, resulting in
\begin{align*}
\int_0^{k_F} \!\!\! \int_0^{k_F} \frac{1}{|\vec p - \vec k|^2} \, \md^3 \vec k \, \md^3 \vec p &=
%
2\pi \int_0^{k_F} \!\!\! \int_0^{k_F} \left[ \frac{k^2 \ln \left( k^2 + p^2 - 2kp\cos \theta \right)}{2kp} \right]_{\theta = 0}^{\theta = \pi} \, \md k \, \md^3 \vec p \\
%
&= \pi \int_0^{k_F} \!\!\! \int_0^{k_F} \frac{k}{p} \ln \left( \frac{(p + k)^2}{(p - k)^2} \right) \, \md k \, \md^3 \vec p \\
%
&= 2\pi \int_0^{k_F} \!\!\! \int_0^{k_F} \frac{k}{p} \ln \left| \frac{p + k}{p - k} \right| \, \md k \, \md^3 \vec p \\
%
&= 2\pi \int_0^{k_F} \!\!\! \int_0^{k_F} \frac{k}{p} \ln | p + k| - \frac{k}{p} \ln|k-p| \, \md k \, \md^3 \vec p .
\end{align*}
We use the following relations
\[ \int k\ln|k+p| = \frac{1}{2} k^2 \ln|k + p| - \frac{k^2}{4} - \frac{1}{2} p^2 \ln|k+p| + \frac{kp}{2} + C , \]
%
which give
\[ \int_0^{k_F} k\ln|k+p| = \frac{1}{2} k_F^2 \ln|k_F + p| - \frac{k_F^2}{4} - \frac{1}{2} p^2 \ln|k_F+p| + \frac{k_F p}{2} + \frac{1}{2} p^2 \ln p , \]
%
and
\[ \int_0^{k_F} k\ln|k-p| = \frac{1}{2} k_F^2 \ln|k_F - p| - \frac{k_F^2}{4} - \frac{1}{2} p^2 \ln|k_F-p| - \frac{k_F p}{2} + \frac{1}{2} p^2 \ln p . \]

Summing up we get
\begin{align*}
\int_0^{k_F} \!\!\! \int_0^{k_F} \frac{1}{|\vec p - \vec k|^2} \, \md^3 \vec k \, \md^3 \vec p &=
%
2\pi \int_0^{k_F} \frac{1}{p} \left( \frac{1}{2} k_F^2 \ln \left| \frac{k_F + p}{k_F - p} \right| - \frac{1}{2} p^2 \ln \left| \frac{k_F + p}{k_F - p} \right| + k_F p \right) \, \md^3 \vec p \\
%
&= 2\pi k_F \frac{4}{3} \pi k_F^3 + \pi \int_0^{k_F} \left( \frac{k_F^2}{p} - p \right) \ln \left| \frac{k_F + p}{k_F - p} \right| \, \md^3 \vec p \\
%
&= \frac{8\pi^2}{3} k_F^4 + 4\pi^2 \int_0^{k_F} \left( k_F^2 p - p^3 \right) \ln \left| \frac{k_F + p}{k_F - p} \right| \, \md p .
\end{align*}
Utilizing
\[ \int_0^{k_F} p \ln |p + k_F| \, \md p = \frac{1}{4} k_F^2 \left( 2\ln k_F + 1 \right), \]
%
\[ \int_0^{k_F} p^3 \ln |p + k_F| \, \md p = \frac{1}{48} k_F^4 \left( 12\ln k_F + 7 \right), \]
%
\[ \int_0^{k_F} p \ln |p - k_F| \, \md p = \frac{1}{4} k_F^2 \left( 2\ln k_F - 3 \right), \]
%
and
\[ \int_0^{k_F} p^3 \ln |p - k_F| \, \md p = \frac{1}{48} k_F^4 \left( 12\ln k_F - 25 \right). \]
This gives
\begin{multline*}
\int_0^{k_F} \!\!\! \int_0^{k_F} \frac{1}{|\vec p - \vec k|^2} \, \md^3 \vec k \, \md^3 \vec p = \\
\frac{8\pi^2}{3} \pi k_F^4 + 4\pi^2 \left( k_F^2 \frac{1}{4} k_F^2 \left( 2\ln k_F + 1 \right) - k_F^2 \frac{1}{4} k_F^2 \left( 2\ln k_F - 3 \right) - \right. \\
\left. \frac{1}{48} k_F^4 \left( 12\ln k_F + 7 \right) + \frac{1}{48} k_F^4 \left( 12\ln k_F - 25 \right) \right),
\end{multline*}
%
which we can bring together to
\[ \int_0^{k_F} \!\!\! \int_0^{k_F} \frac{1}{|\vec p - \vec k|^2} \, \md^3 \vec k \, \md^3 \vec p =
%
\frac{8}{3} \pi^2 k_F^4 + 4\pi^2 \left( k_F^4 - \frac{2}{3} k_F^4 \right) =
4\pi^2 k_F^4 . \]
%
Inserting this in the expression for $\bra{\Phi_0} \hat H_I \ket{\Phi_0}$ we obtain
\[ \bra{\Phi_0} \hat H_I \ket{\Phi_0} =
-\frac{4\pi e^2}{\Omega} \left( \frac{\Omega}{(2\pi)^3} \right)^2 4\pi^2 k_F^4 . \]
%
We get 
\[ \frac{E_0}{N} = \frac{1}{N} \left( \frac{\hbar^2 \Omega}{10\pi^2 m} k_F^5 -\frac{4\pi e^2}{\Omega} \left( \frac{\Omega}{(2\pi)^3} \right)^2 4\pi^2 k_F^4 \right) . \]
Inserting $k_F$ we get
\begin{align*}
\frac{E_0}{N} &=
%
\frac{\hbar^2 \Omega}{10\pi^2 m N} k_F^5 - \frac{4\pi e^2}{\Omega N} \left( \frac{\Omega}{(2\pi)^3} \right)^2 4\pi^2 k_F^4 \\
%
&= \frac{\hbar^2 \Omega}{10\pi^2 m N} k_F^5 - \frac{e^2 \Omega}{4\pi^3 N}k_F^4 \\
%
&= \frac{\hbar^2 \Omega}{10\pi^2 m N} \left( \frac{3\pi^2 N}{\Omega} \right)^{5/3} - \frac{e^2 \Omega}{4\pi^3 N} \left( \frac{3\pi^2 N}{\Omega} \right)^{4/3} \\
%
&= \frac{\hbar^2 N^{2/3}}{\Omega^{2/3}} \frac{(3\pi^2)^{5/3}}{10\pi^2 m} -
\frac{e^2 \Omega^{1/3}}{N^{1/3}} \frac{(3\pi^2)^{4/3}}{4\pi^3} .
\end{align*}

Finally, we introduce
\[ r_0 = \left( \frac{3\Omega}{4\pi N} \right)^{1/3}, \quad \text{og} \quad
a_0 = \frac{\hbar^2}{e^2 m} , \]
%
which gives
\begin{align*}
\frac{E_0}{N} &=
%
\hbar^2 \frac{(3\pi^2)^{5/3}}{10\pi^2 m} \left( \frac{3}{4\pi} \right)^{2/3} \frac{1}{r_0^2} -
e^2 \frac{(3\pi^2)^{4/3}}{4\pi^3} \left( \frac{3}{4\pi} \right)^{1/3} \frac{1}{r_0} \\
%
&= \frac{1}{2} \left( \frac{\hbar^2}{m} \frac{2.21}{r_0^2} -
e^2 \frac{0.916}{r_0} \right)
\end{align*}
Finally we define  $r_s = r_0 / a_0$, and get
\[
\frac{E_0}{N} = \frac{e^2}{2a_0} \left( \frac{2.21}{r_s^2} - \frac{0.916}{r_s} \right).
\]
%
To find the minimum we take the partial derivative
\[ \frac{\partial}{\partial r_s} \left( \frac{E_0}{N} \right) = 0 \quad \Rightarrow \quad
\frac{2 \times 2.21}{r_s^3} - \frac{0.916}{r_s^2} = 0, \]
%
which results in
\[ r_s = \frac{2 \times 2.21}{0.916} \approx 4.83 . \]










