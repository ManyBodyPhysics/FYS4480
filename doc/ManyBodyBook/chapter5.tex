\chapter{Nuclear forces}

\section{Introduction}
Chadwick (1932) discovers the neutron and Heisenberg (1932) proposes the first Phenomenology (Isospin).  
Yukawa (1935) and his Meson Hypothesis       

Discovery of the pion in cosmic ray (1947) and in the Berkeley Cyclotron Lab (1948).
Nobelprize awarded to Yukawa (1949).  Rabi (1948) measures quadrupole moment of the deuteron.

Taketani, Nakamura, Sasaki (1951): 3 ranges.      One-Pion-Exchange (OPE): o.k.
Multi-pion exchanges: Problems!   Taketani, Machida, Onuma (1952);
"Pion Theories" Brueckner, Watson (1953).
Many pions = multi-pion resonances:
$\sigma(600)$,  $\rho(770)$,  $\omega(782)$ etc.One-Boson-Exchange Model.
Refined Meson Theories

Sophisticated models for two-pion exchange:
       Paris Potential (Lacombe et al., Phys.~Rev.~C {\bf 21}, 861 (1980))
       Bonn potential (Machleidt et al., Phys.~Rep.~{\bf 149}, 1 (1987))
Quark cluster models. Begin of effective field theory studies.
1993-2001: High-precision NN potentials: Nijmegen I, II, '93, Reid93 (Stoks et al. 1994), 
Argonne V18 (Wiringa et al, 1995), CD-Bonn (Machleidt et al. 1996 and 2001. 
Advances in effective field theory: Weinberg (1990); Ordonez, Ray, van Kolck and many more.
           Another "pion theory"; but now right: constrained by chiral symmetry. Three-body and higher-body forces
appear naturally at a given order of the chiral expansion. 
Nucleon-nucleon interaction from Lattice QCD, final confirmation of meson hypothesis of Yukawa? 

    \begin{itemize}
\item  Explore the limits of our understanding of the atomic nuclei based
on nucleonic and mesonic degrees of freedom.  CEBAF, J-Parc, FAIR and LHC offer such perspectives.
\item Experimental plans aim at identifying and exploring the transition
from the nucleon/meson description of nuclei to the underlying quark and 
gluon description.
\item Test the short-range behavior of the NN interaction via deep inelastic scattering
\item Effective field theory has made progress in constructing NN and NNN
forces from the underlying symmetries of QCD 
\item Three-body and higher-body forces emerge naturally and have explicit expressions at 
every order in the chiral expansion. 
\item Recent progress in Lattice QCD (LQCD) may hold great promise for constraining
effective field theories.
\item LQCD will be able to tell us about the interactions of systems 
that cannot be probed experimentally, but have relevance to astrophysics (nucleon-hyperon interactions), meson-meson and meson-baryon interactions, and other fields of nuclear physics. 
    \end{itemize}

%\begin{center}
%  \includegraphics[width=0.45\columnwidth]{lattice}
%  \includegraphics[width=0.45\columnwidth]{lattice1}\\[1ex]
%\end{center}

The aim is to give you an overview over central features of the nucleon-nucleon interaction and how
it is constructed, with both technical and theoretical approaches. 
\begin{enumerate}
\item The existence of the deuteron with $J^{\pi}=1^+$ indicates that the force between protons and neutrons is attractive
at least for the $^3S_1$ partial wave. Interference between Coulomb and nuclear scattering for the proton-proton
partial wave $^1S_0$ shows that  the NN force is attractive at least for the $^1S_0$ partial wave. 
\item It has a short range and strong intermediate attraction.
\item Spin dependent, scattering lengths for triplet and singlet states are different,
\item Spin-orbit force. Observation of large polarizations of scattered nucleons perpendicular to the plane of scattering.
\begin{enumerate}
\item Strongly repulsive core. The $s$-wave phase shift becomes negative at $\approx 250$ MeV implying that the singlet $S$
has a hard core with range $0.4-0.5$ fm. 
\item Charge independence (almost). Two nucleons in a given two-body state always (almost) experience the same
force. Modern interactions break charge and isospin symmetry lightly. That means that the pp, neutron-neutron and pn 
parts of the 
interaction will be different for the same quantum numbers. 
\item Non-central. There is a tensor force. First indications from the quadrupole moment of the deuteron pointing to an
admixture in the ground state of both $l=2$ ($^3D_1$) and $l=0$ ($^3S_1$) orbital momenta.
\end{enumerate}

Comparison of the binding energies of
$^2$H (deuteron), $^3$H (triton), $^4$He (alpha - particle)
  show that the nuclear force is of finite range ($1-2$ fm) and
  very strong within that range.
For nuclei with $A>4$, the energy saturates: Volume
  and binding energies of nuclei are proportional to the
  mass number $A$ (as we saw from exercise 1).

Nuclei are also bound. The average distance
between nucleons in nuclei is about $2$ fm which
must roughly correspond to the range of the
attractive part.
\begin{itemize}
\item  After correcting for the electromagnetic interaction, the forces
  between nucleons (pp, nn, or np) in the same state are almost the
  same.
\item  "Almost the same": Charge-independence is slightly broken.
\item  Equality between the pp and nn forces: Charge symmetry.
\item  Equality between pp/nn force and np force: Charge independence.
\item Better notation: Isospin symmetry, invariance under rotations in isospin
\end{itemize}

Charge-symmetry breaking (CSB), after electromagnetic effects
have been removed:
\begin{itemize}
\item $a_{pp}=  -17.3 \pm 0.4 \hspace{0.cm} \mathrm{fm}$
\item $a_{nn}=-18.8 \pm 0.5 \hspace{0.cm} \mathrm{fm}$. Note however discrepancy from $nd$ breakup reactions
resulting in  $a_{nn}=-18.72 \pm 0.13 \pm 0.65 \hspace{0.cm} \mathrm{fm}$
and $\pi^- + d \rightarrow \gamma + 2n$ reactions giving  $a_{nn}=-18.93 \pm 0.27 \pm 0.3 \hspace{0.cm} \mathrm{fm}$.
\end{itemize}
Charge-independence breaking (CIB)
\begin{itemize}
\item $a_{pn}=  -23.74 \pm 0.02 \hspace{0.cm} \mathrm{fm}$ 
\end{itemize}
\begin{enumerate}
\item Translation invariance
\item Galilean invariance
\item  Rotation invariance in space
\item Space reflection invariance
\item Time reversal invariance
\item Invariance under the interchange of particle $1$ and $2$
\item Almost isospin symmetry
\end{enumerate}
\[
V({\bf r})= \left\{ C_c + C_\sigma 
\mbox{\boldmath $\sigma$}_1\cdot\mbox{\boldmath $\sigma$}_2
 + C_T \left( 1 + {3\over m_\alpha r} + {3\over
\left(m_\alpha r\right)^2}\right) S_{12} (\hat r)\right. 
\]
\[
\left. + C_{SL} \left( {1\over m_\alpha r} + {1\over \left( m_\alpha r\right)^2}
\right) {\bf L}\cdot {\bf S}
\right\} \frac{e^{-m_\alpha r}}{m_\alpha r}
\]
How do we derive such terms?  

 Potentials which are based upon the standard non-relativistic operator structure
 are called "Phenomenological Potentials"
Some historically important examples are   
    \begin{itemize}
   \item    Gammel-Thaler potential ( Phys. Rev. {\bf 107}, 291, 1339 (1957) and the 
 Hamada-Johnston potential, Nucl. Phys. {\bf 34}, 382 (1962)), bot with a hard core.
    core.
   \item  Reid potential (Ann. Phys. (N.Y.) {\bf 50}, 411 (1968)), soft core.
\item
Argonne $V_{14}$ potential (Wiringa et al., Phys. Rev. C {\bf 29}, 1207
    (1984)) with 14 operators and  the  Argonne $V_{18}$ potential (Wiringa et al., Phys. Rev. C {\bf 51}, 38
    (1995)), uses 18 operators
\bibitem{ref4}  A good reference: R.~Machleidt, Adv.~Nucl.~Phys~{\bf 19}, 189 (1989).
\end{itemize}

\begin{enumerate}
\item From 1950 till approximately 2000: One-Boson-Exchange (OBE) models dominate. These are models which typically include several low-mass mesons, that is with masses below 1 GeV.
\item Now: models based on chiral perturbation theory. These are effective models with nucleons and pions as degrees of freedom only. The other mesons which appeared in standard one-boson model appear as multi-pion resonances. 
\end{enumerate}




\section{Quantum numbers of the two-body problem and nuclear forces}




In order to understand the basics of the nucleon-nucleon interaction, we need to define the relevant quantum numbers and how we build up a single-particle state and a two-body state. 
\begin{enumerate}
\item For the single-particle states, due to the fact that we have the spin-orbit force, the quantum numbers for the projection of orbital momentum $l$, that is $m_l$, and for spin $s$, that is $m_s$, are no longer so-called good quantum numbers. The total angular momentum $j$ and its projection $m_j$ are then 
so-called good quantum number.
\item This means that the operator $\hat{J}^2$ does not commute with $\hat{L}_z$  or $\hat{S}_z$.  
\item We also start normally with single-particle state functions defined using say the harmonic oscillator. For these functions, we have no explicit dependence on $j$. How can we introduce single-particle wave functions which have $j$ and its projection $m_j$ as quantum numbers? 
\end{enumerate}

We have that the operators for the orbital momentum are given by
\[
L_x=-i\hbar(y\frac{\partial }{\partial z}-z\frac{\partial }{\partial y})=
yp_z-zp_y,
\]
\[
L_y=-i\hbar(z\frac{\partial }{\partial x}-x\frac{\partial }{\partial z})= zp_x-xp_z,
\]
\[
L_z=-i\hbar(x\frac{\partial }{\partial y}-y\frac{\partial }{\partial x})=xp_y-yp_x.
\]
Since we have a spin orbit force which is strong, it is easy to show that 
the total angular momentum operator
\[
   \OP{J}=\OP{L}+\OP{S}
\]
does not commute with $\OP{L}_z$ and $\OP{S}_z$. To see this, we calculate for example
\begin{eqnarray} 
   [\OP{L}_z,\OP{J}^2]&=&[\OP{L}_z,(\OP{L}+\OP{S})^2] \\ \nonumber
   &=&[\OP{L}_z,\OP{L}^2+\OP{S}^2+2\OP{L}\OP{S}]\\ \nonumber 
   &=& [\OP{L}_z,\OP{L}\OP{S}]=[\OP{L}_z,\OP{L}_x\OP{S}_x+\OP{L}_y\OP{S}_y+\OP{L}_z\OP{S}_z]\ne 0, 
\end{eqnarray}
since we have that $[\OP{L}_z,\OP{L}_x]=i\hbar\OP{L}_y$ and $[\OP{L}_z,\OP{L}_y]=i\hbar\OP{L}_x$. 

We have also
\[
   |\OP{J}|=\hbar\sqrt{J(J+1)},
\]
with the the following degeneracy
\[
   M_J=-J, -J+1, \dots, J-1, J.
\]
With a given value of  $L$ and $S$ we can then determine the possible values of 
 $J$ by studying the $z$ component of  $\OP{J}$. 
It is given by
\[
\OP{J}_z=\OP{L}_z+\OP{S}_z.
\]
The operators $\OP{L}_z$ and $\OP{S}_z$ have the quantum numbers
$L_z=M_L\hbar$ and $S_z=M_S\hbar$, respectively, meaning that
\[
   M_J\hbar=M_L\hbar +M_S\hbar,
\]
or
\[
   M_J=M_L +M_S.
\]
Since the max value of  $M_L$ is $L$ and for  $M_S$ is $S$
we obtain
\[
   (M_J)_{\mathrm{maks}}=L+S.
\]

For nucleons we have that the maximum value of $M_S=m_s=1/2$, yielding
\[
   (m_j)_{\mathrm{max}}=l+\frac{1}{2}.
\]
Using this and the fact that the maximum value of  $M_J=m_j$ is $j$ we have
\[
   j=l+\frac{1}{2}, l-\frac{1}{2}, l-\frac{3}{2}, l-\frac{5}{2}, \dots 
\]
To decide where this series terminates, we use the vector inequality
\[
   |\OP{L}+\OP{S}| \ge \left| |\OP{L}|-|\OP{S}|\right|.
\]
Using $\OP{J}=\OP{L}+\OP{S}$ we get 
\[
   |\OP{J}| \ge |\OP{L}|-|\OP{S}|,
\]
or
\[
   |\OP{J}|=\hbar\sqrt{J(J+1)}\ge |\hbar\sqrt{L(L+1)}-
   \hbar\sqrt{S(S+1)}|.
\]

If we limit ourselves to nucleons only with $s=1/2$
we find that
\[
   |\OP{J}|=\hbar\sqrt{j(j+1)}\ge |\hbar\sqrt{l(l+1)}-
   \hbar\sqrt{\frac{1}{2}(\frac{1}{2}+1)}|.
\]
It is then easy to show that for nucleons there are only two possible values of
$j$ which satisfy the inequality, namely
\[
   j=l+\frac{1}{2}\hspace{0.1cm} \mathrm{or} \hspace{0.1cm}j=l-\frac{1}{2},
\]
and with $l=0$ we get 
\[
   j=\frac{1}{2}.
\]

Let us study some selected examples. We need also to keep in mind that parity is conserved.
The strong and electromagnetic Hamiltonians conserve parity. Thus the eigenstates can be
broken down into two classes of states labeled by their parity $\pi= +1$ or $\pi=-1$.
The nuclear interactions do not mix states with different parity.

For nuclear structure the total parity originates
from the intrinsic parity of the nucleon which is $\pi_{\mathrm{intrinsic}}=+1$ 
and the parities associated with
the orbital angular momenta $\pi_l=(-1)^l$ . The total parity is the product over all nucleons
$\pi = \prod_i \pi_{\mathrm{intrinsic}}(i)\pi_l(i) = \prod_i (-1)^{l_i}$

The basis states we deal with are constructed so that they conserve parity and have thus a definite parity. 

Note that we do have parity violating processes, more on this later. 

Consider now the single-particle orbits of the $1s0d$ shell. 
For a $0d$ state we have the quantum numbers $l=2$, $m_l=-2,-1,0,1,2$, $s+1/2$, $m_s=\pm 1/2$,
$n=0$ (the number of nodes of the wave function).   This means that we have positive parity and
\[
j=\frac{3}{2}=l-s\hspace{1cm} m_j=-\frac{3}{2},-\frac{1}{2},\frac{1}{2},\frac{3}{2}.
\]
and
\[
j=\frac{5}{2}=l+s\hspace{1cm} m_j=-\frac{5}{2},-\frac{3}{2},-\frac{1}{2},\frac{1}{2},\frac{3}{2},\frac{5}{2}.
\]

Our single-particle wave functions, if we use the harmonic oscillator, do however not contain the quantum numbers $j$ and $m_j$.
Normally what we have is an eigenfunction for the one-body problem defined as
\[
\phi_{nlm_lsm_s}(r,\theta,\phi)=R_{nl}(r)Y_{lm_l}(\theta,\phi)\xi_{sm_s},
\]
where we have used spherical coordinates (with a spherically symmetric potential) and the spherical harmonics 
\[
    Y_{lm_l}(\theta,\phi)=P(\theta)F(\phi)=\sqrt{\frac{(2l+1)(l-m_l)!}{4\pi (l+m_l)!}}
                      P_l^{m_l}(cos(\theta))\exp{(im_l\phi)},
\]
with $P_l^{m_l}$ being the so-called associated Legendre polynomials. 
Examples are
\[
   Y_{00}=\sqrt{\frac{1}{4\pi}},
\]
for $l=m_l=0$, 
\[
   Y_{10}=\sqrt{\frac{3}{4\pi}}cos(\theta),
\]
for $l=1$ and $m_l=0$, 
\[
   Y_{1\pm 1}=\sqrt{\frac{3}{8\pi}}sin(\theta)exp(\pm i\phi),
\]
for  $l=1$ and $m_l=\pm 1$, 
\[
   Y_{20}=\sqrt{\frac{5}{16\pi}}(3cos^2(\theta)-1)
\]
for $l=2$ and $m_l=0$ etc. 

How can we get a function in terms of $j$ and $m_j$?
Define now
\[
\phi_{nlm_lsm_s}(r,\theta,\phi)=R_{nl}(r)Y_{lm_l}(\theta,\phi)\xi_{sm_s},
\]
and 
\[
\psi_{njm_j;lm_lsm_s}(r,\theta,\phi),
\]
as the state with quantum numbers $jm_j$.
Operating with 
\[
   \OP{j}^2=(\OP{l}+\OP{s})^2=\OP{l}^2+\OP{s}^2+2\OP{l}_z\OP{s}_z+\OP{l}_+\OP{s}_{-}+\OP{l}_{-}\OP{s}_{+},
\]
on the latter state we will obtain admixtures from possible $\phi_{nlm_lsm_s}(r,\theta,\phi)$ states.

To see this, we consider the following example and fix
\[
j=\frac{3}{2}=l-s\hspace{1cm} m_j=\frac{3}{2}.
\]
and
\[
j=\frac{5}{2}=l+s\hspace{1cm} m_j=\frac{3}{2}.
\]
It means we can have, with $l=2$ and $s=1/2$ being fixed, in order to have $m_j=3/2$ either $m_l=1$ and $m_s=1/2$ or
$m_l=2$ and $m_s=-1/2$. The two states    
\[
\psi_{n=0j=5/2m_j=3/2;l=2s=1/2}
\]
and
\[
\psi_{n=0j=3/2m_j=3/2;l=2s=1/2}
\]
will have admixtures from $\phi_{n=0l=2m_l=2s=1/2m_s=-1/2}$ and $\phi_{n=0l=2m_l=1s=1/2m_s=1/2}$. 
How do we find these admixtures? Note that we don't specify the values of $m_l$ and $m_s$ 
in the functions $\psi$ since    
$\OP{j}^2$ does not commute with $\OP{L}_z$ and $\OP{S}_z$. 

We operate with 
\[
   \OP{j}^2=(\OP{l}+\OP{s})^2=\OP{l}^2+\OP{s}^2+2\OP{l}_z\OP{s}_z+\OP{l}_+\OP{s}_{-}+\OP{l}_{-}\OP{s}_{+}
\]
on the two $jm_j$ states, that is
\[
\OP{j}^2\psi_{n=0j=5/2m_j=3/2;l=2s=1/2}= \alpha\hbar^2[l(l+1)+\frac{3}{4}+2m_lm_s]\phi_{n=0l=2m_l=2s=1/2m_s=-1/2}+
\]
\[
\beta\hbar^2\sqrt{l(l+1)-m_l(m_l-1)}\phi_{n=0l=2m_l=1s=1/2m_s=1/2},
\]
and 
\[
\OP{j}^2\psi_{n=0j=3/2m_j=3/2;l=2s=1/2}= \alpha\hbar^2[l(l+1)+\frac{3}{4}+2m_lm_s]+ \phi_{n=0l=2m_l=1s=1/2m_s=1/2}+
\]
\[
\beta\hbar^2\sqrt{l(l+1)-m_l(m_l+1)}\phi_{n=0l=2m_l=2s=1/2m_s=-1/2}.
\]

This means that the eigenvectors $\phi_{n=0l=2m_l=2s=1/2m_s=-1/2}$ etc are not eigenvectors of $\OP{j}^2$. The above problems gives a $2\times2$ matrix that mixes the vectors $\psi_{n=0j=5/2m_j3/2;l=2m_ls=1/2m_s}$ and $\psi_{n=0j=3/2m_j3/2;l=2m_ls=1/2m_s}$ with the states  $\phi_{n=0l=2m_l=2s=1/2m_s=-1/2}$ and
$\phi_{n=0l=2m_l=1s=1/2m_s=1/2}$. The unknown coefficients $\alpha$ and $\beta$ results from eigenvectors of this matrix. That is, inserting all values $m_l,l,m_s,s$ we obtain the matrix 
\[
\left[ \begin{array} {cc} 19/4 & 2 \\ 2 & 31/4 \end{array} \right]\]
whose eigenvectors are the columns of
\[
\left[ \begin{array} {cc} 2/\sqrt{5} &1/\sqrt{5}  \\ 1/\sqrt{5} & -2/\sqrt{5} \end{array}\right]\]  

These numbers define the so-called Clebsch-Gordan coupling coefficients  (the overlaps between the two basis sets). We can thus write
\[
\psi_{njm_j;ls}=\sum_{m_lm_s}\langle lm_lsm_s|jm_j\rangle\phi_{nlm_lsm_s},
\]
where the coefficients $\langle lm_lsm_s|jm_j\rangle$ are the so-called Clebsch-Gordan coeffficients.

The Clebsch-Gordan coeffficients $\langle lm_lsm_s|jm_j\rangle$ have some interesting properties for us, like the following 
orthogonality relations
\[
\sum_{m_1m_2}\langle j_1m_1j_2m_2|JM\rangle\langle j_1m_1j_2m_2|J'M'\rangle=\delta_{J,J'}\delta_{M,M'},
\]
\[
\sum_{JM}\langle j_1m_1j_2m_2|JM\rangle\langle j_1m_1'j_2m_2'|JM\rangle=\delta_{m_1,m_1'}\delta_{m_2,m_2'},
\]
\[
\langle j_1m_1j_2m_2|JM\rangle=(-1)^{j_1+j_2-J}\langle j_2m_2j_1m_1|JM\rangle,
\]
and many others. The latter will turn extremely useful when we are going to define two-body states and interactions in a coupled basis.

For the single-particle case, we have the following eigenfunctions 
\[
\psi_{njm_j;ls}=\sum_{m_lm_s}\langle lm_lsm_s|jm_j\rangle\phi_{nlm_lsm_s},
\]
where the coefficients $\langle lm_lsm_s|jm_j\rangle$ are the so-called Clebsch-Gordan coeffficients.
The relevant quantum numbers are $n$ (related to the principal quantum number and the number of nodes of the wave function) and 
\[
   \OP{j}^2\psi_{njm_j;ls}=\hbar^2j(j+1)\psi_{njm_j;ls},
\]
\[
   \OP{j}_z\psi_{njm_j;ls}=\hbar m_j\psi_{njm_j;ls},
\]
\[
   \OP{l}^2\psi_{njm_j;ls}=\hbar^2l(l+1)\psi_{njm_j;ls},
\]
\[
   \OP{s}^2\psi_{njm_j;ls}=\hbar^2s(s+1)\psi_{njm_j;ls},
\]
but $s_z$ and $l_z$ do not result in good quantum numbers in a basis where we
use the angular momentum $j$.

For a two-body state where we couple two angular momenta $j_1$ and $j_2$ to a final
angular momentum $J$ with projection $M_J$, we can define a similar transformation in terms
of the Clebsch-Gordan coeffficients
\[
\psi_{(j_1j_2)JM_J}=\sum_{m_{j_1}m_{j_2}}\langle j_1m_{j_1}j_2m_{j_2}|JM_J\rangle\psi_{n_1j_1m_{j_1};l_1s_1}\psi_{n_2j_2m_{j_2};l_2s_2}.
\]
We will write these functions in a more compact form hereafter, namely,
\[
|(j_1j_2)JM_J\rangle=\psi_{(j_1j_2)JM_J},
\]
and 
\[
|j_im_{j_i}\rangle=\psi_{n_ij_im_{j_i};l_is_i},
\]
where we have skipped the explicit reference to $l$, $s$ and $n$. The spin of a nucleon is always $1/2$ while the value of $l$ can be deduced from the parity of the state.
It is thus normal to label a state with a given total angular momentum as 
$j^{\pi}$, where $\pi=\pm 1$. 

Our two-body state can thus be written as 
\[
|(j_1j_2)JM_J\rangle=\sum_{m_{j_1}m_{j_2}}\langle j_1m_{j_1}j_2m_{j_2}|JM_J\rangle|j_1m_{j_1}\rangle|j_2m_{j_2}\rangle.
\]
Due to the coupling order of the Clebsch-Gordan coefficient it reads as 
$j_1$ coupled to $j_2$ to yield a final angular momentum $J$. If we invert the order of coupling we would have
\[
|(j_2j_1)JM_J\rangle=\sum_{m_{j_1}m_{j_2}}\langle j_2m_{j_2}j_1m_{j_1}|JM_J\rangle|j_1m_{j_1}\rangle|j_2m_{j_2}\rangle,
\]
and due to the symmetry properties of the Clebsch-Gordan coefficient we have
\[
|(j_2j_1)JM_J\rangle=(-1)^{j_1+j_2-J}\sum_{m_{j_1}m_{j_2}}\langle j_1m_{j_1}j_2m_{j_2}|JM_J\rangle|j_1m_{j_1}\rangle|j_2m_{j_2}\rangle=(-1)^{j_1+j_2-J}|(j_1j_2)JM_J\rangle.
\]

We call the basis $|(j_1j_2)JM_J\rangle$ for the {\bf coupled basis}, or just $j$-coupled basis/scheme. The basis formed by the simple product of single-particle eigenstates 
$|j_1m_{j_1}\rangle|j_2m_{j_2}\rangle$ is called the {\bf uncoupled-basis}, or just the $m$-scheme basis. 

We have thus the coupled basis 
\[
|(j_1j_2)JM_J\rangle=\sum_{m_{j_1}m_{j_2}}\langle j_1m_{j_1}j_2m_{j_2}|JM_J\rangle|j_1m_{j_1}\rangle|j_2m_{j_2}\rangle.
\]
and the uncoupled basis 
\[
|j_1m_{j_1}\rangle|j_2m_{j_2}\rangle.
\]
The latter can easily be generalized to many single-particle states whereas the first 
needs specific coupling coefficients and definitions of coupling orders. 
The $m$-scheme basis is easy to implement numerically and is used in most standard shell-model codes. 
Our coupled basis obeys also the following relations
\[
   \OP{J}^2|(j_1j_2)JM_J\rangle=\hbar^2J(J+1)|(j_1j_2)JM_J\rangle
\]
\[
   \OP{J}_z|(j_1j_2)JM_J\rangle=\hbar M_J|(j_1j_2)JM_J\rangle,
\]
What follows now is a more technical discussion of how we can solve the two-nucleon problem.
This will lead us to the so-called Lippman-Schwinger equation for the scattering problem and a rewrite of Schr\"odinger's equation in relative and center-of-mass coordinates. 

Let us define the latter first. As we did earlier this semester, we define
the center-of-mass (CoM)  momentum as
 \[
    \vec{K}=\sum_{i=1}^A\vec{k}_i,
 \]
with $\hbar=c=1$ the wave number $k_i=p_i$, with $p_i$ the pertinent momentum of a single-particle state. 
We have also the relative momentum
\[
    \vec{k}_{ij}=\frac{1}{2}(\vec{k}_i-\vec{k}_j).
 \]
We will below skip the indices $ij$ and simply write $\vec{k}$

 In a similar fashion we can define the CoM coordinate
 \[
     \vec{R}=\frac{1}{A}\sum_{i=1}^{A}\vec{r}_i,
 \]
 and the relative distance 
\[
    \vec{r}_{ij}=(\vec{r}_i-\vec{r}_j).
 \]
With the definitions
 \[
    \vec{K}=\sum_{i=1}^A\vec{k}_i,
 \]
and 
\[
    \vec{k}_{ij}=\frac{1}{2}(\vec{k}_i-\vec{k}_j).
 \]
we can rewrite the two-particle kinetic energy (note that we use $\hbar=c=1$ as 
\[
\frac{\vec{k}_1^2}{2m_n}+\frac{\vec{k}_2^2}{2m_n}=\frac{\vec{k}^2}{m_n}+\frac{\vec{K}^2}{4m_n},
\]
where $m_n$ is the average of the proton and the neutron masses. 
Since the two-nucleon interaction depends only on the relative distance, this means that we can separate Schr\"odinger's equation in an equation for the center-of-mass motion and one for the relative motion.
With an equation for the relative motion only and a separate one for the center-of-mass motion we need to redefine the two-body quantum numbers.

Previously we had a two-body state vector defined as $|(j_2j_1)JM_J\rangle$ in a coupled basis. 
We will now define the quantum numbers for the relative motion. Here we need to define new orbital momenta (since these are the quantum numbers which change). 
We define 
\[
\OP{l}_1+\OP{l}_2=\OP{\lambda}=\OP{l}+\OP{L},
\]
where $\OP{l}$ is the orbital momentum associated with the relative motion and
$\OP{L}$ the corresponding one linked with the CoM. The total spin $S$ is unchanged since it acts in a different space. We have thus that
\[
\OP{J}=\OP{l}+\OP{L}+\OP{S},
\]
which allows us to define the angular momentum of the relative motion
\[
{ \cal J} =  \OP{l}+\OP{S},
\]
where ${ \cal J}$ is the total angular momentum of the relative motion.

The tensor force is given by
\[
S_{12} (\hat r) = \frac{3}{r^2}\left(\mbox{\boldmath $\sigma$}_1\cdot {\bf r}\right) \left(\mbox{\boldmath $\sigma$}_2\cdot {\bf r}\right) -\mbox{\boldmath $\sigma$}_1\cdot\mbox{\boldmath $\sigma$}_2\]
where the Pauli matrices are defined as
\[
\sigma_x =\begin{pmatrix} 0 & 1 \\ 1 & 0 \end{pmatrix},
\]
\[
\sigma_y =\begin{pmatrix} 0 & -\imath \\ \imath & 0 \end{pmatrix},
\]
and
\[
\sigma_z =\begin{pmatrix} 1 & 0 \\ 0 & -1 \end{pmatrix},
\]
with the properties $\sigma = 2{\bf S}$ (the spin of the system, being $1/2$ for nucleons), 
$\sigma^2_x=\sigma^2_y=\sigma_z={\bf 1}$ and
obeying the anti-commutation and commutation relations $\{\sigma_x,\sigma_y\} =2\delta_{x,y}$, 
$[\sigma_x,\sigma_y] =2\imath\sigma_z$ etc. 
The tensor force has an analogue in the interaction between 
The electromagnetic forces mediated by multiphonon exchanges between particles with a given spin values, leads to
amongst other terms, to the interaction between two magnetic dipoles. 
The potential energy which results from the interaction
between two magnetic dipole moments $\mbox{\boldmath $\mu$}_1$ and  $\mbox{\boldmath $\mu$}_2$  is given by
\[
V_{\mbox{\boldmath $\mu$}_1,\mbox{\boldmath $\mu$}_12} \propto -\frac{1}{r^3}\left(3(\mbox{\boldmath $\mu$}_1{\bf e})(\mbox{\boldmath $\mu$}_2{\bf e})-
\mbox{\boldmath $\mu$}_1\mbox{\boldmath $\mu$}_2\right),
\]
where ${\bf e}$ is a unit vector parallel to the relative distance $r$ between  the two dipoles.  
This expression, except for the different power in the denominator, is to the tensor force component of the nuclear force. The latter, as we will see later, is mediated by complex meson exchanges, of which the simplest one is represented by one-pion exchange. 
 
When we look at the expectation value of 
$\langle \mbox{\boldmath $\sigma$}_1\cdot\mbox{\boldmath $\sigma$}_2\rangle$, we can rewrite this expression in terms of the
spin ${\bf S}={\bf s}_1+{\bf s}_2$, resulting in 
\[
\langle\mbox{\boldmath $\sigma$}_1\cdot\mbox{\boldmath $\sigma$}_2\rangle=2(S^2-s_1^2-s_2^2)=2S(S+1)-3,
\]
where we $s_1=s_2=1/2$ leading to
\[
\left\{ \begin{array}{cc} \langle\mbox{\boldmath $\sigma$}_1\cdot\mbox{\boldmath $\sigma$}_2\rangle=1 &  \mathrm{if} \hspace{0.2cm} S=1\\
\langle\mbox{\boldmath $\sigma$}_1\cdot\mbox{\boldmath $\sigma$}_2\rangle=-3 & \mathrm{if} \hspace{0.2cm} S=0\\\end{array}\right.
\]
Similarly, the expectation value of the spin-orbit term is 
\[
\langle {\bf l}{\bf S} \rangle = \frac{1}{2}\left( {\cal J}({\cal J}+1)-l(l+1)-S(S+1)\right),
\]
which means that for $s$-waves with either $S=0$ and thereby ${\cal J}=0$ or $S=1$ and ${\cal J}=1$, the expectation value for the
spin-orbit force is zero. With the above phenomenological model, the
only contributions to the expectation value of the potential energy
stem from the central and the spin-spin components since the
expectation value of the tensor force is also zero. For $s=1/2$ spin
values only for two nucleons, the expectation value of the tensor
force operator is
\begin{table}[hbtp]
\begin{center}
\begin{tabular}{lccc} 
& & $l'$  &  \\ \hline
 & & & \\ 
$l$  & ${\cal J}+1$    & ${\cal J}$  & ${\cal J}-1$  \\ & & & \\  \hline 
& & & \\ 
${\cal J}+1$ & $-\frac{2{\cal J}({\cal J}+2)}{2{\cal J}+1}$  &0  &$\frac{6\sqrt{{\cal J}({\cal J}+1)}}{2{\cal J}+1}$   \\ 
& & & \\ 
${\cal J}$&  0      &2      &    0     \\ 
& & & \\ 
${\cal J}-1$& $\frac{6\sqrt{{\cal J}({\cal J}+1)}}{2{\cal J}+1}$       & 0     & $-\frac{2(2{\cal J}+1)}{2{\cal J}+1}$        \\
& & & \\ \hline 
\end{tabular}
\end{center}
\end{table}


From mirror nuclei we know that if the Coulomb interaction is
subtracted, the spectra of the excited states are almost at the same
energies. This leads us to the concept of isospin. The nuclear forces
are almost charge independent. If we assume they are, we can introduce
a new quantum number which is conserved. For nucleons only, that is a
proton and neutron, we can limit ourselves to two possible values
which allow us to distinguish between the two particles. If we assign
an isospin value of $\tau=1/2$ for protons and neutrons (they belong
to an isospin doublet, in the same way as we discussed the spin $1/2$
multiplet), we can define the neutron to have isospin projection
$\tau_z=+1/2$ and a proton to have $\tau_z=-1/2$. These assignements
are the standard choices in low-energy nuclear physics. The values are
reversed in particle physics. This means that we can define a
single-nucleon state function in terms of the quantum numbers $n$,
$j$, $m_j$, $l$, $s$, $\tau$ and $\tau_z$. Using our definitions in
terms of an uncoupled basis, we had
\[
\psi_{njm_j;ls}=\sum_{m_lm_s}\langle lm_lsm_s|jm_j\rangle\phi_{nlm_lsm_s},
\]
which we can now extend to
\[
\psi_{njm_j;ls}\xi_{\tau\tau_z}=\sum_{m_lm_s}\langle lm_lsm_s|jm_j\rangle\phi_{nlm_lsm_s}\xi_{\tau\tau_z},
\]
with the isospin spinors defined as 
\[
\xi_{\tau=1/2\tau_z=+1/2}=\left(\begin{array}{c} 1  \\ 0\end{array}\right),
\]
and
\[
\xi_{\tau=1/2\tau_z=-1/2}=\left(\begin{array}{c} 0  \\ 1\end{array}\right).
\]
We can then define the proton state function as 
\[
\psi^p({\bf r})  =\psi_{njm_j;ls}({\bf r})\left(\begin{array}{c} 0  \\ 1\end{array}\right), 
\]
and similarly for neutrons as
\[
\psi^n({\bf r})  =\psi_{njm_j;ls}({\bf r})\left(\begin{array}{c} 1  \\ 0\end{array}\right). 

We can in turn define the isospin Pauli matrices as
\[
\hat{\tau}_x =\left(\begin{array}{cc} 0 & 1 \\ 1 & 0 \end{array}\right),
\]
\[
\hat{\tau}_y =\left(\begin{array}{cc} 0 & -\imath \\ \imath & 0 \end{array}\right),
\]
and
\[
\hat{\tau}_z =\left(\begin{array}{cc} 1 & 0 \\ 0 & -1 \end{array}\right),
\]
and operating with $\hat{\tau}_z$ on the proton state function we have
\[
\hat{\tau}_z\psi^p({\bf r})=-\frac{1}{2}\psi^p({\bf r}),
\]
and for neutrons we have
\[
\hat{\tau}\psi^n({\bf r})=\frac{1}{2}\psi^n({\bf r}).
\]


We can now define the so-called charge operator as 
\[
\frac{\hat{Q}}{e} = \frac{1}{2}\left(1-\hat{\tau}_z\right)=\begin{Bmatrix} 0 & 0 \\ 0 & 1 \end{Bmatrix},
\]
which results in 
\[
\frac{\hat{Q}}{e}\psi^p({\bf r})=\psi^p({\bf r}),
\]
and 
\[
\frac{\hat{Q}}{e}\psi^n({\bf r})=0,
\]
as it should be. 
The total isospin is defined as
\[
\hat{T}=\sum_{i=1}^A\hat{\tau}_i,
\]
and its corresponding isospin projection as
\[
\hat{T}_z=\sum_{i=1}^A\hat{\tau}_{z_i},
\]
with eigenvalues $T(T+1)$ for $\hat{T}$ and $1/2(N-Z)$ for $\hat{T}_z$, where $N$ is the number of neutrons and $Z$ the number of protons. 

If charge is conserved, the Hamiltonian $\hat{H}$ commutes with $\hat{T}_z$ and all members of a given isospin multiplet
(that is the same value of $T$) have the same energy and there is no $T_z$ dependence and we say that $\hat{H}$ is a scalar in isospin space.
If charge is conserved, the Hamiltonian $\hat{H}$ commutes with $\hat{T}_z$ and all members of a given isospin multiplet
(that is the same value of $T$) have the same energy and there is no $T_z$ dependence and we say that $\hat{H}$ is a scalar in isospin space.

If we now add isospin to our simple $V_4$ interaction model, we end up with $8$ operators, popularly dubbed $V_8$ interaction model. The explicit form reads
\[
V({\bf r})= \left\{ C_c + C_\sigma 
\mbox{\boldmath $\sigma$}_1\cdot\mbox{\boldmath $\sigma$}_2
 + C_T \left( 1 + {3\over m_\alpha r} + {3\over
\left(m_\alpha r\right)^2}\right) S_{12} (\hat r)\right. 
\]
\[
\left. + C_{SL} \left( {1\over m_\alpha r} + {1\over \left( m_\alpha r\right)^2}
\right) {\bf L}\cdot {\bf S}
\right\} \frac{e^{-m_\alpha r}}{m_\alpha r}
\]
\[
+ \left\{ C_{c\tau} + C_{\sigma\tau} 
\mbox{\boldmath $\sigma$}_1\cdot\mbox{\boldmath $\sigma$}_2
 + C_{T\tau} \left( 1 + {3\over m_\alpha r} + {3\over
\left(m_\alpha r\right)^2}\right) S_{12} (\hat r)\right. 
\]
\[
\left. + C_{SL\tau} \left( {1\over m_\alpha r} + {1\over \left( m_\alpha r\right)^2}
\right) {\bf L}\cdot {\bf S}
\right\}\mbox{\boldmath $\tau$}_1\cdot\mbox{\boldmath $\tau$}_2 \frac{e^{-m_\alpha r}}{m_\alpha r}
\]


The expectation value of 
$\langle \mbox{\boldmath $\tau$}_1\cdot\mbox{\boldmath $\tau$}_2\rangle$ is calculated along the same lines as we did for
the two spin matrices and we obtain
\[
\left\{ \begin{array}{cc} \langle\mbox{\boldmath $\sigma$}_1\cdot\mbox{\boldmath $\sigma$}_2\rangle\langle\mbox{\boldmath $\tau$}_1\cdot\mbox{\boldmath $\tau$}_2\rangle=9 &  \mathrm{if} \hspace{0.2cm} S=T=0\\
\langle\mbox{\boldmath $\sigma$}_1\cdot\mbox{\boldmath $\sigma$}_2\rangle\langle\mbox{\boldmath $\tau$}_1\cdot\mbox{\boldmath $\tau$}_2\rangle=1 & \mathrm{if} \hspace{0.2cm} S=T=1 \\
\langle\mbox{\boldmath $\sigma$}_1\cdot\mbox{\boldmath $\sigma$}_2\rangle\langle\mbox{\boldmath $\tau$}_1\cdot\mbox{\boldmath $\tau$}_2\rangle=-3 & \mathrm{if} \hspace{0.2cm} (S,T)=(0,1) \vee (S,T)=(1,0)
\end{array}\right.
\]

The total two-nucleon state function has to be anti-symmetric. The total function contains a spatial part, a spin part and an isospin part. If isospin is conserved, this leads to in case we have an $s$-wave with spin $S=0$ to an isospin 
two-body state with $T=1$ since the spatial part is symmetric and the spin part is anti-symmetric. 

Since the projections for $T$ are $T_z=-1,0,1$, we can have a $pp$, an $nn$ and a $pn$ state.

For $l=0$ and $S=1$, a so-called triplet state, $^3S_1$, we must have $T=0$, meaning that we have only one state, a $pn$ state. For other partial waves, see exercises. 
We can systemize this in a table as follows, recalling that $|{\bf l}-{\bf S}| \le |{\bf J}| \le |{\bf l}-{\bf S}|$,  
\begin{table}[hbtp]
\begin{center}
\begin{tabular}{cccccccc} 
$^{2S+1}l_{\cal J}$ &${\cal J}$& $l$ & $S$  & $T$ & $|pp\rangle$ &$|pn\rangle$ &$|nn\rangle$  \\ \hline
$^{1}S_0$&0 &0 &0 &1  &yes &yes &yes \\ 
$^{3}S_1$&1 &0 &1 &0  &no &yes &no \\ 
$^{3}P_0$&0 &1 &1 &1  &yes &yes &yes \\ 
$^{1}P_1$&0 &1 &0 &0  &no &yes &no \\ 
$^{3}P_1$&1 &1 &1 &1  &yes &yes &yes \\ 
$^{3}P_2$&2 &1 &1 &1  &yes &yes &yes \\ 
$^{3}D_1$&1 &1 &1 &0  &no &yes &no \\ 
$^{3}F_2$&2 &3 &1 &1  &yes &yes &yes \\ \hline
\end{tabular}
\end{center}
\end{table}



\section{Constructing a nucleon-nucleon interaction from data}
In this section we derive the equations for solving Schr\"odinger's equation for bound and scattering states. For the two-nucleon sector 
there is only one bound state, namely the deuteron. In order to achieve this, we need to introduce several technicalities, amongst these 
a so-called partial wave expansion of the forces. With a partial wave expansion, we will be able to reduce a six dimensional equation for two particles in thre dimensions, to sets of coupled or uncoupled one-dimensional equations. We start with the bound state problem, ending up with an algorithm 
for solving the two-body Schr\"odinger equation for a given two-body interaction.
\subsection{The bound state problem}

The non-relativistic  Schr\"odinger equation in the abstract vector representation
\begin{equation}
  \left( T + V \right) \vert \psi_n \rangle = E_n \vert\psi_n \rangle 
  \label{eq:vec_rep}
\end{equation}
Here $T$ is the kinetic energy operator and $V$ is the potential operator. 
The eigenstates form a complete orthonormal set according to 
\[ 
{\bf 1} = \sum_n \vert \psi_n\rangle\langle \psi_n \vert, \:\: \langle \psi_n\vert \psi_{n'} \rangle = \delta_{n,n'}
\]

The most commonly used representations of equation~\ref{eq:vec_rep} are the coordinate
and
the momentum space representations. They define the completeness relations 
\begin{eqnarray}
 {\bf 1}&  = &  \int d{\bf r} \:\vert  {\bf r} \rangle \langle {\bf r}\vert, \:\: 
 \langle  {\bf r}\vert  {\bf r'} \rangle = \delta ( {\bf r}-{\bf r'}) \\
{\bf 1} & = & \int d{\bf k} \:\vert  {\bf k} \rangle \langle {\bf k}\vert, \:\: 
  \langle  {\bf k}\vert  {\bf k'} \rangle = \delta ( {\bf k}-{\bf k'}) 
\end{eqnarray}
Here the basis states in  both ${\bf r}$- and ${\bf k}$-space are dirac-delta 
function normalized. From this it follows that the plane-wave states are given by,
\begin{equation}
  \langle  {\bf r}\vert  {\bf k} \rangle = \left( 1\over2\pi \right)^{3/2}\exp \left( i {\bf k\cdot r} \right)
  \label{eq:planewave1}
\end{equation}
which is a transformation function defining the mapping from the abstract 
$ \vert {\bf k} \rangle $ to the abstract $\vert {\bf r}\rangle $ space.

That the ${\bf r}$-space basis states are 
delta-function normalized follows from 
\begin{equation}
  \delta ( {\bf r}-{\bf r'}) = \langle {\bf r} \vert {\bf r}'\rangle = 
  \langle {\bf r} \vert {\bf 1} \vert {\bf r}'\rangle = 
  \int \mathrm{d}{\bf k} \langle {\bf r}\vert {\bf k} \rangle \langle {\bf k}\vert {\bf r}' \rangle = 
  \left( {1\over 2\pi}\right)^3 \int \mathrm{d}{\bf k} e^{i {\bf k}({\bf r} - {\bf r}')} 
\end{equation}
and the same for the momentum space basis states,
\begin{equation}
  \delta ( {\bf k}-{\bf k'}) = \langle {\bf k} \vert {\bf k}'\rangle = 
  \langle {\bf k} \vert {\bf 1} \vert {\bf k}'\rangle = 
  \int \mathrm{d}{\bf r} \langle {\bf k}\vert {\bf r} \rangle \langle {\bf r}\vert {\bf k}' \rangle = 
  \left( {1\over 2\pi}\right)^3 \int \mathrm{d}{\bf r} e^{i {\bf r}({\bf k} - {\bf k}')} 
\end{equation}

Projecting equation~(\ref{eq:vec_rep}) on momentum states
the momentum space Schr\"odinger equation is obtained,
\begin{equation}
  {\hbar^2 \over 2\mu} k^2 \psi_n({\bf k})  + 
  \int d{\bf k'}\: V({\bf k}, {\bf k'}) \psi_n({\bf k'}) = 
  E_n \psi_n({\bf k})
  \label{eq:momspace1}
\end{equation}
Here the notation $\psi_n({\bf k}) = \langle {\bf k} \vert \psi_n\rangle $ and 
$ \langle {\bf k} \vert V \vert {\bf k}' \rangle = V({\bf k}, {\bf k'})$ has been introduced.
The potential in momentum space is given by a double Fourier-transform 
of the potential in coordinate space, i.e.
\begin{equation} 
  V ({\bf k}, {\bf k'}) = \left( {1\over 2\pi}\right)^3 \int \mathrm{d}{\bf r}\int \mathrm{d}{\bf r}'\: 
  e^{-i {\bf kr} }V({\bf r},{\bf r}') e^{i{\bf k}'{\bf r}'}  
\end{equation}

Here it is assumed that the potential interaction does not contain any spin dependence. 
Instead of a differential equation in coordinate space, the Schr\"odinger
equation becomes an integral equation in momentum space. This has 
many tractable features. Firstly, most realistic 
nucleon-nucleon interactions derived from field-theory are given 
explicitly in momentum space. Secondly, the boundary conditions imposed
on the differential equation in coordinate space are automatically built into the
integral equation. And last, but not least, integral equations are easy to numerically 
implement, and convergence is obtained by just increasing the number of integration
points.
Instead of solving the three-dimensional integral equation given in equation~(\ref{eq:momspace1}), an 
infinite set of 1-dimensional equations can be obtained via a  partial wave
expansion. 

The wave function $ \psi_n({\bf k}) $ can be expanded in a complete set of spherical harmonics, i.e. 
\begin{equation}
  \psi_n({\bf k}) = \sum _{lm} \psi_{nlm}(k)Y_{lm} (\hat{k}), \:\:
  \psi_{nlm}(k) = \int d\hat{k} Y_{lm}^*(\hat{k})\psi_n({\bf k}).   
  \label{eq:part_wave1}
\end{equation}
By inserting equation~\ref{eq:part_wave1} in equation~\ref{eq:momspace1}, and projecting from the left
$Y_{lm}(\hat{k})$, the three-dimensional Schr\"odinger equation~(\ref{eq:momspace1}) is reduced
to an infinite set of  1-dimensional angular momentum coupled integral equations, 
\begin{equation}
  \left( {\hbar^2 \over 2\mu} k^2 - E_{nlm}\right) \psi_{nlm}(k) =  
  -\sum_{l'm'} \int_{0}^\infty dk' {k'}^2 V_{lm, l'm'}(k,k') \psi_{nl'm'}(k') 
  \label{eq:part_wave2}
\end{equation}
where the angular momentum projected potential takes the form,
\begin{equation}
  V_{lm, l'm'}(k,k') = \int \mathrm{d}{\hat{k}} \int \mathrm{d}{\hat{k}'}\: 
    Y_{lm}^*(\hat{k})V({\bf k}, {\bf k'})Y_{l'm'}(\hat{k}')
    \label{eq:pot1}
\end{equation}
here $\mathrm{d}{\hat k} = \mathrm{d}\theta \sin\theta \:\mathrm{d}\varphi $.

Often the potential is given in position space, so it is convenient to establish 
the connection between $V_{lm, l'm'}(k,k')$ and $V_{lm, l'm'}(r,r')$. Inserting 
position space completeness in equation~(\ref{eq:pot1}) gives
\begin{eqnarray}
\nonumber
  V_{lm, l'm'}(k,k')& = &\int \mathrm{d}{\bf{r}} \int \mathrm{d}{\bf{r}'}\: 
  \int \mathrm{d}{\hat{k}} \int \mathrm{d}{\hat{k}'}\: 
  Y_{lm}^*(\hat{k})\langle {\bf k}\vert {\bf r} \rangle
  \langle{\bf r}  \vert V \vert {\bf r}' \rangle
  \langle {\bf r'}\vert {\bf k}' \rangle Y_{lm}(\hat{k}') \\ \nonumber
  &=&\int \mathrm{d}{\bf{r}} \int \mathrm{d}{\bf{r}'}\: 
  \left\{ \int \mathrm{d}{\hat{k}}  Y_{lm}^*(\hat{k})\langle {\bf k}\vert {\bf r} \rangle \right\} \\
  &&\times\langle{\bf r}  \vert V \vert {\bf r}' \rangle
  \left\{ \int \mathrm{d}{\hat{k}'}\:   Y_{lm}(\hat{k}') \langle {\bf r'}\vert {\bf k}' \rangle\right\}
  \label{eq:pot2}
\end{eqnarray}

Since the plane waves depend only on the absolute values of position and momentum, 
$ \vert {\bf k} \vert, \vert {\bf r} \vert  $,
and the angle between them, $ \theta_{kr} $, they may be expanded in terms of bipolar harmonics of 
zero rank, i.e.  
\begin{equation} 
  e^{i {\bf k}\cdot {\bf r}} = 4\pi \sum_{l=0}^{\infty} i^l j_l(kr)\left( Y_l(\hat{k}) \cdot Y_l(\hat{r}) \right)
  = \sum_{l=0}^{\infty} (2l+1)i^l j_l(kr) P_l(\cos \theta_{kr}) 
\end{equation}
where the addition theorem for spherical harmonics has been used in order to write
the expansion in terms of Legendre polynomials. The spherical Bessel functions, $j_l(z)$,  
are given in terms of Bessel functions of the first kind with half integer orders,  
\[
j_l(z) = \sqrt{\pi \over 2 z} J_{l+1/2}(z).  
\]
Inserting the plane-wave expansion
into the brackets of equation~(\ref{eq:pot2}) yields, 
\begin{eqnarray}
  \nonumber
  \int \mathrm{d}{\hat{k}}  Y_{lm}^*(\hat{k})\langle {\bf k}\vert {\bf r} \rangle & = &  
  \left( {1\over 2\pi} \right) ^{3/2}4\pi i^{-l} j_l(kr) Y_{lm}^*(\hat{r}), \\  
  \nonumber
  \int \mathrm{d}{\hat{k}'}\:   Y_{lm}(\hat{k}') \langle {\bf r'}\vert {\bf k}' \rangle & = &  
  \left( {1\over 2\pi} \right) ^{3/2}4\pi i^{l'} j_{l'}(k'r') Y_{l'm'}(\hat{r}). 
\end{eqnarray}
The connection between the momentum- and position space angular momentum 
projected potentials are then given, 
\begin{equation}
  V_{lm, l'm'}(k,k') = {2 \over \pi} i^{l' -l}\int_0^\infty dr\: r^2 \int_0^\infty dr'\: {r'}^2 
  j_l(kr) V_{lm,l'm'}(r,r') j_{l'}(k'r')
  \label{eq:pot3}
\end{equation}
which is known as a double Fourier-Bessel transform. The position space angular 
momentum projected potential is given by
\begin{equation}
  V_{lm, l'm'}(r,r') = \int \mathrm{d}{\hat{r}} \int \mathrm{d}{\hat{r}'}\: 
  Y_{lm}^*(\hat{r})V({\bf r}, {\bf r'})Y_{l'm'}(\hat{r}').
  \label{eq:pot4}
\end{equation}
No assumptions of locality/non-locality and deformation of the interaction has so far been made, 
and the result in equation~(\ref{eq:pot3}) is general. In position space the Schr\"odinger equation 
takes form of an integro-differential equation in case of a non-local interaction, 
in momentum space the Schr\"odinger equation is an ordinary integral equation of the Fredholm type, 
see equation~(\ref{eq:part_wave2}). This is a further advantage of the momentum space approach as compared to 
the standard position space approach.  
If we assume that the 
interaction is of local character, i.e. 
\begin{equation}
  \nonumber
  \langle {\bf r}\vert V \vert {\bf r'}\rangle = V({\bf r}) \delta( {\bf r}-{\bf r}' ) = 
  V({\bf r}) {\delta( { r}-{r}' ) \over r^2} \delta ( \cos \theta - \cos \theta' ) \delta (\varphi-\varphi'), 
\end{equation}
then equation~(\ref{eq:pot4}) reduces to 
\begin{equation}
  V_{lm, l'm'}(r,r') = {\delta( { r}-{r}' ) \over r^2}\int \mathrm{d}{\hat{r}}\:
  Y_{lm}^*(\hat{r})V({\bf r})Y_{l'm'}(\hat{r}),
  \label{eq:pot5}
\end{equation}
and equation~(\ref{eq:pot3}) reduces to  
\begin{equation}
  V_{lm, l'm'}(k,k') = {2 \over \pi} i^{l' -l}\int_0^\infty dr\: r^2 \:
  j_l(kr) V_{lm,l'm'}(r) j_{l'}(k'r)
  \label{eq:pot6}
\end{equation}
where 
\begin{equation}
  V_{lm, l'm'}(r) = \int \mathrm{d}{\hat{r}}\:
  Y_{lm}^*(\hat{r})V({\bf r})Y_{l'm'}(\hat{r}),
  \label{eq:pot10}
\end{equation}
In the case that the interaction is central, $V({\bf r}) = V(r)$, then
\begin{equation}
  V_{lm, l'm'}(r) = V(r) \int \mathrm{d}{\hat{r}}\:
  Y_{lm}^*(\hat{r})Y_{l'm'}(\hat{r}) = V(r) \delta_{l,l'}\delta_{m,m'},
  \label{eq:pot7}
\end{equation}
and 
\begin{equation}
  V_{lm, l'm'}(k,k') = {2 \over \pi} \int_0^\infty dr\: r^2 \:
  j_l(kr) V(r) j_{l'}(k'r)\delta_{l,l'}\delta_{m,m'} = 
  V_l(k,k') \delta_{l,l'}\delta_{m,m'}
  \label{eq:pot8}
\end{equation}
where the momentum space representation of the interaction finally reads,
\begin{equation}
  V_{l}(k,k') = {2 \over \pi} \int_0^\infty dr\: r^2 \:
  j_l(kr) V(r) j_{l}(k'r).
  \label{eq:pot9}
\end{equation}
For a local and spherical symmetric potential, 
the coupled momentum space Schr\"odinger equations given in equation~(\ref{eq:part_wave2})
decouples in angular momentum, 
giving
\begin{equation}
  {\hbar^2 \over 2\mu} k^2 \psi_{n l}(k) + 
  \int_{0}^\infty dk' {k'}^2 V_{l}(k,k') \psi_{n l }(k') =
  E_{n l} \psi_{n l}(k) 
  \label{eq:momentum_space}
\end{equation}   
Where we have written $\psi_{n l }(k) = \psi_{nlm}(k)$, since the 
equation becomes independent of the projection $m$ for spherical symmetric interactions. 
The momentum space wave functions $\psi_{n l}(k) $ defines a complete orthogonal set 
of functions, which spans the space of functions with a positive finite Euclidean norm 
 (also called $l^2$-norm), $ \sqrt{ \langle \psi_n \vert \psi_n \rangle} $, which 
is a Hilbert space. The corresponding normalized wave function in coordinate space
is given by the Fourier-Bessel transform 
\begin{equation}
  \phi_{n l}(r)  = \sqrt{ 2\over \pi}\int dk\: k^2 j_l(kr) \psi_{n l}(k)
\end{equation}    
We will thus assume that the interaction is spherically symmetric and use
the partial wave expansion of the plane waves in
terms of spherical harmonics.
This means that we can separate the radial part of the wave function from its
angular dependence. The wave function of the relative motion is described
in terms of plane waves as
\begin{equation}
       e^{\imath {\bf kr}}  =
       \bra{\bf r}{\bf k}\rangle =  4\pi \sum_{lm} \imath ^{l}
        j_{l} (kr) Y_{lm}^{*}({\bf \hat{k}}) Y_{lm}({\bf \hat{r}}),
\end{equation}
where $j_l$ is a spherical Bessel function and $Y_{lm}$ the
spherical harmonics.
In terms of the relative and center-of-mass momenta ${\bf k}$ and
${\bf K}$, the potential in momentum space is related to the nonlocal operator
$V({\bf r},{\bf r}')$ by
\[
      \bra{{\bf k'K'}}V \ket{{\bf kK}} =
       \int d {\bf r}d {\bf r'}
        e^{-\imath {\bf k'r'}}V({\bf r'},{\bf r}) e^{\imath {\bf kr}}
       \delta({\bf K},{\bf K'}).
\]
We will assume that the interaction is spherically symmetric.
Can separate the radial part of the wave function from its
angular dependence. The wave function of the relative motion is described
in terms of plane waves as
\[
       e^{\imath {\bf kr}}  =
       \bra{\bf r}{\bf k}\rangle =  4\pi \sum_{lm} \imath ^{l}
        j_{l} (kr) Y_{lm}^{*}({\bf \hat{k}}) Y_{lm}({\bf \hat{r}}),
\]
where $j_l$ is a spherical Bessel function and $Y_{lm}$ the
spherical harmonic.
This partial wave basis is useful for defining the operator for
the nucleon-nucleon interaction, which
is symmetric with respect to rotations, parity and
isospin transformations. These symmetries imply that the interaction is
diagonal with respect to the quantum numbers of total relative angular
momentum ${\cal J}$, spin $S$ and isospin $T$ (we skip isospin for the moment). Using the above plane wave expansion,
and coupling to final ${\cal J}$ and $S$ and $T$ we get
\[
      \bra{{\bf k'}} V \ket{{\bf k}}
       = (4\pi)^2 \sum_{STll'm_lm_{l'}{\cal J}}
      \imath ^{l+l'} Y_{lm}^{*}({\bf \hat{k}}) Y_{l'm'}({\bf \hat{k}'})
\]
\[
      \langle lm_lSm_S|{\cal J}M\rangle \langle l'm_{l'}Sm_S|{\cal J}M\rangle          
      \bra{k'l'S{\cal J}M}V \ket{klS{\cal J}M},
\]
where we have defined
\begin{equation}
    \bra{k'l'S{\cal J}M}V \ket{klS{\cal J}M}=
    \int   j_{l'}(k'r')\bra{l'S{\cal J}M}V(r',r)\ket{lS{\cal J}M}j_l(kr) {r'}^2 dr' r^2 dr.
\end{equation}
We have omitted the momentum of the center-of-mass motion ${\bf K}$ and the 
corresponding orbital momentum $L$, since the interaction is diagonal
in these variables.

\subsubsection{Algorithm for solving the bound state problem}

\subsection{The scattering problem}

\subsubsection{Derivation of the Lippman-Schwinger equation}


We define the operator 
\[
g_0^{(\pm)}= \lim_{\epsilon\rightarrow 0^+}\frac{1}{E-\hat{H}_0\pm \imath \epsilon},
\]
where the energy $E=(\hbar k)^2/m = E_k$  is the kinetic energy of relative motion of two nucleons. Recall that since we work 
in the center-of-mass system of the two-nucleon system, this energy translates into the relative kinetic energy of two nucleons. $H_0$ is then just the 
kinetic energy operator for two nucleons.
The state 
\[
|\psi_k^{(\pm)}\rangle = |k\rangle +g_0^{(\pm)}\hat{V}|\psi_k^{(\pm)}\rangle,
\]
satisfies the Schr\"odinger 
\[
(E_k-\hat{H}_0)|\psi_k^{(\pm)}\rangle = \hat{V}|\psi_k^{(\pm)}\rangle,
\]
since 
\[
(E_k-\hat{H}_0)|k\rangle = 0,
\]
and 
\[
(E_k-\hat{H}_0)g_0^{(\pm)} = 1.
\]
Generally we have $\psi(\bf r}) = \langle {\bf r} | \psi\rangle$. In case the states are eigenstates of $\hat{H}_0$ we have
$ \langle {\bf r} | {\bf k} \rangle \propto \exp{\imath {\bf k}{\bf r}}$. The momenta states are properly orthogonalized.

e derive now the so-called Lippman-Schwinger equation. We will do this in an operator form first.
Thereafter, we rewrite it in terms of various quantum numbers such as relative momenta, orbital momenta etc. 
The Schr\"odinger equation in abstract vector representation is
\[
  \left( \hat{H}_0 + \hat{V} \right) \vert \psi_n \rangle = E_n \vert\psi_n \rangle. 
\]
In our case for the two-body problem $\hat{H}_0$ is just the kinetic energy. 
We rewrite it as 
\[
\left( \hat{H}_0 -E_n \right)\vert\psi_n \rangle =-\hat{V}\vert \psi_n \rangle . 
\]
We assume that the inverse of $\left( \hat{H}_0 -E_n\right)$ exists and rewrite this equation as
\[
\vert\psi_n \rangle =\frac{1}{\left( E_n -\hat{H}_0\right)}\hat{V}\vert \psi_n \rangle . 
\]

The equation
\[
\vert \psi_n \rangle =\frac{1}{\left( E_n -\hat{H}_0\right)}\hat{V}\vert \psi_n \rangle,
\]
is normally solved in an iterative fashion. 
We assume first that
\[
\vert\psi_n \rangle = \vert\phi_n \rangle,
\] 
where $\vert\phi_n \rangle$ are the eigenfunctions of 
\[
\hat{H}_0\vert \phi_n \rangle=\omega_n\vert \phi_n \rangle
\]
the so-called unperturbed problem. In our case, these will simply be the kinetic energies of the relative motion. 

Inserting  $\vert\phi_n \rangle$  on the right-hand side of 
\[
\vert \psi_n \rangle =\frac{1}{( E_n -\hat{H}_0)}\hat{V}\vert \psi_n \rangle,
\]
yields
\[
\vert \psi_n \rangle =\vert\phi_n \rangle+\frac{1}{\left( E_n -\hat{H}_0\right)}\hat{V}\vert \phi_n \rangle,
\]
as our first iteration. 
Reinserting again gives
\[
\vert \psi_n \rangle =\vert\phi_n \rangle+\frac{1}{\left( E_n -\hat{H}_0\right)}\hat{V}\vert \phi_n \rangle+\frac{1}{( E_n -\hat{H}_0)}\hat{V}\frac{1}{\left( E_n -\hat{H}_0\right)}\hat{V}\vert \phi_n \rangle,
\]
and continuing we obtain
\[
\vert \psi_n \rangle =\sum_{i=0}^{\infty}\left[\frac{1}{( E_n -\hat{H}_0)}\hat{V}\right]^i\vert \phi_n \rangle.
\]
It is easy to see that 
\[
\vert \psi_n \rangle =\sum_{i=0}^{\infty}\left[\frac{1}{(E_n -\hat{H}_0)}\hat{V}\right]^i\vert \phi_n \rangle,
\]
can be rewritten as 
\[
\vert \psi_n \rangle =\vert\phi_n \rangle+\frac{1}{( E_n -\hat{H}_0)}
\hat{V}\left(1+ \frac{1}{(E_n -\hat{H}_0)}\hat{V}+\frac{1}{(E_n -\hat{H}_0)}\hat{V}\frac{1}{(E_n -\hat{H}_0)}\hat{V}+\dots\right]\vert \phi_n \rangle,
\]
which we rewrite as 
\[
\vert \psi_n \rangle =\vert\phi_n \rangle+\frac{1}{(E_n -\hat{H}_0)}\hat{V}\vert \psi_n \rangle.
\]
In operator form we have thus
\[
\vert \psi_n \rangle =\vert\phi_n \rangle+\frac{1}{(E_n -\hat{H}_0)}\hat{V}\vert \psi_n \rangle.
\]
We multiply from the left with $\hat{V}$ and $\langle \phi_m \vert$ and obtain
\[
\langle \phi_m \vert\hat{V}\vert \psi_n \rangle =\langle \phi_m \vert\hat{V}\vert\phi_n \rangle+\langle \phi_m \vert\hat{V}\frac{1}{(E_n -\hat{H}_0)}\hat{V}\vert \psi_n \rangle.
\]
We define thereafter the so-called $T$-matrix as
\[
\langle \phi_m \vert\hat{T}\vert \phi_n \rangle=\langle \phi_m \vert\hat{V}\vert \psi_n \rangle.
\]
We can rewrite our equation as
\[
\langle \phi_m \vert\hat{T}\vert \phi_n \rangle =\langle \phi_m \vert\hat{V}\vert\phi_n \rangle+\langle \phi_m \vert\hat{V}\frac{1}{(E_n -\hat{H}_0)}\hat{T}\vert \phi_n \rangle.
\]
The equation
\[
\langle \phi_m \vert\hat{T}\vert \phi_n \rangle =\langle \phi_m \vert\hat{V}\vert\phi_n \rangle+\langle \phi_m \vert\hat{V}\frac{1}{(E_n -\hat{H}_0)}\hat{T}\vert \phi_n \rangle,
\]
is called the Lippman-Schwinger equation. Inserting the completeness relation
\[ 
{\bf 1} = \sum_n \vert \phi_n\rangle\langle \phi_n \vert, \:\: \langle \phi_n\vert \phi_{n'} \rangle = \delta_{n,n'}
\]
we have 
\[
\langle \phi_m \vert\hat{T}\vert \phi_n \rangle =\langle \phi_m \vert\hat{V}\vert\phi_n \rangle+\sum_k \langle \phi_m \vert\hat{V}\vert \phi_k\rangle\frac{1}{(E_n -\omega_k)}\langle \phi_k \vert\hat{T}\vert \phi_n \rangle,
\]
which is (when we specify the state $\vert\phi_n \rangle$) an integral equation that can actually be solved by matrix inversion easily! The unknown quantity is the $T$-matrix.

Now we wish to introduce a partial wave decomposition in order to solve the Lippman-Schwinger equation. With a partial wave decomposition we can reduce a three-dimensional integral equation to a one-dimensional one. 

We wrote the Lippman-Schwinger equation as
\[
\langle \phi_m \vert\hat{T}\vert \phi_n \rangle =\langle \phi_m \vert\hat{V}\vert\phi_n \rangle+\sum_k \langle \phi_m \vert\hat{V}\vert \phi_k\rangle\frac{1}{(E_n -\omega_k)}\langle \phi_k \vert\hat{T}\vert \phi_n \rangle.
\]
How do we rewrite it in a partial wave expansion with momenta $k$?

The general structure of the $T$-matrix in partial waves is
\[
   T_{ll'}^{\alpha}(kk'K\omega)=V_{ll'}^{\alpha}(kk')
\]
\[
   +{\displaystyle \frac{2}{\pi}\sum_{l''m_{l''}M_S}\int_{0}^{\infty} d {\bf q}
   (\langle l''m_{l''}Sm_S|{\cal J}M\rangle)^2
   \frac{Y_{l''m_{l''}}^*(\hat{{\bf q}})
   Y_{l''m_{l''}}(\hat{{\bf q}}) V_{ll''}^{\alpha}(kq)
   T_{l''l'}^{\alpha}(qk'K\omega)}
   {\omega -H_0}},
   \label{eq:bspartial}
\]
The  shorthand notation
\[
    T_{ll'}^{\alpha}(kk'K\omega)=
   \bra{kKlL{\cal J}S}T(\omega)\ket{k'Kl'L{\cal J}S},
\]
denotes the $T$-matrix
with momenta $k$ and $k'$ and orbital momenta $l$ and $l'$
of the relative motion, and
$K$ is the corresponding momentum of
the center-of-mass motion. Further, $L$, ${\cal J}$, $S$ and $T$
are the orbital momentum of the center-of-mass motion, the
total angular momentum,
spin and isospin, respectively. 
Due to the nuclear tensor force (to be discussed later), the interaction is not diagonal in $ll'$.

Using the orthogonality
properties of the Clebsch-Gordan coefficients and the spherical harmonics,
we obtain the well-known
one-dimensional angle independent
integral equation
\[
   T_{ll'}^{\alpha}(kk'K\omega)=V_{ll'}^{\alpha}(kk')
   +\frac{2}{\pi}\sum_{l''}\int_{0}^{\infty} dqq^2
   \frac{V_{ll''}^{\alpha}(kq)
   T_{l''l'}^{\alpha}(qk'K\omega)}
   {\omega -H_0}.
\]
Inserting the denominator we arrive at 
\[
   \hat{T}_{ll'}^{\alpha}(kk'K)=\hat{V}_{ll'}^{\alpha}(kk')
   +\frac{2}{\pi}\sum_{l''}\int_{0}^{\infty} dqq^2
   \hat{V}_{ll''}^{\alpha}(kq)
   \frac{1}{k^2-q^2 +i\epsilon}
   \hat{T}_{l''l'}^{\alpha}(qk'K).
\]

To parameterize the nucleon-nucleon interaction we solve the Lippman-Scwhinger
equation
\[
   T_{ll'}^{\alpha}(kk'K)=V_{ll'}^{\alpha}(kk')
   +\frac{2}{\pi}\sum_{l''}\int_{0}^{\infty} dqq^2
   V_{ll''}^{\alpha}(kq)
   \frac{1}{k^2-q^2 +i\epsilon}
   T_{l''l'}^{\alpha}(qk'K).
\]
The  shorthand notation
\[
    T(\hat{V})_{ll'}^{\alpha}(kk'K\omega)=
   \bra{kKlL{\cal J}S}T(\omega)\ket{k'Kl'L{\cal J}S},
\]
denotes the $T(V)$-matrix
with momenta $k$ and $k'$ and orbital momenta $l$ and $l'$
of the relative motion, and
$K$ is the corresponding momentum of
the center-of-mass motion. Further, $L$, ${\cal J}$, and $S$
are the orbital momentum of the center-of-mass motion, the
total angular momentum and
spin, respectively. We skip for the moment isospin.




\subsubsection{The optical theorem and relation to data}

For elastic scattering, the scattering potential can only change the outgoing spherical wave function up to a phase. In the asymptotic limit, far away from the scattering potential, we get for the spherical bessel function
\[
j_l(kr) \xrightarrow[]{ r \gg 1} \frac{\sin(kr -l\pi/2)}{kr} =  \frac{1}{2ik}\left( \frac{e^{i(kr-l\pi/2)}}{r} - \frac{e^{-i(kr-l\pi/2)}}{r}\right)
\]
The outgoing wave will change by a phase shift $\delta_l$, from which we can define the S-matrix $S_l(k) = e^{2i\delta_l(k)}$. Thus, we have
\[
 \frac{e^{i(kr-l\pi/2)}}{r} \xrightarrow[]{\textnormal{phase change}}  \frac{S_l(k)e^{i(kr-l\pi/2)}}{r}
\]

The solution to the Schrodinger equation for a spherically symmetric potential, will have the form
\[
\psi_k(r) = e^{ikr} + f(\theta)\frac{e^{ikr}}{r}
\]
where $f(\theta)$ is the scattering amplitude, and related to the differential cross section as
\[
\frac{d\sigma}{d\Omega} = |f(\theta)|^2
\]
Using the expansion of a plane wave in spherical waves, we can relate the scattering amplitude $f(\theta)$ with the partial wave phase shifts $\delta_l$ by identifying the outgoing wave 
\[
\psi_k(r) = e^{ikr} + \left[\frac{1}{2ik}\sum_l i^l (2l+1) (S_l(k)-1)P_l(\cos(\theta))e^{-il\pi/2}\right] \frac{e^{ikr}}{r}
\]
which can be simplified further by cancelling $i^l$ with $e^{-il\pi/2}$ 

From the previous slide we have
\[
\psi_k(r) = e^{ikr} + f(\theta) \frac{e^{ikr}}{r}
\]
with 
\[
f(\theta) = \sum_l (2l+1)f_l(\theta) P_l(\cos(\theta))
\]
where the partial wave scattering amplitude is given by
\[
f_l(\theta) = \frac{1}{k}\frac{(S_l(k)-1)}{2i} = \frac{1}{k}\sin\delta_l(k) e^{i\delta_l(k)}
\]
With Eulers formula for the cotanget, this can also be written as
\[
f_l(\theta) = \frac{1}{k}\frac{1}{\cot \delta_l(k) - i}
\]

%\begin{center}
%\includegraphics[width=11cm]{./phase.png}
%\end{center}

The integrated cross section is given by
\begin{align}
\begin{split}
\sigma = {}& 2\pi \int_0^{\pi} |f(\theta)|^2 \sin \theta\, \textnormal{d}\theta = \\
{}& = 2\pi \sum_l |\frac{(2l+1)}{k} \sin \delta_l |^2 \int_0^{\pi} (P_l(\cos \theta))^2 \sin \theta\, \textnormal{d}\theta = \\
{}& = \frac{4\pi}{k^2} \sum_l (2l+1) \sin^2\delta_l(k) = 4\pi \sum_l (2l+1)|f_l(\theta)|^2 
\end{split}
\end{align}
Where the orthogonality of the Legendre polynomials was used to evaluate the last integral
\[
\int_0^{\pi} P_l(\cos \theta)^2 \sin \theta \, \textnormal{d}\theta = \frac{2}{2l+1}
\]
Thus, the \textit{total} cross section is the sum of the partial-wave cross sections. Note that the differential cross section contains cross-terms from different partial waves. The integral over the full sphere enables the use of the legendre orthogonality, and this kills the cross-terms.

At low energy, $k \rightarrow 0$, S-waves are most important. In this region we can define the scattering length $a$ and the effective range $r$. The $S-$wave scattering amplitude is given by
\[
f_l(\theta) = \frac{1}{k}\frac{1}{\cot \delta_l(k) - i}
\]
Taking the limit $k \rightarrow 0$, gives us the expansion
\[
k \cot \delta_0 = -\frac{1}{a} + \frac{1}{2}r_0 k^2 + \ldots
\]
Thus the low energy cross section is given by
\[
\sigma = 4\pi a^2
\]
If the system contains a bound state, the scattering length will become positive (neutron-proton in $^3S_1$). For the $^1S_0$ wave, the scattering length is negative and large. This indicates that the wave function of the system is at the verge of turning over to get a node, but cannot create a bound state in this wave.%\\
\begin{center}
%\includegraphics[width=11cm]{./scattering_length.png}
\end{center}

It is important to realize that the phase shifts themselves aren't observables. The measurable scattering quantity is the cross section, or the differential cross section. The partial wave phase shifts can be thought of as a parameterization of the (experimental) cross sections. The phase shifts provide insights into the physics of partial wave projected nuclear interactions, and are thus important quantities to know.\\The nucleon-nucleon differential cross section have been measured at almost all energies up to the pion production threshold (290 MeV in the Lab frame), and this experimental data base is what provides us with the constraints on our nuclear interaction models. In order to pin down the unknown coupling constants of the theory, a statistical optimization with respect to cross sections need to be carried out. This is how we constraint the nucleon-nucleon interaction in practice!
\begin{table}
\begin{center}
\begin{tabular}{ccccc} \hline \hline
$T_{\textnormal{lab}}$ bin (MeV) &  N3LO$^1$     & NNLO$^2$   & NLO$^2$  & AV18$^3$   \\ 
                             &          &        &      &         \\ \hline 
0-100                        &  1.05    & 1.7    & 4.5  & 0.95    \\
100-190                      &  1.08    & 22     & 100  & 1.10    \\
190-290                      &  1.15    & 47     & 180  & 1.11    \\ \hline
{\bf 0-290}                   &  \textbf{1.10}    & \textbf{20}     & \textbf{86}   & \textbf{1.04}    \\ \hline \hline 
\end{tabular}
\end{center}
\end{table}



\subsubsection{Algorithm for solving the scattering state problem}
For scattering states, the energy is positive, $E>0$. 
The Lippman-Schwinger equation (a rewrite of the Schr\"odinger equation)
is an integral equation
where we have to deal with the amplitude 
$R(k,k')$ (reaction matrix, which is the real part of  the full
complex $T$-matrix)
defined through the integral equation for one partial wave (no coupled-channels) 
\[
    R_l(k,k') = V_l(k,k') +\frac{2}{\pi}{\cal P}
                \int_0^{\infty}dqq^2V_l(k,q)\frac{1}{E-q^2/m}R_l(q,k').
   \label{eq:ls1}
\]
For negative energies (bound states) and intermediate states scattering states blocked
by  occupied states below the Fermi level.


The symbol ${\cal P}$ in the previous slide indicates that Cauchy's principal-value prescription
is used in order to avoid the singularity arising from the zero of the denominator.


The total kinetic energy of the two 
incoming particles in the center-of-mass system
is 
\[
    E=\frac{k_0^2}{m_n}.
\]


The matrix $R_l(k,k')$ relates to the 
the  phase shifts through its diagonal elements as
\[
     R_l(k_0,k_0)=-\frac{tan\delta_l}{mk_0}.
     \label{eq:shifts}
\]

From now on we will drop the subscript $l$ in all equations.
In order to solve the Lippman-Schwinger equation 
in momentum space, we need first to write 
a function which sets up the mesh points. 
We need to do that since we are going to approximate an integral
through 
\[
   \int_a^bf(x)dx\approx\sum_{i=1}^Nw_if(x_i),
\]
where we have fixed $N$ lattice points through the corresponding weights
$w_i$ and points $x_i$. Typically obtained via methods like Gaussian quadrature.

If you use Gauss-Legendre the points are determined for the interval $x_i\in [-1,1]$
You map these points over to the limits in your integral. You can then
use the following mapping
        \[
          k_i=const\times tan\left\{\frac{\pi}{4}(1+x_i)\right\},
        \]
and 
         \[
            \omega_i= const\frac{\pi}{4}\frac{w_i}{cos^2\left(\frac{\pi}{4}(1+x_i)\right)}.
         \]
If you choose units fm$^{-1}$ for $k$, set $const=1$. If you choose to work
with MeV, set $const\sim 200$ ($\hbar c=197$ MeVfm).

The principal value integral is rather tricky
to evaluate numerically, mainly since computers have limited
precision. We will here use a subtraction trick often used
when dealing with singular integrals in numerical calculations.
We introduce first the calculus relation
\[
  \int_{-\infty}^{\infty} \frac{dk}{k-k_0} =0.
\]
It means that the curve $1/(k-k_0)$ has equal and opposite
areas on both sides of the singular point $k_0$. If we break
the integral into one over positive $k$ and one over 
negative $k$, a change of variable $k\rightarrow -k$ 
allows us to rewrite the last equation as
\[
  \int_{0}^{\infty} \frac{dk}{k^2-k_0^2} =0.
\]

We can then express a principal values integral
as
\[
  {\cal P}\int_{0}^{\infty} \frac{f(k)dk}{k^2-k_0^2} =
  \int_{0}^{\infty} \frac{(f(k)-f(k_0))dk}{k^2-k_0^2},
   \label{eq:trick}
\]
where the right-hand side is no longer singular at 
$k=k_0$, it is proportional to the derivative $df/dk$,
and can be evaluated numerically as any other integral.

We can then use this trick to obtain
\[
    R(k,k') = V(k,k') +\frac{2}{\pi}
                \int_0^{\infty}dq
                \frac{q^2V(k,q)R(q,k')-k_0^2V(k,k_0)R(k_0,k')  }
                     {(k_0^2-q^2)/m}.
   \label{eq:ls2}
\]
This is the equation to solve numerically in order
to calculate the phase shifts. We are interested in obtaining
$R(k_0,k_0)$.

How do we proceed?

Using the mesh points $k_j$ and the weights $\omega_j$,
         we reach
\[
          R(k,k') = V(k,k') +\frac{2}{\pi}
          \sum_{j=1}^N\frac{\omega_jk_j^2V(k,k_j)R(k_j,k')}
                           {(k_0^2-k_j^2)/m}
           -\frac{2}{\pi}k_0^2V(k,k_0)R(k_0,k')
          \sum_{n=1}^N\frac{\omega_n}
                           {(k_0^2-k_n^2)/m}.                
\]

This equation contains now the unknowns $R(k_i,k_j)$
(with dimension $N\times N$) and $R(k_0,k_0)$.

We can turn it into an equation
with dimension $(N+1)\times (N+1)$ with  a mesh
which contains the original mesh points $k_j$ for $j=1,N$
and the point which corresponds to the energy $k_0$.
Consider the latter as the 'observable' point.
The mesh points become then $k_j$ for $j=1,n$ and
$k_{N+1}=k_0$. 

With these new mesh points we define the matrix
\[
      A_{i,j}=\delta_{i,j}-V(k_i,k_j)u_j,
      \label{eq:aeq}
\]
where $\delta$ is the Kronecker $\delta$
and
\[
     u_j=\frac{2}{\pi}
         \frac{\omega_jk_j^2}{(k_0^2-k_j^2)/m}\hspace{1cm}
         j=1,N
\]
and
\[
     u_{N+1}=-\frac{2}{\pi}
          \sum_{j=1}^N\frac{k_0^2\omega_j}{(k_0^2-k_j^2)/m}.
\]

The first task is then to 
set up the matrix $A$ for a given $k_0$. This is an
$(N+1)\times (N+1)$ matrix. It can be convenient
to have an outer loop which runs over the chosen
observable values for the energy $k_0^2/m$.
{\em Note that all mesh points $k_j$ for $j=1,N$ must be
different from $k_0$. Note also that
$V(k_i,k_j)$ is an
$(N+1)\times (N+1)$ matrix}. 

  With the matrix $A$ we can rewrite the problem
  as a matrix problem of dimension $(N+1)\times (N+1)$.
  All matrices $R$, $A$ and $V$ have this dimension
  and we get
\[
    A_{i,l}R_{l,j}=V_{i,j},
\] 
or just
\[
    AR=V.
\] 

Since you already have defined $A$ and $V$
(these are stored as $(N+1)\times (N+1)$ matrices) 
The final equation involves only the unknown
$R$. We obtain it by matrix inversion, i.e.,
\[
    R=A^{-1}V.
    \label{eq:final2}
\] 
Thus, to obtain $R$, you will need to set up the matrices
$A$ and $V$ and invert the matrix $A$. 
With the inverse $A^{-1}$, perform
a matrix multiplication with $V$ results in $R$.


With $R$ you can then evaluate the phase shifts
by noting that 
\[
      R(k_{N+1},k_{N+1})=R(k_0,k_0)=-\frac{tan\delta}{mk_0},
\]
where $\delta$ are the phase shifts.


\section{Standard models for the nuclear forces}




Here we display a typical way to parametrize (non-relativistic expression) the nuclear two-body force
in terms of some operators, the central part, the spin-spin part and the central force.
\[
V({\bf r})= \left\{ C_c + C_\sigma 
\mbox{\boldmath $\sigma$}_1\cdot\mbox{\boldmath $\sigma$}_2
 + C_T \left( 1 + {3\over m_\alpha r} + {3\over
\left(m_\alpha r\right)^2}\right) S_{12} (\hat r)\right. 
\]
\[
\left. + C_{SL} \left( {1\over m_\alpha r} + {1\over \left( m_\alpha r\right)^2}
\right) {\bf L}\cdot {\bf S}
\right\} \frac{e^{-m_\alpha r}}{m_\alpha r}
\]
\subsection{Phenomenology of one-pion exchange}

The one-pion exchange contribution (see derivation below), can be written as 
\[
V_{\pi}({\bf r})= -\frac{f_{\pi}^{2}}{4\pi m_{\pi}^{2}}\mbox{\boldmath $\tau$}_1\cdot\mbox{\boldmath $\tau$}_2
\frac{1}{3}\left\{\mbox{\boldmath $\sigma$}_1\cdot\mbox{\boldmath $\sigma$}_2
 +\left( 1 + {3\over m_\pi r} + {3\over
\left(m_\pi r\right)^2}
\right) S_{12} (\hat r)\right\} \frac{e^{-m_\pi r}}{m_\pi r}.
\]
Here the constant $f_{\pi}^{2}/4\pi\approx 0.08$ and the mass of the pion is $m_\pi\approx 140$ MeV/c$^2$.  

Let us look closer at specific partial waves for which one-pion exchange is applicable. If we have $S=0$ and $T=0$, the 
orbital momentum has to be an odd number in order for the total anti-symmetry to be obeyed. For $S=0$, the tensor force component is zero, meaning that 
the only contribution is 
\[
V_{\pi}({\bf r})=\frac{3f_{\pi}^{2}}{4\pi m_{\pi}^{2}}\frac{e^{-m_\pi r}}{m_\pi r},
\]
since $\langle\mbox{\boldmath $\sigma$}_1\cdot\mbox{\boldmath $\sigma$}_2\rangle=-3$, that is we obtain a repulsive contribution to partial waves like 
$^1P_0$. 

Since $S=0$ yields always a zero tensor force contribution, for the combination of $T=1$ and then even $l$ values, we get an attractive contribution
\[
V_{\pi}({\bf r})=-\frac{f_{\pi}^{2}}{4\pi m_{\pi}^{2}}\frac{e^{-m_\pi r}}{m_\pi r}.
\]

With $S=1$ and $T=0$, $l$ can only take even values in order to obey the anti-symmetry requirements and we get
\[
V_{\pi}({\bf r})= -\frac{f_{\pi}^{2}}{4\pi m_{\pi}^{2}}\left(1+( 1 + {3\over m_\pi r} + {3\over
\left(m_\pi r\right)^2}\right) S_{12} (\hat r)\right) \frac{e^{-m_\pi r}}{m_\pi r},
\]
while for $S=1$ and $T=1$, $l$ can only take odd values, resulting in a repulsive contribution 
\[
V_{\pi}({\bf r})= \frac{1}{3}\frac{f_{\pi}^{2}}{4\pi m_{\pi}^{2}}\left(1+( 1 + {3\over m_\pi r} + {3\over
\left(m_\pi r\right)^2}\right) S_{12} (\hat r)\right) \frac{e^{-m_\pi r}}{m_\pi r},
\]

The central part of one-pion exchange interaction, arising from the spin-spin term,  
is thus attractive for $s$-waves and all even $l$ values. For $p$-waves and all other odd values
it is repulsive. However, its overall strength is weak. 

To see this, we can think of a simplified box potential. 


To derive the above famous form of the nuclear force using field theoretical concepts, we will need some 
elements from relativistic quantum mechanics. I know that many of you have not taken a course in quantum field theory. I hope however that you can see the basic ideas leading to the famous non-relativistic expressions for the nuclear force. 

\begin{table}
\begin{center}
\begin{tabular}{llll}
\\\hline
\multicolumn{1}{c}{Baryons}&
\multicolumn{1}{c}{Mass (MeV)}&
\multicolumn{1}{c}{Mesons}&
\multicolumn{1}{c}{Mass (MeV)}
\\ \hline
$p,n$&938.926&$\pi$&138.03\\
$\Lambda$&1116.0&$\eta$&548.8\\
$\Sigma$&1197.3&$\sigma$&$\approx 550.0$\\
$\Delta$&1232.0&$\rho$&770\\
&&$\omega$&782.6\\
&&$\delta$&983.0\\
&&$K$&495.8\\
&&$K^{\star}$&895.0\\ \hline
\end{tabular}
\end{center}
\end{table}
To describe the interaction between the various baryons and mesons of the previous
table we choose the following phenomenological
lagrangians
for spin $1/2$ baryons
\[
   {\cal L}_{ps} =g^{ps}\overline{\Psi}\gamma^{5}
   \Psi\phi^{(ps)},
   \label{eq:pseudo}
\]
\[
   {\cal L}_{s} =g^{s}\overline{\Psi}\Psi\phi^{(s)},
   \label{eq:scalar}
\]
and
\[
   {\cal L}_{v} =g^{v}\overline{\Psi}\gamma_{\mu}\Psi\phi_{\mu}^{(v)}
   +g^{t}\overline{\Psi}\sigma^{\mu\nu}\Psi\left
   (\partial_{\mu}\phi_{\nu}^{(v)}
   -\partial_{\nu}\phi_{\mu}^{(v)}\right),
   \label{eq:vector}
\]
for pseudoscalar (ps), scalar (s) and vector (v) coupling, respectively.
The factors $g^{v}$ and $g^{t}$ are the vector
and tensor coupling constants, respectively.
For spin $1/2$ baryons, the fields $\Psi$ are expanded
in terms of the Dirac spinors (positive energy
solution shown here with $\overline{u}u=1$)
\[
   u(k\sigma)=\sqrt{\frac{E(k)+m}{2m}}
	  \left(\begin{array}{c} \chi\\ \\
	  \frac{\mbox{\boldmath $\sigma$}{\bf k}}{E(k)+m}\chi
	  \end{array}\right), 
   \label{eq:freespinor}
\]
with $\chi$ the familiar Pauli spinor and $E(k) =\sqrt{m^2 +|{\bf k}|^2}$. 
The positive energy part of the field $\Psi$ reads
\[
\Psi (x)={\displaystyle \frac{1}{(2\pi )^{3/2}}
        \sum_{{\bf k}\sigma}u(k\sigma)e^{-ikx}a_{{\bf k}\sigma}},
\]
with $a$ being a fermion annihilation operator.

Expanding the free Dirac spinors
in terms of $1/m$ ($m$ is here the mass of the relevant baryon) 
results, to lowest order, in the familiar non-relativistic
expressions for baryon-baryon potentials.
The configuration space version of the interaction can be approximated as
\[
V({\bf r})= \left\{ C^0_C + C^1_C + C_\sigma 
\mbox{\boldmath $\sigma$}_1\cdot\mbox{\boldmath $\sigma$}_2
 + C_T \left( 1 + {3\over m_\alpha r} + {3\over
\left(m_\alpha r\right)^2}
\right) S_{12} (\hat r)\right.
\]
\[
+ C_{SL}\left. \left( {1\over m_\alpha r} + {1\over \left( m_\alpha r\right)^2}
\right) {\bf L}\cdot {\bf S}
\right\} \frac{e^{-m_\alpha r}}{m_\alpha r},
\]
where $m_{\alpha}$ is the mass of the relevant meson and
$S_{12}$ is the familiar tensor term.

We derive now the non-relativistic one-pion exchange interaction.

Here $p_{1}$, $p_{1}'$, $p_{2}$, $p_{2}'$ and $k=p_{1}-p_{1}'$ denote 
four-momenta.  
The vertices are 
given by the pseudovector Lagrangian
\[
{\cal L}_{pv}=\frac{f_{\pi}}{m_{\pi}}\overline{\psi}\gamma_{5}\gamma_{\mu}
\psi\partial^{\mu}\phi_{\pi}.
\]
 From the Feynman diagram rules we can write the two-body interaction as  
\[
V^{pv}=\frac{f_{\pi}^{2}}{m_{\pi}^{2}}\frac{\overline{u}(p_{1}')\gamma_{5}
\gamma_{\mu}(p_{1}-p_{1}')^{\mu}u(p_{1})\overline{u}(p_{2}')\gamma_{5}
\gamma_{\nu}(p_{2}'-p_{2})^{\nu}u(p_{2})}{(p_{1}-p_{1}')^{2}-m_{\pi}^{2}}.
\]

The factors $p_{1}-p_{1}'=p_{2}'-p_{2}$ are both the four-momentum of the 
exchanged meson and come from the derivative of the meson field in 
the interaction Lagrangian. 
The Dirac spinors obey 
\begin{eqnarray}
\gamma_{\mu}p^{\mu}u(p)&=&mu(p) \nonumber \\
\overline{u}(p)\gamma_{\mu}p^{\mu}&=&m\overline{u}(p). \nonumber
\end{eqnarray} 
Using these relations, together with $\{\gamma_{5},\gamma_{\mu}\}=0$, 
we find 
\begin{eqnarray}
\overline{u}(p_{1}')\gamma_{5}\gamma_{\mu}(p_{1}-p_{1}')^{\mu}u(p_{1})
&=&m\overline{u}(p_{1}')\gamma_{5}u(p_{1})+\overline{u}(p_{1}')\gamma_{\mu}
p_{1}'^{\mu}\gamma_{5}u(p_{1}) \nonumber \\
 &=&2m\overline{u}(p_{1}')\gamma_{5}u(p_{1}) \nonumber
\end{eqnarray}
and 
\[
\overline{u}(p_{2}')\gamma_{5}\gamma_{\mu}(p_{2}'-p_{2})^{\mu}=
-2m\overline{u}(p_{2}')\gamma_{5}u(p_{1}).
\]
We get 
\[
V^{pv}=-\frac{f_{\pi}^{2}}{m_{\pi}^{2}}4m^{2}\frac{\overline{u}(p_{1}')
\gamma_{5}u(p_{1})\overline{u}(p_{2}')\gamma_{5}u(p_{2})}{(p_{1}-p_{1}')
^{2}-m_{\pi}^{2}}.
\]
By inserting expressions for the Dirac spinors, we find
\begin{eqnarray*}
\overline{u}(p_{1}')\gamma_{5}u(p_{1})&=&\sqrt{\frac{(E_{1}'+m)(E_{1}+m)}
{4m^{2}}}\left(\begin{array}{cc}\chi^{\dagger}&-\frac{\sigma_{1}\cdot{
\bf p_{1}}}{E_{1}'
+m}\chi^{\dagger}\end{array}\right)\left(\begin{array}{cc}0&1\\1&0\end{array}
\right)\nonumber \\
 &&\times \left(\begin{array}{c}\chi\\ \frac{\sigma_{1}\cdot{\bf p_{1}}}{E_{1}+m}\chi
\end{array}\right) 
\nonumber \\
 &=&\sqrt{\frac{(E_{1}'+m)(E_{1}+m)}{4m^{2}}}\left(\frac{\sigma_{1}\cdot
{\bf p_{1}}}{E_{1}+m}-\frac{\sigma_{1}\cdot{\bf p_{1}'}}{E_{1}'+m}\right) 
\nonumber 
\end{eqnarray*}
Similarly
\[
\overline{u}(p_{2}')\gamma_{5}u(p_{2})=\sqrt{\frac{(E_{2}'+m)(E_{2}+m)}
{4m^{2}}}\left(\frac{\sigma_{2}\cdot {\bf p}_{2}}{E_{2}+m}-
\frac{\sigma_{2}\cdot{\bf p'}_{2}}{E_{2}'+m}\right).
\]
In the CM system we have ${\bf p}_{2}=-{\bf p}_{1}$, ${\bf p'}_{2}=
-{\bf p'}_{1}$ and so $E_{2}=E_{1}$, $E_{2}'=E_{1}'$.  
We can then write down the relativistic contribution 
to the NN potential in the CM system: 
\begin{eqnarray}
V^{pv}&=&-\frac{f_{\pi}^{2}}{m_{\pi}^{2}}4m^{2}\frac{1}{(p_{1}-p_{1}')^{2}-
m_{\pi}^{2}}\frac{(E_{1}+m)(E_{1}'+m)}{4m^{2}} \nonumber \\ 
 &\times&\left(\frac{\sigma_{1}\cdot{\bf p}_{1}}{E_{1}+m}-\frac{\sigma_{1}
\cdot{\bf p'}_{1}}{E_{1}'+m}\right)\left(\frac{\sigma_{2}\cdot{\bf p}_{1}}
{E_{1}+m}-\frac{\sigma_{2}\cdot{\bf p'}_{1}}{E_{1}'+m}\right). \nonumber
\end{eqnarray}

In the non-relativistic limit we have to lowest order 
\[
E_{1}=\sqrt{{\bf p}_{1}^{2}+m^{2}}\approx m \approx E_{1}'
\]
and then $(p_{1}-p_{1}')^{2}=-{\bf k}^{2}$, so we get 
for the contribution to the NN potential
\begin{eqnarray}
V^{pv}&=&-\frac{f_{\pi}^{2}}{m_{\pi}^{2}}4m^{2}\frac{1}{{\bf k}^{2}+m^{2}}
\frac{2m\cdot 2m}{4m^{2}}\frac{\sigma_{1}}{2m}\cdot({\bf p}_{1}-{\bf p'}_{1})
\frac{\sigma_{2}}{2m}\cdot ({\bf p}_{1}-{\bf p'}_{1}) \nonumber \\ 
 &=&-\frac{f_{\pi}^{2}}{m_{\pi}^{2}}
\frac{(\sigma_{1}\cdot{\bf k})(\sigma_{2}\cdot{\bf k})}{{\bf k}^{2}+m_{\pi}^{2}}.
\nonumber
\end{eqnarray}
We have omitted exchange terms and the isospin term $\mbox{\boldmath $\tau$}_1\cdot\mbox{\boldmath $\tau$}_2$.

We have
\[
V^{pv}(k)=-\frac{f_{\pi}^{2}}{m_{\pi}^{2}}
\frac{(\sigma_{1}\cdot{\bf k})(\sigma_{2}\cdot{\bf k})}{{\bf k}^{2}+m_{\pi}^{2}}.
\]
In coordinate space we have
\[
V^{pv}(r)=\int\frac{d^3k}{(2\pi)^3}e^{i{\bf kr}}V^{pv}(k)
\]
resulting in
\[
  V^{pv}(r)=-\frac{f_{\pi}^{2}}{m_{\pi}^{2}}
\sigma_{1}\cdot{\nabla}\sigma_{2}\cdot{\nabla}
\int\frac{d^3k}{(2\pi)^3}e^{i{\bf kr}}\frac{1}{{\bf k}^{2}+m_{\pi}^{2}}.
\]
We obtain
\[
V^{pv}(r)=-\frac{f_{\pi}^{2}}{m_{\pi}^{2}}\sigma_{1}\cdot{\nabla}\sigma_{2}\cdot{\nabla}\frac{e^{-m_{\pi}r}}{r}.
\]

Carrying out the differentation of
\[
V^{pv}(r)=-\frac{f_{\pi}^{2}}{m_{\pi}^{2}}\sigma_{1}\cdot{\nabla}\sigma_{2}\cdot{\nabla}\frac{e^{-m_{\pi}r}}{r}.
\]
we arrive at the famous one-pion exchange potential with central and tensor parts
\[
V({\bf r})= -\frac{f_{\pi}^{2}}{m_{\pi}^{2}}\left\{\mbox{\boldmath $\sigma$}_1\cdot\mbox{\boldmath $\sigma$}_2
 + C_T \left( 1 + {3\over m_\alpha r} + {3\over
\left(m_\alpha r\right)^2}
\right) S_{12} (\hat r)\right\} \frac{e^{-m_\pi r}}{m_\pi r}.
\]
For the full potential add the exchange part and the $\mbox{\boldmath $\tau$}_1\cdot\mbox{\boldmath $\tau$}_2$ term as well. (Subtle point: there is a divergence which gets cancelled by using cutoffs).
When we perform similar non-relativistic expansions for scalar and vector mesons we obtain
for the $\sigma$ meson
\[
V^{\sigma}= g_{\sigma NN}^{2}\frac{1}{{\bf k}^{2}+m_{\sigma}^{2}}\left (-1+\frac{{\bf q}^{2}}{2M_N^2}
-\frac{{\bf k}^{2}}{8M_N^2}-\frac{{\bf LS}}{2M_N^2}\right).
\]
We note an attractive central force and spin-orbit force. This term has an intermediate range.
We have defined $1/2(p_{1}+p_{1}')={\bf q}$.
For the full potential add the exchange part and the isospin dependence as well.

We obtain
for the $\omega$ meson
\[
V^{\omega}= g_{\omega NN}^{2}\frac{1}{{\bf k}^{2}+m_{\omega}^{2}}\left (1-3\frac{{\bf LS}}{2M_N^2}\right).
\]
We note a repulsive central force and an attractive spin-orbit force. This term has  short range.
For the full potential add the exchange part and the isospin dependence as well.

Finally 
for the $\rho$ meson
\[
V^{\rho}= g_{\rho NN}^{2}\frac{{\bf k}^{2}}{{\bf k}^{2}+m_{\rho}^{2}}\left (
-2\sigma_{1}\sigma_{2}+S_{12}(\hat{k})\right)\tau_{1}\tau_{2}.
\]
We note a tensor force with sign opposite to that of the pion. This term has  short range. For the full potential add the exchange part and the isospin dependence as well.

    \begin{enumerate}
\item Can use a one-boson exchange picture to construct a nucleon-nucleon
interaction a la QED
\item Non-relativistic approximation yields amongst other things a spin-orbit
force which is much stronger than in atoms.
\item At large intermediate distances pion exchange dominates while 
pion resonances (other mesons) dominate at intermediate and short range 
\item  Potentials are parameterized to fit selected two-nucleon data, binding energies and scattering phase shifts.
\item Nowaydays, chiral perturbation theory gives an effective theory that allows a systematic expansion in terms of contrallable parameters. Good basis for many-body physics
    \end{enumerate}


\section{Links to data}




\section{Exercises}

\begin{Exercise}
List all {\em allowed according to the Pauli principle} partial waves with isospin $T$, its 
projection $T_z$, spin $S$, orbital angular momentum $l$ and total spin $J$ for $J\le 2$.
Use the standard spectroscopic notation $^{2S+1}L_J$ to label different partial waves. A proton-proton state
has $T_Z=-1$, a proton-neutron state has $T_z=0$ and a neutron-neutron state has $T_z=1$.
\end{Exercise}
\begin{Exercise}
\begin{enumerate}
\item[a)] Find the closed form expression for the sin-orbit force.
Show that the spin-orbit force {\bf LS} gives a zero
contribution for $S$-waves (orbital angular momentum $l=0$).   What is the value of the spin-orbit force for spin-singlet states ($S=0$)?  
\item[b)] Find thereafter the expectation value of 
$\mbox{\boldmath $\sigma$}_1\cdot\mbox{\boldmath $\sigma$}_2$, where $\mbox{\boldmath $\sigma$}_i$
are so-called Pauli matrices. 
\item[c)] Add thereafter isospin and find the expectation value of 
$\mbox{\boldmath $\sigma$}_1\cdot\mbox{\boldmath $\sigma$}_2\mbox{\boldmath $\tau$}_1\cdot\mbox{\boldmath $\tau$}_2$, where $\mbox{\boldmath $\tau$}_i$
are also so-called Pauli matrices. List all the cases with $S=0,1$ and $T=0,1$.
\end{enumerate}
\end{Exercise}

\begin{Exercise}
A simple parametrization of the nucleon-nucleon force is given by what is called the $V_8$ potential model,
where we have kept eight different operators. These operators contain a central force, a spin-orbit force,
a spin-spin force and a tensor force. Several features of the nuclei can be explained in terms of these four components. Without the Pauli matrices for isospin the final form of such an interaction model results in the following form: 
\[
V({\bf r})= \left\{ C^0_C + C^1_C + C_\sigma 
\mbox{\boldmath $\sigma$}_1\cdot\mbox{\boldmath $\sigma$}_2
 + C_T \left( 1 + {3\over m_\alpha r} + {3\over
\left(m_\alpha r\right)^2}
\right) S_{12} (\hat r)\right.
\]
\[
+ C_{SL}\left. \left( {1\over m_\alpha r} + {1\over \left( m_\alpha r\right)^2}
\right) {\bf L}\cdot {\bf S}
\right\} \frac{e^{-m_\alpha r}}{m_\alpha r},
\]
where $m_{\alpha}$ is the mass of the relevant meson and
$S_{12}$ is the familiar tensor term. The various coefficients $C_i$ are normally fitted so that the potential reproduces experimental scattering cross sections. By adding terms which include the isospin Pauli matrices 
results in an interaction model with eight operators.

The expectaction value of the tensor operator is non-zero only for $S=1$. We will show this in a forthcoming lecture, after that we have derived the Wigner-Eckart theorem. 
Here it suffices to know that the expectaction value of the tensor force for different partial values is  (with $l$ the orbital angular momentum and ${\cal J}$ the total angular momentum in the relative and center-of-mass frame of motion)
\[
\langle l {\cal J}S=1| S_{12} | l' {\cal J}S=1\rangle = -\frac{2{\cal J}({\cal J}+2)}{2{\cal J}+1} \hspace{0.5cm} l= {\cal J}+1 \hspace{0.1cm}\mathrm{and} \hspace{0.1cm} l'={\cal J}+1,
\]
\[
\langle l {\cal J}S=1| S_{12} | l' {\cal J}S=1\rangle = \frac{6\sqrt{{\cal J}({\cal J}+1)}}{2{\cal J}+1} \hspace{0.5cm} l= {\cal J}+1 \hspace{0.1cm}\mathrm{and} \hspace{0.1cm} l'={\cal J}-1,
\]
\[
\langle l {\cal J}S=1| S_{12} | l' {\cal J}S=1\rangle = \frac{6\sqrt{{\cal J}({\cal J}+1)}}{2{\cal J}+1} \hspace{0.5cm} l= {\cal J}-1 \hspace{0.1cm}\mathrm{and} \hspace{0.1cm} l'={\cal J}+1,
\]
\[
\langle l {\cal J}S=1| S_{12} | l' {\cal J}S=1\rangle = -\frac{2({\cal J}-1)}{2{\cal J}+1} \hspace{0.5cm} l= {\cal J}-1 \hspace{0.1cm}\mathrm{and} \hspace{0.1cm} l'={\cal J}-1,
\]
\[
\langle l {\cal J}S=1| S_{12} | l' {\cal J}S=1\rangle = 2 \hspace{0.5cm} l= {\cal J} \hspace{0.1cm}\mathrm{and} \hspace{0.1cm} l'={\cal J},
\]
and zero else.   
In this exercise we will focus only on the one-pion exchange term of the nuclear force, namely
\[
V({\bf r})= -\frac{f_{\pi}^{2}}{4\pi m_{\pi}^{2}}\mbox{\boldmath $\tau$}_1\cdot\mbox{\boldmath $\tau$}_2
\frac{1}{3}\left\{\mbox{\boldmath $\sigma$}_1\cdot\mbox{\boldmath $\sigma$}_2
 +\left( 1 + {3\over m_\pi r} + {3\over
\left(m_\pi r\right)^2}
\right) S_{12} (\hat r)\right\} \frac{e^{-m_\pi r}}{m_\pi r}.
\]
Here the constant $f_{\pi}^{2}/4\pi\approx 0.08$ and the mass of the pion is $m_\pi\approx 140$ MeV/c$^2$. 

\begin{enumerate}
\item[a)] Compute the expectation value of the tensor force and the spin-spin  and isospin operators for the one-pion exchange potential for all partial waves you found in exercise 9. Comment your results. How does the one-pion exchange part behave as function of different $l$, ${\cal J}$ and $S$ values? Do you see some patterns?
\item[b)]  For the binding energy of the deuteron, with the ground state defined by the quantum numbers 
$l=0$, $S=1$ and ${\cal J}=1$, the tensor force plays an important role due to the admixture from the $l=2$ state. Use the expectation values of the different operators of the one-pion exchange potential and plot the ratio of the tensor force component over the spin-spin component of the one-pion exchange part as function of $x=m_\pi r$ for the $l=2$ state (that is the $l,l'={\cal J}+1$. Comment your results. 
\end{enumerate}
\end{Exercise}




