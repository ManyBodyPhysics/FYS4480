\documentclass[prc]{revtex4}
\usepackage[dvips]{graphicx}
\usepackage{mathrsfs}
\usepackage{amsfonts}
\usepackage{lscape}

\usepackage{epic,eepic}
\usepackage{amsmath}
\usepackage{amssymb}
\usepackage[dvips]{epsfig}
\usepackage[T1]{fontenc}
\usepackage{hyperref}
\usepackage{bezier}
\usepackage{pstricks}
\usepackage{dcolumn}% Align table columns on decimal point
\usepackage{bm}% bold math
%\usepackage{braket}
\usepackage[dvips]{graphicx}
\usepackage{pst-plot}

\newcommand{\One}{\hat{\mathbf{1}}}
\newcommand{\eff}{\text{eff}}
\newcommand{\Heff}{\hat{H}_\text{eff}}
\newcommand{\Veff}{\hat{V}_\text{eff}}
\newcommand{\braket}[1]{\langle#1\rangle}
\newcommand{\Span}{\operatorname{sp}}
\newcommand{\tr}{\operatorname{trace}}
\newcommand{\diag}{\operatorname{diag}}
\newcommand{\bra}[1]{\left\langle #1 \right|}
\newcommand{\ket}[1]{\left| #1 \right\rangle}
\newcommand{\element}[3]
    {\bra{#1}#2\ket{#3}}

\newcommand{\normord}[1]{
    \left\{#1\right\}
}

\usepackage{amsmath}
\begin{document}

\title{Exercises FYS4480, week 36, September 4-8, 2023}
%\author{}
\maketitle
\subsection*{Exercise 1, Condon-Slater rules for expectation values}
Consider three $N$-particle 
Slater determinants $|SD\rangle$, $|SD_i^j\rangle$ and $|SD_{ij}^{kl}\rangle$, where the notation means that 
Slater determinant $|SD_i^j\rangle$ differs from $|SD\rangle$ by one single-particle state, that is a single-particle
state $\psi_i$ is replaced by a single-particle state $\psi_j$. Similarly, the Slater determinant $|SD_{ij}^{kl}\rangle$
differs by two single-particle states from $|SD\rangle$.

We define thereafter a general onebody operator $\hat{F} = \sum_{i}^N\hat{f}(x_{i})$ and a general 
twobody operator $\hat{G}=\sum_{i>j}^N\hat{g}(x_{i},x_{j})$
with $g$ being invariant under the interchange of the coordinates of two particles.
The single-particle states $\psi_i$ are not necessarily eigenstates of $\hat{f}$.
\begin{enumerate}
\item[a)] Find the expectation values of 
\[
\langle SD |\hat{F}|SD\rangle,
\]
and
\[
\langle SD\hat{G}|SD\rangle.
\]
\item[b)] Find thereafter t
\[
\langle SD |\hat{F}|SD_i^j\rangle,
\]
and
\[
\langle SD\hat{G}|SD_i^j\rangle,
\]
and finally
\item[c)] find 
\[
\langle SD |\hat{F}|SD_{ij}^{kl}\rangle,
\]
and
\[
\langle SD\hat{G}|SD_{ij}^{kl}\rangle.
\]
What happens with the two-body operator if we have a transition probability  of the type
\[
\langle SD\hat{G}|SD_{ijk}^{lmn}\rangle,
\]
where the Slater determinant to the right of the operator differs by more than two single-particle states?
\end{enumerate}
\subsection*{Exercise 2, first second quantization encounter}
\begin{enumerate}
\item[a)] Show that the density of particles with coordinates $\mathbf{x}$, is given by
\[
  n(\mathbf{x}) = N \int d\mathbf{x}_2 \dots d\mathbf{x}_N |\Psi_{AS}(\mathbf{x},\mathbf{x}_2,\dots,\mathbf{x}_N)|^2 
\]
can be written in terms of the single-particle states $\psi_k$ as 
\[
 n(\mathbf{x}) = \sum_k|\psi_k(\mathbf{x})|^2.
\]
\item[b)] Calculate the matrix elements (second quantization)
\[
\bra{\alpha_{1}\alpha_{2}}\hat{F}\ket{\alpha_{1}\alpha_{2}}
\]
and
\[
\bra{\alpha_{1}\alpha_{2}}\hat{G}\ket{\alpha_{1}\alpha_{2}}
\]
with
\[
\ket{\alpha_{1}\alpha_{2}}=a_{\alpha_{1}}^{\dagger}
a_{\alpha_{2}}^{\dagger}\ket{0} ,
\]
\[
\hat{F}=\sum_{\alpha\beta}\bra{\alpha}f\ket{\beta}
a_{\alpha}^{\dagger}a_{\beta}  ,
\]
\[
\bra{\alpha}f\ket{\beta}=\int \psi_{\alpha}^{*}(x)f(x)\psi_{\beta}(x)dx ,
\]
\[
\hat{G} = \frac{1}{2}\sum_{\alpha\beta\gamma\delta}
\bra{\alpha\beta}g\ket{\gamma\delta}
a_{\alpha}^{\dagger}a_{\beta}^{\dagger}a_{\delta}a_{\gamma} ,
\]
and
\[
\bra{\alpha\beta}g\ket{\gamma\delta}=
\int\int \psi_{\alpha}^{*}(x_{1})\psi_{\beta}^{*}(x_{2})g(x_{1},
x_{2})\psi_{\gamma}(x_{1})\psi_{\delta}(x_{2})dx_{1}dx_{2}
\]
Compare these results with those from exercise 1c) from the exercise set of week 35.

\end{enumerate}





\end{document}
