\documentclass[prc]{revtex4}
\usepackage[dvips]{graphicx}
\usepackage{mathrsfs}
\usepackage{amsfonts}
\usepackage{lscape}

\usepackage{epic,eepic}
\usepackage{amsmath}
\usepackage{amssymb}
\usepackage[dvips]{epsfig}
\usepackage[T1]{fontenc}
\usepackage{hyperref}
\usepackage{bezier}
\usepackage{pstricks}
\usepackage{dcolumn}% Align table columns on decimal point
\usepackage{bm}% bold math
%\usepackage{braket}
\usepackage[dvips]{graphicx}
\usepackage{pst-plot}

\newcommand{\One}{\hat{\mathbf{1}}}
\newcommand{\eff}{\text{eff}}
\newcommand{\Heff}{\hat{H}_\text{eff}}
\newcommand{\Veff}{\hat{V}_\text{eff}}
\newcommand{\braket}[1]{\langle#1\rangle}
\newcommand{\Span}{\operatorname{sp}}
\newcommand{\tr}{\operatorname{trace}}
\newcommand{\diag}{\operatorname{diag}}
\newcommand{\bra}[1]{\left\langle #1 \right|}
\newcommand{\ket}[1]{\left| #1 \right\rangle}
\newcommand{\element}[3]
    {\bra{#1}#2\ket{#3}}

\newcommand{\normord}[1]{
    \left\{#1\right\}
}

\usepackage{amsmath}
\begin{document}
\title{Exercises FYS4480, week 38, September 25-29, 2023}
%\author{}
\maketitle



\subsection*{Exercise 1}
We define the one-particle operator
\[
\hat{T}={\displaystyle
\sum_{\alpha\beta}}\bra{\alpha}t\ket{\beta}a_{\alpha}^
{\dagger}a_{\beta},
\]
and the two-particle operator
\[
\hat{V}=
\frac{1}{2}{\displaystyle
\sum_{\alpha\beta\gamma\delta}}\bra{\alpha\beta}
v\ket{\gamma\delta}a_{\alpha}^{\dagger}a_{\beta}^{\dagger}
a_{\delta}a_{\gamma}.
\]
We have defined a single-particle basis with quantum numbers given by the set of greek letters $\alpha,\beta,\gamma,\dots$

\begin{enumerate}
\item[a)] Show that the form of these operators remain unchanged under 
a transformation  of the single-particle basis given by 
\[
\ket{i}=\sum_{\lambda}\ket{\lambda}\left\langle \lambda | i \right\rangle,
\]
with $\lambda\in \left\{\alpha,\beta,\gamma,\dots\right\}$. 
Show also that
$a_{i}^{\dagger}a_{i}$ is the number operator
for  the orbital $\ket{i}$. 
\item[b)] Find also the expressions for the operators
$T$ and $V$ when $T$ is diagonal in the representation
$i$. 
\end{enumerate}



\subsection*{Exercise 2}
Consider a Slater determinant built up of single-particle orbitals $\psi_{\lambda}$, 
with $\lambda = 1,2,\dots,N$.

The unitary transformation
\[
\psi_a  = \sum_{\lambda} C_{a\lambda}\phi_{\lambda},
\]
brings us into the new basis.  
The new basis has quantum numbers $a=1,2,\dots,N$.
Show that the new basis is orthonormal given that the old basis is orthonormal.
Show that the new Slater determinant constructed from the new single-particle wave functions can be
written as the determinant based on the previous basis and the determinant of the matrix $C$.
Show that the old and the new Slater determinants are equal up to a complex constant with absolute value unity.
(Hint, $C$ is a unitary matrix).  Show also that the Slater determinants are orthogonal if we employ a single-particle basis which is orthogonal.




\end{document}

