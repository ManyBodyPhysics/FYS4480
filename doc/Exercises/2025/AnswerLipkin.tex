\documentclass[11pt]{article}

    \usepackage[breakable]{tcolorbox}
    \usepackage{parskip} % Stop auto-indenting (to mimic markdown behaviour)
    

    % Basic figure setup, for now with no caption control since it's done
    % automatically by Pandoc (which extracts ![](path) syntax from Markdown).
    \usepackage{graphicx}
    % Keep aspect ratio if custom image width or height is specified
    \setkeys{Gin}{keepaspectratio}
    % Maintain compatibility with old templates. Remove in nbconvert 6.0
    \let\Oldincludegraphics\includegraphics
    % Ensure that by default, figures have no caption (until we provide a
    % proper Figure object with a Caption API and a way to capture that
    % in the conversion process - todo).
    \usepackage{caption}
    \DeclareCaptionFormat{nocaption}{}
    \captionsetup{format=nocaption,aboveskip=0pt,belowskip=0pt}

    \usepackage{float}
    \floatplacement{figure}{H} % forces figures to be placed at the correct location
    \usepackage{xcolor} % Allow colors to be defined
    \usepackage{enumerate} % Needed for markdown enumerations to work
    \usepackage{geometry} % Used to adjust the document margins
    \usepackage{amsmath} % Equations
    \usepackage{amssymb} % Equations
    \usepackage{textcomp} % defines textquotesingle
    % Hack from http://tex.stackexchange.com/a/47451/13684:
    \AtBeginDocument{%
        \def\PYZsq{\textquotesingle}% Upright quotes in Pygmentized code
    }
    \usepackage{upquote} % Upright quotes for verbatim code
    \usepackage{eurosym} % defines \euro

    \usepackage{iftex}
    \ifPDFTeX
        \usepackage[T1]{fontenc}
        \IfFileExists{alphabeta.sty}{
              \usepackage{alphabeta}
          }{
              \usepackage[mathletters]{ucs}
              \usepackage[utf8x]{inputenc}
          }
    \else
        \usepackage{fontspec}
        \usepackage{unicode-math}
    \fi

    \usepackage{fancyvrb} % verbatim replacement that allows latex
    \usepackage{grffile} % extends the file name processing of package graphics
                         % to support a larger range
    \makeatletter % fix for old versions of grffile with XeLaTeX
    \@ifpackagelater{grffile}{2019/11/01}
    {
      % Do nothing on new versions
    }
    {
      \def\Gread@@xetex#1{%
        \IfFileExists{"\Gin@base".bb}%
        {\Gread@eps{\Gin@base.bb}}%
        {\Gread@@xetex@aux#1}%
      }
    }
    \makeatother
    \usepackage[Export]{adjustbox} % Used to constrain images to a maximum size
    \adjustboxset{max size={0.9\linewidth}{0.9\paperheight}}

    % The hyperref package gives us a pdf with properly built
    % internal navigation ('pdf bookmarks' for the table of contents,
    % internal cross-reference links, web links for URLs, etc.)
    \usepackage{hyperref}
    % The default LaTeX title has an obnoxious amount of whitespace. By default,
    % titling removes some of it. It also provides customization options.
    \usepackage{titling}
    \usepackage{longtable} % longtable support required by pandoc >1.10
    \usepackage{booktabs}  % table support for pandoc > 1.12.2
    \usepackage{array}     % table support for pandoc >= 2.11.3
    \usepackage{calc}      % table minipage width calculation for pandoc >= 2.11.1
    \usepackage[inline]{enumitem} % IRkernel/repr support (it uses the enumerate* environment)
    \usepackage[normalem]{ulem} % ulem is needed to support strikethroughs (\sout)
                                % normalem makes italics be italics, not underlines
    \usepackage{soul}      % strikethrough (\st) support for pandoc >= 3.0.0
    \usepackage{mathrsfs}
    

    
    % Colors for the hyperref package
    \definecolor{urlcolor}{rgb}{0,.145,.698}
    \definecolor{linkcolor}{rgb}{.71,0.21,0.01}
    \definecolor{citecolor}{rgb}{.12,.54,.11}

    % ANSI colors
    \definecolor{ansi-black}{HTML}{3E424D}
    \definecolor{ansi-black-intense}{HTML}{282C36}
    \definecolor{ansi-red}{HTML}{E75C58}
    \definecolor{ansi-red-intense}{HTML}{B22B31}
    \definecolor{ansi-green}{HTML}{00A250}
    \definecolor{ansi-green-intense}{HTML}{007427}
    \definecolor{ansi-yellow}{HTML}{DDB62B}
    \definecolor{ansi-yellow-intense}{HTML}{B27D12}
    \definecolor{ansi-blue}{HTML}{208FFB}
    \definecolor{ansi-blue-intense}{HTML}{0065CA}
    \definecolor{ansi-magenta}{HTML}{D160C4}
    \definecolor{ansi-magenta-intense}{HTML}{A03196}
    \definecolor{ansi-cyan}{HTML}{60C6C8}
    \definecolor{ansi-cyan-intense}{HTML}{258F8F}
    \definecolor{ansi-white}{HTML}{C5C1B4}
    \definecolor{ansi-white-intense}{HTML}{A1A6B2}
    \definecolor{ansi-default-inverse-fg}{HTML}{FFFFFF}
    \definecolor{ansi-default-inverse-bg}{HTML}{000000}

    % common color for the border for error outputs.
    \definecolor{outerrorbackground}{HTML}{FFDFDF}

    % commands and environments needed by pandoc snippets
    % extracted from the output of `pandoc -s`
    \providecommand{\tightlist}{%
      \setlength{\itemsep}{0pt}\setlength{\parskip}{0pt}}
    \DefineVerbatimEnvironment{Highlighting}{Verbatim}{commandchars=\\\{\}}
    % Add ',fontsize=\small' for more characters per line
    \newenvironment{Shaded}{}{}
    \newcommand{\KeywordTok}[1]{\textcolor[rgb]{0.00,0.44,0.13}{\textbf{{#1}}}}
    \newcommand{\DataTypeTok}[1]{\textcolor[rgb]{0.56,0.13,0.00}{{#1}}}
    \newcommand{\DecValTok}[1]{\textcolor[rgb]{0.25,0.63,0.44}{{#1}}}
    \newcommand{\BaseNTok}[1]{\textcolor[rgb]{0.25,0.63,0.44}{{#1}}}
    \newcommand{\FloatTok}[1]{\textcolor[rgb]{0.25,0.63,0.44}{{#1}}}
    \newcommand{\CharTok}[1]{\textcolor[rgb]{0.25,0.44,0.63}{{#1}}}
    \newcommand{\StringTok}[1]{\textcolor[rgb]{0.25,0.44,0.63}{{#1}}}
    \newcommand{\CommentTok}[1]{\textcolor[rgb]{0.38,0.63,0.69}{\textit{{#1}}}}
    \newcommand{\OtherTok}[1]{\textcolor[rgb]{0.00,0.44,0.13}{{#1}}}
    \newcommand{\AlertTok}[1]{\textcolor[rgb]{1.00,0.00,0.00}{\textbf{{#1}}}}
    \newcommand{\FunctionTok}[1]{\textcolor[rgb]{0.02,0.16,0.49}{{#1}}}
    \newcommand{\RegionMarkerTok}[1]{{#1}}
    \newcommand{\ErrorTok}[1]{\textcolor[rgb]{1.00,0.00,0.00}{\textbf{{#1}}}}
    \newcommand{\NormalTok}[1]{{#1}}

    % Additional commands for more recent versions of Pandoc
    \newcommand{\ConstantTok}[1]{\textcolor[rgb]{0.53,0.00,0.00}{{#1}}}
    \newcommand{\SpecialCharTok}[1]{\textcolor[rgb]{0.25,0.44,0.63}{{#1}}}
    \newcommand{\VerbatimStringTok}[1]{\textcolor[rgb]{0.25,0.44,0.63}{{#1}}}
    \newcommand{\SpecialStringTok}[1]{\textcolor[rgb]{0.73,0.40,0.53}{{#1}}}
    \newcommand{\ImportTok}[1]{{#1}}
    \newcommand{\DocumentationTok}[1]{\textcolor[rgb]{0.73,0.13,0.13}{\textit{{#1}}}}
    \newcommand{\AnnotationTok}[1]{\textcolor[rgb]{0.38,0.63,0.69}{\textbf{\textit{{#1}}}}}
    \newcommand{\CommentVarTok}[1]{\textcolor[rgb]{0.38,0.63,0.69}{\textbf{\textit{{#1}}}}}
    \newcommand{\VariableTok}[1]{\textcolor[rgb]{0.10,0.09,0.49}{{#1}}}
    \newcommand{\ControlFlowTok}[1]{\textcolor[rgb]{0.00,0.44,0.13}{\textbf{{#1}}}}
    \newcommand{\OperatorTok}[1]{\textcolor[rgb]{0.40,0.40,0.40}{{#1}}}
    \newcommand{\BuiltInTok}[1]{{#1}}
    \newcommand{\ExtensionTok}[1]{{#1}}
    \newcommand{\PreprocessorTok}[1]{\textcolor[rgb]{0.74,0.48,0.00}{{#1}}}
    \newcommand{\AttributeTok}[1]{\textcolor[rgb]{0.49,0.56,0.16}{{#1}}}
    \newcommand{\InformationTok}[1]{\textcolor[rgb]{0.38,0.63,0.69}{\textbf{\textit{{#1}}}}}
    \newcommand{\WarningTok}[1]{\textcolor[rgb]{0.38,0.63,0.69}{\textbf{\textit{{#1}}}}}
    \makeatletter
    \newsavebox\pandoc@box
    \newcommand*\pandocbounded[1]{%
      \sbox\pandoc@box{#1}%
      % scaling factors for width and height
      \Gscale@div\@tempa\textheight{\dimexpr\ht\pandoc@box+\dp\pandoc@box\relax}%
      \Gscale@div\@tempb\linewidth{\wd\pandoc@box}%
      % select the smaller of both
      \ifdim\@tempb\p@<\@tempa\p@
        \let\@tempa\@tempb
      \fi
      % scaling accordingly (\@tempa < 1)
      \ifdim\@tempa\p@<\p@
        \scalebox{\@tempa}{\usebox\pandoc@box}%
      % scaling not needed, use as it is
      \else
        \usebox{\pandoc@box}%
      \fi
    }
    \makeatother

    % Define a nice break command that doesn't care if a line doesn't already
    % exist.
    \def\br{\hspace*{\fill} \\* }
    % Math Jax compatibility definitions
    \def\gt{>}
    \def\lt{<}
    \let\Oldtex\TeX
    \let\Oldlatex\LaTeX
    \renewcommand{\TeX}{\textrm{\Oldtex}}
    \renewcommand{\LaTeX}{\textrm{\Oldlatex}}
    % Document parameters
    % Document title
    \title{AnswerLipkin}
    
    
    
    
    
    
    
% Pygments definitions
\makeatletter
\def\PY@reset{\let\PY@it=\relax \let\PY@bf=\relax%
    \let\PY@ul=\relax \let\PY@tc=\relax%
    \let\PY@bc=\relax \let\PY@ff=\relax}
\def\PY@tok#1{\csname PY@tok@#1\endcsname}
\def\PY@toks#1+{\ifx\relax#1\empty\else%
    \PY@tok{#1}\expandafter\PY@toks\fi}
\def\PY@do#1{\PY@bc{\PY@tc{\PY@ul{%
    \PY@it{\PY@bf{\PY@ff{#1}}}}}}}
\def\PY#1#2{\PY@reset\PY@toks#1+\relax+\PY@do{#2}}

\@namedef{PY@tok@w}{\def\PY@tc##1{\textcolor[rgb]{0.73,0.73,0.73}{##1}}}
\@namedef{PY@tok@c}{\let\PY@it=\textit\def\PY@tc##1{\textcolor[rgb]{0.24,0.48,0.48}{##1}}}
\@namedef{PY@tok@cp}{\def\PY@tc##1{\textcolor[rgb]{0.61,0.40,0.00}{##1}}}
\@namedef{PY@tok@k}{\let\PY@bf=\textbf\def\PY@tc##1{\textcolor[rgb]{0.00,0.50,0.00}{##1}}}
\@namedef{PY@tok@kp}{\def\PY@tc##1{\textcolor[rgb]{0.00,0.50,0.00}{##1}}}
\@namedef{PY@tok@kt}{\def\PY@tc##1{\textcolor[rgb]{0.69,0.00,0.25}{##1}}}
\@namedef{PY@tok@o}{\def\PY@tc##1{\textcolor[rgb]{0.40,0.40,0.40}{##1}}}
\@namedef{PY@tok@ow}{\let\PY@bf=\textbf\def\PY@tc##1{\textcolor[rgb]{0.67,0.13,1.00}{##1}}}
\@namedef{PY@tok@nb}{\def\PY@tc##1{\textcolor[rgb]{0.00,0.50,0.00}{##1}}}
\@namedef{PY@tok@nf}{\def\PY@tc##1{\textcolor[rgb]{0.00,0.00,1.00}{##1}}}
\@namedef{PY@tok@nc}{\let\PY@bf=\textbf\def\PY@tc##1{\textcolor[rgb]{0.00,0.00,1.00}{##1}}}
\@namedef{PY@tok@nn}{\let\PY@bf=\textbf\def\PY@tc##1{\textcolor[rgb]{0.00,0.00,1.00}{##1}}}
\@namedef{PY@tok@ne}{\let\PY@bf=\textbf\def\PY@tc##1{\textcolor[rgb]{0.80,0.25,0.22}{##1}}}
\@namedef{PY@tok@nv}{\def\PY@tc##1{\textcolor[rgb]{0.10,0.09,0.49}{##1}}}
\@namedef{PY@tok@no}{\def\PY@tc##1{\textcolor[rgb]{0.53,0.00,0.00}{##1}}}
\@namedef{PY@tok@nl}{\def\PY@tc##1{\textcolor[rgb]{0.46,0.46,0.00}{##1}}}
\@namedef{PY@tok@ni}{\let\PY@bf=\textbf\def\PY@tc##1{\textcolor[rgb]{0.44,0.44,0.44}{##1}}}
\@namedef{PY@tok@na}{\def\PY@tc##1{\textcolor[rgb]{0.41,0.47,0.13}{##1}}}
\@namedef{PY@tok@nt}{\let\PY@bf=\textbf\def\PY@tc##1{\textcolor[rgb]{0.00,0.50,0.00}{##1}}}
\@namedef{PY@tok@nd}{\def\PY@tc##1{\textcolor[rgb]{0.67,0.13,1.00}{##1}}}
\@namedef{PY@tok@s}{\def\PY@tc##1{\textcolor[rgb]{0.73,0.13,0.13}{##1}}}
\@namedef{PY@tok@sd}{\let\PY@it=\textit\def\PY@tc##1{\textcolor[rgb]{0.73,0.13,0.13}{##1}}}
\@namedef{PY@tok@si}{\let\PY@bf=\textbf\def\PY@tc##1{\textcolor[rgb]{0.64,0.35,0.47}{##1}}}
\@namedef{PY@tok@se}{\let\PY@bf=\textbf\def\PY@tc##1{\textcolor[rgb]{0.67,0.36,0.12}{##1}}}
\@namedef{PY@tok@sr}{\def\PY@tc##1{\textcolor[rgb]{0.64,0.35,0.47}{##1}}}
\@namedef{PY@tok@ss}{\def\PY@tc##1{\textcolor[rgb]{0.10,0.09,0.49}{##1}}}
\@namedef{PY@tok@sx}{\def\PY@tc##1{\textcolor[rgb]{0.00,0.50,0.00}{##1}}}
\@namedef{PY@tok@m}{\def\PY@tc##1{\textcolor[rgb]{0.40,0.40,0.40}{##1}}}
\@namedef{PY@tok@gh}{\let\PY@bf=\textbf\def\PY@tc##1{\textcolor[rgb]{0.00,0.00,0.50}{##1}}}
\@namedef{PY@tok@gu}{\let\PY@bf=\textbf\def\PY@tc##1{\textcolor[rgb]{0.50,0.00,0.50}{##1}}}
\@namedef{PY@tok@gd}{\def\PY@tc##1{\textcolor[rgb]{0.63,0.00,0.00}{##1}}}
\@namedef{PY@tok@gi}{\def\PY@tc##1{\textcolor[rgb]{0.00,0.52,0.00}{##1}}}
\@namedef{PY@tok@gr}{\def\PY@tc##1{\textcolor[rgb]{0.89,0.00,0.00}{##1}}}
\@namedef{PY@tok@ge}{\let\PY@it=\textit}
\@namedef{PY@tok@gs}{\let\PY@bf=\textbf}
\@namedef{PY@tok@ges}{\let\PY@bf=\textbf\let\PY@it=\textit}
\@namedef{PY@tok@gp}{\let\PY@bf=\textbf\def\PY@tc##1{\textcolor[rgb]{0.00,0.00,0.50}{##1}}}
\@namedef{PY@tok@go}{\def\PY@tc##1{\textcolor[rgb]{0.44,0.44,0.44}{##1}}}
\@namedef{PY@tok@gt}{\def\PY@tc##1{\textcolor[rgb]{0.00,0.27,0.87}{##1}}}
\@namedef{PY@tok@err}{\def\PY@bc##1{{\setlength{\fboxsep}{\string -\fboxrule}\fcolorbox[rgb]{1.00,0.00,0.00}{1,1,1}{\strut ##1}}}}
\@namedef{PY@tok@kc}{\let\PY@bf=\textbf\def\PY@tc##1{\textcolor[rgb]{0.00,0.50,0.00}{##1}}}
\@namedef{PY@tok@kd}{\let\PY@bf=\textbf\def\PY@tc##1{\textcolor[rgb]{0.00,0.50,0.00}{##1}}}
\@namedef{PY@tok@kn}{\let\PY@bf=\textbf\def\PY@tc##1{\textcolor[rgb]{0.00,0.50,0.00}{##1}}}
\@namedef{PY@tok@kr}{\let\PY@bf=\textbf\def\PY@tc##1{\textcolor[rgb]{0.00,0.50,0.00}{##1}}}
\@namedef{PY@tok@bp}{\def\PY@tc##1{\textcolor[rgb]{0.00,0.50,0.00}{##1}}}
\@namedef{PY@tok@fm}{\def\PY@tc##1{\textcolor[rgb]{0.00,0.00,1.00}{##1}}}
\@namedef{PY@tok@vc}{\def\PY@tc##1{\textcolor[rgb]{0.10,0.09,0.49}{##1}}}
\@namedef{PY@tok@vg}{\def\PY@tc##1{\textcolor[rgb]{0.10,0.09,0.49}{##1}}}
\@namedef{PY@tok@vi}{\def\PY@tc##1{\textcolor[rgb]{0.10,0.09,0.49}{##1}}}
\@namedef{PY@tok@vm}{\def\PY@tc##1{\textcolor[rgb]{0.10,0.09,0.49}{##1}}}
\@namedef{PY@tok@sa}{\def\PY@tc##1{\textcolor[rgb]{0.73,0.13,0.13}{##1}}}
\@namedef{PY@tok@sb}{\def\PY@tc##1{\textcolor[rgb]{0.73,0.13,0.13}{##1}}}
\@namedef{PY@tok@sc}{\def\PY@tc##1{\textcolor[rgb]{0.73,0.13,0.13}{##1}}}
\@namedef{PY@tok@dl}{\def\PY@tc##1{\textcolor[rgb]{0.73,0.13,0.13}{##1}}}
\@namedef{PY@tok@s2}{\def\PY@tc##1{\textcolor[rgb]{0.73,0.13,0.13}{##1}}}
\@namedef{PY@tok@sh}{\def\PY@tc##1{\textcolor[rgb]{0.73,0.13,0.13}{##1}}}
\@namedef{PY@tok@s1}{\def\PY@tc##1{\textcolor[rgb]{0.73,0.13,0.13}{##1}}}
\@namedef{PY@tok@mb}{\def\PY@tc##1{\textcolor[rgb]{0.40,0.40,0.40}{##1}}}
\@namedef{PY@tok@mf}{\def\PY@tc##1{\textcolor[rgb]{0.40,0.40,0.40}{##1}}}
\@namedef{PY@tok@mh}{\def\PY@tc##1{\textcolor[rgb]{0.40,0.40,0.40}{##1}}}
\@namedef{PY@tok@mi}{\def\PY@tc##1{\textcolor[rgb]{0.40,0.40,0.40}{##1}}}
\@namedef{PY@tok@il}{\def\PY@tc##1{\textcolor[rgb]{0.40,0.40,0.40}{##1}}}
\@namedef{PY@tok@mo}{\def\PY@tc##1{\textcolor[rgb]{0.40,0.40,0.40}{##1}}}
\@namedef{PY@tok@ch}{\let\PY@it=\textit\def\PY@tc##1{\textcolor[rgb]{0.24,0.48,0.48}{##1}}}
\@namedef{PY@tok@cm}{\let\PY@it=\textit\def\PY@tc##1{\textcolor[rgb]{0.24,0.48,0.48}{##1}}}
\@namedef{PY@tok@cpf}{\let\PY@it=\textit\def\PY@tc##1{\textcolor[rgb]{0.24,0.48,0.48}{##1}}}
\@namedef{PY@tok@c1}{\let\PY@it=\textit\def\PY@tc##1{\textcolor[rgb]{0.24,0.48,0.48}{##1}}}
\@namedef{PY@tok@cs}{\let\PY@it=\textit\def\PY@tc##1{\textcolor[rgb]{0.24,0.48,0.48}{##1}}}

\def\PYZbs{\char`\\}
\def\PYZus{\char`\_}
\def\PYZob{\char`\{}
\def\PYZcb{\char`\}}
\def\PYZca{\char`\^}
\def\PYZam{\char`\&}
\def\PYZlt{\char`\<}
\def\PYZgt{\char`\>}
\def\PYZsh{\char`\#}
\def\PYZpc{\char`\%}
\def\PYZdl{\char`\$}
\def\PYZhy{\char`\-}
\def\PYZsq{\char`\'}
\def\PYZdq{\char`\"}
\def\PYZti{\char`\~}
% for compatibility with earlier versions
\def\PYZat{@}
\def\PYZlb{[}
\def\PYZrb{]}
\makeatother


    % For linebreaks inside Verbatim environment from package fancyvrb.
    \makeatletter
        \newbox\Wrappedcontinuationbox
        \newbox\Wrappedvisiblespacebox
        \newcommand*\Wrappedvisiblespace {\textcolor{red}{\textvisiblespace}}
        \newcommand*\Wrappedcontinuationsymbol {\textcolor{red}{\llap{\tiny$\m@th\hookrightarrow$}}}
        \newcommand*\Wrappedcontinuationindent {3ex }
        \newcommand*\Wrappedafterbreak {\kern\Wrappedcontinuationindent\copy\Wrappedcontinuationbox}
        % Take advantage of the already applied Pygments mark-up to insert
        % potential linebreaks for TeX processing.
        %        {, <, #, %, $, ' and ": go to next line.
        %        _, }, ^, &, >, - and ~: stay at end of broken line.
        % Use of \textquotesingle for straight quote.
        \newcommand*\Wrappedbreaksatspecials {%
            \def\PYGZus{\discretionary{\char`\_}{\Wrappedafterbreak}{\char`\_}}%
            \def\PYGZob{\discretionary{}{\Wrappedafterbreak\char`\{}{\char`\{}}%
            \def\PYGZcb{\discretionary{\char`\}}{\Wrappedafterbreak}{\char`\}}}%
            \def\PYGZca{\discretionary{\char`\^}{\Wrappedafterbreak}{\char`\^}}%
            \def\PYGZam{\discretionary{\char`\&}{\Wrappedafterbreak}{\char`\&}}%
            \def\PYGZlt{\discretionary{}{\Wrappedafterbreak\char`\<}{\char`\<}}%
            \def\PYGZgt{\discretionary{\char`\>}{\Wrappedafterbreak}{\char`\>}}%
            \def\PYGZsh{\discretionary{}{\Wrappedafterbreak\char`\#}{\char`\#}}%
            \def\PYGZpc{\discretionary{}{\Wrappedafterbreak\char`\%}{\char`\%}}%
            \def\PYGZdl{\discretionary{}{\Wrappedafterbreak\char`\$}{\char`\$}}%
            \def\PYGZhy{\discretionary{\char`\-}{\Wrappedafterbreak}{\char`\-}}%
            \def\PYGZsq{\discretionary{}{\Wrappedafterbreak\textquotesingle}{\textquotesingle}}%
            \def\PYGZdq{\discretionary{}{\Wrappedafterbreak\char`\"}{\char`\"}}%
            \def\PYGZti{\discretionary{\char`\~}{\Wrappedafterbreak}{\char`\~}}%
        }
        % Some characters . , ; ? ! / are not pygmentized.
        % This macro makes them "active" and they will insert potential linebreaks
        \newcommand*\Wrappedbreaksatpunct {%
            \lccode`\~`\.\lowercase{\def~}{\discretionary{\hbox{\char`\.}}{\Wrappedafterbreak}{\hbox{\char`\.}}}%
            \lccode`\~`\,\lowercase{\def~}{\discretionary{\hbox{\char`\,}}{\Wrappedafterbreak}{\hbox{\char`\,}}}%
            \lccode`\~`\;\lowercase{\def~}{\discretionary{\hbox{\char`\;}}{\Wrappedafterbreak}{\hbox{\char`\;}}}%
            \lccode`\~`\:\lowercase{\def~}{\discretionary{\hbox{\char`\:}}{\Wrappedafterbreak}{\hbox{\char`\:}}}%
            \lccode`\~`\?\lowercase{\def~}{\discretionary{\hbox{\char`\?}}{\Wrappedafterbreak}{\hbox{\char`\?}}}%
            \lccode`\~`\!\lowercase{\def~}{\discretionary{\hbox{\char`\!}}{\Wrappedafterbreak}{\hbox{\char`\!}}}%
            \lccode`\~`\/\lowercase{\def~}{\discretionary{\hbox{\char`\/}}{\Wrappedafterbreak}{\hbox{\char`\/}}}%
            \catcode`\.\active
            \catcode`\,\active
            \catcode`\;\active
            \catcode`\:\active
            \catcode`\?\active
            \catcode`\!\active
            \catcode`\/\active
            \lccode`\~`\~
        }
    \makeatother

    \let\OriginalVerbatim=\Verbatim
    \makeatletter
    \renewcommand{\Verbatim}[1][1]{%
        %\parskip\z@skip
        \sbox\Wrappedcontinuationbox {\Wrappedcontinuationsymbol}%
        \sbox\Wrappedvisiblespacebox {\FV@SetupFont\Wrappedvisiblespace}%
        \def\FancyVerbFormatLine ##1{\hsize\linewidth
            \vtop{\raggedright\hyphenpenalty\z@\exhyphenpenalty\z@
                \doublehyphendemerits\z@\finalhyphendemerits\z@
                \strut ##1\strut}%
        }%
        % If the linebreak is at a space, the latter will be displayed as visible
        % space at end of first line, and a continuation symbol starts next line.
        % Stretch/shrink are however usually zero for typewriter font.
        \def\FV@Space {%
            \nobreak\hskip\z@ plus\fontdimen3\font minus\fontdimen4\font
            \discretionary{\copy\Wrappedvisiblespacebox}{\Wrappedafterbreak}
            {\kern\fontdimen2\font}%
        }%

        % Allow breaks at special characters using \PYG... macros.
        \Wrappedbreaksatspecials
        % Breaks at punctuation characters . , ; ? ! and / need catcode=\active
        \OriginalVerbatim[#1,codes*=\Wrappedbreaksatpunct]%
    }
    \makeatother

    % Exact colors from NB
    \definecolor{incolor}{HTML}{303F9F}
    \definecolor{outcolor}{HTML}{D84315}
    \definecolor{cellborder}{HTML}{CFCFCF}
    \definecolor{cellbackground}{HTML}{F7F7F7}

    % prompt
    \makeatletter
    \newcommand{\boxspacing}{\kern\kvtcb@left@rule\kern\kvtcb@boxsep}
    \makeatother
    \newcommand{\prompt}[4]{
        {\ttfamily\llap{{\color{#2}[#3]:\hspace{3pt}#4}}\vspace{-\baselineskip}}
    }
    

    
    % Prevent overflowing lines due to hard-to-break entities
    \sloppy
    % Setup hyperref package
    \hypersetup{
      breaklinks=true,  % so long urls are correctly broken across lines
      colorlinks=true,
      urlcolor=urlcolor,
      linkcolor=linkcolor,
      citecolor=citecolor,
      }
    % Slightly bigger margins than the latex defaults
    
    \geometry{verbose,tmargin=1in,bmargin=1in,lmargin=1in,rmargin=1in}
    
    

\begin{document}
    
    \maketitle
    
    

    
    

    \hypertarget{solution-to-exercises-week-37-and-38-lipkin-model}{%
\section{Solution to exercises week 37 and 38, Lipkin
model}\label{solution-to-exercises-week-37-and-38-lipkin-model}}

\textbf{Morten Hjorth-Jensen}, Department of Physics, University of Oslo

Date: \textbf{March 5, 2025}

    \hypertarget{lipkin-model}{%
\subsection{Lipkin model}\label{lipkin-model}}

We will study a schematic model (the Lipkin model, see Nuclear Physics
\textbf{62} (1965) 188), for the interaction among \(2\) and more
fermions that can occupy two different energy levels.

In project 1 we consider a two-fermion case and a four-fermion case.

For four fermions, the case we consider in the examples here, each
levels has degeneration \(d=4\), leading to different total spin values.
The two levels have quantum numbers \(\sigma=\pm 1\), with the upper
level having \(2\sigma=+1\) and energy
\(\varepsilon_{1}= \varepsilon/2\). The lower level has \(2\sigma=-1\)
and energy \(\varepsilon_{2}=-\varepsilon/2\). That is, the lowest
single-particle level has negative spin projection (or spin down), while
the upper level has spin up. In addition, the substates of each level
are characterized by the quantum numbers \(p=1,2,3,4\).

    \hypertarget{four-fermion-case}{%
\subsection{Four fermion case}\label{four-fermion-case}}

We define the single-particle states (for the four fermion case which we
will work on here)

    \[
\vert u_{\sigma =-1,p}\rangle=a_{-p}^{\dagger}\vert 0\rangle
\hspace{1cm}
\vert u_{\sigma =1,p}\rangle=a_{+p}^{\dagger}\vert 0\rangle.
\]

    The single-particle states span an orthonormal basis.

    \hypertarget{hamiltonian}{%
\subsection{Hamiltonian}\label{hamiltonian}}

The Hamiltonian of the system is given by

    \[
\begin{array}{ll}
\hat{H}=&\hat{H}_{0}+\hat{H}_{1}+\hat{H}_{2}\\
&\\
\hat{H}_{0}=&\frac{1}{2}\varepsilon\sum_{\sigma ,p}\sigma
a_{\sigma,p}^{\dagger}a_{\sigma ,p}\\
&\\
\hat{H}_{1}=&\frac{1}{2}V\sum_{\sigma ,p,p'}
a_{\sigma,p}^{\dagger}a_{\sigma ,p'}^{\dagger}
a_{-\sigma ,p'}a_{-\sigma ,p}\\
&\\
\hat{H}_{2}=&\frac{1}{2}W\sum_{\sigma ,p,p'}
a_{\sigma,p}^{\dagger}a_{-\sigma ,p'}^{\dagger}
a_{\sigma ,p'}a_{-\sigma ,p}\\
&\\
\end{array}
\]

    where \(V\) and \(W\) are constants. The operator \(H_{1}\) can move
pairs of fermions while \(H_{2}\) is a spin-exchange term. The latter
moves a pair of fermions from a state \((p\sigma ,p' -\sigma)\) to a
state \((p-\sigma ,p'\sigma)\).

    \hypertarget{quasispin-operators}{%
\subsection{Quasispin operators}\label{quasispin-operators}}

We are going to rewrite the above Hamiltonian in terms of so-called
quasispin operators

    \[
\begin{array}{ll}
\hat{J}_{+}=&\sum_{p}
a_{p+}^{\dagger}a_{p-}\\
&\\
\hat{J}_{-}=&\sum_{p}
a_{p-}^{\dagger}a_{p+}\\
&\\
\hat{J}_{z}=&\frac{1}{2}\sum_{p\sigma}\sigma
a_{p\sigma}^{\dagger}a_{p\sigma}\\
&\\
\hat{J}^{2}=&J_{+}J_{-}+J_{z}^{2}-J_{z}\\
&\\
\end{array}
\]

    We show here that these operators obey the commutation relations for
angular momentum.

    \hypertarget{including-the-number-operator}{%
\subsection{Including the number
operator}\label{including-the-number-operator}}

We can in turn express \(\hat{H}\) in terms of the above quasispin
operators and the number operator

    \[
\hat{N}=\sum_{p\sigma}
a_{p\sigma}^{\dagger}a_{p\sigma}.
\]

    We have the following quasispin operators

    \hypertarget{eq:Jpm}{}

\[
\begin{equation}
J_{\pm} = \sum_p a_{p\pm}^\dagger a_{p\mp},
\label{eq:Jpm} \tag{1} 
\end{equation}
\]

    \hypertarget{eq:Jz}{}

\[
\begin{equation} 
J_{z} = \frac{1}{2}\sum_{p,\sigma} \sigma a_{p\sigma}^\dagger a_{p\sigma},
\label{eq:Jz} \tag{2} 
\end{equation}
\]

    \hypertarget{eq:J2}{}

\[
\begin{equation} 
J^{2} = J_+ J_- + J_z^2 - J_z,
\label{eq:J2} \tag{3}
\end{equation}
\]

    and we want to compute the commutators

    \[
[J_z,J_\pm], \quad [J_+,J_-], \quad [J^2,J_\pm] \quad \text{og} \quad 
[J^2,J_z].
\]

    \hypertarget{angular-momentum-magics-i}{%
\subsection{Angular momentum magics I}\label{angular-momentum-magics-i}}

Let us start with the first one and inserting for \(J_z\) and \(J_\pm\)
given by the equations (\hyperref[eqjz]{2}) and (\hyperref[eqjpm]{1}),
respectively, we obtain

    \[
\begin{align*}
[J_z,J_\pm] &= J_z J_\pm - J_\pm J_z \\
%
&= \left( \frac{1}{2}\sum_{p,\sigma} \sigma a_{p\sigma}^\dagger a_{p\sigma} \right)
\left( \sum_{p'} a_{p'\pm}^\dagger a_{p'\mp} \right) -
\left( \sum_{p'} a_{p'\pm}^\dagger a_{p'\mp} \right)
\left( \frac{1}{2}\sum_{p,\sigma} \sigma a_{p\sigma}^\dagger a_{p\sigma} \right) \\
&= \frac{1}{2} \sum_{p,p',\sigma} \sigma \left( a_{p\sigma}^\dagger a_{p\sigma} a_{p'\pm}^\dagger a_{p'\mp} - a_{p'\pm}^\dagger a_{p'\mp} a_{p\sigma}^\dagger a_{p\sigma} \right).
\end{align*}
\]

    \hypertarget{angular-momentum-magics-ii}{%
\subsection{Angular momentum magics
II}\label{angular-momentum-magics-ii}}

Using the commutation relations for the creation and annihilation
operators

    \hypertarget{eq:alux2cak}{}

\[
\begin{equation}
\{ a_l,a_k \} = 0, \label{eq:al,ak} \tag{4} 
\end{equation}
\]

    \hypertarget{eq:aldux2cakd}{}

\[
\begin{equation} 
\{ a_l^\dagger , a_k^\dagger \} = 0, \label{eq:ald,akd} \tag{5} 
\end{equation}
\]

    \hypertarget{eq:aldux2cak}{}

\[
\begin{equation} 
\{ a_l^\dagger , a_k \} = \delta_{lk}, \label{eq:ald,ak} \tag{6}
\end{equation}
\]

    in order to move the operators in the right product to be in the same
order as those in the lefthand product

    \[
\begin{align*}
[J_z,J_\pm] &= \frac{1}{2} \sum_{p,p',\sigma} \sigma \left(
a_{p\sigma}^\dagger a_{p\sigma} a_{p'\pm}^\dagger a_{p'\mp} -
a_{p'\pm}^\dagger \left( \delta_{p' p} \delta_{\mp \sigma} - a_{p\sigma}^\dagger a_{p'\mp} \right) a_{p\sigma} \right) \\
&= \frac{1}{2} \sum_{p,p',\sigma} \sigma \left(
a_{p\sigma}^\dagger a_{p\sigma} a_{p'\pm}^\dagger a_{p'\mp} -
a_{p'\pm}^\dagger \delta_{p' p} \delta_{\mp \sigma} a_{p\sigma} +
a_{p'\pm}^\dagger a_{p\sigma}^\dagger a_{p'\mp} a_{p\sigma} \right). \\
\end{align*}
\]

    \hypertarget{angular-momentum-magics-iii}{%
\subsection{Angular momentum magics
III}\label{angular-momentum-magics-iii}}

It results in

    \[
\begin{align*}
[J_z,J_\pm]
&= \frac{1}{2} \sum_{p,p',\sigma} \sigma \left(
a_{p\sigma}^\dagger a_{p\sigma} a_{p'\pm}^\dagger a_{p'\mp} -
a_{p'\pm}^\dagger \delta_{pp'} \delta_{\mp \sigma} a_{p\sigma} +
a_{p\sigma}^\dagger a_{p'\pm}^\dagger a_{p\sigma} a_{p'\mp} \right) \\
&= \frac{1}{2} \sum_{p,p',\sigma} \sigma \left(
a_{p\sigma}^\dagger a_{p\sigma} a_{p'\pm}^\dagger a_{p'\mp} -
a_{p'\pm}^\dagger \delta_{pp'} \delta_{\mp \sigma} a_{p\sigma} +
a_{p\sigma}^\dagger \left( \delta_{pp'} \delta_{\pm \sigma} - a_{p\sigma} a_{p'\pm}^\dagger \right) a_{p'\mp} \right) \\
&= \frac{1}{2} \sum_{p,p',\sigma} \sigma \left(
a_{p\sigma}^\dagger \delta_{pp'} \delta_{\pm \sigma} a_{p'\mp} -
a_{p'\pm}^\dagger \delta_{pp'} \delta_{\mp \sigma} a_{p\sigma} \right). \\
\end{align*}
\]

    \hypertarget{angular-momentum-magics-iv}{%
\subsection{Angular momentum magics
IV}\label{angular-momentum-magics-iv}}

The last equality leads to

    \[
\begin{align*}
[J_z,J_\pm] &= \frac{1}{2} \sum_p \left(
(\pm 1) a_{p\pm}^\dagger a_{p\mp} - (\mp 1)
a_{p\pm}^\dagger a_{p\mp} \right) =
\pm \frac{1}{2} \sum_p \left(
a_{p\pm}^\dagger a_{p\mp} + (\pm 1)
a_{p\pm}^\dagger a_{p\mp} \right) \\
&= \pm \sum_p a_{p\pm}^\dagger a_{p\mp} = \pm J_\pm,
\end{align*}
\]

    where the last results follows from comparing with Eq.
(\hyperref[eqjpm]{1}).

    \hypertarget{angular-momentum-magics-v}{%
\subsection{Angular momentum magics V}\label{angular-momentum-magics-v}}

We can then continue with the next commutation relation, using Eq.
(\hyperref[eqjpm]{1}),

    \[
\begin{align*}
[J_+,J_-] &= J_+ J_- - J_- J_+ \\
&= \left( \sum_p a_{p'+}^\dagger a_{p-} \right)
\left( \sum_{p'} a_{p'-}^\dagger a_{p'+} \right) -
\left( \sum_{p'} a_{p'-}^\dagger a_{p'+} \right)
\left( \sum_p a_{p+}^\dagger a_{p-} \right) \\
&= \sum_{p,p'} \left(
a_{p'+}^\dagger a_{p-} a_{p'-}^\dagger a_{p'+} -
a_{p'-}^\dagger a_{p'+} a_{p+}^\dagger a_{p-} \right) \\
&= \sum_{p,p'} \left(
a_{p'+}^\dagger a_{p-} a_{p'-}^\dagger a_{p'+} -
a_{p'-}^\dagger \left( \delta_{++} \delta_{pp'} -
a_{p+}^\dagger a_{p'+} \right) a_{p-} \right) \\
&= \sum_{p,p'} \left(
a_{p'+}^\dagger a_{p-} a_{p'-}^\dagger a_{p'+} -
a_{p'-}^\dagger \delta_{pp'} a_{p-} +
a_{p'-}^\dagger a_{p+}^\dagger a_{p'+} a_{p-} \right) \\
&= \sum_{p,p'} \left(
a_{p'+}^\dagger a_{p-} a_{p'-}^\dagger a_{p'+} -
a_{p'-}^\dagger \delta_{pp'} a_{p-} +
a_{p+}^\dagger a_{p'-}^\dagger a_{p-} a_{p'+} \right) \\
&= \sum_{p,p'} \left(
a_{p'+}^\dagger a_{p-} a_{p'-}^\dagger a_{p'+} -
a_{p'-}^\dagger \delta_{pp'} a_{p-} +
a_{p+}^\dagger \left( \delta_{--} \delta_{pp'} -
a_{p-} a_{p'-}^\dagger \right) a_{p'+} \right) \\
&= \sum_{p,p'} \left(
a_{p+}^\dagger \delta_{pp'} a_{p'+} -
a_{p'-}^\dagger \delta_{pp'} a_{p-} \right), \\
\end{align*}
\]

    \hypertarget{angular-momentum-magics-vi}{%
\subsection{Angular momentum magics
VI}\label{angular-momentum-magics-vi}}

Which results in

    \[
[J_+,J_-] = \sum_p \left(
a_{p+}^\dagger a_{p+} -
a_{p-}^\dagger a_{p-} \right) = 2J_z,
\]

    It is straightforward to show that

    \[
[J^2, J_\pm] = [J_+ J_- + J_z^2 - J_z, J_\pm] =
[J_+ J_-, J_\pm] + [J_z^2, J_\pm] - [J_z, J_\pm].
\]

    \hypertarget{angular-momentum-magics-vii}{%
\subsection{Angular momentum magics
VII}\label{angular-momentum-magics-vii}}

Using the relations

    \hypertarget{eq:abux2cc}{}

\[
\begin{equation}
[AB,C] = A[B,C] + [A,C]B, \label{eq:ab,c} \tag{7} 
\end{equation}
\]

    \hypertarget{eq:aux2cbc}{}

\[
\begin{equation} 
[A,BC] = [A,B]C + B[A,C], \label{eq:a,bc} \tag{8}
\end{equation}
\]

    we obtain

    \[
[J^2, J_\pm] =
J_+ [J_-,J_\pm] + [J_+,J_\pm] J_- + J_z [J_z,J_\pm] + [J_z,J_\pm] J_z - [J_z,J_\pm].
\]

    \hypertarget{angular-momentum-magics-viii}{%
\subsection{Angular momentum magics
VIII}\label{angular-momentum-magics-viii}}

Finally, from the above it follows that

    \[
\begin{align*}
[J^2, J_+] &= -2J_+ J_z + J_z [J_z,J_+] + [J_z,J_+] J_z - [J_z,J_+] \\
&= -2J_+ J_z + J_z J_+ + J_+ J_z - J_+ \\
&= -2J_+ J_z + J_+ + J_+ J_z + J_+ J_z - J_+ = 0,
\end{align*}
\]

    and

    \[
\begin{align*}
[J^2, J_-] &= 2J_z J_- + J_z [J_z,J_-] + [J_z,J_-] J_z - [J_z,J_-] \\
&= 2J_z J_- - J_z J_- - J_- J_z + J_- \\
&= J_z J_- - (J_z J_- + J_-) + J_- = 0.
\end{align*}
\]

    \hypertarget{angular-momentum-magics-ix}{%
\subsection{Angular momentum magics
IX}\label{angular-momentum-magics-ix}}

Our last commutator is given by

    \[
\begin{align*}
[J^2,J_z] &= [J_+ J_- + J_z^2 - J_z, J_z] \\
&= [J_+ J_-, J_z] + [J_z^2, J_z] - [J_z, J_z] \\
&= J_+ [J_-, J_z] + [J_+,J_z] J_- \\
&= J_+ J_- - J_+ J_- = 0
\end{align*}
\]

    \hypertarget{angular-momentum-magics-x}{%
\subsection{Angular momentum magics X}\label{angular-momentum-magics-x}}

Summing up we have

    \hypertarget{eq:kJzJpm}{}

\[
\begin{equation}
[J_z, J_\pm] = \pm J_\pm, \label{eq:kJzJpm} \tag{9} 
\end{equation}
\]

    \hypertarget{eq:kJpJm}{}

\[
\begin{equation} 
[J_+, J_-] = 2J_z, \label{eq:kJpJm} \tag{10} 
\end{equation}
\]

    \hypertarget{eq:kJ2Jpm}{}

\[
\begin{equation} 
[J^2, J_\pm] = 0, \label{eq:kJ2Jpm} \tag{11} 
\end{equation}
\]

    \hypertarget{eq:kJ2Jz}{}

\[
\begin{equation} 
[J^2,J_z] = 0, \label{eq:kJ2Jz} \tag{12}
\end{equation}
\]

    which are the standard commutation relations for angular (or orbital)
momentum \(L_\pm\), \(L_z\) og \(L^2\).

    \hypertarget{rewriting-the-hamiltonian}{%
\subsection{Rewriting the Hamiltonian}\label{rewriting-the-hamiltonian}}

We wrote the above Hamiltonian as

    \[
H = H_0 + H_1 +H_2,
\]

    with

    \[
H_0 = \frac{1}{2} \varepsilon \sum_{p\sigma}\sigma a_{p\sigma}^{\dagger}a_{p\sigma},
\]

    and

    \[
H_1 = \frac{1}{2} V \sum_{p,p',\sigma} a_{p\sigma}^\dagger a_{p'\sigma}^\dagger a_{p'-\sigma} a_{p-\sigma},
\]

    and

    \[
H_{2} = \frac{1}{2} W \sum_{p,p',\sigma}a_{p\sigma}^\dagger a_{p'-\sigma}^\dagger a_{p'\sigma}a_{p-\sigma}.
\]

    \hypertarget{hamiltonian-and-angular-momentum-magics-i}{%
\subsection{Hamiltonian and angular momentum magics
I}\label{hamiltonian-and-angular-momentum-magics-i}}

We will now rewrite the Hamiltonian in terms of the above quasi-spin
operators and the number operator

    \hypertarget{eq:N}{}

\[
\begin{equation}
N = \sum_{p,\sigma} a_{p\sigma}^\dagger a_{p\sigma}.
\label{eq:N} \tag{13}
\end{equation}
\]

    Going through each term of the Hamiltonian and using the expressions for
the quasi-spin operators we obtain

    \hypertarget{eq:H0ny}{}

\[
\begin{equation}
H_0 = \varepsilon J_z.
\label{eq:H0ny} \tag{14}
\end{equation}
\]

    \hypertarget{hamiltonian-and-angular-momentum-magics-ii}{%
\subsection{Hamiltonian and angular momentum magics
II}\label{hamiltonian-and-angular-momentum-magics-ii}}

Moving over to \(H_1\) and using the anti-commutation relations
(\hyperref[eqalak]{4}) through (\hyperref[eqaldak]{6}) we obtain

    \[
\begin{align*}
H_1 &= \frac{1}{2} V \sum_{p,p',\sigma}
a_{p\sigma}^\dagger a_{p'\sigma}^\dagger a_{p'-\sigma} a_{p-\sigma} \\
&= \frac{1}{2} V \sum_{p,p',\sigma}
-a_{p\sigma}^\dagger a_{p'\sigma}^\dagger a_{p-\sigma} a_{p'-\sigma} \\
&= \frac{1}{2} V \sum_{p,p',\sigma}
-a_{p\sigma}^\dagger \left( \delta_{pp'} \delta_{\sigma -\sigma} - a_{p-\sigma} a_{p'\sigma}^\dagger \right) a_{p'-\sigma} \\
&= \frac{1}{2} V \sum_{p,p',\sigma}
a_{p\sigma}^\dagger a_{p-\sigma} a_{p'\sigma}^\dagger a_{p'-\sigma} \\
\end{align*}
\]

    \hypertarget{hamiltonian-and-angular-momentum-magics-iii}{%
\subsection{Hamiltonian and angular momentum magics
III}\label{hamiltonian-and-angular-momentum-magics-iii}}

Rewriting the sum over \(\sigma\) we arrive at

    \[
\begin{align*}
H_1 &= \frac{1}{2} V\sum_{p,p'}
a_{p+}^\dagger a_{p-} a_{p'+}^\dagger a_{p'-} +
a_{p-}^\dagger a_{p+} a_{p'-}^\dagger a_{p'+} \\
&= \frac{1}{2} V \left[ \sum_p \left( a_{p+}^\dagger a_{p-} \right)
\sum_{p'} \left( a_{p'+}^\dagger a_{p'-} \right) +
\sum_p \left( a_{p-}^\dagger a_{p+} \right)
\sum_{p'} \left( a_{p'-}^\dagger a_{p'+} \right) \right] \\
&= \frac{1}{2} V \left[ J_+ J_+ + J_- J_- \right] = \frac{1}{2} V \left[ J_+^2 + J_-^2 \right] ,
\end{align*}
\]

    which leads to

    \hypertarget{eq:H1ny}{}

\[
\begin{equation}
H_1 = \frac{1}{2} V \left( J_+^2 + J_-^2 \right).
\label{eq:H1ny} \tag{15}
\end{equation}
\]

    \hypertarget{hamiltonian-and-angular-momentum-magics-iv}{%
\subsection{Hamiltonian and angular momentum magics
IV}\label{hamiltonian-and-angular-momentum-magics-iv}}

Finally, we rewrite the last term

    \[
\begin{align*}
H_2 &= \frac{1}{2} W \sum_{p,p',\sigma}
a_{p\sigma}^\dagger a_{p'-\sigma}^\dagger a_{p'\sigma} a_{p-\sigma} \\
&= \frac{1}{2} W \sum_{p,p',\sigma}
-a_{p\sigma}^\dagger a_{p'-\sigma}^\dagger a_{p-\sigma} a_{p'\sigma} \\
&= \frac{1}{2} W \sum_{p,p',\sigma}
-a_{p\sigma}^\dagger \left( \delta_{pp'} \delta_{-\sigma -\sigma} -
a_{p-\sigma} a_{p'-\sigma}^\dagger \right) a_{p'\sigma} \\
&= \frac{1}{2} W \sum_{p,p',\sigma}
-a_{p\sigma}^\dagger \delta_{pp'} a_{p'\sigma} +
a_{p\sigma}^\dagger a_{p-\sigma} a_{p'-\sigma}^\dagger a_{p'\sigma} \\
&= \frac{1}{2} W \left( -\sum_{p,\sigma}
a_{p\sigma}^\dagger a_{p\sigma} +
\sum_{p,p',\sigma} a_{p\sigma}^\dagger a_{p-\sigma} a_{p'-\sigma}^\dagger a_{p'\sigma} \right) \\
\end{align*}
\]

    \hypertarget{hamiltonian-and-angular-momentum-magics-v}{%
\subsection{Hamiltonian and angular momentum magics
V}\label{hamiltonian-and-angular-momentum-magics-v}}

Using the expression for the number operator we obtain

    \[
\begin{align*}
\sum_{p,p',\sigma} a_{p\sigma}^\dagger a_{p-\sigma} a_{p'-\sigma}^\dagger a_{p'\sigma}
&= \sum_{p,p'} a_{p+}^\dagger a_{p-} a_{p'-}^\dagger a_{p'+} +
a_{p-}^\dagger a_{p+} a_{p'+}^\dagger a_{p'-} \\
&= \sum_p \left( a_{p+}^\dagger a_{p-} \right)
\sum_{p'} \left( a_{p'-}^\dagger a_{p'+} \right) +
\sum_p \left( a_{p-}^\dagger a_{p+} \right)
\sum_{p'} \left( a_{p'+}^\dagger a_{p'-} \right) \\
&= J_+ J_- + J_- J_+,
\end{align*}
\]

    resulting in

    \hypertarget{eq:H2ny}{}

\[
\begin{equation}
H_2 = \frac{1}{2} W \left( -N + J_+ J_- + J_- J_+ \right).
\label{eq:H2ny} \tag{16}
\end{equation}
\]

    We have thus expressed the Hamiltonian in term of the quasi-spin
operators. Below, we will show how we can rewrite these expressions in
terms of Pauli \(X\), \(Y\) and \(Z\) matrices.

    \hypertarget{commutation-relations-for-the-hamiltonian}{%
\subsection{Commutation relations for the
Hamiltonian}\label{commutation-relations-for-the-hamiltonian}}

The above expressions can in turn be used to show that the Hamiltonian
commutes with the various quasi-spin operators. This leads to quantum
numbers which are conserved. Let us first show that \([H,J^2]=0\), which
means that \(J\) is a so-called \emph{good} quantum number and that the
total spin is a conserved quantum number.

We have

    \[
\begin{align*}
[H,J^2] &= [H_0 + H_1 + H_2,J^2] \\
&= [H_0,J^2] + [H_1,J^2] + [H_2,J^2] \\
&= \varepsilon [J_z,J^2] + \frac{1}{2} V [J_+^2 + J_-^2,J^2] +
\frac{1}{2} W [-N + J_+ J_- + J_- J_+,J^2]. \\
\end{align*}
\]

    \hypertarget{hamiltonian-and-commutators}{%
\subsection{Hamiltonian and
commutators}\label{hamiltonian-and-commutators}}

We have previously shown that

    \[
[H,J^2] = \frac{1}{2} V \left( [J_+^2,J^2] + [J_-^2,J^2] \right) +
\frac{1}{2} W \left( -[N,J^2] + [J_+ J_-,J^2] + [J_- J_+, J^2] \right)
\]

    Using that \([J_\pm,J^2] = 0\), it follows that \([J_\pm^2,J^2] = 0\).
We can then see that \([J_+ J_-,J^2] = 0\) and \([J_- J_+, J^2] = 0\)
which leads to

    \[
\begin{align*}
[H,J^2] &= -\frac{1}{2} W [N,J^2] \\
&= \frac{1}{2} W \left( -[N,J_+ J_-] - [N,J_z^2] + [N,J_z] \right) \\
&= \frac{1}{2} W \left( -[N,J_+]J_- - J_+[N,J_-] - [N,J_z]J_z - J_z[N,J_z] + [N,J_z] \right).
\end{align*}
\]

    \hypertarget{including-the-number-operator}{%
\subsection{Including the number
operator}\label{including-the-number-operator}}

Combining with the number operator we have

    \[
\begin{align*}
[N,J_\pm] &= N J_\pm - J_\pm N \\
&= \left( \sum_{p,\sigma} a_{p\sigma}^\dagger a_{p\sigma} \right)
\left( \sum_{p'} a_{p'\pm}^\dagger a_{p'\mp} \right) -
\left( \sum_{p'} a_{p'\pm}^\dagger a_{p'\mp} \right)
\left( \sum_{p,\sigma} a_{p\sigma}^\dagger a_{p\sigma} \right) \\
&= \sum_{p,p',\sigma}
a_{p\sigma}^\dagger a_{p\sigma} a_{p'\pm}^\dagger a_{p'\mp} -
a_{p'\pm}^\dagger a_{p'\mp} a_{p\sigma}^\dagger a_{p\sigma} \\
&= \sum_{p,p',\sigma}
a_{p\sigma}^\dagger a_{p\sigma} a_{p'\pm}^\dagger a_{p'\mp} -
a_{p'\pm}^\dagger \left( \delta_{\mp \sigma} \delta_{pp'} - a_{p\sigma}^\dagger a_{p'\mp} \right) a_{p\sigma} \\
&= \sum_{p,p',\sigma}
a_{p\sigma}^\dagger a_{p\sigma} a_{p'\pm}^\dagger a_{p'\mp} -
a_{p'\pm}^\dagger \delta_{\mp \sigma} \delta_{pp'} a_{p\sigma} +
a_{p'\pm}^\dagger a_{p\sigma}^\dagger a_{p'\mp} a_{p\sigma} \\
&= \sum_{p,p',\sigma}
a_{p\sigma}^\dagger a_{p\sigma} a_{p'\pm}^\dagger a_{p'\mp} +
a_{p\sigma}^\dagger a_{p'\pm}^\dagger a_{p\sigma} a_{p'\mp} -
\sum_{p} a_{p\pm}^\dagger  a_{p\mp} \\
&= \sum_{p,p',\sigma}
a_{p\sigma}^\dagger a_{p\sigma} a_{p'\pm}^\dagger a_{p'\mp} +
a_{p\sigma}^\dagger \left( \delta_{pp'} \delta_{\pm \sigma} -
a_{p\sigma} a_{p'\pm}^\dagger \right) a_{p'\mp} -
\sum_{p} a_{p\pm}^\dagger  a_{p\mp} \\
&= \sum_p a_{p\pm}^\dagger a_{p\mp} -
\sum_{p} a_{p\pm}^\dagger  a_{p\mp} = 0. \\
\end{align*}
\]

    \hypertarget{hamiltonian-and-angular-momentum-commutators}{%
\subsection{Hamiltonian and angular momentum
commutators}\label{hamiltonian-and-angular-momentum-commutators}}

We obtain then

    \[
\begin{align*}
[N,J_z] &= N J_z - J_z N \\
&= \left( \sum_{p,\sigma} a_{p\sigma}^\dagger a_{p\sigma} \right)
\left( \frac{1}{2}\sum_{p',\sigma} \sigma a_{p'\sigma}^\dagger a_{p'\sigma} \right) -
\left( \frac{1}{2}\sum_{p',\sigma} \sigma a_{p'\sigma}^\dagger a_{p'\sigma} \right)
\left( \sum_{p,\sigma} a_{p\sigma}^\dagger a_{p\sigma} \right) \\
&= \sum_{p,p',\sigma} 
\sigma a_{p\sigma}^\dagger a_{p\sigma} a_{p'\sigma}^\dagger a_{p'\sigma} -
\sigma a_{p'\sigma}^\dagger a_{p'\sigma} a_{p\sigma}^\dagger a_{p\sigma} = 0,
\end{align*}
\]

    which leads to

    \hypertarget{eq:kHJ2}{}

\[
\begin{equation}
[H,J^2] = 0,
\label{eq:kHJ2} \tag{17}
\end{equation}
\]

    and \(J\) is a good quantum number.

    \hypertarget{constructing-the-hamiltonian-matrix-for-j2}{%
\subsection{\texorpdfstring{Constructing the Hamiltonian matrix for
\(J=2\)}{Constructing the Hamiltonian matrix for J=2}}\label{constructing-the-hamiltonian-matrix-for-j2}}

We start with the state (unique) where all spins point down

    \hypertarget{eq:2ux2c-2}{}

\[
\begin{equation}
\vert 2,-2\rangle = a_{1-}^{\dagger} a_{2-}^{\dagger}
a_{3-}^{\dagger} a_{4-}^{\dagger} \vert 0\rangle
\label{eq:2,-2} \tag{18}
\end{equation}
\]

    which is a state with \(J_z = -2\) and \(J = 2\). (we label the states
as \(\vert J,J_z\rangle\)). For \(J = 2\) we have the spin projections
\(J_z = -2,-1,0,1,2\). We can use the lowering and raising operators for
spin in order to define the other states

    \hypertarget{eq:J+ket}{}

\[
\begin{equation}
J_+ \vert J,J_z\rangle = \sqrt{J(J+1) - J_z(J_z + 1)} \vert J,J_z + 1\rangle,
\label{eq:J+ket} \tag{19} 
\end{equation}
\]

    \hypertarget{eq:J-ket}{}

\[
\begin{equation} 
J_- \vert J,J_z\rangle = \sqrt{J(J+1) - J_z(J_z - 1)} \vert J,J_z - 1\rangle.
\label{eq:J-ket} \tag{20}
\end{equation}
\]

    \hypertarget{constructing-the-hamiltonian-matrix-for-the-other-states}{%
\subsection{Constructing the Hamiltonian matrix for the other
states}\label{constructing-the-hamiltonian-matrix-for-the-other-states}}

We can then construct all other states with \(J=2\) using the raising
operator \(J_+\) on \(\vert 2,-2\rangle\)

    \[
J_+ \vert 2,-2\rangle = \sqrt{2(2+1) - (-2)(-2+1)} \vert 2,-2+1\rangle =\sqrt{6 - 2} \vert 2,-1\rangle = 2\vert 2,-1\rangle,
\]

    which gives

    \hypertarget{_auto1}{}

\[
\begin{equation}
\vert 2,-1\rangle = \frac{1}{2} J_+ \vert 2,-2\rangle \notag 
\label{_auto1} \tag{21}
\end{equation}
\]

    \hypertarget{_auto2}{}

\[
\begin{equation} 
= \frac{1}{2} \sum_p a_{p+}^\dagger a_{p-} a_{1-}^{\dagger} a_{2-}^{\dagger}
a_{3-}^{\dagger} a_{4-}^{\dagger} \vert 0\rangle \notag 
\label{_auto2} \tag{22}
\end{equation}
\]

    \hypertarget{eq:2ux2c-1}{}

\[
\begin{equation} 
= \frac{1}{2} \left(
a_{1+}^{\dagger} a_{2-}^{\dagger} a_{3-}^{\dagger} a_{4-}^{\dagger} +
a_{1-}^{\dagger} a_{2+}^{\dagger} a_{3-}^{\dagger} a_{4-}^{\dagger} +
a_{1-}^{\dagger} a_{2-}^{\dagger} a_{3+}^{\dagger} a_{4-}^{\dagger} +
a_{1-}^{\dagger} a_{2-}^{\dagger} a_{3-}^{\dagger} a_{4+}^{\dagger}
\right) \vert 0\rangle. \label{eq:2,-1} \tag{23}
\end{equation}
\]

    \hypertarget{constructing-the-hamiltonian-matrix}{%
\subsection{Constructing the Hamiltonian
matrix}\label{constructing-the-hamiltonian-matrix}}

We can construct all the other states in the same way. That is

    \[
J_+ \vert 2,-1\rangle = \sqrt{2(2+1) - (-1)(-1+1)} \vert 2,-1+1\rangle = \sqrt{6} \vert 2,0\rangle,
\]

    which results in

    \hypertarget{eq:2ux2c0}{}

\[
\begin{equation}
\begin{aligned}
\vert 2,0\rangle &= \frac{1}{\sqrt{6}} \left(
a_{1+}^{\dagger} a_{2+}^{\dagger} a_{3-}^{\dagger} a_{4-}^{\dagger} +
a_{1+}^{\dagger} a_{2-}^{\dagger} a_{3+}^{\dagger} a_{4-}^{\dagger} +
a_{1+}^{\dagger} a_{2-}^{\dagger} a_{3-}^{\dagger} a_{4+}^{\dagger} +
a_{1-}^{\dagger} a_{2+}^{\dagger} a_{3+}^{\dagger} a_{4-}^{\dagger} + \right. \\
&\quad\,\, \left.
a_{1-}^{\dagger} a_{2+}^{\dagger} a_{3-}^{\dagger} a_{4+}^{\dagger} +
a_{1-}^{\dagger} a_{2-}^{\dagger} a_{3+}^{\dagger} a_{4+}^{\dagger} \right) \vert 0\rangle
\end{aligned}
\label{eq:2,0} \tag{24}
\end{equation}
\]

    \hypertarget{constructing-the-hamiltonian-matrix-last-two-states}{%
\subsection{Constructing the Hamiltonian matrix, last two
states}\label{constructing-the-hamiltonian-matrix-last-two-states}}

The two remaining states are

    \hypertarget{eq:2ux2c1}{}

\[
\begin{equation}
\vert2,1\rangle  = \frac{1}{2} \left(
a_{1+}^{\dagger} a_{2+}^{\dagger} a_{3+}^{\dagger} a_{4-}^{\dagger} +
a_{1+}^{\dagger} a_{2+}^{\dagger} a_{3-}^{\dagger} a_{4+}^{\dagger} +
a_{1+}^{\dagger} a_{2-}^{\dagger} a_{3+}^{\dagger} a_{4+}^{\dagger} +
a_{1-}^{\dagger} a_{2+}^{\dagger} a_{3+}^{\dagger} a_{4+}^{\dagger}
 \right).
\label{eq:2,1} \tag{25}
\end{equation}
\]

    and

    \hypertarget{eq:2ux2c2}{}

\[
\begin{equation}
\vert 2,2\rangle  = a_{1+}^{\dagger} a_{2+}^{\dagger} a_{3+}^{\dagger} a_{4+}^{\dagger} \vert 0\rangle.
\label{eq:2,2} \tag{26}
\end{equation}
\]

    \hypertarget{final-hamiltonian-matrix}{%
\subsection{Final Hamiltonian matrix}\label{final-hamiltonian-matrix}}

These five states can in turn be used as computational basis states in
order to define the Hamiltonian matrix to be diagonalized. The matrix
elements are given by \(\langle J,J_z \vert H \vert J',J_z' \rangle\).
The Hamiltonian is hermitian and we obtain after all this labor of ours

    \hypertarget{eq:HJ=2}{}

\[
\begin{equation}
H_{J = 2} =
\begin{bmatrix}
-2\varepsilon & 0 & \sqrt{6}V & 0 & 0 \\
0 & -\varepsilon + 3W & 0 & 3V & 0 \\
\sqrt{6}V & 0 & 4W & 0 & \sqrt{6}V \\
0 & 3V & 0 & \varepsilon + 3W & 0 \\
0 & 0 & \sqrt{6}V & 0 & 2\varepsilon
\end{bmatrix}
\label{eq:HJ=2} \tag{27}
\end{equation}
\]

    \hypertarget{comparing-with-standard-diagonalization}{%
\subsection{Comparing with standard
diagonalization}\label{comparing-with-standard-diagonalization}}

We can now select a set of parameters and diagonalize the above matrix.
We select \(\epsilon = 2\), \(V = -1/3\), \(W = -1/4\) and our matrix
becoes

    \[
H_{J=2}^{(1)} =
\begin{bmatrix}
-4 & 0 & -\sqrt{6}/3 & 0 & 0 \\
0 & -2 - 3/4 & 0 & -1 & 0 \\
-\sqrt{6}/3 & 0 & -1 & 0 & -\sqrt{6}/3 \\
0 & -1 & 0 & 2 + -3/4 & 0 \\
0 & 0 & -\sqrt{6}/3 & 0 & 4
\end{bmatrix},
\]

    which gives the eigenvalue

    \[
D = \begin{bmatrix}
-4.21288 &  0 &  0 &  0 &  0 \\
0 & -2.98607  & 0  & 0  & 0 \\
0 &  0 & -0.91914  & 0  & 0 \\
0 &  0 & 0   & 1.48607  & 0 \\
0 &  0  & 0  & 0  & 4.13201
\end{bmatrix}.
\]

    The lowest state has an admixture of basis states given by

    \[
\vert \psi_0\rangle = 0.96735\vert2,-2\rangle + 0.25221\vert 2,0\rangle + 0.02507\vert 2,2\rangle,
\]

    with energy \(E_0 = -4.21288\).

    \hypertarget{comparing-with-standard-diagonalization-other-parameters}{%
\subsection{Comparing with standard diagonalization, other
parameters}\label{comparing-with-standard-diagonalization-other-parameters}}

We can now change the parameters to \(\varepsilon = 2\), \(V = -4/3\),
\(W = -1\). Our matrix reads then

    \[
H_{J=2}^{(2)} =
\begin{bmatrix}
-4 & 0 & -4\sqrt{6}/3 & 0 & 0 \\
0 & -5 & 0 & -4 & 0 \\
-4\sqrt{6}/3 & 0 & -4 & 0 & -4\sqrt{6}/3 \\
0 & -4 & 0 & -1 & 0 \\
0 & 0 & -4\sqrt{6}/3 & 0 & 4
\end{bmatrix},
\]

    with the following eigenvalues

    \[
D = \begin{bmatrix}
-7.75122 &  0 &  0 &  0  & 0 \\
0 & -7.47214  & 0  & 0  & 0 \\
0 &  0  & -1.55581 &  0  & 0 \\
0 &  0  & 0  & 1.47214  & 0 \\
0 &  0  & 0  & 0  & 5.30704
\end{bmatrix}.
\]

    The new ground state (lowest state) has the following admixture of
computational basis states

    \[
\vert \psi_0\rangle = 0.64268\vert 2,-2\rangle + 0.73816\vert 2,0\rangle + 0.20515 \vert 2,2\rangle,
\]

    with energy \(E_0 = -7.75122\).

    \hypertarget{analysis}{%
\subsection{Analysis}\label{analysis}}

For the first set of parameters, the likelihood for observing the system
in the computational basis state \(\vert 2,-2 \rangle\) is rather large.
This is expected since the interaction matrix elements are smaller than
the single-particle energies. For the second case, with larger matrix
elements, we see a much stronger mixing of the other states, again as
expected due to the ratio of the interaction matrix elements and the
single-particle energies.


    % Add a bibliography block to the postdoc
    
    
    
\end{document}
