\documentclass[prc]{revtex4}
\usepackage[dvips]{graphicx}
\usepackage{mathrsfs}
\usepackage{amsfonts}
\usepackage{lscape}

\usepackage{epic,eepic}
\usepackage{amsmath}
\usepackage{amssymb}
\usepackage[dvips]{epsfig}
\usepackage[T1]{fontenc}
\usepackage{hyperref}
\usepackage{bezier}
\usepackage{pstricks}
\usepackage{dcolumn}% Align table columns on decimal point
\usepackage{bm}% bold math
%\usepackage{braket}
\usepackage[dvips]{graphicx}
\usepackage{pst-plot}

\newcommand{\One}{\hat{\mathbf{1}}}
\newcommand{\eff}{\text{eff}}
\newcommand{\Heff}{\hat{H}_\text{eff}}
\newcommand{\Veff}{\hat{V}_\text{eff}}
\newcommand{\braket}[1]{\langle#1\rangle}
\newcommand{\Span}{\operatorname{sp}}
\newcommand{\tr}{\operatorname{trace}}
\newcommand{\diag}{\operatorname{diag}}
\newcommand{\bra}[1]{\left\langle #1 \right|}
\newcommand{\ket}[1]{\left| #1 \right\rangle}
\newcommand{\element}[3]
    {\bra{#1}#2\ket{#3}}

\newcommand{\normord}[1]{
    \left\{#1\right\}
}

\usepackage{amsmath}
\begin{document}
\title{Exercises FYS4480/9480, week 40, September 29-October 3, 2025}
%\author{}
\maketitle



\subsection*{Exercise 1}

Last week we considered a Slater determinant built up of
single-particle orbitals $\psi_{\lambda}$, with $\lambda =
1,2,\dots,N$.

The unitary transformation
\[
\vert a\rangle  = \sum_{\lambda} C_{a\lambda}\vert \lambda\rangle,
\]
brings us into the new basis.  The new basis has quantum numbers
$a=1,2,\dots,N$.  We showed that the new basis is orthonormal given
that the old basis is orthonormal and that the new Slater determinant
constructed from the new single-particle wave functions can be written
as the determinant based on the previous basis and the determinant of
the matrix $C$.  We showed then that the old and the new Slater
determinants are equal up to a complex constant with absolute value
unity.  The resulting Slater determinants are orthogonal if we employ
a single-particle basis which is orthogonal.

Define a Slater determinant $\vert\Phi_0\rangle$ as an ansatz for the ground state
using the single-particle basis functions $\vert\mu\rangle$.  Assume
that you have fille all states $\mu$ up to the Fermi level and show
that the expectation value for the ground state with a Hamiltonian
that contains at most two-body interactions can be written as

\[
 \langle \Phi_0 \vert \hat{H}\vert \Phi_0\rangle 
  = \sum_{\mu=1}^N \langle \mu | \hat{h}_0 | \mu \rangle +
  \frac{1}{2}\sum_{{\mu}=1}^N\sum_{{\nu}=1}^N \langle \mu\nu|\hat{v}|\mu\nu\rangle_{AS}.
\]

Explain what the different terms stand for and express the above
equation in a diagrammatic form.

We define then a new Slater determinant $\vert\Psi_0\rangle$ defined by the
single-particle basis function $\vert a\rangle$, where the Fermi level is
define by filling all single-particle states $a$ below the Fermi level. The new basis is also orthonormal.

Show that you can write the expectation value as

\[
  \langle \Psi_0 \vert \hat{H}\vert \Psi_0\rangle 
  = \sum_{i=1}^N \langle i | h | i \rangle +
  \frac{1}{2}\sum_{ij=1}^N\langle ij|\hat{v}|ij\rangle_{AS}.
\]

Using the new single-particle basis $\vert a\rangle$ (romans), show that you can rewrite the last equation in terms of the
basis functions $\vert \lambda\rangle$ (greeks)
\begin{equation}
  \langle \Psi_0 \vert \hat{H}\vert \Psi_0\rangle 
  = \sum_{i=1}^N \sum_{\alpha\beta} C^*_{i\alpha}C_{i\beta}\langle \alpha | h | \beta \rangle +
  \frac{1}{2}\sum_{ij=1}^N\sum_{{\alpha\beta\gamma\delta}} C^*_{i\alpha}C^*_{j\beta}C_{i\gamma}C_{j\delta}\langle \alpha\beta|\hat{v}|\gamma\delta\rangle_{AS}. 
\end{equation}

\end{document}



