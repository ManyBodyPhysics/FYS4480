\documentclass[a4wide,10pt]{article}

\listfiles               %  print all files needed to compile this document

\usepackage{relsize,makeidx,color,setspace,amsmath,amsfonts,amssymb}
\usepackage[table]{xcolor}
\usepackage{bm,ltablex,microtype}

\usepackage[pdftex]{graphicx}

\usepackage{fancyvrb} % packages needed for verbatim environments

\usepackage[T1]{fontenc}
%\usepackage[latin1]{inputenc}
\usepackage{ucs}
\usepackage[utf8x]{inputenc}

\usepackage{lmodern}         % Latin Modern fonts derived from Computer Modern

% Hyperlinks in PDF:
\definecolor{linkcolor}{rgb}{0,0,0.4}
\usepackage{hyperref}
\hypersetup{
    breaklinks=true,
    colorlinks=true,
    linkcolor=linkcolor,
    urlcolor=linkcolor,
    citecolor=black,
    filecolor=black,
    %filecolor=blue,
    pdfmenubar=true,
    pdftoolbar=true,
    bookmarksdepth=3   % Uncomment (and tweak) for PDF bookmarks with more levels than the TOC
    }
%\hyperbaseurl{}   % hyperlinks are relative to this root

\setcounter{tocdepth}{2}  % levels in table of contents

% --- fancyhdr package for fancy headers ---
\usepackage{fancyhdr}
\fancyhf{} % sets both header and footer to nothing
\renewcommand{\headrulewidth}{0pt}
\pagestyle{fancy}


% prevent orhpans and widows
\clubpenalty = 10000
\widowpenalty = 10000

% --- end of standard preamble for documents ---


% insert custom LaTeX commands...

\raggedbottom
\makeindex
\usepackage[totoc]{idxlayout}   % for index in the toc
\usepackage[nottoc]{tocbibind}  % for references/bibliography in the toc

%-------------------- end preamble ----------------------

\begin{document}

% matching end for #ifdef PREAMBLE

\newcommand{\exercisesection}[1]{\subsection*{#1}}


% ------------------- main content ----------------------



% ----------------- title -------------------------

\thispagestyle{empty}

\begin{center}
{\LARGE\bf
\begin{spacing}{1.25}
First midterm FYS4480 – Quantum mechanics for many-particle systems, deadline October 20 at midnight
\end{spacing}
}
\end{center}

% ----------------- author(s) -------------------------

\begin{center}
{\bf \href{{http://www.uio.no/studier/emner/matnat/fys/FYS4480/index-eng.html}}{FYS4480 – Quantum mechanics for many-particle systems}}
\end{center}

    \begin{center}
% List of all institutions:
\centerline{{\small Department of Physics, University of Oslo, Norway}}
\end{center}

\section*{Introduction}

\end{document}


\documentclass[prc]{revtex4}
\usepackage[dvips]{graphicx}
\usepackage{mathrsfs}
\usepackage{amsfonts}
\usepackage{lscape}

\usepackage{epic,eepic}
\usepackage{amsmath}
\usepackage{amssymb}
\usepackage[dvips]{epsfig}
\usepackage[T1]{fontenc}
\usepackage{hyperref}
\usepackage{bezier}
\usepackage{pstricks}
\usepackage{dcolumn}% Align table columns on decimal point
\usepackage{bm}% bold math
%\usepackage{braket}
\usepackage[dvips]{graphicx}
\usepackage{pst-plot}

\newcommand{\One}{\hat{\mathbf{1}}}
\newcommand{\eff}{\text{eff}}
\newcommand{\Heff}{\hat{H}_\text{eff}}
\newcommand{\Veff}{\hat{V}_\text{eff}}
\newcommand{\braket}[1]{\langle#1\rangle}
\newcommand{\Span}{\operatorname{sp}}
\newcommand{\tr}{\operatorname{trace}}
\newcommand{\diag}{\operatorname{diag}}
\newcommand{\bra}[1]{\left\langle #1 \right|}
\newcommand{\ket}[1]{\left| #1 \right\rangle}
\usepackage{amsmath}
\begin{document}

\subsection*{Oppgave 1, homogene system av fermioner}
Vi skal i denne oppgava holde oss til et system av  
fermioner med spinn $1/2$.  Her kan det v\ae re nyttig \aa\ se over for eksempel kapittel
3.4-3.6 i Raimes eller Gross, Runge og Heinonen kapittel 9 og 10.

Vi skal avgrense oss til det som kalles uendelig materie, et homogent system 
hvor en-partikkel tilstandene er gitt ved plan b\o lge funksjoner normert til et volum $\Omega$ for en boks med lengde $L$ (grensa $L\rightarrow \infty$ skal tas ved slutten av rekninga av ulike integral)
\[
\psi_{{\bf k}\sigma}({\bf r})= \frac{1}{\sqrt{\Omega}}\exp{(i{\bf kr})}\xi_{\sigma}
\]
hvor ${\bf k}$ er b\o lgetallet og $\xi_{\sigma}$ er  standard spinnfunksjoner for spinn opp eller ned  
\[ 
\xi_{\sigma=+1/2}=\left(\begin{array}{c} 1 \\ 0 \end{array}\right) \hspace{0.5cm}
\xi_{\sigma=-1/2}=\left(\begin{array}{c} 0 \\ 1 \end{array}\right).\]

Periodiske randbetingelser avgrenser de tillatte b\o lgetalla til 
\[
k_i=\frac{2\pi n_i}{L}\hspace{0.5cm} i=x,y,z \hspace{0.5cm} n_i=0,\pm 1,\pm 2, \dots
\]
Vi antar f\o rst at partiklene vekselvirker via ei sentral-symmetrisk
og translasjons invariant vekselvirkning $V(r_{12})$ med 
$r_{12}=|{\bf r}_1-{\bf r}_2|$. 
Vekselvirkninga er spinn uavhengig.

Den totale Hamilton operatoren best\aa r 
av kinetisk og potensiell energi 
\[
\hat{H} = \hat{T}+\hat{V}.
\]\newline
1a) Vis at operatoren for kinetisk energi kan skrives som
\[
\hat{T}=\sum_{{\bf k}\sigma}\frac{\hbar^2k^2}{2m}a_{{\bf k}\sigma}^{\dagger}a_{{\bf k}\sigma}.
\]
Finn antallsoperatoren $\hat{N}$ ogs\aa\ og
sett  deretter opp det tilsvarende uttrykket for vekselvirkninga 
$\hat{V}$ uttrykt ved kreasjons og annihilasjons operatorer.
Du skal ogs\aa\ sette opp uttrykket for vekselvirkninga i $k$-rommet
sj\o l om  $V$ ikke er spesifisert, det vil si sett opp uttrykket for den Fourier transformerte vekselvirkninga $\langle {\bf k}_i{\bf k}_j| V | {\bf k}_m{\bf k}_n\rangle$.\newline
1b) Vi antar n\aa\ at $V(r_{12}) < 0$ og at integralet 
$\int |V(x)| d^3x < \infty$.

Bruk operatorforma til $\hat{H}$ fra forrige oppgave og rekn ut 
$E_0=\bra{\Phi_{0}}H\ket{\Phi_{0}}$ for
dette systemet til f\o rste orden i perturbasjonsteori som funksjon av tettheten $\rho=N/\Omega$. 
Tilstanden $\ket{\Phi_{0}}$  er Slater determinanten gitt ved \aa\ fylle opp
alle tilstander til og med Fermi niv\aa et.
Vis at systemet vil kollapse. Kommenter resultatet.\newline
1c) Vi skal n\aa\ studere den degenererte elektrongassen. Hamilton operatoren
er n\aa\ gitt ved 
\[
\hat{H}=\hat{H}_{el}+\hat{H}_{b}+\hat{H}_{el-b},
\]
med den elektroniske delen gitt ved 
\[
\hat{H}_{el}=\sum_{i=1}^N\frac{p_i^2}{2m}+\frac{e^2}{2}\sum_{i\ne j}\frac{e^{-\mu |{\bf r}_i-{\bf r}_j|}}{|{\bf r}_i-{\bf r}_j|},
\]
hvor vi har innf\o rt en eksplisitt konvergensfaktor (grensa $\mu\rightarrow 0$ tas etter utrekning av ulike integral), tilsvarende er 
\[
\hat{H}_{b}=\frac{e^2}{2}\int\int d{\bf r}d{\bf r}'\frac{n({\bf r})n({\bf r}')e^{-\mu |{\bf r}-{\bf r}'|}}{|{\bf r}-{\bf r}'|},
\]
energien til den positive bakgrunnsladninga med tetthet $n({\bf r})=N/\Omega$,
og
\[
\hat{H}_{el-b}=-\frac{e^2}{2}\sum_{i=1}^N\int d{\bf r}\frac{n({\bf r})e^{-\mu |{\bf r}-{\bf x}_i|}}{|{\bf r}-{\bf x}_i|},
\]
er  vekselvirkninga mellom elektrona og den positive bakgrunnen. 

Vis at 
\[
\hat{H}_{b}=\frac{e^2}{2}\frac{N^2}{\Omega}\frac{4\pi}{\mu^2},
\]
og
\[
\hat{H}_{el-b}=-e^2\frac{N^2}{\Omega}\frac{4\pi}{\mu^2}.
\]
Vis deretter at den endelige Hamilton operatoren kan skrives som
\[
H=H_{0}+H_{I},
\]
med
\[
H_{0}={\displaystyle\sum_{{\bf k}\sigma}}
\frac{\hbar^{2}k^{2}}{2m}a_{{\bf k}\sigma}^{\dagger}
a_{{\bf k}\sigma},
\]
og
\[
H_{I}=\frac{e^{2}}{2\Omega}{\displaystyle\sum_{\sigma_{1}
\sigma_{2}}}{\displaystyle
\sum_{{\bf q}\neq 0,{\bf k},{\bf p}}}\frac{4\pi}{q^{2}}
a_{{\bf k}+{\bf q},\sigma_{1}}^{\dagger}
a_{{\bf p}-{\bf q},\sigma_{2}}^{\dagger}
a_{{\bf p}\sigma_{2}}a_{{\bf k}\sigma_{1}}.
\] \newline
1d) Rekn deretter ut
$E_0/N=\bra{\Phi_{0}}H\ket{\Phi_{0}}/N$ for
dette systemet til f\o rste orden i perturbasjonsteori. 
Vis at, ved  \aa\ bruke
\[
\rho= \frac{k_F^3}{3\pi^2}=\frac{3}{4\pi r_0^3},
\]
med $\rho=N/\Omega$ og $r_0$ 
som er radien til ei kule som representerer volumet et ledningselektron opptar
og  Bohr radien $a_0=\hbar^2/e^2m$, s\aa\ kan energien per elektron
skrives som
\[
E_0/N=\frac{e^2}{2a_0}\left[\frac{2.21}{r_s^2}-\frac{0.916}{r_s}\right].
\]
Her har vi definert $r_s=r_0/a_0$ som er en dimensjonl\o s st\o rrelse.
Plott resultatet og tolk det, spesielt i sammenheng med det du fant i
oppgave 1b). Hva er det som gj\o r at dette systemet ikke kollapser?

Rekn ogs\aa\ ut termodynamiske st\o rrelser som trykket, gitt ved
\[
P=-\left(\frac{\partial E}{\partial \Omega}\right)_N,\]
og bulk modulusen
\[
B=-\Omega\left(\frac{\partial P}{\partial \Omega}\right)_N,\]
og kommenter resultatene.\newline
1e) En-partikkel Hartree-Fock energien er gitt ved
\[
\varepsilon_{k}^{HF}=\frac{\hbar^{2}k^{2}}{2m}-\frac{e^{2}
k_{F}}{2\pi}
\left[
2+\frac{k_{F}^{2}-k^{2}}{kk_{F}}ln\left\vert\frac{k+k_{F}}
{k-k_{F}}\right\vert
\right].
\]
(Du skal ikke utlede dette uttrykket).
Hvordan kan du bruke Hartree-Fock energien 
til \aa\ finne grunntilstands energien?
Er det forskjeller mellom resultatene fra ei slik rekning og det du fant
i forrige punkt? Kommenter. 

\subsection*{Oppgave 2, perturbasjonsteori}
La $H=H_0 +V$ og $\ket{\phi_n}$ er egentilstandene til $H_0$ og
at $\ket{\psi_n}$ er de tilsvarende for $H$. Anta at grunntilstandene
$\ket{\phi_0}$ og $\ket{\psi_0}$ ikke er degenererte. \newline
2a) Vis at
\[
E_0 -\varepsilon_0 =\frac{\bra{\phi_0} V\ket{\psi_0}}
{\left\langle \phi_0 | \psi_0 \right\rangle},
\]
med $H\ket{\psi_0} =E_0\ket{\psi_0}$ og
$H_0\ket{\phi_0} =\varepsilon_0\ket{\phi_0}$.
Definer operatorene $P=\ket{\phi_0}\bra{\phi_0}$ og $Q=1-P$. Vis at
disse operatorene er idempotente.
Vis ogs\aa\ at for enhver $z$ s\aa\ gjelder
\[
\ket{\psi_0}=
	    \left\langle \phi_0 | \psi_0 \right\rangle
	    \sum_{n=0}^{\infty}\left(\frac{Q}{z-H_0}(z-E_0+V)\right)^n
	    \ket{\phi_0},
\]
og
\[
E_0=\varepsilon_0+
    \sum_{n=0}^{\infty}\bra{\phi_0}V
    \left(\frac{Q}{z-H_0}(z-E_0+V)\right)^n
    \ket{\phi_0}.
\]
Diskuter disse resultata for $z=E_0$ (Brillouin-Wigner pert. teori)
og $z=\varepsilon_0$ (Rayleigh-Schr\"{o}dinger pert. teori).
Sammenhold de f\o rste ledd i utviklinga (til tredje orden i vekselvirkninga).\newline
2b) Betrakt deretter et system av to fermioner i parorbitalene
$\ket{m_0}$ og $\ket{-m_0}$ i et enkelt skall $j$ med $2j+1>2$.
Projeksjonen av $j$ tar verdiene $m=-j,-j+1,\dots j-1,j$.
Anta at matriseelementa for vekselvirkninga mellom partiklene er av
forma
\[
\bra{m,-m}v\ket{m',-m'}=-G.
\]
Dette er et enkelt eksempel p\aa\ ei s\aa kalla parkraft, som dukker opp
i for eksempel kjernefysikk og teorier om superledning.

Hvordan vil du definere en-partikkel delen av Hamilton operatoren?  

Vis at Brillouin-Wigner utviklinga  fra 2a) kan summeres til
\aa\ gi
\[
E_0=-(j+1/2)G.
\]
Vis ved ei direkte diagonalisering av Hamilton funksjonen at dette
er den eksakte energien og finn den eksakte ikke-degenererte grunntilstanden. 
\newline
2c)
Bruk deretter Rayleigh-Schr\"{o}dinger
perturbasjonsteori og diskuter forskjellene.






\end{document}









