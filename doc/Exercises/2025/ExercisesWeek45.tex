
\documentclass[prc]{revtex4}
\usepackage[dvips]{graphicx}
\usepackage{mathrsfs}
\usepackage{amsfonts}
\usepackage{lscape}

\usepackage{epic,eepic}
\usepackage{amsmath}
\usepackage{amssymb}
\usepackage[dvips]{epsfig}
\usepackage[T1]{fontenc}
\usepackage{hyperref}
\usepackage{bezier}
\usepackage{pstricks}
\usepackage{dcolumn}% Align table columns on decimal point
\usepackage{bm}% bold math
%\usepackage{braket}
\usepackage[dvips]{graphicx}
\usepackage{pst-plot}

\newcommand{\One}{\hat{\mathbf{1}}}
\newcommand{\eff}{\text{eff}}
\newcommand{\Heff}{\hat{H}_\text{eff}}
\newcommand{\Veff}{\hat{V}_\text{eff}}
\newcommand{\braket}[1]{\langle#1\rangle}
\newcommand{\Span}{\operatorname{sp}}
\newcommand{\tr}{\operatorname{trace}}
\newcommand{\diag}{\operatorname{diag}}
\newcommand{\bra}[1]{\left\langle #1 \right|}
\newcommand{\ket}[1]{\left| #1 \right\rangle}
\newcommand{\element}[3]
    {\bra{#1}#2\ket{#3}}

\newcommand{\normord}[1]{
    \left\{#1\right\}
}

\usepackage{amsmath}
\begin{document}

\title{Exercises FYS4480/9480, week 45, November 3-7, 2025}
%\author{}
\maketitle
Let $\hat{H}=\hat{H}_0 +\hat{H}_I$ and $\ket{\Phi_n}$ be the eigenstates of $\hat{H}_0$ and that
$\ket{\Psi_n}$ are the corresponding ones for $\hat{H}$. 
Assume that the ground states
$\ket{\Phi_0}$ and $\ket{\Psi_0}$ are not degenerate. 
We can then write the energy of the ground state as
\[
E_0 -\varepsilon_0 =\frac{\bra{\Phi_0} \hat{H}_I\ket{\Psi_0}}
{\left\langle \Phi_0 | \Psi_0 \right\rangle},
\]
with $\hat{H}\ket{\Psi_0} =E_0\ket{\Psi_0}$ and
$H_0\ket{\Phi_0} =\varepsilon_0\ket{\Phi_0}$. We 
define also the projection operators $\hat{P}=\ket{\Phi_0}\bra{\Phi_0}$ and $\hat{Q}=1-\hat{P}$. 
These operators satisfy $\hat{P}^2=\hat{P}$, $\hat{Q}^2=\hat{Q}$ and $\hat{P}\hat{Q}=0$.
\begin{enumerate}
\item[a)]
Show that for any  $\omega$ we have can write the ground state energy as
\[
E_0=\varepsilon_0+
    \sum_{n=0}^{\infty}\bra{\Phi_0}\hat{H}_I
    \left(\frac{\hat{Q}}{\omega-\hat{H}_0}(\omega-E_0+\hat{H}_I)\right)^n
    \ket{\Phi_0}.
\]
\item[b)]
Discuss these results  for $\omega=E_0$ (Brillouin-Wigner perturbation theory)
and $\omega=\varepsilon_0$ (Rayleigh-Schr\"{o}dinger perturbation theory).
Compare the first few terms in these expansions and discuss the differences.
\item[c)] Show that the onebody part of the Hamiltonian
    \[
        \hat{H}_0 = \sum_{pq} \element{p}{\hat{h}_0}{q} a^\dagger_p a_q
    \]
can be written, using standard annihilation and creation operators, in normal-ordered form as 
    \[
        \hat{H}_0 = \sum_{pq} \element{p}{\hat{h}_0}{q} a^\dagger_p a_q=\sum_{pq} \element{p}{\hat{h}_0}{q} \left\{a^\dagger_p a_q\right\} +
                \sum_i \element{i}{\hat{h}_0}{i}, \]
and that the two-body Hamiltonian 
    \[
        \hat{H}_I = \frac{1}{4} \sum_{pqrs} \element{pq}{\hat{v}}{rs} a^\dagger_p a^\dagger_q a_s  a_r,
    \]
can be written 
\[
\hat{H}_I=\frac{1}{4} \sum_{pqrs} \element{pq}{\hat{v}}{rs} \normord{a^\dagger_p a^\dagger_q a_s  a_r}
            + \sum_{pqi} \element{pi}{\hat{v}}{qi} \normord{a^\dagger_p a_q} + \frac{1}{2} \sum_{ij} \element{ij}{\hat{v}}{ij}
\]
Explain the meaning of the various symbols. Which reference 
vacuum has been used?  Write down the diagrammatic representation of all these terms. 
\item[d)]  Use the diagrammatic representation of the Hamiltonian operator 
from the previous exercise to set up all diagrams (use either 
anti-symmetrized Goldstone diagrams or Hugenholz diagrams)
to second order (including the reference energy) in Rayleigh Schr\"odinger perturbation theory 
that contribute to the expectation value of $E_0$.

Use the diagram rules to write down their closed-form expressions.
\end{enumerate}
We consider now a one-particle system with the following Hamiltonian
$\hat{H}=\hat{H}_{0}+\hat{H}_{I}$ where
\[
\hat{H}_{0}=\sum_{p}\varepsilon_{p}a_{p}^{\dagger}a_{p},
\]
and
\[
\hat{H}_{I}=g\sum_{pq}a_{p}^{\dagger}a_{q}.
\]
The strength parameter $g$ is a real constant.  
The first part of the Hamiltonian plays the role of the unperturbed part,
with 
\[
\element{p}{\hat{h}_0}{q}=\delta_{p,q}\varepsilon_{p}.
\]
We have only two one-particle states, with $\varepsilon_{1}< \varepsilon_{2}$, and we will let the first state $p=1$ correspond to the
model space and the other, $p=2$, correspond to the excluded space.  Use labels $ijk\dots$ for hole
states (below the Fermi level) and labels $abc\dots$ for particle (virtual) 
states (above the Fermi level).
\begin{enumerate}
\item[e)] Use the results from exercise c) to write down the above  Hamiltonian 
in a normal-ordered form and set up all diagrams.  Use an $X$  to indicate the interaction part $H_I$.
\item[f)] Define the ground state (which is our model space) as 
\[
\ket{\Phi_0}=a_{i}^{\dagger}\ket{0}=a_{1}^{\dagger}\ket{0},
\]
and the excited state as
\[
\ket{\Phi_i^a}=a_{a}^{\dagger}a_i\ket{\Phi_0},
\]
where $a=2$ and $i=1$.  Set up the Hamiltonian matrix (a $2\times 2$ matrix) and find the 
exact energy 
and expand the exact result for the ground state in terms of the
parameter $g$.  
\item[g)] Find the ground state energy to third order in Rayleigh-Sch\"{o}dinger 
perturbation theory and compare the results with the expansion of the exact energy from the previous exercise. Write down all diagrams which contribute and comment your results.
\end{enumerate}


\end{document}


