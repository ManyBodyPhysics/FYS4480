\documentclass[prc]{revtex4}
\usepackage[dvips]{graphicx}
\usepackage{mathrsfs}
\usepackage{amsfonts}
\usepackage{lscape}

\usepackage{epic,eepic}
\usepackage{amsmath}
\usepackage{amssymb}
\usepackage[dvips]{epsfig}
\usepackage[T1]{fontenc}
\usepackage{hyperref}
\usepackage{bezier}
\usepackage{pstricks}
\usepackage{dcolumn}% Align table columns on decimal point
\usepackage{bm}% bold math
%\usepackage{braket}
\usepackage[dvips]{graphicx}
\usepackage{pst-plot}

\newcommand{\One}{\hat{\mathbf{1}}}
\newcommand{\eff}{\text{eff}}
\newcommand{\Heff}{\hat{H}_\text{eff}}
\newcommand{\Veff}{\hat{V}_\text{eff}}
\newcommand{\braket}[1]{\langle#1\rangle}
\newcommand{\Span}{\operatorname{sp}}
\newcommand{\tr}{\operatorname{trace}}
\newcommand{\diag}{\operatorname{diag}}
\newcommand{\bra}[1]{\left\langle #1 \right|}
\newcommand{\ket}[1]{\left| #1 \right\rangle}
\newcommand{\element}[3]
    {\bra{#1}#2\ket{#3}}

\newcommand{\normord}[1]{
    \left\{#1\right\}
}

\usepackage{amsmath}
\begin{document}

\title{Exercises FYS4480, week 34, August 19-23, 2024}
%\author{}
\maketitle
The exercises this week are mainly meant as reminders of specific linear algebra elements which we will use throughout the course.

\subsection*{Exercise 1, unitary transformations and orthogonality}

In many-body theories 
it is common to  expand a new basis using a known basis  and vary the expansion coefficients. Often these expansion coefficients represent the matrix elements of a unitary or orthogonal matrix., 

We define our new basis by performing a unitary transformation 
on our original orthogonal and normalized (orthonormal for short)   basis, labelled with greek indices and with a Dirac notation, as
\[
\vert \psi_p\rangle  = \sum_{\lambda} C_{p\lambda}\vert \phi_{\lambda}\rnagle. 
\]
Assuming that the  coefficients $C_{p\lambda}$ belong to a unitary or orthogonal trasformation and 
show that the new basis is orthogonal and normalized as well.


\subsection*{Exercise 2, spectral decomposition}


\subsection{Exercise 1, unitary matrices and orthogonality}

\end{enumerate}

\end{document}





Before we develop for example the Hartree-Fock equations, there is another very useful property of determinants that we will use both in connection with Hartree-Fock calculations and later shell-model calculations.  

Consider the following determinant
!bt
\[
\left| \begin{array}{cc} \alpha_1b_{11}+\alpha_2sb_{12}& a_{12}\\
                         \alpha_1b_{21}+\alpha_2b_{22}&a_{22}\end{array} \right|=\alpha_1\left|\begin{array}{cc} b_{11}& a_{12}\\
                         b_{21}&a_{22}\end{array} \right|+\alpha_2\left| \begin{array}{cc} b_{12}& a_{12}\\b_{22}&a_{22}\end{array} \right|
\]
!et







We can generalize this to  an $n\times n$ matrix and have 
!bt
\[
\left| \begin{array}{cccccc} a_{11}& a_{12} & \dots & \sum_{k=1}^n c_k b_{1k} &\dots & a_{1n}\\
a_{21}& a_{22} & \dots & \sum_{k=1}^n c_k b_{2k} &\dots & a_{2n}\\
\dots & \dots & \dots & \dots & \dots & \dots \\
\dots & \dots & \dots & \dots & \dots & \dots \\
a_{n1}& a_{n2} & \dots & \sum_{k=1}^n c_k b_{nk} &\dots & a_{nn}\end{array} \right|=
\sum_{k=1}^n c_k\left| \begin{array}{cccccc} a_{11}& a_{12} & \dots &  b_{1k} &\dots & a_{1n}\\
a_{21}& a_{22} & \dots &  b_{2k} &\dots & a_{2n}\\
\dots & \dots & \dots & \dots & \dots & \dots\\
\dots & \dots & \dots & \dots & \dots & \dots\\
a_{n1}& a_{n2} & \dots &  b_{nk} &\dots & a_{nn}\end{array} \right| .
\]
!et
This is a property we will use in our Hartree-Fock discussions. 




We can generalize the previous results, now 
with all elements $a_{ij}$  being given as functions of 
linear combinations  of various coefficients $c$ and elements $b_{ij}$,
!bt
\[
\left| \begin{array}{cccccc} \sum_{k=1}^n b_{1k}c_{k1}& \sum_{k=1}^n b_{1k}c_{k2} & \dots & \sum_{k=1}^n b_{1k}c_{kj}  &\dots & \sum_{k=1}^n b_{1k}c_{kn}\\
\sum_{k=1}^n b_{2k}c_{k1}& \sum_{k=1}^n b_{2k}c_{k2} & \dots & \sum_{k=1}^n b_{2k}c_{kj} &\dots & \sum_{k=1}^n b_{2k}c_{kn}\\
\dots & \dots & \dots & \dots & \dots & \dots \\
\dots & \dots & \dots & \dots & \dots &\dots \\
\sum_{k=1}^n b_{nk}c_{k1}& \sum_{k=1}^n b_{nk}c_{k2} & \dots & \sum_{k=1}^n b_{nk}c_{kj} &\dots & \sum_{k=1}^n b_{nk}c_{kn}\end{array} \right|=det(\mathbf{C})det(\mathbf{B}),
\]
!et
where $det(\mathbf{C})$ and $det(\mathbf{B})$ are the determinants of $n\times n$ matrices
with elements $c_{ij}$ and $b_{ij}$ respectively.  
This is a property we will use in our Hartree-Fock discussions. Convince yourself about the correctness of the above expression by setting $n=2$. 






With our definition of the new basis in terms of an orthogonal basis we have
!bt
\[
\psi_p(x)  = \sum_{\lambda} C_{p\lambda}\phi_{\lambda}(x).
\]
!et
If the coefficients $C_{p\lambda}$ belong to an orthogonal or unitary matrix, the new basis
is also orthogonal. 
Our Slater determinant in the new basis $\psi_p(x)$ is written as
!bt
\[
\frac{1}{\sqrt{N!}}
\left| \begin{array}{ccccc} \psi_{p}(x_1)& \psi_{p}(x_2)& \dots & \dots & \psi_{p}(x_N)\\
                            \psi_{q}(x_1)&\psi_{q}(x_2)& \dots & \dots & \psi_{q}(x_N)\\  
                            \dots & \dots & \dots & \dots & \dots \\
                            \dots & \dots & \dots & \dots & \dots \\
                     \psi_{t}(x_1)&\psi_{t}(x_2)& \dots & \dots & \psi_{t}(x_N)\end{array} \right|=\frac{1}{\sqrt{N!}}
\left| \begin{array}{ccccc} \sum_{\lambda} C_{p\lambda}\phi_{\lambda}(x_1)& \sum_{\lambda} C_{p\lambda}\phi_{\lambda}(x_2)& \dots & \dots & \sum_{\lambda} C_{p\lambda}\phi_{\lambda}(x_N)\\
                            \sum_{\lambda} C_{q\lambda}\phi_{\lambda}(x_1)&\sum_{\lambda} C_{q\lambda}\phi_{\lambda}(x_2)& \dots & \dots & \sum_{\lambda} C_{q\lambda}\phi_{\lambda}(x_N)\\  
                            \dots & \dots & \dots & \dots & \dots \\
                            \dots & \dots & \dots & \dots & \dots \\
                     \sum_{\lambda} C_{t\lambda}\phi_{\lambda}(x_1)&\sum_{\lambda} C_{t\lambda}\phi_{\lambda}(x_2)& \dots & \dots & \sum_{\lambda} C_{t\lambda}\phi_{\lambda}(x_N)\end{array} \right|,
\]
!et
which is nothing but $det(\mathbf{C})det(\Phi)$, with $det(\Phi)$ being the determinant given by the basis functions $\phi_{\lambda}(x)$. 



In our discussions hereafter we will use our definitions of single-particle states above and below the Fermi ($F$) level given by the labels
$ijkl\dots \le F$ for so-called single-hole states and $abcd\dots > F$ for so-called particle states.
For general single-particle states we employ the labels $pqrs\dots$. 




The energy functional is
!bt
\[
  E[\Phi] 
  = \sum_{\mu=1}^N \langle \mu | h | \mu \rangle +
  \frac{1}{2}\sum_{{\mu}=1}^N\sum_{{\nu}=1}^N \langle \mu\nu|\hat{v}|\mu\nu\rangle_{AS},
\]
!et 
we found the expression for the energy functional in terms of the basis function $\phi_{\lambda}(\mathbf{r})$. We then  varied the above energy functional with respect to the basis functions $|\mu \rangle$. 
Now we are interested in defining a new basis defined in terms of
a chosen basis as defined in Eq.~(ref{eq:newbasis}). We can then rewrite the energy functional as
!bt
\begin{equation}
  E[\Phi^{New}] 
  = \sum_{i=1}^N \langle i | h | i \rangle +
  \frac{1}{2}\sum_{ij=1}^N\langle ij|\hat{v}|ij\rangle_{AS}, label{FunctionalEPhi2}
\end{equation}
!et
where $\Phi^{New}$ is the new Slater determinant defined by the new basis of Eq.~(ref{eq:newbasis}). 





Using Eq.~(ref{eq:newbasis}) we can rewrite Eq.~(ref{FunctionalEPhi2}) as 
!bt
\begin{equation}
  E[\Psi] 
  = \sum_{i=1}^N \sum_{\alpha\beta} C^*_{i\alpha}C_{i\beta}\langle \alpha | h | \beta \rangle +
  \frac{1}{2}\sum_{ij=1}^N\sum_{{\alpha\beta\gamma\delta}} C^*_{i\alpha}C^*_{j\beta}C_{i\gamma}C_{j\delta}\langle \alpha\beta|\hat{v}|\gamma\delta\rangle_{AS}. label{FunctionalEPhi3}
\end{equation}
!et





