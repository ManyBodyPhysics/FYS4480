\documentclass[prc]{revtex4}
\usepackage[dvips]{graphicx}
\usepackage{mathrsfs}
\usepackage{amsfonts}
\usepackage{lscape}

\usepackage{epic,eepic}
\usepackage{amsmath}
\usepackage{amssymb}
\usepackage[dvips]{epsfig}
\usepackage[T1]{fontenc}
\usepackage{hyperref}
\usepackage{bezier}
\usepackage{pstricks}
\usepackage{dcolumn}% Align table columns on decimal point
\usepackage{bm}% bold math
%\usepackage{braket}
\usepackage[dvips]{graphicx}
\usepackage{pst-plot}

\newcommand{\One}{\hat{\mathbf{1}}}
\newcommand{\eff}{\text{eff}}
\newcommand{\Heff}{\hat{H}_\text{eff}}
\newcommand{\Veff}{\hat{V}_\text{eff}}
\newcommand{\braket}[1]{\langle#1\rangle}
\newcommand{\Span}{\operatorname{sp}}
\newcommand{\tr}{\operatorname{trace}}
\newcommand{\diag}{\operatorname{diag}}
\newcommand{\bra}[1]{\left\langle #1 \right|}
\newcommand{\ket}[1]{\left| #1 \right\rangle}
\newcommand{\element}[3]
    {\bra{#1}#2\ket{#3}}

\newcommand{\normord}[1]{
    \left\{#1\right\}
}

\usepackage{amsmath}
\begin{document}
\title{Exercises FYS4480, week 43, October 23-27, 2023}
%\author{}
\maketitle



\subsection*{Exercise 1}
The electron gas model allows closed form solutions for quantities like the 
single-particle Hartree-Fock energy.  The latter quantity is given by the following expression
\[
\varepsilon_{k}^{HF}=\frac{\hbar^{2}k^{2}}{2m}-\frac{e^{2}}
{V^{2}}\sum_{k'\leq
k_{F}}\int d\vec{r}e^{i(\vec{k'}-\vec{k})\vec{r}}\int
d\vec{r'}\frac{e^{i(\vec{k}-\vec{k'})\vec{r'}}}
{\vert\vec{r}-\vec{r'}\vert}
\]
a) Show that
\[
\varepsilon_{k}^{HF}=\frac{\hbar^{2}k^{2}}{2m}-\frac{e^{2}
k_{F}}{2\pi}
\left[
2+\frac{k_{F}^{2}-k^{2}}{kk_{F}}ln\left\vert\frac{k+k_{F}}
{k-k_{F}}\right\vert
\right]
\]
(Hint: Introduce the convergence factor 
$e^{-\mu\vert\vec{r}-\vec{r'}\vert}$
in the potential and use  $\sum_{\vec{k}}\rightarrow
\frac{V}{(2\pi)^{3}}\int d\vec{k}$ )
\newline
b) Rewrite the above result as a function of the density
\[
n= \frac{k_F^3}{3\pi^2}=\frac{3}{4\pi r_s^3},
\]
where $n=N/V$, $N$ being the number of particles, and $r_s$ is the radius of a sphere which represents the volum per conducting electron.  
It can be convenient to use the Bohr radius $a_0=\hbar^2/e^2m$.

For most metals we have a relation $r_s/a_0\sim 2-6$.

Make a plot of the free electron energy and the Hartree-Fock energy
and discuss the behavior around the Fermi surface. Extract also  
the Hartree-Fock band width $\Delta\varepsilon^{HF}$ defined as
\[ \Delta\varepsilon^{HF}=\varepsilon_{k_{F}}^{HF}-
\varepsilon_{0}^{HF}.\]
Compare this results with the corresponding one for a free electron and comment
your results. How large is the contribution due to the exchange term in the Hartree-Fock equation?\newline
c) We will now define a quantity called the effective mass.
For $\vert\vec{k}\vert$ near $k_{F}$, we can Taylor expand the Hartree-Fock energy as  
\[
\varepsilon_{k}^{HF}=\varepsilon_{k_{F}}^{HF}+
\left(\frac{\partial
\varepsilon_{k}^{HF}}{\partial k}\right)_{k_{F}}(k-k_{F})+\dots
\]
If we compare the latter with the corresponding expressiyon for the non-interacting system
\[
\varepsilon_{k}^{(0)}=\frac{\hbar^{2}k^{2}_{F}}{2m}+
\frac{\hbar^{2}k_{F}}{m}\left(k-k_{F}\right)+\dots ,
\]
we can define the so-called effective Hartree-Fock mass as
\[
m_{HF}^{*}\equiv\hbar^{2}k_{F}\left(
\frac{\partial\varepsilon_{k}^{HF}}
{\partial k}\right)_{k_{F}}^{-1}
\]
Compute $m_{HF}^{*}$ and comment your results after you have done 
point d). \newline
d) Show that the level density (the number of single-electron states
per unit energy) can be written as
\[
n(\varepsilon)=\frac{Vk^{2}}{2\pi^{2}}\left(
\frac{\partial\varepsilon}{\partial k}\right)^{-1}
\]
Calculate $n(\varepsilon_{F}^{HF})$ and comment the results from c) and d).




\end{document}



