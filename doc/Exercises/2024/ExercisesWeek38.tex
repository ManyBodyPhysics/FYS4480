\documentclass[prc]{revtex4}
\usepackage[dvips]{graphicx}
\usepackage{mathrsfs}
\usepackage{amsfonts}
\usepackage{lscape}

\usepackage{epic,eepic}
\usepackage{amsmath}
\usepackage{amssymb}
\usepackage[dvips]{epsfig}
\usepackage[T1]{fontenc}
\usepackage{hyperref}
\usepackage{bezier}
\usepackage{pstricks}
\usepackage{dcolumn}% Align table columns on decimal point
\usepackage{bm}% bold math
%\usepackage{braket}
\usepackage[dvips]{graphicx}
\usepackage{pst-plot}

\newcommand{\One}{\hat{\mathbf{1}}}
\newcommand{\eff}{\text{eff}}
\newcommand{\Heff}{\hat{H}_\text{eff}}
\newcommand{\Veff}{\hat{V}_\text{eff}}
\newcommand{\braket}[1]{\langle#1\rangle}
\newcommand{\Span}{\operatorname{sp}}
\newcommand{\tr}{\operatorname{trace}}
\newcommand{\diag}{\operatorname{diag}}
\newcommand{\bra}[1]{\left\langle #1 \right|}
\newcommand{\ket}[1]{\left| #1 \right\rangle}
\newcommand{\element}[3]
    {\bra{#1}#2\ket{#3}}

\newcommand{\normord}[1]{
    \left\{#1\right\}
}

\usepackage{amsmath}
\begin{document}

\title{Exercises FYS4480, week 38, September 18-22, 2023}
%\author{}
\maketitle




\subsection*{Exercise 1}

This exercise is a continuation of the exercises from last week on the so-called Lipkin model.
We considered
a state with all fermions in the lowest
single-particle state
\[
\ket{\Phi_{J_z=-2}} =a_{1-}^{\dagger}a_{2-}^{\dagger}
a_{3-}^{\dagger}a_{4-}^{\dagger}\ket{0}.
\]

This state has $J_{z}=-2$ and belongs to the set of projections for
$J=2$.  We will use the shorthand notation $\ket{J,J_z}$ for states
with different spon $J$ and spin projection $J_z$.  The other possible
states have $J_{z}=-1$, $J_{z}=0$, $J_{z}=1$ and $J_{z}=2$.

Use the raising or lowering operators $J_{+}$ and $J_{-}$  in order to construct the 
states for spin $J_{z}=-1$ $J_{z}=0$, $J_{z}=1$
and $J_{z}=2$.
The action of these two operators on a given state with spin $J$ and projection $J_z$
is given by ($\hbar = 1$) by
$J_+\ket{J,J_z}=\sqrt{J(J+1)-J_z(J_z+1)}\ket{J,J_z+1}$ and
$J_-\ket{J,J_z}=\sqrt{J(J+1)-J_z(J_z-1)}\ket{J,J_z-1}$.

\subsection*{Exercise 2}

\begin{enumerate}
\item[a)] Show that the onebody part of the Hamiltonian
    \begin{equation*}
        \hat{H}_0 = \sum_{pq} \element{p}{\hat{h}_0}{q} a^\dagger_p a_q
    \end{equation*}
can be written, using standard annihilation and creation operators, in normal-ordered form as 
    \begin{align*}
        \hat{H}_0 &= \sum_{pq} \element{p}{\hat{h}_0}{q} a^\dagger_p a_q \nonumber \\
            &= \sum_{pq} \element{p}{\hat{h}_0}{q} \left\{a^\dagger_p a_q\right\} + 
                \delta_{pq\in i} \sum_{pq} \element{p}{\hat{h}_0}{q} \nonumber \\
            &= \sum_{pq} \element{p}{\hat{h}_0}{q} \left\{a^\dagger_p a_q\right\} +
                \sum_i \element{i}{\hat{h}_0}{i}
    \end{align*}
Explain the meaning of the various symbols. Which reference 
vacuum has been used?


\item[b)] Show that the twobody part of the Hamiltonian
    \begin{equation*}
        \hat{H}_I = \frac{1}{4} \sum_{pqrs} \element{pq}{\hat{v}}{rs} a^\dagger_p a^\dagger_q a_s  a_r
    \end{equation*}
can be written, using standard annihilation and creation operators, in normal-ordered form as 
    \begin{align*}
    \hat{H}_I &= \frac{1}{4} \sum_{pqrs} \element{pq}{\hat{v}}{rs} a^\dagger_p a^\dagger_q a_s  a_r \nonumber \\
        &= \frac{1}{4} \sum_{pqrs} \element{pq}{\hat{v}}{rs} \normord{a^\dagger_p a^\dagger_q a_s  a_r}
            + \sum_{pqi} \element{pi}{\hat{v}}{qi} \normord{a^\dagger_p a_q} 
            + \frac{1}{2} \sum_{ij} \element{ij}{\hat{v}}{ij}
    \end{align*}
Explain again the meaning of the various symbols. The two-body matrix elements are anti-symmetrized.
\end{enumerate}


\end{document}


