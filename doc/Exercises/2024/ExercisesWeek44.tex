\documentclass[prc]{revtex4}
\usepackage[dvips]{graphicx}
\usepackage{mathrsfs}
\usepackage{amsfonts}
\usepackage{lscape}

\usepackage{epic,eepic}
\usepackage{amsmath}
\usepackage{amssymb}
\usepackage[dvips]{epsfig}
\usepackage[T1]{fontenc}
\usepackage{hyperref}
\usepackage{bezier}
\usepackage{pstricks}
\usepackage{dcolumn}% Align table columns on decimal point
\usepackage{bm}% bold math
%\usepackage{braket}
\usepackage[dvips]{graphicx}
\usepackage{pst-plot}

\newcommand{\One}{\hat{\mathbf{1}}}
\newcommand{\eff}{\text{eff}}
\newcommand{\Heff}{\hat{H}_\text{eff}}
\newcommand{\Veff}{\hat{V}_\text{eff}}
\newcommand{\braket}[1]{\langle#1\rangle}
\newcommand{\Span}{\operatorname{sp}}
\newcommand{\tr}{\operatorname{trace}}
\newcommand{\diag}{\operatorname{diag}}
\newcommand{\bra}[1]{\left\langle #1 \right|}
\newcommand{\ket}[1]{\left| #1 \right\rangle}
\newcommand{\element}[3]
    {\bra{#1}#2\ket{#3}}

\newcommand{\normord}[1]{
    \left\{#1\right\}
}

\usepackage{amsmath}
\begin{document}
\title{Exercises FYS4480 for week 44, October 28-November 1, 2024}

%\author{}
\maketitle

\subsection*{Exercise 1}
Let $H=H_0 +V$ and $\ket{\phi_n}$ be the eigenstates of $H_0$ and that
$\ket{\psi_n}$ are the corresponding ones for $H$. 
Assume that the ground states
$\ket{\phi_0}$ and $\ket{\psi_0}$ are not degenerate. Show that
\[
E_0 -\varepsilon_0 =\frac{\bra{\phi_0} V\ket{\psi_0}}
{\left\langle \phi_0 | \psi_0 \right\rangle},
\]
with $H\ket{\psi_0} =E\ket{\psi_0}$ and
$H_0\ket{\phi_0} =\varepsilon_0\ket{\phi_0}$.
\begin{enumerate}
\item[a)]
Define the new operators $P=\ket{\phi_0}\bra{\phi_0}$ and $Q=1-P$. Show that these operators are idempotent.
\item[b)]
Show that for any  $z$ we have 
\[
\ket{\psi_0}=
	    \left\langle \phi_0 | \psi_0 \right\rangle
	    \sum_{n=0}^{\infty}\left(\frac{Q}{z-H_0}(z-E_0+V)\right)^n
	    \ket{\phi_0},
\]
and
\[
E_0=\varepsilon_0+
    \sum_{n=0}^{\infty}\bra{\phi_0}V
    \left(\frac{Q}{z-H_0}(z-E_0+V)\right)^n
    \ket{\phi_0}.
\]
\item[c)]
Discuss these results  for $z=E_0$ (Brillouin-Wigner perturbation theory)
and $z=\varepsilon_0$ (Rayleigh-Schr\"{o}dinger perturbation theory).
Compare the first few terms in these expansions.
\end{enumerate}
\subsection*{Exercise 2}
Consider a system of two fermions spin $s=1/2$ in the pair-orbitals
$\ket{m_0}$ and $\ket{-m_0}$ in a single shell $j=l+s$ with $2j+1>2$, where $l$ is the orbital momentum with $l=0,1,2,\dots$.
Assume that the matrix elements for the interaction between the particles
takes the form
\[
\bra{m,-m}v\ket{m',-m'}=-G.
\]
\begin{enumerate}
\item[a)]
Show that the Brillouin-Wigner expansion from the previous exercise can be used
to give
\[
E_0=-(j+1/2)G.
\]
\item[b)] Show thereafter by direct diagonalization of the  Hamiltonian
matrix that this is the exact energy. Use thereafter 
Rayleigh-Schr\"{o}dinger perturbation theory and discuss the differences.
\end{enumerate}
\end{document}
