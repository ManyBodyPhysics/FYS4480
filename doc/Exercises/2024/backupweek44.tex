\documentclass[prc]{revtex4}
\usepackage[dvips]{graphicx}
\usepackage{mathrsfs}
\usepackage{amsfonts}
\usepackage{lscape}

\usepackage{epic,eepic}
\usepackage{amsmath}
\usepackage{amssymb}
\usepackage[dvips]{epsfig}
\usepackage[T1]{fontenc}
\usepackage{hyperref}
\usepackage{bezier}
\usepackage{pstricks}
\usepackage{dcolumn}% Align table columns on decimal point
\usepackage{bm}% bold math
%\usepackage{braket}
\usepackage[dvips]{graphicx}
\usepackage{pst-plot}

\newcommand{\One}{\hat{\mathbf{1}}}
\newcommand{\eff}{\text{eff}}
\newcommand{\Heff}{\hat{H}_\text{eff}}
\newcommand{\Veff}{\hat{V}_\text{eff}}
\newcommand{\braket}[1]{\langle#1\rangle}
\newcommand{\Span}{\operatorname{sp}}
\newcommand{\tr}{\operatorname{trace}}
\newcommand{\diag}{\operatorname{diag}}
\newcommand{\bra}[1]{\left\langle #1 \right|}
\newcommand{\ket}[1]{\left| #1 \right\rangle}
\newcommand{\element}[3]
    {\bra{#1}#2\ket{#3}}

\newcommand{\normord}[1]{
    \left\{#1\right\}
}

\usepackage{amsmath}
\begin{document}
\title{Exercises FYS4480, week 43, October 30-November 3, 2023}
%\author{}
\maketitle



\subsection*{Exercise 1}

We consider a system of electrons in infinite matter, the so-called
electron gas. This is a homogeneous system and the one-particle states
are given by plane wave function normalized to a volume $\Omega$ for a
box with length $L$ (the limit $L\rightarrow \infty$ is to be taken
after we have computed various expectation values)
\[
\psi_{{\bf k}\sigma}({\bf r})= \frac{1}{\sqrt{\Omega}}\exp{(i{\bf kr})}\xi_{\sigma}
\]
where ${\bf k}$ is the wave number and  $\xi_{\sigma}$ is a spin function for either spin up or down
\[ 
\xi_{\sigma=+1/2}=\left(\begin{array}{c} 1 \\ 0 \end{array}\right) \hspace{0.5cm}
\xi_{\sigma=-1/2}=\left(\begin{array}{c} 0 \\ 1 \end{array}\right).\]

We assume that we have periodic boundary conditions which limit the allowed wave numbers to
\[
k_i=\frac{2\pi n_i}{L}\hspace{0.5cm} i=x,y,z \hspace{0.5cm} n_i=0,\pm 1,\pm 2, \dots
\]
We assume first that the particles interact via a central, symmetric and translationally invariant
interaction  $V(r_{12})$ with
$r_{12}=|{\bf r}_1-{\bf r}_2|$.  The interaction is spin independent.

The total Hamiltonian consists then of kinetic and potential energy
\[
\hat{H} = \hat{T}+\hat{V}.
\]
\begin{enumerate}
\item[a)] Show that the operator for the kinetic energy can be written as
\[
\hat{T}=\sum_{{\bf k}\sigma}\frac{\hbar^2k^2}{2m}a_{{\bf k}\sigma}^{\dagger}a_{{\bf k}\sigma}.
\]
Find also the number operator $\hat{N}$ and find a corresponding expression for the interaction
$\hat{V}$ expressed with creation and annihilation operators.   The expression for the interaction
has to be written in  $k$ space, even though $V$ depends only on the relative distance. It means that you ned to set up the Fourier transform $\langle {\bf k}_i{\bf k}_j| V | {\bf k}_m{\bf k}_n\rangle$.
\item[b)] We assume that  $V(r_{12}) < 0$ and that the integral 
$\int |V(x)| d^3x < \infty$.

Use the operator form for $\hat{H}$ from the previous exercise and calculate
$E_0=\bra{\Phi_{0}}H\ket{\Phi_{0}}$ for this system to first order in perturbation theory
and express the result as a function of the density $\rho=N/\Omega$. 
The state $\ket{\Phi_{0}}$  is a Slater determinant determined by filling all states up to Fermi level.
Show that the system will collapse (you wil not be able to find an energy minimum). Comment your results.
\item[c)] We will now study the electron gas. The Hamilton operator is given by
\[
\hat{H}=\hat{H}_{el}+\hat{H}_{b}+\hat{H}_{el-b},
\]
with the electronic part
\[
\hat{H}_{el}=\sum_{i=1}^N\frac{p_i^2}{2m}+\frac{e^2}{2}\sum_{i\ne j}\frac{e^{-\mu |{\bf r}_i-{\bf r}_j|}}{|{\bf r}_i-{\bf r}_j|},
\]
where we have introduced an explicit convergence factor
(the limit $\mu\rightarrow 0$ is performed after having calculated the various integrals).
Correspondingly, we have
\[
\hat{H}_{b}=\frac{e^2}{2}\int\int d{\bf r}d{\bf r}'\frac{\rho({\bf r})\rho({\bf r}')e^{-\mu |{\bf r}-{\bf r}'|}}{|{\bf r}-{\bf r}'|},
\]
which is the energy contribution from the positive background charge with density
$\rho({\bf r})=N/\Omega$. Finally,
\[
\hat{H}_{el-b}=-\frac{e^2}{2}\sum_{i=1}^N\int d{\bf r}\frac{\rho({\bf r})e^{-\mu |{\bf r}-{\bf x}_i|}}{|{\bf r}-{\bf x}_i|},
\]
is the interaction between the electrons and the positive background.

Show that
\[
\hat{H}_{b}=\frac{e^2}{2}\frac{N^2}{\Omega}\frac{4\pi}{\mu^2},
\]
and
\[
\hat{H}_{el-b}=-e^2\frac{N^2}{\Omega}\frac{4\pi}{\mu^2}.
\]
Show thereafter that the final Hamiltonian can be written as 
\[
H=H_{0}+H_{I},
\]
with
\[
H_{0}={\displaystyle\sum_{{\bf k}\sigma}}
\frac{\hbar^{2}k^{2}}{2m}a_{{\bf k}\sigma}^{\dagger}
a_{{\bf k}\sigma},
\]
and
\[
H_{I}=\frac{e^{2}}{2\Omega}{\displaystyle\sum_{\sigma_{1}
\sigma_{2}}}{\displaystyle
\sum_{{\bf q}\neq 0,{\bf k},{\bf p}}}\frac{4\pi}{q^{2}}
a_{{\bf k}+{\bf q},\sigma_{1}}^{\dagger}
a_{{\bf p}-{\bf q},\sigma_{2}}^{\dagger}
a_{{\bf p}\sigma_{2}}a_{{\bf k}\sigma_{1}}.
\] 
\item[d)] Calculate
$E_0/N=\bra{\Phi_{0}}H\ket{\Phi_{0}}/N$ for
for this system to first order in the interaction.
Show that, by using
\[
\rho= \frac{k_F^3}{3\pi^2}=\frac{3}{4\pi r_0^3},
\]
with $\rho=N/\Omega$, $r_0$
being the radius of a sphere representing the volume an electron occupies 
and the Bohr radius $a_0=\hbar^2/e^2m$, 
that the energy per electron can be written as 
\[
E_0/N=\frac{e^2}{2a_0}\left[\frac{2.21}{r_s^2}-\frac{0.916}{r_s}\right].
\]
Here we have defined
$r_s=r_0/a_0$ to be a dimensionless quantity.

Plot your results and link your discussion to the result in exercise b). Why is this system stable?
\item[e)] 
Calculate thermodynamical quantities like the pressure, given by
\[
P=-\left(\frac{\partial E}{\partial \Omega}\right)_N,\]
and the bulk modulus
\[
B=-\Omega\left(\frac{\partial P}{\partial \Omega}\right)_N,\]
and comment your results.
\item[f)] 
The single-particle Hartree-Fock energies are given by the expression 
\[
\varepsilon_{k}^{HF}=\frac{\hbar^{2}k^{2}}{2m}-\frac{e^{2}
k_{F}}{2\pi}
\left[
2+\frac{k_{F}^{2}-k^{2}}{kk_{F}}ln\left\vert\frac{k+k_{F}}
{k-k_{F}}\right\vert
\right].
\]
(You don't need to calculate this quantity).  How can you use the Hartree-Fock energy to 
find the ground state energy? Are there differences between the Hartree-Fock results and those you 
found in exercise d)?  Comment your results.
\end{enumerate}



\end{document}



